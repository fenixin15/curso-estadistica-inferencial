\part{Estad\'{\i}stica descriptiva
\index{estadistica descriptiva@estad\'{\i}stica descriptiva}}

\chapter{Variables unidimensionals\index{variable!unidimensional}}

\section{Resum te\`oric}

L'estad\'{\i}stica descriptiva \'es la part de l'estad\'{\i}stica que 
s'encarrega de donar una descripci\'o num\`erica, 
\index{descripcio numerica@descripci\'o num\`erica} 
una ordenaci\'o\index{ordenacio@ordenaci\'o} 
i una simplificaci\'o de la informaci\'o recollida.

S'anomena {\bf poblaci\'o}\index{poblacio@poblaci\'o} 
el conjunt de refer\`encia sobre el qual recauran les
observacions.\index{observacio@observaci\'o} 
Direm {\bf unitat estad\'{\i}stica}\index{unitat estadistica@unitat estad\'{\i}stica} 
o {\bf individu}\index{individu} a cada element de
la poblaci\'o. Una {\bf mostra}\index{mostra} 
\'es un subconjunt de la poblaci\'o. Quan
l'observaci\'o es pot quantificar, els valors num\`erics de l'observaci\'o
s'anomenen {\bf variables estad\'{\i}stiques}.
\index{variable!estadistica@estad\'{\i}stica} 
Aquestes poden ser {\bf discretes},\index{variable!discreta}
si prenen un conjunt finit o numerable de valors, i {\bf cont\'{\i}nues},
\index{variable!continua@cont\'{\i}nua} si prenen
un conjunt no numerable de valors.


Per tal d'ordenar les dades obtengudes en l'observaci\'o d'una mostra 
o poblaci\'o, necessitam con\`eixer els conceptes seg\"uents:

\begin{enumerate}

\item {\bf Freq\"u\`encia absoluta $n_i$} 
\index{frequencia@freq\"u\`encia!absoluta}d'un valor $x_i$: el nombre de vegades
que apareix en l'observaci\'o.

\item {\bf Freq\"u\`encia relativa $f_i$} 
\index{frequencia@freq\"u\`encia!relativa}d'un valor $x_i$: 
la freq\"u\`encia absoluta
dividida pel nombre d'observacions fetes.

\item {\bf Freq\"u\`encia absoluta acumulada $N_i$} 
\index{frequencia@freq\"u\`encia!absoluta!acumulada}en el valor $x_i$: la suma de
les freq\"u\`encies absolutes dels valors menors o iguals al $x_i$.

\item {\bf Freq\"u\`encia relativa acumulada $F_i$} 
\index{frequencia@freq\"u\`encia!relativa!acumulada}en el valor $x_i$: la
freq\"u\`encia absoluta acumulada dividida pel nombre d'observacions fetes.

\end{enumerate}

{\bf Propietats de les freq\"u\`encies}

Sigui $N$ el nombre total d'observacions fetes i $k$, el nombre
d'observacions diferents.

\begin{enumerate}

\item $\displaystyle \sum_{i=1}^k n_i = N$.

\item $\displaystyle \sum_{i=1}^k f_i = 1$.

\item $N_k = N$.

\item $F_k = 1$.

\item $0 \leq n_i \leq N, \ i = 1, \ldots , k$.

\item $0 \leq f_i \leq 1, \ i = 1, \ldots , k$.

\item $n_i = N_i - N_{i-1}, \ i = 2, \ldots , k$.

\item El percentatge corresponent a un valor $x_i$ s'obt\'e com $f_i \times 100$.

\end{enumerate}

Una {\bf taula de freq\"u\`encies} 
\index{taula de frequencies@taula de freq\"u\`encies}s'obt\'e ordenant els valors de menor a major i
anotant les diferents freq\"u\`encies al costat de cada valor $x_i$:

$$
\begin{tabular}{|c|c|c|c|c|}
\hline
$x_i$ & $n_i$ & $N_i$ & $f_i$ & $F_i$ \\
\hline
\hline
$x_1$ & $n_1$ & $N_1$ & $f_1$ & $F_1$ \\
\hline
$x_2$ & $n_2$ & $N_2$ & $f_2$ & $F_2$ \\
\hline
$\vdots$ & $\vdots$ & $\vdots$ & $\vdots$ & $\vdots$ \\
\hline
$x_k$ & $n_k$ & $N_k$ & $f_k$ & $F_k$ \\
\hline
\hline
Suma $\sum$ & N & & 1 & \\
\hline
\end{tabular} 
$$

Quan s'han fet moltes observacions i la variable estad\'{\i}stica 
\index{variable!estadistica@estad\'{\i}stica}pren molts de
valors diferents, l'estudi es fa tractant d'agrupar els valors de la variable
estad\'{\i}stica en intervals anomenats {\bf intervals de classe},
\index{interval de classe} que poden ser
d'amplitud constant o variable. Aquests intervals es poden elegir de dos tipus:
a) o b\'e intervals semioberts de la forma $[a,b)$ que no s'encavalquin, de manera
que l'extrem obert d'un interval coincideixi amb l'extrem tancat del seg\"uent i
que recobreixin tot el recorregut de la variable; o b\'e b) quan ens donin intervals
que no s'encavalquin per\`o de manera que l'extrem obert d'un interval no
coincideixi amb l'extrem tancat del seg\"uent, prendrem el punt mitj\`a 
entre aquests dos extrems dels intervals consecutius com a nou extrem per als dos intervals.

Direm {\bf marca de classe}\index{marca de classe} 
al punt mitj\`a de cada interval. Sempre que vulguem resumir la informaci\'o 
contenguda dins cada interval, agafarem la seva marca de
classe.

Les diferents representacions gr\`afiques 
\index{representacio grafica@representaci\'o gr\`afica}que permeten observar 
la informaci\'o recollida s\'on:

\begin{enumerate}

\item {\bf Diagrama de barres:} 
\index{diagrama!de barres}posam en l'eix de les abscisses els valors de
la variable i damunt cada un una l\'{\i}nia perpendicular d'al\c cada igual a
la freq\"u\`encia absoluta (o relativa) del valor en q\"uesti\'o.

\item {\bf Histograma:}\index{histograma} 
quan les variables estad\'{\i}stiques estan agrupades en
intervals de classe, aixecam damunt cada interval un rectangle d'\`area igual a
la freq\"u\`encia absoluta (o relativa) de l'interval.

\item {\bf Pol\'{\i}gon de freq\"u\`encies:} 
\index{poligon@pol\'{\i}gon!de frequencies@de freq\"u\`encies} 
s'obt\'e unint els extrems superiors de les
barres en el diagrama de barres.\index{diagrama!de barres} 
Si la variable est\`a agrupada en 
intervals, el pol\'{\i}gon de freq\"u\`encies 
\index{poligon@pol\'{\i}gon!de frequencies@de freq\"u\`encies}
s'obt\'e unint els punts mitjans 
de les bases superiors de cada interval.

\item {\bf Diagrama de freq\"u\`encies acumulades:} 
\index{diagrama!de frequencies acumulades@de freq\"u\`encies acumulades}
posam a les abscisses els valors de la variable i sobre cada un una l\'{\i}nia perpendicular d'al\c cada
igual a la freq\"u\`encia (absoluta o relativa) acumulada. Finalment s'uneixen amb
segments horitzontals l'extrem superior de cada l\'{\i}nia amb la l\'{\i}nia seg\"uent.

\item {\bf Pol\'{\i}gon de freq\"u\`encies acumulades:} 
\index{poligon@pol\'{\i}gon!de frequencies@de freq\"u\`encies!acumulades}
si la variable est\`a agrupada en
intervals, co{\lgem}ocam en les abscisses els intervals de classe,
\index{interval de classe} despr\'es posam
damunt l'extrem dret de cada interval una l\'{\i}nia perpendicular d'al\c{c}ada
equivalent a la seva freq\"u\`encia (absoluta o relativa) acumulada.
\index{frequencia@freq\"u\`encia!absoluta!acumulada} 
\index{frequencia@freq\"u\`encia!relativa!acumulada} 
Finalment, unim els extrems superiors de totes aquestes l\'{\i}nies.
\end{enumerate}

De vegades interessa resumir la informaci\'o recollida a un o uns quants valors
per tal de poder comparar diferents mostres o poblacions. 
Veurem ara quins s\'on els valors m\'es utilitzats per a aix\`o. 
En tot el que segueix, suposarem que la variable estad\'{\i}stica $X$ 
\index{variable!estadistica@estad\'{\i}stica}
pren els valors $x_1, \ldots , x_k$ amb freq\"u\`encies
absolutes respectives $n_1, \ldots , n_k$. 
\index{frequencia@freq\"u\`encia!absoluta}
Sigui $N = n_1 + \cdots + n_k$.

\begin{itemize}

\item {\bf Mitjana aritm\`etica:} 
\index{mitjana!aritmetica@aritm\`etica}
$\displaystyle \bar{x} = {\sum\limits_{i=1}^k x_i n_i \over N} = 
\sum_{i=1}^k x_i f_i$.

\item {\bf Mitjana geom\`etrica:} 
\index{mitjana!geometrica@geom\`etrica}
$\displaystyle \bar{x}_G = \sqrt[N]{x_1^{n_1}
\cdots x_k^{n_k}}$.

\item {\bf Mitjana quadr\`atica:} 
\index{mitjana!quadratica@quadr\`atica}
$\displaystyle \bar{x}_Q = {\sum\limits_{i=1}^k
x_i^2 n_i \over N}$.

\item {\bf Mitjana harm\`onica:} 
\index{mitjana!harmonica@harm\`onica}
$\displaystyle \bar{x}_A = {N \over\sum\limits_{i=1}^k {n_i \over x_i}}$.

\end{itemize}

Resulta que $\bar{x}_A \leq \bar{x}_G \leq \bar{x} \leq \bar{x}_Q$.

\begin{itemize}

\item {\bf Mediana:} 
\index{mediana}\'es el valor $M_e$ que, una vegada ordenats els valors de
la variable, deixa igual nombre d'observacions a la seva esquerra que a la seva
dreta. Si hi ha un nombre parell de valors, la mediana \'es la mitjana aritm\`etica
dels dos valors centrals.

\item {\bf Moda:}\index{moda} \'es el valor $M_d$ amb freq\"u\`encia m\'es alta. No t\'e perqu\`e
ser \'unica. Si la variable ve agrupada en intervals de classe, parlam de
l'interval modal, que en l'histograma correspon al rectangle d'\`area m\'es gran
per unitat de base.

\item {\bf Quartils:}\index{quartil} s\'on tres valors que divideixen les observacions ordenades
en quatre parts iguals. El primer quartil $P_{1/4}$ \'es el que deixa la quarta
part de les observacions menors o iguals que ell i les tres quartes parts
majors que ell. El segon quartil $P_{2/4}$ \'es la mediana. El tercer quartil
$P_{3/4}$ \'es el rec\'{\i}proc del primer.

\item {\bf Decils:}\index{decil} el decil $i$-\`essim \'es el valor de la variable que deixa
$i/10$ parts de les observacions menors o iguals que ell. An\`alogament es 
defineix el {\bf percentil} $i$-\`essim com el valor de la variable que deixa
$i/100$ parts de les observacions menors o iguals que ell.

\item {\bf Vari\`ancia:}\index{variancia@vari\`ancia} 
la d\'ona $\displaystyle S_x^2 =
{\sum\limits_{i=1}^k (x_i - \bar{x})^2 n_i \over N}$. La seva arrel quadrada
positiva s'anomena la {\bf desviaci\'o t\'{\i}pica} o {\bf est\`andard}: $\displaystyle
S_x = + \sqrt{S_x^2}$.

\item {\bf Desviaci\'o mitjana:}\index{desviacio@desviaci\'o!mitjana}

\begin{itemize}

\item respecte d'una mitjana $p$: $\displaystyle D_{M_p} = {\sum\limits_{i=1}^k
| x_i - p | n_i \over N}$.

\item respecte de la mitjana aritm\`etica: $\displaystyle D_{M_{\bar{x}}} =
{\sum\limits_{i=1}^k | x_i - \bar{x} | n_i \over N}$.

\item respecte de la mediana: $\displaystyle D_{M_{M_e}} = {\sum\limits_{i=1}^k
| x_i - M_e | n_i \over N}$.

\end{itemize}

\item {\bf Coeficient de variaci\'o de Pearson:} $\displaystyle CV = {S_x
\over \bar{x}}$.\index{coeficient!de variacio@de variaci\'o!de Pearson}

\item {\bf Coeficient de variaci\'o mitjana:}
\index{coeficient!de variacio@de variaci\'o!mitjana}

\begin{itemize}

\item respecte d'una mitjana $p$: $\displaystyle CVM_p = {D_{M_p} \over | p |}$.

\item respecte de la mitjana aritm\`etica: $\displaystyle CVM_{\bar{x}} =
{D_{M_{\bar{x}}} \over | \bar{x} |}$.

\item respecte de la mediana: $\displaystyle CVM_{M_e} = {D_{M_{M_e}} \over |
M_e |}$.

\end{itemize}

\item {\bf Recorregut:}\index{recorregut} \'es la difer\`encia $R$ entre el 
valor m\`axim que pren la
variable i el valor m\'{\i}nim.

\item {\bf Recorregut interquart\'{\i}lic:} 
\index{recorregut!interquartilic@interquart\'{\i}lic}
$\displaystyle R_I = P_{3/4} - P_{1/4}$.

\item {\bf Recorregut semiinterquart\'{\i}lic:} 
\index{recorregut!semiinterquartilic@semiinterquart\'{\i}lic}
$\displaystyle R_{SI} = {P_{3/4} - P_{1/4} \over 2}$.

\item {\bf Moment d'ordre $r$ respecte del par\`ametre $c$:} 
\index{moment} $\displaystyle M_r(c) = {\sum\limits_{i=1}^k 
(x_i - c)^r n_i \over N}$.

\item {\bf Moments centrals d'ordre $r$:} 
\index{moment!central} quan $c = 0$: $\displaystyle a_r =
{\sum\limits_{i=1}^k x_i^r n_i \over N}. \ $ Per exemple, $a_1 = \bar{x}$.

\item {\bf Moments d'ordre $r$ respecte de la mitjana:} $\displaystyle m_r =
{\sum\limits_{i=1}^k (x_i - \bar{x})^r n_i \over N}. \ $ Per exemple, $m_2 =
S_x^2$.

\end{itemize}

Finalment parlarem de mesures d'asimetria i apuntament. 
\index{mesura de simetria}
Direm que una distribuci\'o de freq\"u\`encies 
\index{distribucio@distribuci\'o!de frequencies@de freq\"u\`encies}
\'es {\bf sim\`etrica}\index{simetrica@sim\`etrica} si els valors 
equidistants d'un valor central tenen les mateixes freq\"u\`encies. 
En aquest cas, resulta: $\bar{x} = M_e$. En el cas de distribucions
unimodals, tenim que $\bar{x} =M_d = M_e$.
\index{distribucio@distribuci\'o!unimodal}
Una distribuci\'o asim\`etrica pot ser:

\begin{itemize}

\item {\bf asim\`etrica per la dreta} o {\bf positiva}, 
\index{asimetrica@asim\`etrica!per la dreta}
\index{asimetrica@asim\`etrica!positiva} quan la gr\`afica de les
freq\"u\`encies presenta cua a la dreta, \'es a dir, les freq\"u\`encies descendeixen m\'es
lentament per la dreta que per l'esquerra. En aquest cas, resulta: $\bar{x}
\geq M_e$. Si la distribuci\'o \'es unimodal, resulta 
$\bar{x}\geq M_e \geq M_d$.
\index{distribucio@distribuci\'o!unimodal}

\item {\bf asim\`etrica per l'esquerra} o {\bf negativa}, 
\index{asimetrica@asim\`etrica!per l'esquerra}
\index{asimetrica@asim\`etrica!negativa} en cas rec\'{\i}proc. En
aquest cas, resulta: $\bar{x} \leq M_e$.
Si la distribuci\'o \'es unimodal, resulta $\bar{x} \leq M_e \leq M_d$.
\index{distribucio@distribuci\'o!unimodal}
\end{itemize}

Els seg\"uents coeficients permeten con\`eixer l'asimetria d'una distribuci\'o sense
haver-la de representar.

\begin{itemize}

\item {\bf Coeficient d'asimetria de Pearson:} (nom\'es per a distribucions
unimodals)
\index{coeficient!d'asimetria!de Pearson} $ A_P = {\bar{x} - M_d \over
S_x}$.

\begin{itemize}

\item Si $A_P > 0$, la distribuci\'o \'es asim\`etrica per la dreta.

\item Si $A_P = 0$, la distribuci\'o \'es sim\`etrica.

\item Si $A_P < 0$, la distribuci\'o \'es asim\`etrica per l'esquerra.

\end{itemize}

\item {\bf Coeficient d'asimetria de Fisher:} 
\index{coeficient!d'asimetria!de Fisher} $A_F = {m_3 \over S_x^3}$.

\begin{itemize}

\item Si $A_F > 0$, la distribuci\'o \'es asim\`etrica per la dreta.

\item Si $A_F = 0$, la distribuci\'o \'es sim\`etrica.

\item Si $A_F < 0$, la distribuci\'o \'es asim\`etrica per l'esquerra.

\end{itemize}

\end{itemize}

Per acabar, donarem un coeficient d'apuntament o {\bf curtosi}, 
\index{coeficient!d'apuntament}\index{curtosi} que ens indica
si la gr\`afica d'una distribuci\'o, comparada amb la de la normal (campana de
Gauss),\index{campana de Gauss} \'es poc apuntada ({\bf platic\'urtica}
\index{platicurtica@platic\'urtica} o apuntament baix), punxeguda
({\bf leptoc\'urtica}\index{leptocurtica@leptoc\'urtica} 
o apuntament alt) o mitjanament apuntada ({\bf mesoc\'urtica}). 
\index{mesocurtica@mesoc\'urtica}
El coeficient es defineix com:

$$g_2 = {m_4 \over S_x^4}.$$

Aleshores resulta:
\begin{itemize}
\item $g_2 > 3 \to $ leptoc\'urtica.\index{leptocurtica@leptoc\'urtica}
\item $g_2 = 3 \to $ mesoc\'urtica.\index{mesocurtica@mesoc\'urtica}
\item $g_2 < 3 \to $ platic\'urtica.\index{platicurtica@platic\'urtica}
\end{itemize}

\section{Problemes resolts}

\begin{probres}
{
Considerem les puntuacions de $50$ aspirants a un lloc 
de treball:
$$\begin{tabular}{cccccccccc}
8 & 11 & 11 & 8 & 9 & 10 & 16 & 6 & 12 & 19 \\
13 & 6 & 9 & 13 & 15 & 9 & 12 & 16 & 8 & 7 \\
14 & 11 & 15 & 6 & 14 & 14 & 17 & 11 & 6 & 9 \\
10 & 19 & 12 & 11 & 12 &  6 & 15 & 16 & 16 & 12 \\ 
13 & 12 & 12 & 8 & 17 & 13 & 7 & 12 & 14 & 12 
\end{tabular}
$$
Trobau la taula 
de freq\"u\`encies prenent intervals 
d'amplada $3$.  Trobau tamb\'e l'histograma 
de freq\"u\`encies absolutes amb el 
corresponent pol\'{\i}gon
 
de freq\"u\`encies.
}
\etiqueta{PROBUESTDESC}
\end{probres}
\newpage

\res{
El valor m\`axim \'es $19$ i el m\'{\i}nim, $6$. 
Tenint en compte que 
consideram l\'{\i}mits reals per als 
intervals de classe, la taula ser\`a:
$$
\begin{tabular}{|l|r|r|r|r|r|}
\hline
intervals  & $X_j$ & $n_j$ & $N_j$ &  $f_j$ &  $F_j$ \\
\hline\hline
$[5.5,8.5)$  & 7  & 11 & 11 & 0.22 &  0.22 \\\hline
$[8.5,11.5)$ & 10  & 11  & 22  & 0.22  & 0.44 \\\hline
$[11.5,14.5)$ & 13 & 17  & 39  & 0.34  & 0.78 \\\hline
$[14.5,17.5)$ & 16 &  9  & 48  & 0.18  & 0.96 \\\hline
$[17.5,20.5)$ & 19   & 2  & 50 & 0.04  & 1.00 \\
\hline
\end{tabular}
$$

L'histograma de freq\"u\`encies absolutes amb el 
corresponent pol\'{\i}gon
de freq\"u\`encies ser\`a com s'indica en el 
gr\`afic~\ref{GRAFHISTO}.

Fixau-vos que les al\c{c}ades dels rectangles depenen del fet que
l'amplada dels intervals \'es~$3$:  \[ \begin{array}{rlcrl} h_1 =&
\frac{n_1}{3}=\frac{11}{3}=3.6666,& & h_2=& \frac{n_2}{3}=
\frac{11}{3}=3.666,\\ &&&&\\ h_3 =& \frac{n_3}{3}=\frac{17}{3}=5.6666,&
& h_4=& \frac{n_4}{3}= \frac{9}{3}=3, \\ &&&&\\ h_5 = &
\frac{n_5}{3}=\frac{2}{3}=0.666.&&& \end{array} \]

\begin{figure}
$$
\setcoordinatesystem units <0.5cm,0.25cm>
\beginpicture
\setplotarea x from 5.5 to 20.5, y from 0 to 20
\axis bottom shiftedto y=0 ticks
in withvalues $5.5$ $8.5$ $11.5$ $14.5$ $17.5$ $20.5$ /
 quantity 6 /
\axis left shiftedto x=5.5 ticks
in withvalues {} $2$ $4$ $6$ /
 quantity 4 /
\sethistograms
\plot 5.5 0
8.5 11
11.5 11
14.5 17
17.5 9
20.5 2 /
\setlinear
\plot 4 0
7 11
10 11
13 17
16 9
19 2
22 0 /
\endpicture
$$
\caption{Histograma de freq\"u\`encies absolutes per al problema 
\ref{PROBUESTDESC}}
\label{GRAFHISTO}
\end{figure}
}

\begin{probres}
{
Considerem les dades seg\"uents:
$$
\begin{tabular}{cccccccccc}
10 & 5 & 2 & 7 & 9 & 5 & 7 & 6 & 5 & 9 \\
12 & 2 & 6 & 6 & 9 & 12 & 6 & 6 & 6 & 4 \\
 9 & 7 & 12 & 11 & & & & & & 
\end{tabular}
$$
Trobau la mitjana 
aritm\`etica de les dades anteriors sense agrupar en 
intervals i agrupant les dades en intervals
 d'amplada $3$.

Trobau tamb\'e la mediana i els percentils 
$P_{1/4}$ i $P_{3/4}$ de les dades 
agrupades.}
\end{probres}

\res{
La mitjana aritm\`etica
de les dades anteriors sense agrupar 
en intervals ser\`a:

$$\overline{X}= \frac{10+5+2+\cdots +12+11}{24}=\frac{173}{24}=7.20833   
$$   
                                          
Si les agrupam en intervals d'amplada $3$, 
la mitjana ser\`a (fem primer 
la taula de
freq\"u\`encies):
$$
\begin{tabular}{|l|r|r|r|}
\hline
intervals & $X_j$ & $n_j$ & $n_j X_j$ \\
\hline\hline
$[1.5,4.5)$&3&3&9\\
\hline
$[4.5,7.5)$&6&12&72\\ 
\hline
$[7.5,10.5)$&9&5&45\\ 
\hline
$[10.5,13.5)$ &12&4&48\\ 
\hline\hline
  Suma     & 24  &    & 174\\\hline
\end{tabular}
$$
La mitjana ser\`a, doncs:
$$\overline{X}= \frac{174}{24}=7.25$$

Notem que els valors difereixen ja que l'agrupament provoca una 
p\`erdua d'informaci\'o.

A continuaci\'o trobam la mediana i els 
percentils $P_{1/4}$ i $P_{3/4}$:

Tenim que $N=24$. Per tant,
$\frac{N}{2}=12$. L'interval cr\'{\i}tic, 
ser\`a, en aquest cas: $[4.5,7.5)$.
La mediana valdr\`a:
$$Md=4.5+3\frac{12-3}{12}=6.75.$$

Percentil $25$: $P_{1/4}\Rightarrow  
\frac{NP}{100}= 6$. Interval cr\'{\i}tic: 

$[4.5,7.5)$.
$$P_{1/4}=4.5+3\frac{(6-3)}{12}=5.25$$
Percentil $75$: $P_{3/4} \Rightarrow  
\frac{NP}{100}= 18$. Interval cr\'{\i}tic:  

$[7.5,10.5)$.
$$
P_{3/4}=7.5+3\frac{(18-15)}{5}=9.3
$$
}

\begin{probres}
{
Considerem la distribuci\'o
de freq\"u\`encies seg\"uent:
$$
\begin{tabular}{|l|r|}
\hline
intervals & $n_j$ \\
\hline\hline
$[9.5,29.5)$    &  38\\\hline  
$[29.5,49.5)$   &  18\\\hline
$[49.5,69.5)$   &  31 \\\hline
$[69.5,89.5)$   &  20  \\\hline
\end{tabular}
$$
Trobau la vari\`ancia i el coeficient
 de variaci\'o respecte de la mitjana.
}
\end{probres}

\res{
Primer hem de trobar la mitjana.

Per fer-ho, constru\"{\i}m la taula seg\"uent:

$$
\begin{tabular}{|l|r|r|r|}
\hline
intervals    & $X_j$ & $n_j$ & $n_jX_j$ \\
\hline\hline
$[9.5,29.5)$   & 19.5 &  38 &   741 \\\hline  
$[29.5,49.5)$  & 39.5 &  18 &   711 \\\hline
$[49.5,69.5)$  & 59.5 &  31 &  1844.5 \\\hline
$[69.5,89.5)$  & 79.5 &  20 &  1590   \\\hline
\hline
    Sumes      &      & 107 &  4886.50 \\\hline
\end{tabular}
$$
La mitjana valdr\`a:
$$\overline{X}=\frac{4886.5}{107}=45.6682.$$

Per trobar la vari\`ancia hem d'afegir dues 
columnes m\'es a la taula anterior:
$$
\begin{tabular}{|c|r|r|}
\hline
 $X_j$ & $X_j^2$ & $n_jX_j^2$ \\
\hline\hline
 19.5  & 380.25  & 14449.50 \\\hline
 39.5  & 1560.25 &  28084.50 \\\hline
 59.5  & 3540.25 & 109747.75 \\\hline
 79.5  & 6320.25 & 126405.00 \\
\hline\hline
Suma   &          & 278686.75  \\\hline
\end{tabular}
$$
La vari\`ancia i la desviaci\'o 
t\'{\i}pica valen:

\begin{eqnarray*}
S_X^2 & = & \frac{278686.75}{107}-45.6682^2=518.962, \\
S_X & = & \sqrt{518.962}=22.7807.
\end{eqnarray*}
El coeficient 
de variaci\'o val:
$$CV=\frac{S_X}{\overline{X}}=\frac{22.6807}{45.6682}=0.4988.$$
}

\begin{probres}
{
Trobau els coeficients de simetria 
i apuntament de la 
distribuci\'o
de freq\"u\`encies seg\"uent:
$$
\begin{tabular}{|l|r|r|r|r|}
\hline
intervals  &  $ n_j$  \\ 
\hline\hline
$[14.5,19.5)$  &  4  \\\hline
$[19.5,24.5)$  &  6 \\\hline
$[24.5,29.5)$  &  8 \\\hline
$[29.5,34.5)$  & 11  \\\hline
$[34.5,39.5)$  & 35  \\\hline
$[39.5,44.5)$  & 100 \\\hline
$[44.5,49.5)$  & 218 \\
\hline
\end{tabular}
$$
}
\end{probres}

\newpage

\res{
Trobem primer la mitjana i la vari\`ancia.
Per fer-ho constru\"{\i}m la 
taula seg\"uent:
$$
\begin{tabular}{|l|r|r|r|r|}
\hline
intervals &  $X_j$ & $ n_j$ &  $n_jX_j$  &  $n_jX_j^2$ \\ 
\hline\hline
$[14.5,19.5)$ & 17  &  4    & 68   & 1156  \\\hline
$[19.5,24.5)$ & 22  &  6   & 132   & 2904 \\\hline
$[24.5,29.5)$ & 27  &  8  &  216   & 5832 \\\hline
$[29.5,34.5)$ & 32  & 11  &  352   & 11264 \\\hline
$[34.5,39.5)$ & 37  & 35  & 1295   & 47915 \\\hline
$[39.5,44.5)$ & 42  &100  & 4200  & 176400 \\\hline
$[44.5,49.5)$ & 47  & 218 & 10246 & 481562 \\\hline
\hline
  Sumes       & &   382 & 16509 & 727033\\\hline
\end{tabular}
$$

La mitjana i la vari\`ancia valen:

$$
\overline{X}=  \frac{16509}{382}=43.2173,\quad
S_{X}^2=  \frac{727033}{382}-\left(\frac{16509}{382}\right)^2=35.49.
$$

Trobem ara el coeficient d'asimetria
$A_F$. Per fer-ho, hem d'afegir una
columna m\'es a la taula anterior: 

$$
\begin{tabular}{|r|r|r|}
\hline
        $X_j$ &  $ n_j$ &    $n_j {(X_j-\overline{X})}^3$ \\
\hline\hline
        17 &   4  & -72081.33  \\\hline
        22  &  6  & -57308.66 \\\hline
        27  &  8  & -34121.16 \\\hline
        32  & 11  & -15525.84 \\\hline
        37  & 35  &  -8411.41 \\\hline
        42  & 100  &   -180.37 \\\hline
        47  & 218  &  11799.67 \\\hline
\hline
 Sumes       & 382  &-175829.09 \\\hline
\end{tabular}
$$
El moment de tercer ordre val:

$$m_3=\frac{-175829.09}{382}=-460.285.$$

A continuaci\'o calculam el coeficient d'asimetria de Fisher:


$$A_F=\frac{m_3}{S_X^3}=\frac{-460.285}{\left(\sqrt{35.49}\right)^3}=-2.18.$$

Per tant, podem dir que es tracta d'una 
distribuci\'o asim\`etrica per l'esquerra o asim\`etrica 
negativa.

A continuaci\'o trobem el coeficient d'apuntament
 $g_2$.
Per fer-ho, hem d'afegir una columna m\'es a la taula anterior:
$$
\begin{tabular}{|l|r|r|r|}
\hline
intervals   &  $X_j$  & $n_j $  & $n_j(X_j-X)^4$   \\   
\hline\hline
$[14.5,19.5)$ & 17  & 4    & 1889776.11 \\\hline
$[19.5,24.5)$ & 22  & 6    & 1215933.65 \\\hline
$[24.5,29.5)$ & 27 &   8   &  553352.38 \\\hline
$[29.5,34.5)$ & 32 &  11   & 174157.64 \\\hline
$[34.5,39.5)$ & 37 &  35   & 52296.07 \\\hline
$[39.5,44.5)$ & 42 & 100   &  219.56 \\\hline
$[44.5,49.5)$ & 47 & 218  &  44634.88 \\\hline
\hline
  Sumes     & &    382  & 3930370.29 \\\hline
\end{tabular}
$$
El moment de quart ordre val:

$$m_4=\frac{3930370.29}{382}=10288.93.$$

A continuaci\'o, calculam el coeficient d'apuntament

$$g_2=\frac{m_4}{S_X^4} -3 =\frac{10288.93}{35.49^2}-3=5.17.$$
Per tant, es tracta d'una distribuci\'o 
punxeguda o leptoc\'urtica.
}

\section{Problemes proposats}

\begin{prob}
{
A 50 aspirants a un determinat lloc de treball se'ls 
va sotmetre a una
prova. Les puntuacions obtengudes varen ser:
\begin{center}
\begin{tabular}{rrrrrrrrrr}
4,&4,&2,&10,&1,&9,&5,&3,&4,&5,\\
6,&6,&7,&6,&8,&7,&6,&8,&7,&6,\\
5,&4,&4,&4,&5,&6,&6,&7,&5,&6,\\
6,&7,&5,&6,&6,&7,&5,&6,&4,&3,\\
2,&6,&6,&7,&7,&8,&8,&9,&8,&7.\\
\end{tabular}
\end{center}
\begin{itemize}
\item[a)] {Constru\"{\i}u la taula de freq\"u\`encies 
i la representaci\'o gr\`afica corresponent.}
\item[b)] {Trobau la puntuaci\'o que 
seleccioni el $20\%$ dels millors candidats.}
\end{itemize}
}
\end{prob}

\begin{prob}
{
En la poblaci\'o 
d'estudiants de la facultat es va seleccionar una mostra
de 20 alumnes i es varen obtenir les seg\"uents talles en cent\'{\i}metres:
\begin{center}
\begin{tabular}{cccccccccc}
162,&168,&174,&168,&166,&170,&168,&166,&170,&172,\\
188,&182,&178,&180,&176,&168,&164,&166,&164,&172.\\
\end{tabular}
\end{center}
Es demana:
\begin{itemize}
\item[a)] {Descripci\'o num\`erica i representaci\'o gr\`afica.
}
\item[b)] {Mitjana aritm\`etica, 
mediana i moda.}
\end{itemize}
}
\etiqueta{POBESTFAC}
\end{prob}

\begin{prob}
{
Agrupant les dades de l'exercici \ref{POBESTFAC} en intervals d'amplada 
$10$ cm., es demana:
\begin{itemize}
\item[a)] {Descripci\'o num\`erica i representaci\'o gr\`afica.
}
\item[b)] {Mitjana aritm\`etica, 
mediana i moda.}
\item[c)] {Analitzau els c\`alculs fets i els errors d'agregaci\'o 
i comparau-los
amb els de l'exercici anterior.}
\end{itemize}}
\end{prob}

\begin{prob}
{
Les tres factories d'una ind\'ustria
 han produ\"{\i}t en l'\'ultim any el seg\"uent
nombre de motocicletes per trimestre:
\begin{center}
\begin{tabular}{|c||r|r|r|}
\hline
 & factoria 1 & factoria 2 & 
factoria 3 \\\hline\hline 
1r. trimestre & 600 & 650 & 550 \\\hline 
2n. trimestre & 750 & 1200 & 900 \\\hline
3r. trimestre & 850 &1250 & 1050 \\\hline
4t. trimestre & 400 & 800 & 650 \\\hline
\end{tabular}
\end{center}
Obteniu:
\begin{itemize}
\item[a)] {Producci\'o mitjana 
trimestral de cada factoria i de tota la
ind\'ustria.}
\item[b)] {Producci\'o mitjana 
di\`aria de cada factoria i 
de tota la ind\'ustria
tenint en compte que durant el primer trimestre hi va 
haver $68$ dies laborables, durant el segon, $78$, durant el tercer, $54$ i 
durant el quart, $74$.}
\end{itemize}
}
\end{prob}

\begin{prob}
{
Una empresa ha pagat per un cert article: 
$225$, $250$, $300$ i $200$ ptes. de preu. 

Determinau el preu mitj\`a en les hip\`otesis seg\"uents:
\begin{itemize}
\item[a)] {Comprava per valor de $38.250$, $47.500$, $49.500$ i $42.000$ ptes.,
respectivament.}
\item[b)] {Comprava cada vegada un mateix import global.}
\item[c)] {Comprava $174$, $186$, $192$ i $214$ unitats, respectivament.}
\end{itemize}
}
\end{prob}

\begin{prob}
{
D'una mostra de $56$ botigues 
distintes, es varen obtenir els seg\"uents preus de venda d'un determinat 
article:
\begin{center}
\begin{tabular}{ccccccc}
3260 & 3510 & 3410 & 3180 & 3300 & 3540 & 3320 \\
3450 & 3840 & 3760 & 3340 & 3260 & 3720 & 3430 \\
3320 & 3460 & 3600 & 3700 & 3670 & 3610 & 3910 \\
3610 & 3610 & 3620 & 3150 & 3520 & 3430 & 3330 \\
3370 & 3620 & 3750 & 3220 & 3400 & 3520 & 3360 \\
3300 & 3340 & 3410 & 3600 & 3320 & 3670 & 3420 \\
3320 & 3290 & 3550 & 3750 & 3710 & 3530 & 3500 \\
3290 & 3410 & 3100 & 3860 & 3560 & 3440 & 3620 \\
\end{tabular}
\end{center}
Es demana:
\begin{itemize}
\item[a)] {Agrupau la informaci\'o en sis intervals d'igual amplada
 i representaci\'o gr\`afica
 corresponent.}
\item[b)] {Mitjana geom\`etrica, 
quadr\`atica, 
aritm\`etica 
i harm\`onica.}
\item[c)] {Desviaci\'o de les quatre mitjanes.}
\end{itemize}
}
\end{prob}

\begin{prob}
{
La seg\"uent distribuci\'o correspon al capital 
pagat per les $420$ empreses de la construcci\'o 
 amb adre\c{c}a social en una regi\'o 
determinada:
\begin{center}
\begin{tabular}{|r|r|}
\hline
Capital (milions de ptes.) & Nombre d'empreses\\\hline\hline
menys de 5 & 12 \\\hline
de 5 a 13 & 66 \\\hline
de 13 a 20 & 212 \\\hline
de 20 a 30 & 84 \\\hline
de 30 a 50 & 30 \\\hline
de 50 a 100 & 14 \\\hline
m\'es de 100 & 2 \\\hline
\end{tabular}
\end{center}
\begin{itemize}
\item[a)] {Fent servir com a marques de classe 
del primer i \'ultim interval $4$ i
$165$, respectivament, trobau la mitjana 
aritm\`etica i la 
desviaci\'o t\'{\i}pica.}
\item[b)] {Calculau la moda i la mediana.}
\item[c)] {Coeficient d'asimetria de Fisher.
}
\end{itemize}
}
\end{prob}

