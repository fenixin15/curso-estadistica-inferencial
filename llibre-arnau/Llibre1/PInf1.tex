\part{Probabilitats}

\chapter{Probabilitat elemental}

\section{Resum te\`oric}

Molts de sistemes d'inter\`es presenten fen\`omens que mostren variacions
impredictibles, degudes a l'atzar. Aquests fen\`omens, anomenats {\bf fen\`omens
aleatoris}\index{fenomen aleatori@fen\`omen aleatori} o {\bf experiments
aleatoris}\index{experiment aleatori}, es caracteritzen perqu\`e les
condicions sota les quals es realitzen no permeten precisar-ne {\it a priori} el
resultat. Per tant, tendrem tot un conjunt de resultats possibles. Per exemple,
en el llan\c{c}ament d'un dau, el conjunt de resultats possibles \'es
$\{1,2,3,4,5,6\}$. En aquests casos, els models  deterministes no s\'on
adequats, ja que aquests, cada vegada que es repeteixen les condicions inicials,
donen el mateix resultat. Per tant, s'han d'utilitzar els anomenats {\bf models
probabil\'{\i}stics}\index{model probabilistic@model probabil\'{\i}stic}.

D'altra banda, a cada experiment aleatori hi podem associar
successos\index{succes@succ\'es}, que s\'on q\"uestions que es poden contestar
amb ``s\'{\i}'' o ``no'' i, una vegada realitzat l'experiment, podrem observar
quina \'es la resposta correcta. En el llan\c{c}ament d'un dau, podem estar
interessats, per exemple, en si s'obt\'e un nombre parell. Associat amb 
``s\'{\i}'' tenim el conjunt de resultats possibles $\{2,4,6\}$ i associat 
amb ``no'' tenim $\{1,3,5\}$.

Finalment, sovint estarem interessats a avan\c{c}ar, abans de realitzar
l'experiment, quines opcions t\'e un determinat succ\'es de produir-se
(fonament dels jocs d'atzar). Aix\`o ho indicam amb frases com ``la
probabilitat\index{probabilitat} de $A$ \'es $p$'', on $A$ \'es un succ\'es com
el d'abans, o ``el Bar\c{c}a guanyar\`a la lliga'', o ``dem\`a far\`a sol'', 
i $p$ \'es el nombre o adjectiu que descriu una quantitat tal com 
``$1/2$'', ``alta'' o ``nu{\lgem}a''.

Aix\'{\i}, amb cada experiment aleatori, ens apareixen tres conceptes: el conjunt
dels seus possibles resultats, una s\`erie de q\"uestions o successos 
relacionats amb l'experiment i unes quantitats o adjectius, associats amb
cada succ\'es, que anomenam la probabilitat\index{probabilitat} del succ\'es en
q\"uesti\'o i que ens permeten con\`eixer {\it a priori} les opcions que t\'e el
succ\'es de produir-se. A continuaci\'o definirem amb rigor aquests tres 
conceptes.

\begin{defin}
Donat un experiment aleatori, direm {\bf espai mostral}\index{espai!mostral}
associat a l'experiment al conjunt de tots els seus resultats possibles. 
L'indicarem amb $\Omega$, i els seus elements, anomenats {\bf successos
elementals}\index{succes@succ\'es!elemental}, amb $\omega, \omega_i, \ldots$
\end{defin}

\begin{defin}
Donat un conjunt $\Omega$, anomenarem {\bf $\sigma$-\`algebra} de successos
sobre $\Omega$\index{$\sigma$-algebra de successos@$\sigma$-\`algebra de
successos} a tot subconjunt ${\cal F} \subset \Omega$ tal que:

\begin{enumerate}

\item $\Omega \in {\cal F}$.

\item $A \in {\cal F} \Longrightarrow A^c \in {\cal F}.$

\item $\displaystyle A_1, A_2, \ldots \in {\cal F} \Longrightarrow
\mathop{\cup}\limits_{n=1}^\infty A_n \in {\cal F}.$

\end{enumerate}
\end{defin}

{\bf Notes.}

\begin{enumerate}

\item Anomenarem {\bf succ\'es}\index{succes@succ\'es} a tot element de ${\cal F}$. 
Direm que un succ\'es $A\in {\cal F}$ {\bf ha ocorregut} (o ha tengut lloc)
si el resultat $\omega$ de l'experiment \'es $\omega\in A$.

\item Observem que els successos sobre $\Omega$ s\'on subconjunts de $\Omega$.
El fet de dotar la co{\lgem}ecci\'o de successos de l'estructura de 
$\sigma$-\`algebra \'es degut al fet que, per poder definir de manera autoconsistent
la probabilitat sobre ${\cal F}$, necessitam que les 
operacions de conjunts tals 
com complementaci\'o, uni\'o, intersecci\'o, difer\`encia, etc. siguin 
operacions internes sobre ${\cal F}$.

\item Es pot provar que donada qualsevol co{\lgem}ecci\'o de successos (les
q\"uestions relacionades amb l'experiment aleatori que ens interessi
investigar), existeix la m\'{\i}nima $\sigma$-\`algebra que cont\'e la
co{\lgem}ecci\'o, \'es a dir, que aquesta co{\lgem}ecci\'o es pot completar afegint el
nombre m\'{\i}nim de successos de manera que el nou conjunt de successos
obtengut tengui l'estructura de $\sigma$-\`algebra.

\item El conjunt buit $\emptyset$ s'anomena el {\bf succ\'es
impossible}\index{succes@succ\'es!impossible} (mai ocorre) i el conjunt total 
$\Omega$ s'anomena el {\bf succ\'es segur}\index{succes@succ\'es!segur}
(sempre ocorre).

\end{enumerate}

\newpage

{\bf Propietats}

\begin{enumerate}

\item $\emptyset \in {\cal F}.$

\item $\displaystyle A_1, A_2, \ldots \in {\cal F} \Longrightarrow
\mathop{\cap}\limits_{i=1}^\infty \in {\cal F}.$

\item Si $A_1, \ldots , A_n \in {\cal F} \Longrightarrow A_1 \cup \cdots \cup
A_n \in {\cal F}.$

\item Si $A_1, \ldots , A_n \in {\cal F} \Longrightarrow A_1 \cap \cdots \cap
A_n \in {\cal F}.$

\item $A, B \in {\cal F} \Longrightarrow A \setminus B = A \cap B^c \in {\cal
F}.$

\item La $\sigma$-\`algebra m\'es petita \'es 
${\cal F} = \{ \emptyset, \Omega \}$ i
la m\'es gran \'es ${\cal F} = P(\Omega)$.

\end{enumerate}

\begin{defin}
Considerem un conjunt $\Omega$ i sigui ${\cal F}$ una $\sigma$-\`algebra 
de successos sobre~$\Omega$. Una
aplicaci\'o $p : {\cal F} \to [0,1] $ s'anomena una {\bf mesura de
probabilitat}\index{mesura de probabilitat}
(o {\bf probabilitat}) si verifica:

\begin{enumerate}
\item $\pp{\Omega} = 1.$

\item Si $A_1, A_2, \ldots \in {\cal F}$ s\'on disjunts dos a dos ($A_i \cap A_j
=
\emptyset \ \ \forall i \not = j$), aleshores $\displaystyle 
\pp{\mathop{\cup}\limits_{n \geq 1} A_n} = \sum_{n \geq 1} \pp{A_n}.$

\end{enumerate}
\end{defin}

{\bf Propietats}

\begin{enumerate}

\item $\pp{A^c} = 1 - \pp{A}.$

\item $\pp{\emptyset} = 0.$

\item $A, B \in {\cal F} \Longrightarrow \pp{A \setminus B} = \pp{A} - 
\pp{A \cap B}.
$

\item $\pp{A \cup B} = \pp{A}+ \pp{B} - \pp{A \cap B}.$

\item $A \subset B \Longrightarrow \pp{A} \leq \pp{B}.$

\item {\bf Continu\"{\i}tat seq\"uencial cap a
dalt}:
\index{continuitat sequencial@continu\"{\i}tat seq\"uencial!de la probabilitat} Si $A_1 \subset A_2 \subset \cdots$ \'es una successi\'o creixent de
successos de ${\cal F}$, definim el seu l\'{\i}mit com $A = \lim A_n = \mathop{
\cup}\limits_{n=1}^\infty A_n$. Aleshores resulta que
$$\pp{A} = \lim_{n\to\infty} \pp{A_n}.$$

\item {\bf Continu\"{\i}tat seq\"uencial cap a baix}: Si 
$A_1 \supset A_2
\supset \cdots$ \'es una successi\'o decreixent de successos de ${\cal F}$,
definim el seu l\'{\i}mit com $A = \lim A_n = \mathop{ \cap}\limits_{n=1}^\infty
A_n$. Aleshores resulta que $$\pp{A} = \lim_{n\to\infty} \pp{A_n}.$$

\item $p$ \'es {\bf subadditiva}: $\displaystyle A_1, \ldots ,A_n \in {\cal F}
\Longrightarrow \pp{\mathop{\cup}\limits_{i=1}^n A_i}
 \leq \sum\limits_{i=1}^n \pp{A_i}.$

\end{enumerate}

{\bf Notes.}

\begin{enumerate}

\item Direm que $A \in {\cal F}$ \'es un {\bf succ\'es nul}
\index{succes@succ\'es!nul} si $\pp{A} = 0$. No s'han de confondre els successos 
nuls amb el succ\'es impossible $\emptyset$: els successos
nuls poden oc\'orrer, encara que tenen probabilitat 0. Per exemple, encertar el
centre geom\`etric d'una diana amb un dard t\'e probabilitat 0, per\`o no \'es
el succ\'es impossible.

\item Quan $\Omega = \{ \omega_1, \ldots , \omega_n \}$ \'es finit i tots els
seus elements s\'on successos {\bf
equiprobables}:\index{equiprobabilitat} 
$\pp{\omega_1} = \cdots =\pp{\omega_n} = 1/n$, aleshores donat un succ\'es $A \in
{\cal F}$, tenim l'anomenada {\bf
definici\'o cl\`assica} de
probabilitat
\index{definicio@definici\'o!classica de probabilitat@cl\`assica de probabilitat}:

$$\pp{A} = {card(A) \over n} = {nombre \> de \> casos \> favorables \> a \> A
\over nombre \> de \> casos \> possibles}.$$

Aquesta definici\'o falla quan $\Omega$ no \'es finit (quants de
llan\c{c}aments d'una moneda s\'on necessaris per obtenir la primera cara?)
i quan no hi ha equiprobabilitat dels successos elementals (en el
llan\c{c}ament de dues monedes, la probabilitat d'obtenir una cara \'es el doble
de la d'obtenir-ne dues). De totes formes, quan els requisits es compleixen,
\'es una definici\'o molt \'util.

\item D'altra banda, la {\bf definici\'o freq\"uentista}
\index{definicio@definici\'o!frequentista de probabilitat@freq\"uentista de probabilitat}
defineix la probabilitat d'un succ\'es com el l\'{\i}mit de les freq\"u\`encies
relatives d'ocurr\`encies
del succ\'es en una llarga s\`erie de realitzacions de l'experiment:

$$\pp{A} = \lim_{n \to \infty} f_A(n) = \lim_{n \to \infty} {nombre \> de \>
vegades \> que \> s'ha \> donat \> A \over nombre \> de \> realitzacions \> de
\> l'experiment}.$$

Aquesta definici\'o t\'e un clar problema pr\`actic i, a m\'es, qu\`e passa si
l'ex\-pe\-ri\-ment no \'es repetible, com a m\'{\i}nim sota les mateixes condicions? De
totes formes, en certes situacions ser\`a molt \'util.

\end{enumerate}

\begin{defin}
Direm {\bf espai de probabilitat}\index{espai!de probabilitat} (o {\bf model
probabil\'{\i}stic})\index{model probabilistic@model probabil\'{\i}stic} associat a un
experiment aleatori donat a una terna $(\Omega, {\cal F}, p)$, on:

\begin{itemize}

\item $\Omega$ \'es l'espai mostral\index{espai!mostral}, o conjunt de tots els
resultats possibles de l'experiment,

\item ${\cal F}$ \'es una $\sigma$-\`algebra de successos sobre $\Omega,$

\item $p$ \'es una mesura de probabilitat sobre ${\cal F}.$

\end{itemize}
\end{defin}

\begin{defin}
Sigui $B$ un succ\'es no nul. Definim la {\bf probabilitat condicionada de
$A$ donat $B$}\index{probabilitat!condicionada} com:

$$\pp{A/B} = {\pp{A \cap B} \over \pp{B}}.$$
\end{defin}

{\bf Propietats.}

\begin{enumerate}

\item $P(A \cap B) = \pp{B} \cdot \pp{A/B} = \pp{A} \cdot \pp{B/A}.$

\item En general, donats $A_1, \ldots , A_n \in {\cal F}$ tals que 
$\pp{A_1 \cap \cdots \cap A_{n-1}} > 0$, resulta:

$$\pp{A_1 \cap \cdots \cap A_n} = \pp{A_1} \cdot \pp{A_2/A_1} 
\cdot \pp{A_3/A_1 \cap A_2} \cdots \pp{A_n/A_1 \cap \cdots A_{n-1}}.$$

\item {\bf F\'ormula de les probabilitats totals:}\index{Formula@F\'ormula de
les probabilitats totals}

Considerem una partici\'o finita (o numerable) de $\Omega$, $\displaystyle \pi =
\{ A_i \} \subset {\cal F}$\break $(\Omega = \mathop{\cup}\limits_i A_i, \ A_i \cap
A_j = \emptyset \ \forall i \not = j)$ tal que $\pp{A_i} > 0 \ \forall i$. Donat un
succ\'es $A \in {\cal F}$, resulta:

$$\pp{A} = \sum_i \pp{A \cap A_i} = \sum_i \pp{A_i} \cdot \pp{A/A_i}.$$

\item {\bf Regla de Bayes:}\index{Regla de Bayes}

Donada una partici\'o finita (o numerable) de $\Omega$, $\pi = \{ A_i \} \subset
{\cal F}$ de successos no nuls, i $B \in {\cal F}$,

$$\pp{A_i/B} = {\pp{A_i \cap B} \over \pp{B}} = {\pp{A_i} \cdot 
\pp{B/A_i} \over \sum\limits_j \pp{A_j} \cdot \pp{B/A_j}}.$$

En el cas de dos successos $A$ i $B$, resulta:

$$\pp{A/B} = {\pp{A} \cdot \pp{B/A} \over
\pp{A} \cdot \pp{B/A} + \pp{A^c} \cdot \pp{B/A^c}}.$$

Aquesta propietat \'es \'util quan $B$ \'es un succ\'es observable i $A$ no ho
\'es, per\`o del qual coneixem la seva probabilitat {\it a priori}
\index{probabilitat!a priori} $\pp{A}$. 
Volem con\`eixer la seva probabilitat {\it a posteriori}
\index{probabilitat!a posteriori}
$\pp{A/B}$, una vegada que hem observat $B$.

\item Si $B \in {\cal F}$ \'es un succ\'es no nul, l'aplicaci\'o 
$\pp{\bullet/B} : {\cal F}
\to [0,1]$ donada per:

$$\pp{\bullet/B}(A) = \pp{A/B} \ \ \forall A \in {\cal F}$$

\'es una nova probabilitat sobre ${\cal F}$.

\end{enumerate}

\begin{defin}
Direm que dos successos $A, B \in {\cal F}$ s\'on {\bf
independents}\index{successos independents}
si \break $ \pp{A \cap B} = \pp{A} \cdot \pp{B}$.
\end{defin}

{\bf Propietats.}

\begin{enumerate}

\item S\'on equivalents:

\begin{enumerate}

\item $A, B$ independents,

\item $A, B^c$ independents,

\item $A^c, B$ independents.

\end{enumerate}

\item $\Omega$ i $\emptyset$ s\'on independents de qualsevol succ\'es $A$.

\item Si $A$ \'es independent d'ell mateix, aleshores o b\'e 
$\pp{A} = 0$ o b\'e $\pp{A} = 1$.

\end{enumerate}

Com ja hem dit, donat un succ\'es no nul~$B$, la probabilitat
condicionada~$\pp{\bullet/B}$
\index{probabilitat!condicionada} tamb\'e \'es una probabilitat sobre~${\cal
F}$. Aleshores podem definir el seg\"uent tipus d'independ\`encia: 

\begin{defin}
Dos successos $A,B\in {\cal F}$ s\'on {\bf condicionalment independents}
\index{independencia condicionada@independ\`encia condicionada} donat un altre
succ\'es no nul~$C\in {\cal F}$ si:
\[
\pp{A\cap B/C}=\pp{A/C}\cdot \pp{B/C}.
\]
\end{defin}

{\bf Nota:}

En principi, pareix que hi hauria d'haver relaci\'o entre la independ\`encia 
de dos successos i la independ\`encia condicionada. Per exemple, si
$C=\Omega$, coincideixen les dues independ\`encies. De totes formes, en
general, no hi ha cap relaci\'o entre elles. (Vegeu problemes
proposats.)

\section{Problemes resolts}

\begin{probres}
{Trobau l'espai mostral per a l'experiment que consisteix a treure
dues bolles amb reposici\'o i sense reposici\'o d'una urna que cont\'e 5 bolles
(suposau que les bolles estan numerades d'1 a 5).}
\end{probres}

\res{
Sigui $\Omega$ l'espai mostral. Aleshores tenim que:
\begin{itemize}
\item {$\Omega=\{(i,j),i=1,\ldots,5,j=1,\ldots,5\}$, si l'experiment \'es amb 
reposici\'o.}
\item {$\Omega=\{(i,j),i=1,\ldots,5,j=1,\ldots,5,j\not = i\}$, si l'experiment
\'es    sense reposici\'o.}
\end{itemize}
}

\begin{probres}
{Per al problema anterior, especificau els successos seg\"uents:
\begin{itemize}
\item {$A=``\hbox{Treure exactament una bolla parell''}$}
\item {$B=``\hbox{Treure la segona bolla parell''}$}
\end{itemize}}
\end{probres}

\res{
\begin{itemize}
\item {$A=\{(1,2),(1,4),(2,1),(2,3),(2,5),(3,2),(3,4),(4,1),(4,3),(4,5),
(5,2),(5,4)\}$, tant si l'experiment \'es amb reposici\'o com sense.}
\item {\begin{itemize}
\item $B=\{(1,2),(1,4),(2,2),(2,4),(3,2),(3,4),(4,2),(4,4),(5,2),(5,4)\}$, si  
l'experiment \'es amb reposici\'o.
\item $B=\{(1,2),(1,4),(2,4),(3,2),(3,4),(4,2),(5,2),(5,4)\}$, si  
l'experiment \'es sense reposici\'o.
\end{itemize}}
\end{itemize}
}

\begin{probres}
{Donats $\Omega=\{1,2,3\}$, $A=\{1\}$, $B=\{3\}$, $C=\{2\}$ i suposant
que $\pp{A}={1\over 3}$, $\pp{B}={1\over 3}$, trobau: $\pp{C}$, 
$\pp{A\cup B}$, $\pp{A^c}$, $\pp{A^c \cap B^c}$, 
$\pp{A^c \cup B^c}$ i $\pp{B\cup C}$.}
\end{probres}

\res{
\begin{itemize}
\item $\pp{C}=1-\pp{A}-\pp{B}=1-\frac{1}{3}-\frac{1}{3}=\frac{1}{3}.$
\item $\pp{A\cup B}=\pp{A}+\pp{B}=\frac{2}{3}$, ja que $A\cap B=\emptyset$.
\item $\pp{{A}^c}=1-\pp{A}=\frac{2}{3}$. 
\item $\pp{{A}^c\cap {B}^c}=
\pp{{(A\cup B)}^c}=1-\pp{A\cup B}= \frac{1}{3}$.
\item $\pp{{A}^c\cup {B}^c}=\pp{{A}^c}+
\pp{{B}^c}- \pp{{A}^c \cap {B}^c}=
\frac{2}{3}+\frac{2}{3}-\frac{1}{3}=1$.
\item $\pp{B\cup C}=\pp{B}+\pp{C}=\frac{2}{3}$, ja que $B\cap C=\emptyset$.
\end{itemize}
}

\begin{probres}
{Es pinta un ``1'' en una cara d'una moneda i un ``2'' en l'altra
cara. Es tiren tres monedes pintades d'aquesta manera. Trobau la probabilitat
que la suma dels tres nombres que surtin sigui 3, que sigui 4, que sigui
5 i que sigui~6.}
\end{probres}

\res{
L'espai mostral ser\`a en aquest cas:
\[ 
\begin{array}{rl}
\Omega=& \{ (1,1,1),(1,1,2),(1,2,1),(1,2,2),\\
& (2,1,1),(2,1,2),(2,2,1),(2,2,2)\}
\end{array}
\]
Fixau-vos que, en aquest cas, tots els successos elementals tenen la
mateixa probabilitat de sortir. Aix\'{\i}, doncs,
\begin{itemize}
\item $\pp{\mbox{``Suma igual a 3''}}=\pp{(1,1,1)}=\frac{1}{8}$.
\item $\pp{\mbox{``Suma igual a 4''}}=\pp{(1,1,2),(1,2,1),(2,1,1)}=
\frac{3}{8}$.
\item $\pp{\mbox{``Suma igual a 5''}}=\pp{(1,2,2),(2,2,1),(2,1,2)}=
\frac{3}{8}$.
\item $\pp{\mbox{``Suma igual a 6''}}=\pp{(2,2,2)}=\frac{1}{8}$.
\end{itemize}
}

\begin{probres}
{En un formiguer hi ha formigues vermelles i negres. En un
passad\'{\i}s del formiguer, on l'espai \'es tan estret que nom\'es pot passar
una formiga a la vegada, passen 4 formigues una darrera l'altra. Quantes
combinacions diferents de color es poden produir suposant que les formigues
vermelles no es poden distingir entre elles igualment que les negres?}
\end{probres}

\res{
Hem de trobar totes les variacions amb repetici\'o del conjunt
\break $\{$vermelles,  negres$\}$ presos els elements de 4 en 4. 
 Per tant, el nombre de combinacions ser\`a:
 \[ VR_2^4 =2^4=16.\]}

\begin{probres}
{Vint autom\`obils han de participar en una carrera, 8 dels quals s\'on
de la marca A, 7 de la marca B i la resta de la marca C. Si nom\'es es mira la
marca del cotxe a l'arribada, de quantes maneres poden arribar els cotxes
a la meta? De totes les possibles formes d'arribar a la meta, quantes
n'hi ha en qu\`e un cotxe de la marca A estigui en primer lloc? Quantes en qu\`e
hi hagi dos cotxes de la marca A en els dos primers llocs?}
\end{probres}

\res{
\begin{itemize}
\item Maneres d'arribar a la meta $= PR_{20}^{8,7,5} =\frac{20!}{8!\cdot 7!
\cdot 5!}=99768240$ maneres.
\item Maneres en qu\`e un cotxe de la marca $A$ arriba en primer lloc 
$= PR_{19}^{7,7,5} =\frac{19!}{7!\cdot 7!
\cdot 5!}=39907296$ maneres.
\item Maneres en qu\`e dos cotxes de la marca $A$ arriben en primer lloc 
$= PR_{18}^{6,7,5} =\frac{18!}{6!\cdot 7!
\cdot 5!}=14702688$ maneres.
\end{itemize}}

\newpage

\begin{probres}
{Trobau la probabilitat que 5 cartes escollides a l'atzar d'una
baralla de 52 cartes contenguin:
\begin{itemize}
\item[a)]{exactament dues parelles.}
\item[b)]{{\it full} (3 cartes d'un mateix 
nombre i 2 d'un altre).}
\item[c)]{Flor (5 cartes d'un mateix pal).}
\item[d)]{Correguda (5 cartes en seq\"u\`encia ascendent).}
\end{itemize}}
\end{probres}

\res{
\begin{itemize}
\item[a)] $\pp{\mbox{2 parelles }}=
\frac{Casos\  favorables}{Casos\ possibles}
=\frac{13\cdot {4\choose 2}\cdot 12\cdot {4\choose 2}\cdot 11 {4\choose 1}}{
{52\choose 5}}=
\frac{247104}{2598960}\approx 0.0951$

L'explicaci\'o de la f\'ormula anterior \'es la seg\"uent:

Primer hem de 
tenir en compte que hem de pensar en nombres en lloc de cartes. 

Aix\'{\i} doncs, tenim $13$ nombres per formar la primera parella i $4$ 
pals per triar per cada nombre. Per tant, les maneres que tenim per formar
la primera parella seran: $13\cdot {4\choose 2}$.

Ara ens queden $12$ nombres per formar la segona parella i $4$ pals per 
triar per cada nombre. Per tant, les maneres que tenim per formar 
la segona parella seran: $12\cdot {4\choose 2}$.

Finalment, ens queda un nombre per triar d'entre $4$ pals. Per tant, les
maneres que tenim de triar l'\'ultima carta seran: $11\cdot {4\choose 1}$.

Combinant els tres resultats anteriors, surt la f\'ormula.

Tots els altres casos s'han resolt seguint la mateixa filosofia.
\item[b)] $\pp{\mbox{``full''}}=
\frac{Casos\  favorables}{Casos\ possibles}
=\frac{13\cdot {4\choose 3}\cdot 12\cdot {4\choose 2}}{
{52\choose 5}}=\frac{3744}{2598960}\approx 0.00144$ 
\item[c)] $\pp{\mbox{``Flor''}}=
\frac{Casos\  favorables}{Casos\ possibles}
=\frac{4\cdot {13\choose 5}}{
{52\choose 5}}=\frac{1287}{2598960}\approx 0.000495$ 
\item[d)] $\pp{\mbox{``Correguda''}}=
\frac{Casos\  favorables}{Casos\ possibles}
=\frac{{4\choose 1}\cdot {4\choose 1}\cdot {4\choose 1}\cdot {4\choose 1}
\cdot {4\choose 1}\cdot 9}{
{52\choose 5}}=\frac{9216}{2598960}\approx 0.003546$ 

L'\'ultima f\'ormula necessita una mica d'explicaci\'o.

Fixau-vos que si posam els nombres en ordre ascendent:

{\tt 1 2 3 4 5 6 7 8 9 10 J Q K}

podem fer en total 9 corregudes, suposant que no es pot 
donar la volta, o sigui, no s'admet el cas per exemple {\tt J Q K 1 2}.
\end{itemize}}

\begin{probres}
{Una urna cont\'e 2 bolles negres i 5 de blanques. Se selecciona
una bolla a l'atzar. Si la bolla \'es blanca, es torna a l'urna i s'afegeixen
dues bolles blanques m\'es. Si la bolla \'es negra, no es torna a
l'urna i no s'afegeix cap bolla m\'es. Es treu una segona bolla. Quina \'es
la probabilitat que la bolla sigui blanca?}
\end{probres}

\res{
Siguin els successos:
\begin{itemize}
\item[] $B_1=\{$ La primera bolla \'es blanca $\}$.
\item[] $N_1=\{$ La primera bolla \'es negra $\}$.
\item[] $B_2=\{$ La segona bolla \'es blanca $\}$.
\end{itemize}
Aplicant el teorema de les probabilitats totals, tenim:
\begin{eqnarray*}
\pp{B_2} &= &\pp{B_2/B_1}\cdot \pp{B_1}+\pp{B_2/N_1}\cdot \pp{N_1} \\ &=&
\frac{7}{9}\cdot \frac{5}{7}+\frac{5}{6}\cdot \frac{2}{7}=
\frac{35}{63}+\frac{10}{42}\approx 0.7936 
\end{eqnarray*}
}

\begin{probres}
{Dos fabricants, A i B, entreguen la mateixa pe\c{c}a a un
altre fabricant, que guarda totes les exist\`encies en el mateix lloc. Els
antecedents demostren que el 5\% de les peces entregades per A estan
defectuoses i que el 9\% de les peces entregades per B tamb\'e estan
defectuoses. A m\'es a m\'es, A entrega 4 vegades m\'es peces que B. Si es treu a
l'atzar una pe\c{c}a i es troba que no est\`a defectuosa, quina \'es la
probabilitat que l'hagi fabricada A.}
\end{probres}

\res{Considerem els successos:
\begin{itemize}
\item[] $A=\{$``Pe\c{c}a entregada per A''$\}$.
\item[] $B=\{$``Pe\c{c}a entregada per B''$\}$.
\item[] $D=\{$``Pe\c{c}a defectuosa''$\}$.
\end{itemize}
Com que A entrega 4 vegades m\'es peces que B, tenim que:
$\pp{A}=\frac{4}{5},\ \pp{B}=\frac{1}{5}$.
A m\'es a m\'es, tenim les probabilitats seg\"uents:

\noindent $\pp{D/A}=0.05,\ \pp{D/B}=0.09$.

Aplicant la f\'ormula de Bayes, podem trobar la probabilitat demanada:
\begin{eqnarray*}
\pp{A/{D}^c} & = & \frac{\pp{A\cap {D}^c}}{\pp{{D}^c}}=
\frac{\pp{A}\cdot \pp{{D}^c/A}}{\pp{A}\cdot \pp{{D}^c/A}+
\pp{B}\cdot \pp{{D}^c/B}} \\ 
& = & \frac{0.8\cdot 0.95}{0.8\cdot 0.95+0.2 
\cdot 0.91}\approx 0.8068.
\end{eqnarray*}}

\begin{probres}
{Provau que si els successos $A$ i $B$ s\'on independents, tamb\'e ho
s\'on ${A}^c$ i~${B}^c$.}
\end{probres}

\res{Estam suposant que $\pp{A\cap B}=\pp{A}\cdot \pp{B}$.

Per tant, tenim:
\begin{eqnarray*}
\pp{{A}^c\cap {B}^c} & = & \pp{ {(A\cup B)}^c}=1-\pp{A\cup
B}  = 1-(\pp{A}+\pp{B}-\pp{A\cap B})
\\ &=&  1-\pp{A}-\pp{B}+\pp{A}\cdot\pp{B} 
  = (1-\pp{A})\cdot (1-\pp{B}) \\ & = & \pp{{A}^c}\cdot \pp{{B}^c}
\end{eqnarray*}}

\begin{probres}
{Considerem una urna amb $N$ bolles, de les quals n'hi ha $n$ de blanques. Traiem
una bolla de l'urna. Si surt blanca, la tornam a posar a l'urna, per\`o si no
\'es blanca no la hi tornam a posar. Traiem una bolla per segona vegada i surt
blanca. Trobau la probabilitat $p$ que hagi sortit blanca la primera
vegada.\newline{\footnotesize Final. Setembre 94.}}
\end{probres}

\res{
Considerem els successos:

\begin{itemize}
\item[] $B_1=\{$``Surt blanca la primera vegada''$\}$.
\item[] $B_2=\{$``Surt blanca la segona vegada''$\}$.
\item[] $N_1=\{$``No surt blanca la primera vegada''$\}$.
\end{itemize}

Aplicant la f\'ormula de Bayes, tenim:

\begin{eqnarray*}
p & = & \pp{B_1/B_2}=\frac{\pp{B_1\cap B_2}}{\pp{B_2}}
=\frac{\pp{B_1}\cdot \pp{B_2/B_1}}{\pp{B_1}\cdot \pp{B_2/B_1}+
\pp{N_1}\cdot \pp{B_2/N_1}} \\
& = & \frac{\frac{n}{N}\cdot\frac{n}{N}}{\frac{n}{N}\cdot
\frac{n}{N}+\frac{N-n}{N}\cdot \frac{n}{N-1}}=
\frac{\frac{n}{N}}{\frac{n}{N}+\frac{N-n}{N-1}} \\
& = & \frac{\frac{n}{N}}{\frac{n (N-1)+N (N-n)}{N (N-1)}}=
\frac{n (N-1)}{n N-n-n N+N^2}=\frac{n (N-1)}{N^2-n}
\end{eqnarray*}
}

\begin{probres} {Una quarta part de la poblaci\'o ha estat vacunada contra una
malaltia contagiosa. Durant una epid\`emia, s'observa que d'entre els malalts
n'hi ha un que ha estat vacunat per cada quatre que no hi estan. \begin{itemize}
\item[a)] Ha tengut gens d'efic\`acia la vacuna? \item[b)] D'altra banda, se sap
que hi ha un malalt entre cada 12 persones vacunades. Quina \'es la probabilitat
que estigui malalta una persona que no s'ha vacunat? \end{itemize}}
\end{probres}

\res{ \begin{itemize} \item[a)] S\'{\i} que ha tengut efic\`acia, ja que $$ \pp{
\mbox{estar vacunat}} = \frac{1}{4} > \pp{\mbox{estar vacunat / malalt}} =
\frac{1}{5}. $$

\item[b)] Posem:
\[ 
V : \mbox{estar vacunat}, \quad
M : \mbox{estar malalt}.
\]

Sabem:

\begin{eqnarray*} \pp{V} = \frac{1}{4}, & & 
\pp{ V / M } = \frac{1}{5}, \\ 
\pp{ V^c } = \frac{3}{4}, & & 
\pp{ V^c / M } = \frac{4}{5}. 
\end{eqnarray*}

Aleshores $$ \pp{ M / V^c } = {\pp{ M \cap V^c } \over \pp{ V^c}} = {\pp{ M} 
\cdot \pp{ V^c / M } \over \pp{ V^c}}. $$ Falta con\`eixer $\pp{ M }.$
Sabem per\`o que $\pp{ M / V } = \frac{1}{12}.$ Per tant $$ \pp{ V / M } = 
{\pp{ V \cap M } \over \pp{ M }} \Longrightarrow \pp{ M } = {\pp{ V \cap M}
\over \pp{ V / M }}. $$ I ara, $$ \pp{ V \cap M} = \pp{ V } \cdot \pp{ M / V} = 
\frac{1}{4} \cdot \frac{1}{12} = \frac{1}{48}, $$ d'on $\pp{ M } = {1/48
\over 1/5} = \frac{5}{48}$. Finalment, $$ \pp{ M / V^c } = {5/48 \cdot 4/5 \over
3/4} = {1/12 \over 3/4} = \frac{4}{36} = \frac{1}{9}. $$ \end{itemize}}

\begin{probres} {Un llarg missatge s'ha codificat en termes de dos s\'{\i}mbols
$A$ i $B$ per a transmetre'l a trav\'es d'un canal de comunicaci\'o. La
codificaci\'o \'es tal que $A$ apareix el doble de vegades que $B$ en el missatge
codificat. El soroll del canal \'es tal que, quan $A$ es transmet, es rep com a
$A$ amb probabilitat 0.8 i com a $B$ amb probabilitat 0.2; quan $B$ es transmet,
es rep com a $B$ amb probabilitat 0.7 i com a $A$ amb probabilitat 0.3.
\begin{itemize} \item[a)] Quina \'es la freq\"u\`encia relativa de $A$ en el
missatge rebut? \item[b)] Si la darrera lletra del missatge que s'ha rebut \'es
una $A$, quina \'es la probabilitat que s'hagi enviat una $A$? 
\end{itemize}}
\end{probres}

\newpage

\res{\begin{itemize} \item[a)] Posem: \begin{eqnarray*} A_t : \mbox{$A$
transm\`es}, & \> & A_r : \mbox{$A$ rebut}, \\ B_t : \mbox{$B$ transm\`es}, & \>
& B_r : \mbox{$B$ rebut}. \end{eqnarray*}

Aleshores tenim: $$ \pp{ A_t } = 2 \cdot \pp{ B_t } \Longrightarrow \pp{ A_t }
= 2/3, \> p\{ B_t \} = 1/3. $$ Tamb\'e,
\[
\begin{array}{llcll}
\pp{ A_r / A_t} = &  0.8, & & \pp{ B_r / A_t } = & 0.2, \\ 
\pp{ B_r / A_t} = &  0.3, & & \pp{ B_r / B_t } = & 0.7. 
\end{array}
\]
Aleshores,
 \begin{eqnarray*} \pp{ A_r } & = & \pp{ A_t } \cdot \pp{ A_r / A_t}
+ \pp{ B_t} \cdot \pp{ A_r / B_t} \\ & = & \frac{2}{3} \cdot \frac{8}{10} +
\frac{1}{3} \cdot \frac{3}{10} = \frac{19}{30} = 0.633. \end{eqnarray*}

\item[b)] 
\[
 \pp{ A_t / A_r }  =  {\pp{ A_t } \cdot \pp{ A_r /
A_t } \over \pp{ A_r}} =  {\frac{2}{3} \times \frac{8}{10} \over
\frac{19}{30}} = \frac{16}{19} = 0.84.
\]
\end{itemize}}

\begin{probres} {Un comerciant ha de viatjar en avi\'o entre Bangkok i Bagdad.
Preocupat, demana a la companyia a\`eria quina \'es la probabilitat que hi
hagi com a m\'{\i}nim una bomba dins l'avi\'o i li diuen que \'es 0.1. M\'es
preocupat encara, demana quina seria la probabilitat que n'hi hagu\'es com a
m\'{\i}nim dues, i li diuen que seria 0.01. M\'es tranqui{\lgem}itzat, decideix dur una
bomba en el seu equipatge. Quina valoraci\'o estad\'{\i}stica podem fer de la
seva decisi\'o?} \end{probres}

\res{ Posem: \begin{eqnarray*} A & : & \mbox{com a m\'{\i}nim una bomba}, \\ B &
: & \mbox{com a m\'{\i}nim dues bombes}. \end{eqnarray*}

Sabem que $\pp{ A} = 0.1$ i $ \pp{ B } = 0.01$, i per tant $\pp{ A \cap B } =
\pp{ B } = 0.01$. Aleshores,
 $$ \pp{ B / A } = {\pp{ A \cap B } \over \pp{ A}} = 
{0.01 \over 0.1} = 0.1 = \pp{ A} !! $$ Per tant, de res serveix que dugui
la seva bomba, com no sigui per augmentar la probabilitat que una bomba
exploti per casualitat. El comerciant ha conf\'os $\pp{ B}$ amb $\pp{ B / A}$.}

\begin{probres} {Un auditor ha estat informat de l'exist\`encia d'un error
deliberat en un dels documents del departament de comptabilitat d'una empresa. La
probabilitat que l'error es trobi en un lot de 10 documents \'es 0.05 i \'es
igualment probable que el defecte estigui en qualsevol d'aquests documents. Si ja
n'ha inspeccionat 9 sense trobar l'error, quina \'es la probabilitat que
l'error no es trobi en aquest lot?}
\end{probres}

\res{Posem:

\begin{eqnarray*} D_i & : & 
\mbox{l'error es troba en el document $i$-\`essim ( i
= 1, \ldots , 10)}, \\ L & : & \mbox{l'error es troba en el lot}. \end{eqnarray*}

Aleshores ens demanam $$ \pp{ D_{10}^c / D_1^c \cap \cdots \cap D_9^c } = {\pp{
D_1^c \cap \cdots \cap D_{10}^c} \over \pp{ D_1^c \cap \cdots \cap D_9^c}}.
$$ I ara,
\begin{eqnarray*} \pp{ D_1^c \cap \cdots \cap D_{10}^c} & = & \pp{ L^c} =
0.95 \\ \pp{ D_1^c \cap \cdots \cap D_9^c} & = & \pp{ L } \cdot \pp{ D_1^c
\cap \cdots \cap D_9^c / L}  +  \pp{ L^c } \cdot \pp{ D_1^c \cap \cdots
\cap D_9^c / L^c} \\ & = & 0.05 \times \frac{1}{10} + 0.95 \times 1 = 0.955.
\end{eqnarray*}

Finalment, 
$$
\pp{ D_{10}^c / D_1^c \cap \cdots \cap D_9^c} = {0.95 \over
0.955} = 0.995.
$$
}

\newpage

\section{Problemes proposats}

\begin{prob}
{Se seleccionen a l'atzar tres cartes sense reposici\'o d'una baralla
que cont\'e 3 cartes vermelles, 3 de blaves, 3 de verdes i 3 de negres. Especificau
un espai mostral per a aquest experiment i trobau els successos seg\"uents:
\begin{itemize}
\item {$A=``\hbox{Totes les cartes seleccionades s\'on vermelles''}$}
\item {$B=``\hbox{1 carta \'es vermella, 1 \'es verda i l'altra blava''}$} 
\item {$C=``\hbox{Surten tres cartes de colors diferents''}$}
\end{itemize}}
\end{prob}

\begin{prob}
{Es llancen a l'aire dues monedes iguals. Trobau la probabilitat 
que surtin dues cares iguals.}
\end{prob}

\begin{prob}
{Suposau que s'ha trucat un dau de tal manera que la probabilitat 
que surti un determinat nombre sigui proporcional al nombre. Trobau la
probabilitat dels successos elementals, que surti nombre parell i 
que surti nombre senar. Feu el mateix problema per\`o ara suposant que la
probabilitat que surti un determinat nombre sigui inversament
proporcional al nombre.}
\end{prob}

\begin{prob}
{De quantes maneres diferents es
poden co{\lgem}ocar tres llibres distints en una taula?}
\end{prob}

\begin{prob}
{Sis persones es disposen a entrar al cinema. De quantes maneres
diferents es poden co{\lgem}ocar en fila?}
\end{prob}

\begin{prob}
{Tres ciutadans destacats han de rebre premis. Si hi ha 4 candidats a
aquests premis, de quantes maneres diferents es poden distribuir aquests
si un ciutad\`a nom\'es pot rebre com a m\`axim 1 premi? I si en pot rebre m\'es
d'un?}
\end{prob}

\begin{prob}
{Donat un conjunt de 15 punts en el pla, quantes l\'{\i}nies es 
necessiten per juntar totes les possibles parelles de punts?}
\end{prob}

\begin{prob}
{Donada una caixa amb els seg\"uents focus: 2 de 25 vats, 4 de 40 vats
i 4 de 100 vats, de quantes maneres se'n poden escollir 3?}
\end{prob}

\begin{prob}
{Suposem que les plaques de matr\'{\i}cula dels cotxes es componen de 3
lletres seguides de 3 d\'{\i}gits. Si es poden emprar totes les combinacions
possibles, quantes plaques diferents es poden formar?}
\end{prob}

\begin{prob}
{De quantes maneres diferents es poden formar $2$ equips d'una lliga que
tengui $8$ equips?}
\end{prob}

\begin{prob}
{En un magatzem hi ha caixes vermelles i verdes. De
quantes maneres es poden co{\lgem}ocar en fila 20 caixes si 15 s\'on
vermelles i 5 verdes? I si n'hi ha 10 de cada color?}
\end{prob}

\begin{prob}
{En una pres\'o de 100 presos, es varen seleccionar a l'atzar dues
persones per posar-les en llibertat. Quina \'es la probabilitat que el m\'es
vell dels presos sigui un dels elegits? I que se seleccioni la
parella formada pel m\'es vell i el m\'es jove?}
\end{prob}

\begin{prob}
{S'apunten A, B i C en una cursa. Quina \'es la probabilitat 
que A acabi abans que C, si tots tenen la mateixa habilitat i no hi pot haver
empats? Quina \'es la probabilitat que A acabi abans que B i C?}
\end{prob}

\begin{prob}
{En una sala es troben $n$ persones. Trobau la probabilitat que
hi hagi almenys 2 persones amb el mateix mes de naixement. Donau aquest
valor per a $n=3,4,5,6$.}
\end{prob}

\begin{prob}
{Una urna cont\'e 4 bolles numerades amb 1, 2, 3 i 4, respectivament. Es
treuen 2 bolles sense reposici\'o. Sigui $A$ el succ\'es que la suma sigui
5, i sigui $B_i$ el succ\'es que la primera bolla treta tengui una $i$,
amb $i=1,2,3,4$. Trobau $\pp{A/B_i},\ i=1,2,3,4$ i $\pp{B_i/A},\
i=1,2,3,4$.}
\end{prob}

\begin{prob}
{Es llan\c{c}a a l'aire una moneda no trucada. Quina \'es la
probabilitat que la quarta vegada surti cara, si surt cara en les tres
primeres tirades? I si surten 2 cares en les 4 tirades?}
\end{prob}

\begin{prob}
{La urna 1 cont\'e 2 bolles vermelles i 4 de blaves. La urna 2 cont\'e 10
bolles vermelles i 2 de blaves. Si escollim a l'atzar una urna i en traiem una
bolla, quina \'es la probabilitat que la bolla seleccionada
sigui blava? I que sigui vermella?}
\end{prob}

\begin{prob}
{Suposem que la ci\`encia m\`edica ha desenvolupat una prova per al
diagn\`ostic de c\`ancer que t\'e un 95\% d'exactitud, tant en els que tenen
c\`ancer com en els que no. Si el 5 per mil de la poblaci\'o realment t\'e
c\`ancer, trobau la probabilitat que un determinat individu tengui
c\`ancer, si la prova ha donat positiu.}
\end{prob}

\begin{prob}
{Es tira una sola vegada un parell de daus. Si la suma dels dos \'es
com a m\'{\i}nim 7, trobau la probabilitat que sigui igual a $i$, per a
$i=7,8,9,10,11,12$.}
\end{prob}

\begin{prob}
{Una moneda no trucada es tira a l'aire 2 vegades. Considerem els
seg\"uents successos:
\begin{itemize}
\item {A: Surt una cara en la primera tirada.}
\item {B: Surt una cara en la segona tirada.}
\end{itemize}
S\'on els dos successos independents?}
\end{prob}

\begin{prob}
{Una urna cont\'e 4 bolles numerades amb 1, 2, 3 i 4, respectivament. Traiem
dues bolles sense reposici\'o. Sigui A el succ\'es que la primera bolla
treta tengui un 1 marcat i sigui B el succ\'es que la segona bolla treta
tengui un 1 marcat. Es pot dir que A i B s\'on independents? I si
l'experiment \'es amb reposici\'o?}
\end{prob}

\begin{prob}
{Se sap que els ${2\over 3}$ dels interns d'una certa pres\'o s\'on menors
de 25 anys. Tamb\'e se sap que els ${3\over 5}$ s\'on homes i que els
${5\over8}$ dels interns o s\'on dones o majors de 25 anys. Quina \'es
la probabilitat que
un presoner escollit a l'atzar sigui dona i de menys de 25
anys?\newline{\footnotesize Primer parcial Curs 92-93.}}
\end{prob}

\begin{prob}
{
Considerem una urna amb $2n$ bolles numerades de $1$ a $2n$. Traiem
$2$ bolles de l'urna sense reposici\'o. Sabent que la segona bolla \'es
parell, quina \'es la probabilitat que la primera bolla sigui senar?
\newline{\footnotesize Final. Juny 95.}
}
\end{prob}

\begin{prob}
{
Considerem el seg\"uent experiment aleatori: traiem $5$ nombres a l'atzar
sense reposici\'o a partir dels nombres naturals $1,2,\ldots,20$. Trobeu la 
probabilitat $p$ que exactament dos dels nombres siguin m\'ultiples de
$3$. \newline{\footnotesize Final. Setembre 95.}
}
\end{prob}

\begin{prob}
{
Sigui $\Omega$ un espai mostral i $A$, $B$ i $C$ tres successos. Provau que:
\begin{itemize}
\item[a)] Si $A$ i $B$ s\'on independents, tamb\'e ho s\'on $A$ i 
${B}^c$.

\item[b)] Si $A$, $B$ i $C$ s\'on independents, tamb\'e ho s\'on $A$,
$B$ i ${C}^c$.

\item[c)] \'Es cert que si $A$, $B$ i $C$ s\'on independents, tamb\'e
ho s\'on $A$, ${B}^c$ i ${C}^c$? I ${A}^c$, 
${B}^c$ i ${C}^c$? En el cas que la resposta sigui
negativa, donau contraexemples on la propietat falli.
\end{itemize}
{\footnotesize Final. Setembre 95.}
}
\end{prob}

\begin{prob}
En una urna hi ha~$10$ bolles, numerades d'$1$ fins a~$10$. Les~$4$ primeres
bolles, o sigui, les bolles $1$, $2$, $3$ i~$4$ s\'on blanques. Les bolles~$5$ 
i $6$ s\'on negres i les bolles $7$, $8$, $9$ i $10$ s\'on vermelles.
Traiem dues bolles sense reposici\'o. Sabent que la segona bolla \'es
negre, trobau la probabilitat~$p$ que la primera bolla sigui blanca.
\newline{\footnotesize Final. Juny 96.}
\end{prob}

\begin{prob}
{
Llan\c{c}am un dau no trucat~$3$ vegades. Trobau la probabilitat~$p$ 
que la suma de les~$3$ cares sigui~$10$.
\newline{\footnotesize Final. Setembre 96.}
}
\end{prob}

\begin{prob} {Quatre cartes numerades de l'1 al 4 estan girades cap avall damunt
d'una taula. Una persona, suposadament clarivident, anir\`a endevinant els valors
de les 4 cartes una a una. Suposant que \'es un farsant i que el que fa \'es dir
els quatre nombres a l'atzar, quina \'es la probabilitat que n'encerti com a
m\'{\i}nim un? (\`Obviament, no repeteix cap nombre.)} \end{prob}

\begin{prob} {Una manera d'augmentar la fiabilitat d'un sistema \'es
mitjan\c{c}ant la introducci\'o d'una c\`opia dels components en una
configuraci\'o para{\lgem}ela. Suposem que la NASA vol una probabilitat no menor que
0.99999 que el transbordador espacial entri en \`orbita al voltant de la Terra
amb \`exit. Quants de motors s'han de configurar en para{\lgem}el per tal d'assolir
aquesta fiabilitat, si se sap que la probabilitat que un qualsevol dels motors
funcioni adequadament \'es 0.95? Suposem que els motors funcionen de manera
independent els uns amb els altres.} 
\end{prob}

\begin{prob}
{
Dues empreses $A$ i $B$ fabriquen el mateix producte. L'empresa~$A$ t\'e
un $2\%$ de productes defectuosos mentre que l'empresa~$B$ en t\'e un
$1\%$. Un client rep una comanda d'una de les empreses (no sap de
quina) i comprova que la primera pe\c{c}a funciona. Si suposam que
l'estat de les peces de cada empresa \'es independent, quina \'es la
probabilitat que la segona pe\c{c}a que provi sigui bona? Comprovau
que l'estat de les dues peces no \'es independent, per\`o en canvi \'es
condicionalment independent donada l'empresa que les fabrica.
}
\end{prob}

\begin{prob}
{
Trobau un exemple de tres successos~$A$, $B$ i $C$ tals que $A$ i $B$
siguin independents per\`o en canvi no siguin condicionalment
independents donat~$C$.
}
\end{prob}
