\chapter{Estudi d'algunes distribucions conegudes}

\section{Resum te\`oric}

\begin{enumerate}

\item {\bf Bernoulli, $B(1,p)$.}

Considerem un experiment aleatori amb dos resultats possibles, $a_1$ i $a_2$,
amb probabilitats respectives $p \in (0,1)$ i $q = 1 - p$.

Definim la variable aleat\`oria $X : \Omega \to \RR$ donada per $X(a_1) = 1,$
\break $X(a_2) = 0$. $a_1$ se sol anomenar {\bf encert} i $a_2$ {\bf fallada}. La
funci\'o
de probabilitat de $X$ \'es $\pp{X = 1} = p, \ \pp{X = 0} = q$, i $X$ s'anomena 
{\bf variable aleat\`oria de Bernoulli amb par\`ametre
$p$},\index{variable aleatoria@variable aleat\`oria!de Bernoulli}
\index{distribucio@distribuci\'o!de Bernoulli} 
indicada amb $B(1,p)$. Cada experiment s'anomena una {\bf prova de
Bernoulli.}\index{prova de Bernoulli}
Resulta: $\EE X = p$ i $\Var X = p \cdot q$.

\item {\bf Binomial, $B(n,p)$.}

Considerem una successi\'o de $n$ repeticions independents d'una prova de
Bernoulli amb par\`ametre $p$. Sigui $X$ la variable aleat\`oria que d\'ona el
nombre
d'encerts en les $n$ proves. Aleshores $X(\Omega) = \{ 0, 1, \ldots , n \}$, i

$$\pp{X = k} = {n \choose k} p^k q^{n-k}.$$

$X$ s'anomena {\bf variable aleat\`oria binomial amb par\`ametres $n$ i
$p$}\index{variable aleatoria@variable aleat\`oria!binomial}
\index{distribucio@distribuci\'o!binomial}, i 
s'indica amb $B(n,p)$. Resulta: $\EE X = np, \Var X = npq$.

\newpage
\item {\bf Geom\`etrica, $Geom (p)$.}

Considerem una successi\'o de repeticions independents d'una prova de Bernoulli
\index{prova de Bernoulli}
amb par\`ametre $p$. Sigui $X$ la variable aleat\`oria que d\'ona el nombre de
repeticions de l'experiment fins que s'obt\'e el primer encert. Aleshores
$X(\Omega) = \{ 1, 2, \ldots \}$, i

$$\pp{X = k} = p \cdot q^{k-1}.$$

$X$ s'anomena {\bf variable aleat\`oria geom\`etrica amb par\`ametre
$p$}.\index{variable aleatoria@variable aleat\`oria!geometrica@geom\`etrica}
\index{distribucio@distribuci\'o!geometrica@geom\`etrica}
Resulta:
$\EE X = 1/p, \Var X = 1/p^2$.

De vegades, es pren com a $X$ el nombre de repeticions de l'experiment abans
d'obtenir el primer encert. Aleshores $X(\Omega) = \{ 0, 1, 2, \ldots \}$, i
$\pp{X = k} = p \cdot q^k$. En aquest cas, E$X = q/p$, i $\Var X = q/p^2$.

$X$ tamb\'e s'anomena el {\bf temps d'espera}\index{temps d'espera}. Aquesta
variable aleat\`oria queda caracteritzada dins les variables aleat\`ories discretes per la propietat
anomenada de {\bf falta de mem\`oria}\index{falta de memoria@falta de
mem\`oria}
:

$$\pp{X \geq m+n / X \geq m} = \pp{X \geq n+1} \ \forall m,n.$$

(La probabilitat que un determinat component superi m\'es de $m+n$ proves
sabent que n'ha superat m\'es de $m$ \'es igual a la que superi m\'es de $n$
proves).

\item {\bf Poisson, $Poiss(\lambda)$.}

Direm que $X$, una variable aleat\`oria discreta amb $X(\Omega) = \{ 0, 1, 2,
\ldots \}$, t\'e la {\bf distribuci\'o de Poisson amb par\`ametre $\lambda >
0$}\index{variable aleatoria@variable aleat\`oria!de Poisson}
\index{distribucio@distribuci\'o!de Poisson},
indicada amb $Poiss(\lambda$), si

$$\pp{X = k} = {\lambda^k \over k!} \e^{-\lambda}, \ k \geq 0.$$

Resulta: $\EE X = \lambda, \Var X = \lambda$.

Aquesta distribuci\'o \'es molt important i pot apar\`eixer de dues maneres:

\begin{enumerate}

\item Com a l\'{\i}mit d'una $B(n,p)$ quan $n \to \infty$ i $p \to 0$ de manera
que
$\lambda = np$ es conserva constant. A la pr\`actica, l'aproximaci\'o \'es bona
per a
$n \geq 30$ i $p \leq 0.1$.

\item Com a model d'un proc\'es aleatori que d\'ona lloc a successos al llarg de
l'espai o del temps amb les condicions seg\"uents:

\begin{itemize}

\item el nombre de successos que tenen lloc en una unitat arbitr\`aria de temps
(espai) \'es independent del nombre de successos que tenen lloc en una altra
unitat de la  mateixa mida,

\item la probabilitat que ocorri un succ\'es en una unitat molt petita \'es
proporcional a la mida de la unitat,

\item la probabilitat que tengui lloc m\'es d'un succ\'es en una unitat molt
petita \'es despreciable.

\end{itemize}

Si $\lambda$ \'es el ritme d'aparici\'o dels successos, resulta que el nombre de
successos que apareixen en l'interval $(0,t]$ \'es una variable aleat\`oria
$Poiss(\lambda t)$.

\end{enumerate}

\item {\bf Hipergeom\`etrica.}

S'aplica a les extraccions sense reempla\c{c}ament. 
Considerem una urna amb $N$ bolles, de les quals $N_1$ s\'on negres i 
$N_2$ blanques ($N_1 + N_2 = N$). Extraiem $n$ bolles sense reempla\c{c}ament. 
Considerem la variable aleat\`oria $X$ que d\'ona el nombre de bolles negres 
que han sortit. Aleshores resulta que $X$ t\'e una funci\'o de probabilitat 
donada per:
$$
\pp{X=k} = {{N_1 \choose k} \> {N_2 \choose n-k} \over {N \choose n}}.
$$
$X$ s'anomena una {\bf variable aleat\`oria hipergeom\`etrica}.
\index{variable aleatoria@variable aleat\`oria!hipergeometrica@hipergeom\`etrica}
\index{distribucio@distribuci\'o!hipergeometrica@hipergeom\`etrica} Resulta: 
\newline
$\EE X = np$, i $\Var X = npq{N-n \over N-1}, $ on $p={N_1 \over N}$ i $q=1-p$.

\item {\bf Uniforme, $U[a,b]$.}

Siguin $a, b \in \RR,\  a < b$. Definim $f : \RR \to \RR$ com

$$f(x) = \left\{ 
\begin{array}{ll} 
{1 \over b - a}, & {\rm si } \ x \in [a,b],\\ 
0, & {\rm en \> cas \> contrari}. 
\end{array} 
\right.$$

$f$ \'es una funci\'o de densitat i la corresponent funci\'o de distribuci\'o
\'es

$$F(x) = \left\{
\begin{array}{ll} 
0, & {\rm si } \ x \leq a,\\ 
{x-a \over b-a}, &  {\rm si } \ a \leq x \leq b,\\ 
1, & {\rm si } \ x \geq b.
\end{array} \right.$$

La variable aleat\`oria $X$ que t\'e per funci\'o de distribuci\'o $F$ s'anomena
{\bf
variable aleat\`oria uniforme en l'interval
$[a,b]$}\index{variable aleatoria@variable aleat\`oria!uniforme}
\index{distribucio@distribuci\'o!uniforme} 
i s'indica amb $U[a,b]$. Resulta
$$\EE X = {a + b \over 2}, \ \Var X = {(b - a)^2 \over 12}.$$

\item {\bf Exponencial, Exp($\lambda$).}

Fixem $\lambda > 0$. Direm que una variable aleat\`oria absolutament
cont\'{\i}nua $X$
t\'e una {\bf distribuci\'o exponencial amb par\`ametre $\lambda$},
\index{variable aleatoria@variable aleat\`oria!exponencial}
\index{distribucio@distribuci\'o!exponencial} indicada amb 
Exp($\lambda$), si la seva densitat la d\'ona:
$$f(x) = \lambda \cdot \e^{-\lambda x}, \ x \geq 0.$$

La funci\'o de distribuci\'o ser\`a aleshores:
$$F(x) = 1 - \e^{-\lambda x}, \ x \geq 0.$$

Una manera d'introduir la distribuci\'o exponencial \'es a partir de la de
Poisson.\index{variable aleatoria@variable aleat\`oria!de Poisson}
Considerem un flux de successos aleatoris com en un proc\'es de Poisson
amb ritme
$\lambda$. Sigui $X$ la variable aleat\`oria que d\'ona l'interval de temps entre
dos successos consecutius. Aleshores $X$ t\'e una distribuci\'o Exp($\lambda$).
Resulta: E$X = 1/\lambda, \Var X = 1/\lambda^2$.

Dins les variables aleat\`ories absolutament cont\'{\i}nues, les exponencials
queden
caracteritzades per la propietat de falta de mem\`oria.
\index{falta de memoria@falta de
mem\`oria}

\item {\bf Normal, $N(\mu,\sigma^2)$.}

\'Es probablement la distribuci\'o absolutament cont\'{\i}nua m\'es important.
\'Es la
que t\'e per densitat la funci\'o:

$$f(x) ={1 \over \sqrt{2 \pi} \sigma} \cdot \e^{-{(x - \mu)^2} \over 2
\sigma^2},$$

la gr\`afica de la qual \'es la coneguda campana de Gauss.\index{campana de
Gauss}\index{campana de Gauss} Es diu que $X$ t\'e la {\bf distribuci\'o normal} 
o {\bf gaussiana}.\index{variable aleatoria@variable aleat\`oria!normal}
\index{variable aleatoria@variable aleat\`oria!gaussiana}
\index{distribucio@distribuci\'o!normal}
\index{distribucio@distribuci\'o!gaussiana} Resulta: $\EE X = \mu$ i
$\Var X = \sigma^2$.

Si consideram $\mu = 0$ i $\sigma^2 = 1$, obtenim la 
{\bf distribuci\'o normal est\`andard}
\index{variable aleatoria@variable aleat\`oria!normal!estandard@est\`andard}
\index{distribucio@distribuci\'o!normal!estandard@est\`andard}
$N(0,1)$.

Donada una variable aleat\`oria $X$ amb distribuci\'o $N(\mu,\sigma^2)$, la
variable
aleat\`oria $\displaystyle Y = {X - \mu \over \sigma}$ t\'e una distribuci\'o
$N(0,1)$. Rec\'{\i}procament, si $X$ t\'e una distribuci\'o $N(0,1)$, aleshores
$Y = \sigma \cdot X + \mu$ t\'e una distribuci\'o $N(\mu, \sigma^2)$.

\end{enumerate}

\section{Problemes resolts}

\begin{probres}
{En una urna hi ha 8 bolles
vermelles i 2 de negres. Es treuen 20 bolles amb reposici\'o. Sigui $Y$ la
variable aleat\`oria que d\'ona
el nombre de bolles vermelles que surten. Calculau $\mu_Y$, $\sigma_Y$, 
\mbox{$\pp{Y=16}$},
\mbox{$\pp{Y<14}$} i \mbox{$\pp{Y>18}$.}} 
\end{probres}

\res{La variable aleat\`oria $Y$ \'es una variable binomial amb par\`ametres 
$n=20$ i $p=$``probabilitat que una bolla sigui 
vermella''$=\frac{8}{10}=\frac{4}{5}$.

Per tant, 
\begin{eqnarray*}
	\mu_Y  & = & n p=20\cdot\frac{4}{5}=16,  \\
	\sigma_Y & = & \sqrt{n p 
	(1-p)}=\sqrt{20\cdot\frac{4}{5}\cdot\frac{1}{5}}=\sqrt{\frac{80}{25}}\approx 
	1.7888, \\
	\pp{Y=16} & = & {20\choose 16} {\left(\frac{4}{5}\right)}^{16} 
	{\left(\frac{1}{5}\right)}^{4}\approx 0.2182, \\
	\pp{Y<14} & = & 1- \pp{ Y\geq 14} = 1-\sum\limits_{i=14}^{20} {20\choose 
	i} {\left(\frac{4}{5}\right)}^{i} 
	{\left(\frac{1}{5}\right)}^{20-i}\approx 0.0867, \\ 
	\pp{Y>18} & = & {20\choose 19} {\left(\frac{4}{5}\right)}^{19} 
	{\left(\frac{1}{5}\right)} + {20\choose 20} {\left(\frac{4}{5}\right)}^{20} 
	\approx 0.06917.
\end{eqnarray*}
 }

\begin{probres}
{Trobau la funci\'o generatriu dels moments
per a una variable aleat\`oria binomial amb par\`ametres $n$ i $p$.} 
\end{probres}

\res{Si $X$ \'es una variable aleat\`oria binomial amb par\`ametres $n$ i $p$, 
aleshores el rang de $X$ \'es:
\[
X(\Omega)=\{0,1,\ldots,n\},
\]
amb funci\'o de probabilitat:
\[
f_X (i)={n\choose i} p^i {(1-p)}^{n-i}:= {n\choose i} p^i q^{n-i}\ (q=1-p).
\]
Per tant, la funci\'o generatriu de moments ser\`a:
\begin{eqnarray*}
	m_X (t) & = & \EE\left( \e^{t x}\right) =\sum\limits_{i=0}^n \e^{t i} f_X 
	(i) = \sum\limits_{i=0}^n \e^{t i}{n\choose i} p^i q^{n-i} \\
	 & = & \sum\limits_{i=0}^n {\left( p \e^{t}\right)}^i {n\choose i} 
	 q^{n-i}= {\left( p \e^t +q\right)}^n.
\end{eqnarray*}
}

\begin{probres}
{Es llan\c{c}a una moneda no trucada
fins que surti cara. Quina \'es la probabilitat que es necessitin menys de
3 intents? I la que es necessitin menys de 4 intents?} 
\end{probres}

\res{La variable aleat\`oria $X=$``nombre d'intents necessaris fins que 
surti cara'' \'es una variable aleat\`oria geom\`etrica amb par\`ametre 
$p=\frac{1}{2}$, ja que la moneda no est\`a trucada.

El rang de $X$ \'es
\[
X(\Omega)=\{1,2,\ldots\}=\{ n\in \ZZ | n\geq 1\},
\]
amb funci\'o de probabilitat:
\[ 
f_X(n)= p\cdot (1-p)^{n-1}:=p q^{n-1},\ n\geq 1.
\]
Per tant:
\[
	\pp{ X\leq 3}  =  \sum\limits_{i=1}^3 p\cdot q^{i-1}=\frac{7}{8},  
\quad
	\pp{ X\leq 4}  =  \sum\limits_{i=1}^4 p\cdot q^{i-1}=\frac{15}{16}.
\]
}

\begin{probres}
{Una caixa
cont\'e 5 bolles, 3 de les quals estan fetes malb\'e. Escollim dues bolles a
l'atzar sense reposici\'o. Quina \'es la funci\'o de probabilitat per al
nombre de bolles fetes malb\'e en la mostra?} 
\end{probres}

\res{Considerem la variable aleat\`oria $X=$``nombre de bolles escollides 
fetes malb\'e''. Fixau-vos que $X$ \'es una variable hipergeom\`etrica amb 
par\`ametres $M=5$ (nombre total de bolles), $W=3$ (nombre de bolles d'un 
determinat tipus; en aquest cas, fetes malb\'e) i $n=2$ (nombre de bolles 
escollides).

El rang de $X$ ser\`a: $X(\Omega)=\{0,1,2\}$ ja que com a m\`axim escollim dues
 bolles.

La funci\'o de probabilitat ser\`a:
\[
f_X (k)=\pp{ X=k}=\frac{{3\choose k}\cdot {2\choose 2-k}}{{5\choose 2}}.
\]
La f\'ormula anterior surt d'aplicar la f\'ormula $\frac{\mbox{casos 
favorables}}{\mbox{casos possibles}}$ ja que, en aquest cas, tenim en total 
${5\choose 2}$ casos possibles d'escollir dues bolles qualsevol i 
${3\choose k}\cdot {2\choose 2-k}$ casos d'escollir $k$ bolles fetes malb\'e 
d'un total de $3$ i $2-k$ bolles no fetes malb\'e d'un total de $2$.

La funci\'o de probabilitat queda reflectida en la taula seg\"uent:
$$
\renewcommand{\arraystretch}{1.5}
\begin{tabular}{|c|c|c|c|}
	\hline
	$k$ & $0$ & $1$ & $2$  \\
	\hline
	$f_X (k)$ & $0.1$ & $0.6$ & $0.3$  \\
	\hline
\end{tabular}
$$
}

\begin{probres}
{Es fan proves de
Bernoulli amb probabilitat $p$ d'\`exit. Sigui $X$ la variable aleat\`oria que
ens d\'ona el nombre de proves necess\`aries per tenir \`exit $r$ vegades.
Trobau la funci\'o de probabilitat de $X$ (en aquest cas 
direm que la variable aleat\`oria $X$ t\'e una 
{\bf distribuci\'o de Pascal} o {\bf binomial negativa}).
\index{variable aleatoria@variable aleat\`oria!de Pascal}
\index{distribucio@distribuci\'o!de Pascal}
\index{variable aleatoria@variable aleat\`oria!binomial!negativa}
\index{distribucio@distribuci\'o!binomial!negativa}
}  
\end{probres}

\res{El rang de la variable aleat\`oria $X$ ser\`a:
\[
X(\Omega)=\{ r, r+1,\ldots\}=\{n\in \ZZ\ |\ n\geq r\},
\]
ja que hem de fer com a m\'{\i}nim $r$ proves per poder obtenir \`exit $r$ 
vegades.

De cara a trobar la funci\'o de probabilitat, hem de tenir en compte que si 
$X=n$, la 
darrera prova de Bernoulli ha de ser \`exit ja que, en cas contrari,
haur\'{\i}em  obtengut \`exit en un nombre de proves m\'es petit que $n$. En les
$n-1$ proves restants s'ha obtengut \`exit $r-1$ vegades. La funci\'o de 
probabilitat, ser\`a, doncs:
\[
f_X (n)=\pp{X=n}={n-1\choose r-1} p^r {(1-p)}^{n-r}:=
{n-1\choose r-1} p^r q^{n-r},\ n\geq r.
\]}

\begin{probres}
{En una determinada f\`abrica han
tengut lloc accidents a ra\'o d'1 cada 2 mesos. Suposant que tenen lloc de
forma independent, quin \'es el nombre d'accidents esperats en un any? Quina
\'es la desviaci\'o est\`andard del nombre d'accidents a l'any? Quina \'es la
probabilitat que no hi hagi cap accident en un determinat mes?} 
\end{probres}

\res{Podem dir que en aquesta f\`abrica hi ha hagut $0.5$ 
accidents per mes. 

Sigui la variable aleat\`oria $X=$``nombre d'accidents en una any''. $X$ \'es 
una variable de Poisson amb par\`ametre $\lambda s$, amb $\lambda =0.5$ i 
$s=12$ mesos (1 any). Per tant, \mbox{$X$ \'es $Poiss(6)$.}

El nombre d'accidents esperats en un any ser\`a 
\[
 \EE X=\lambda s =6 \mbox{ accidents.}
\]
i la desviaci\'o est\`andard del nombre d'accidents a l'any ser\`a:
\[
\sigma =\sqrt{\mbox{Var }X}=\sqrt{\lambda s}=\sqrt{6}\approx 2.4494 
\mbox{ accidents.}
\]
Considerem la variable aleat\`oria $Y=$``nombre d'accidents en un mes''. 
$Y$ \'es 
una variable aleat\`oria de Poisson amb par\`ametre $\lambda s$, amb $\lambda
=0.5$ i  $s=1$ mes. Per tant, $Y$ �s $Poiss(0.5)$.

La probabilitat que no hi hagi cap accident en un determinat mes ser\`a:
\[
\pp{Y=0}=f_Y(0)=\frac{{(\lambda s)}^0}{0!}\e^{-\lambda s}= 
\e^{-0.5}\approx 0.6065
\]}

\begin{probres}
{Si
$X$ \'es una variable aleat\`oria de Poisson amb par\`ametre $\lambda s$,
provau que el valor de $X$ m\'es probable \'es $\lambda s$.
\newline{\footnotesize  Indicaci\'o: Considerau $\scriptstyle {f_X(k)\over
f_X(k-1)},\ k=1,2,\ldots$}} 
\end{probres}

\res{La funci\'o de probabilitat d'una variable de Poisson amb par\`ametre 
$\lambda s$ \'es:
\[
f_X (k)=\frac{{(\lambda s)}^k}{k!} \e^{-\lambda s}.
\]
Considerem
\[
\frac{f_X(k)}{f_X (k-1)}=\frac{\frac{{(\lambda s)}^k}{k!}}
{\frac{{(\lambda s)}^{k-1}}{(k-1)!}}=\frac{\lambda s}{k}.
\]
Per tant:
\begin{itemize}
	\item[] Si $\lambda s\geq k\ \Rightarrow\ f_X(k)\geq f_X(k-1)\ 
	\Rightarrow f_X(k)$ creix.
	
	\item[] Si $\lambda s\leq k\ \Rightarrow\ f_X(k)\leq f_X(k-1)\ 
	\Rightarrow f_X(k)$ decreix.
\end{itemize}
Per a $k=\lambda s$, $f_X (k)$ pren el m\`axim.

{\footnotesize Nota: si $\lambda s\not\in \ZZ$, aleshores:
\[
\max\limits_{k\in \ZZ} f_X (k)=\max\{ f_X ([\lambda s]),f_X([\lambda s]+1)\},
\]
on $[\lambda s]$ \'es la part entera de $\lambda s$.}
}

\begin{probres}
{Trobau la funci\'o generatriu de moments
per a la variable aleat\`oria de Poisson de par�metre~$\lambda s$.} 
\end{probres}

\res{
\begin{eqnarray*}
	m_X (t) & = & \EE\left(\e^{t x}\right)=\sum\limits_{k=0}^\infty \e^{t k}
	\frac{{(\lambda s)}^k}{k!} \e^{-\lambda s} \\
	 & = & \sum\limits_{k=0}^\infty 
	\frac{{\left(\lambda s \e^{t}\right)}^k}{k!} \e^{-\lambda s}=\e^{-\lambda 
	s}\cdot \e^{\lambda s \e^t} = \e^{\lambda s \left( \e^t -1\right)}.
\end{eqnarray*}
}

\begin{probres}
{Trobau la funci\'o generatriu cumulativa per a 
una variable aleat\`oria de Poisson i feu-la servir per trobar $\mu_X$ i
$\sigma_X^2$.} 
\end{probres}

\res{Tenim que la funci\'o generatriu d'una variable aleat\`oria de Poisson 
val:
\[
m_X(t)=\e^{\lambda s \left( \e^t -1\right)}.
\]
Per tant, la funci\'o generatriu cumulativa valdr\`a:
\[
c_X(t)=\ln m_X(t)=\lambda s \left( \e^t -1\right).
\]
Trobem a continuaci\'o $\EE X$ i $\mbox{Var }X$:
\begin{itemize}
	\item[] $c_X'(t)=\lambda s \e^t,\ \Rightarrow\ \EE X=c_X'(0)=\lambda s$,
	
	\item[] $c_X''(t)=\lambda s \e^t,\ \Rightarrow\ \mbox{Var 
	}X=c_X''(0)=\lambda s$.
\end{itemize}
}

\begin{probres}
{Suposau que $X$ est\`a distribu\"{\i}da
uniformement en l'interval $(1,2)$ i que es forma un quadrat amb costats de
llargada $X$. Trobau la funci\'o de densitat de la variable $Y=X^2$ que \'es
l'\`area del quadrat i trobau $\pp{Y>2}$.}
\end{probres}

\res{Si $X$ \'es $U(1,2)$, les funcions de distribuci\'o i de densitat seran:
\[
F_X(t)=
\left\{\begin{array}{ll}
0, & \text{si $t\leq 1$}, \\
t-1, & \text{si $1<t\leq 2$}, \\
1, & \text{si 
$t\geq 2$},
\end{array}\right.
\quad
f_X(t)=
\left\{\begin{array}{ll}
1, & \text{si $t\in (1,2)$},\\
0, & \text{en cas contrari.}
\end{array}\right.
\]
Observem que el rang de $X$ \'es l'interval $(1,2)$: $X(\Omega)=(1,2)$. Per 
tant, el rang de $Y=X^2$ ser\`a l'interval $(1,4)$: $Y(\Omega)=(1,4)$.
D'aqu\'{\i}  dedu\"{\i}m que si $t \leq 1$, $F_Y (t)=0$ i si $t \geq 4$, 
$F_Y (t)=1$. Trobem
$F_Y(t)$ per a $t\in (1,4)$:
\[
F_Y(t)=\pp{Y\leq t}=\pp{X^2\leq t}=\pp{1<X\leq\sqrt{t}}=F_X 
(\sqrt{t}) - F_X(1)=\sqrt{t}-1.
\]
Les funcions de distribuci\'o i de densitat de $Y$ seran:
\[
F_Y(t)=
\left\{\begin{array}{ll}
0, & \text{si $t\leq 1$}, \\
\sqrt{t}-1, & \text{si $1 \leq t \leq 4$}, \\
1, & \text{si 
$t\geq 4$,}
\end{array}\right.
\quad
f_Y(t)=
\left\{\begin{array}{ll}
\frac{1}{2\sqrt{t}}, & \text{si $t\in (1,4)$},\\ & \\
0, & \text{en cas contrari.}
\end{array}\right.
\]
Per calcular $\pp{Y>2}$, ho podem posar en termes de la funci\'o de 
distribuci\'o:
\[
\pp{Y>2}=1-F_Y (2)=1-\left(\sqrt{2}-1\right)=2-\sqrt{2}.
\]}

\begin{probres}
{$X$ \'es una
variable geom\`etrica amb par\`ametre $p$ i $Y$ \'es exponencial amb
par\`ametre $\lambda$. Trobau el valor de $\lambda$ tal que
$\pp{X>1}=\pp{Y>1}$.} 
\end{probres}

\res{La funci\'o de probabilitat de la variable aleat\`oria $X$ \'es:
\[
f_X (n)=p\cdot {(1-p)}^{n-1},\ n\in\ZZ,\ n\geq 1.
\]
Les funcions de distribuci\'o i de densitat de la variable aleat\`oria $Y$ s\'on:
\[
F_Y(t)=1-\e^{-\lambda t},\ t\geq 0,\quad f_Y (t)=\lambda \e^{-\lambda t},\ 
t\geq 0.
\]
Per tant:
\begin{eqnarray*}
	\pp{ X >1 } & = & 1-\pp{X=1}=1-p,  \\
	\pp{Y >1 } & = & 1- F_Y(1) =1-\left( 1-\e^{-\lambda}\right)=\e^{-\lambda}.
\end{eqnarray*}
L'equaci\'o que hem de resoldre per trobar $\lambda$ \'es:
\[
\e^{-\lambda}=1-p,
\]
d'on resulta que $\lambda =\ln\left(\frac{1}{1-p}\right)$.
}

\begin{probres}
{Es d\'ona un proc\'es de Poisson amb
par\`ametre $\lambda$. Es comen\c{c}a a observar el proc\'es a partir del temps
zero. Sigui $S$ el temps passat fins que t\'e lloc la segona incid\`encia.
Trobau
la funci\'o de densitat per a $S$.} 
\end{probres}

\res{Considerem les variables aleat\`ories:
\begin{itemize}
	\item[] $S=$ ``temps fins que t\'e lloc la segona incid\`encia en el
	proc\'es de Poisson'',
	
	\item[] $X_s =$ ``nombre de successos en l'interval $(0,s]$'' $\to
	Poiss(\lambda 	s)$. 
\end{itemize}
Observem que els dos successos seg\"uents s\'on el mateix:
\[
\{S>s\} = \{ X_s \leq 1 \},
\]
ja que si per a $t=s$ no ha tengut lloc la segona incid\`encia en el
proc\'es de Poisson, 
el valor de $X_s$ (nombre d'incid\`encies en el proc\'es de Poisson) \'es com
a  m\`axim $1$.

Per tant,
\[
\pp{ S>s} =\pp{X_s =0}+\pp{X_s=1}=\e^{-\lambda s} (1+\lambda s), \ 
\mbox{per a } s>0.
\]
De la darrera expressi\'o podem trobar les funcions de distribuci\'o i densitat
de la 
variable aleat\`oria $S$:
\begin{eqnarray*}
	F_S (s) & = & 1- \pp{S>s}=1- \e^{-\lambda s} (1+\lambda s), \mbox{ per a } 
	s>0,  \\
	f_S (s) & = & F_S'(s)= \lambda \e^{-\lambda s} (1+\lambda s) -\lambda
\e^{-\lambda 
	s}= \lambda^2 s \e^{-\lambda s}, \mbox{ per a } s>0.
\end{eqnarray*}
}

\newpage

\begin{probres}
{Generalitzaci\'o de l'exercici
anterior. Suposau que se d\'ona un proc\'es de Poisson amb par\`ametre
$\lambda$ i sigui $T_r$ el temps passat fins que t\'e lloc la $r-$\`essima
incid\`encia ($r\geq 1$). Trobau la funci\'o de densitat de la variable
aleat\`oria
$T_r$ (en aquest cas direm que la variable $T_r$ t\'e la {\bf distribuci\'o 
gamma.})\index{variable aleatoria@variable aleat\`oria!gamma}
\index{distribucio@distribuci\'o!gamma}\newline {\footnotesize
Ind.: El succ\'es $\scriptstyle T_r>t$ \'es equivalent al succ\'es: ``observar
$\scriptstyle r-1$ incid\`encies en l'interval de temps $\scriptstyle
(0,t)$''.}} 
\end{probres}

\res{Considerem les variables aleat\`ories:
\begin{itemize}
	\item[] $T_r =$ ``temps fins que t\'e lloc la $r-$\`essima 
	incid\`encia en el proc\'es de Poisson'',
	
	\item[] $X_s = Poiss(\lambda s)$.
\end{itemize}
Observem que els dos successos seg\"uents s\'on el mateix:
\[
\{T_r >s\} = \{ X_s \leq r-1 \},
\]
ja que si per a $t=s$ no ha tengut lloc la $r-$\`essima incid\`encia en 
el proc\'es de Poisson, 
el valor de $X_s$ (nombre d'incid\`encies en el proc\'es de Poisson) \'es com
a  m\`axim $r-1$.

Per tant,
\[
\pp{ T_r >s} =\sum\limits_{i=0}^{r-1} \e^{-\lambda s}\cdot\frac{{(\lambda 
s)}^i}{i!}, \ 
\mbox{per a } s>0.
\]
De la darrera expressi\'o podem trobar les funci\'o de distribuci\'o i densitat
de la 
variable aleat\`oria $T_r$:
\begin{eqnarray*}
	F_{T_r} (s) & = & 1- \pp{S>s}=1- \sum\limits_{i=0}^{r-1} \e^{-\lambda
s}\cdot\frac{{(\lambda 
s)}^i}{i!}, \mbox{ per a } 
	s>0,  \\
	f_{T_r} (s) & = & F_{T_r}'(s)= \lambda \e^{-\lambda s} 
	\sum\limits_{i=0}^{r-1} \frac{{(\lambda 
s)}^i}{i!} - \lambda  \e^{-\lambda s}\sum\limits_{i=1}^{r-1} \frac{{(\lambda 
s)}^{i-1}}{(i-1)!}\\ & = & \lambda  \e^{-\lambda s}\left(\sum\limits_{i=0}^{r-1}
\frac{{(\lambda 
s)}^i}{i!}-\sum\limits_{i=0}^{r-2} \frac{{(\lambda 
s)}^i}{i!}\right)=\frac{\lambda^r s^{r-1}}{(r-1)!}  \e^{-\lambda s}, \mbox{ per
a } s>0.
\end{eqnarray*}}

\begin{probres}
{Si $X$ \'es una variable aleat\`oria normal amb mitjana
$\mu$ i vari\`ancia $\sigma^2$, trobau la funci\'o de distribuci\'o de la
variable aleat\`oria $U=\vert X\vert$ en funci\'o de la funci\'o de
distribuci\'o de la normal est\`andard.\hfill\break\indent Com a aplicaci\'o,
trobau $\pp{\vert X\vert \leq 3}$ si $X$ \'es $N(\mu =1,\sigma^2 =4)$.}
\end{probres}

\res{Siguin les variables aleat\`ories \mbox{$X \> N(\mu,\sigma^2)$}, 
\mbox{$Z \> N(0,1)$} i \mbox{$U= |X|$}.

La relaci\'o entre la funci\'o de distribuci\'o de $U$ i de $Z$ \'es la
seg\"uent:

\begin{eqnarray*}
	F_U (t) & = & \pp{ U = |X|\leq t}=\pp{-t\leq X\leq t}  \\
	 & = & F_X (t)-F_X (-t)= F_Z \left(\frac{t-\mu}{\sigma}\right)
	 -F_Z \left(\frac{-t-\mu}{\sigma}\right).
\end{eqnarray*}

En el cas que $\mu =1$, $\sigma^2 =4$ i $t=3$, tendrem que:

\begin{eqnarray*}
\pp{\vert X\vert \leq 3} & = & F_U (3) =F_Z \left(\frac{3-1}{2}\right)
	 -F_Z \left(\frac{-3-1}{2}\right) \\ 
	 & = &  F_Z (1)- F_Z(-2)\approx 
	 0.8413-(1-0.9772)=0.8185.
\end{eqnarray*}
}

\begin{probres}
{Se sap que el percentil 90\% d'una distribuci\'o normal \'es igual a 50 i
que el seu percentil 15\% \'es 25.
\begin{itemize}
\item[a)] {Trobau $\mu$ i $\sigma$.}
\item[b)] {Quin \'es el percentil 40\% ?}
\end{itemize}
}
\end{probres}

\res{\begin{itemize}
\item[a)] Per trobar $\mu$ i $\sigma$, hem de resoldre el sistema
d'equacions seg\"uent:
\[
\left.
\begin{array}{rl}
\pp{X\leq 50 } & = \pp{ Z\leq \frac{50-\mu}{\sigma}}=0.9, \\
& \\
\pp{X\leq 25 } & = \pp{ Z\leq \frac{25-\mu}{\sigma}}=0.15. \\
\end{array}
\right\}
\]
Mirant a les taules, resulta que:
\[
\left.
\begin{array}{rl}
\frac{50-\mu}{\sigma} & = 1.28, \\
& \\
\frac{25 -\mu}{\sigma} & = -1.04.
\end{array}
\right\}
\]
Les solucions del sistema anterior s\'on: $\mu \approx 36.2068$ i $\sigma 
\approx 10.7758$.
\item[b)] Hem de trobar el valor $t_{0.4}$ tal que $\pp{ X\leq 
t_{0.4}}=0.4$. Si posam la condici\'o anterior en termes de la variable 
aleat\`oria $Z=N(0,1)$, podem dir que hem de trobar el valor $t_{0.4}$ tal 
que 
\[
\pp{Z\leq\frac{t_{0.4}-36.2068}{10.7758}}=0.4
\]
Mirant a les taules, resulta que $\frac{t_{0.4}-36.2068}{10.7758}\approx 
-0.26$, d'on resulta que \break $t_{0.4}\approx 33.4051.$
\end{itemize}}

\begin{probres}
{Sigui $T$ la variable aleat\`oria que ens d\'ona el temps d'arribada d'una
persona al seu lloc de treball a partir de les 8.00 h. del mat\'{\i}. Suposem que
$T$ \'es $\hbox{Exp }(\lambda =1\hbox{ h.})$. Escollim a l'atzar 10 dies.
Trobau: \begin{itemize}
\item[a)] {La probabilitat que arribi 5 dies m\'es tard de les 8.30 h.}
\item[b)] {La probabilitat que arribi com a m\`axim un dia m\'es prest de les
9.00~h.}
\item[c)] {El nombre mitj\`a de dies que arriba m\'es tard de les 8.30 h.}
\end{itemize}
{\footnotesize Final. Juny 94.}}
\end{probres}

\res{
\begin{itemize}
	\item[a)] Considerem la variable aleat\`oria:
	\[
	X_a =\mbox{`` nombre de dies que arriba m\'es tard de les 8.30 h ''.}
	\]
	Fixau-vos que $X_a$ \'es una variable aleat\`oria binomial de par\`ametres
	 	$n=10$ dies i $p_a =$ probabilitat que un dia a l'atzar arribi m\'es 
	tard de les 8.30 h. Aquest darrer valor ens el donar\`a la variable 
	aleat\`oria $T$:
	\[
    p_a = \pp{T\geq\frac{1}{2}}=\e^{-0.5}\approx 0.6065.
	\]
    Per tant, la probabilitat que arribi 5 dies m\'es tard de les 8.30 h
    val:     \[
    \pp{ X_a =5}= {10 \choose 5} p_a^5 {(1-p_a)}^5\approx 0.1950.
    \]
	\item[b)] Considerem la variable aleat\`oria:
	\[
	X_b =\mbox{`` nombre de dies que arriba m\'es prest de les 9h.''.}
	\]
	Fixau-vos que $X_b$ \'es una variable aleat\`oria binomial de par\`ametres
	 	$n=10$ dies i $p_b =$ probabilitat que un dia a l'atzar arribi m\'es 
	prest de les 9h. Aquest darrer valor ens el donar\`a la variable 
	aleat\`oria $T$:
	\[
    p_b = \pp{T\leq 1}=1-\e^{-1}\approx 0.6321.
	\]
	Per tant, la probabilitat que arribi com a m\`axim un dia m\'es prest de
les
    9.00 h. val:
    \[
    \pp{ X_b\leq 1}= {10 \choose 0} {(1 -p_b)}^{10}+{10\choose 1} p_b 
    {(1-p_b)}^9 \approx 0.000825.
    \]
	\item[c)] El nombre mitj\`a de dies que arriba m\'es tard de les 8.30 h
s'obt\'e amb: 
\[
\EE X_a = 10 p_a\approx 6.06\mbox{ dies.}
\]
\end{itemize}
}

\begin{probres}
{Un canal de transmissi\'o accepta un voltatge arbitrari $V$ com a {\it input}, i
treu com a {\it output} un voltatge $Y=V+N$, on $N$ \'es una variable aleat\`oria
$N(0, 1)$. Suposem que el canal s'utilitza per transmetre informaci\'o
bin\`aria de la seg\"uent manera:

\begin{itemize}
\item[-] per transmetre $0 \to V=-1$,
\item[-] per transmetre $1 \to V=1$.
\end{itemize}

El receptor decideix que s'ha enviat un 0 si el voltatge rebut \'es negatiu i
un 1 si \'es positiu. Trobau la probabilitat de cometre un error si s'ha enviat
un 0; el mateix si s'ha enviat un 1.}
\end{probres}

\res{
Tenim:
\begin{eqnarray*}
	p\{ \mbox{error / 0 transm\`es} \} & = & p\{ Y>0/V=-1 \} =
	 p\{ V+N>0/V=-1\} \\
	& = & {p\{ -1+N>0 \} \over P\{ V=-1 \}} = {p\{ N>1\} \over 1/2} = 
	{0.159\over 1/2} = 0.318,
\end{eqnarray*}
i, an\`alogament,
\begin{eqnarray*}
	p\{ \mbox{error / 1 transm\`es} \} & = & p\{ Y<0/V=1 \} = 
	p\{ V+N<0/V=1\} \\
	& = & {p\{ 1+N<0 \} \over P\{ V=1 \}} = {p\{ N<-1\} \over 1/2} = 
	{0.159\over 1/2} = 0.318.
\end{eqnarray*}
}

\section{Problemes proposats}

\begin{prob}
{Es tiren una sola vegada 5 daus no trucats. Sigui $X$ el nombre 
d'uns que surten. Calculau l'esperan\c{c}a de $X$, la vari\`ancia de $X$,
$\pp{1\leq X<4}$ i $\pp{X\geq 2}$.} 
\end{prob}

\begin{prob}
{Se sap que el 10\% dels tassons fabricats
per una determinada m\`aquina t\'e algun defecte. Si se seleccionen a
l'atzar 10 dels tassons fabricats per aquesta m\`aquina, quina \'es la
probabilitat que cap sigui defectu\'os? Quants tassons defectuosos 
s'esperaria trobar?} 
\end{prob}

\begin{prob}
{Se sap que $Y$ \'es una variable binomial amb mitjana
$\mu_Y=6$ i vari\`ancia $\sigma_Y^2=4$. Trobau la distribuci\'o de $Y$, o
sigui, trobau $n$ i $p$.} 
\end{prob}

\begin{prob}
{Un
fabricant de peces les envia en paquets de 20 als seus clients.
Suposau que la probabilitat que una pe\c{c}a sigui defectuosa \'es
$0.05$. 
\begin{itemize}
\item[a)] {Quin \'es el nombre esperat de peces defectuoses per
paquet?} 
\item[b)] {Quina \'es la probabilitat que un determinat paquet no
tengui cap pe\c{c}a defectuosa?}
\end{itemize}} 
\end{prob}

\begin{prob}
{Suposau que una
urna cont\'e 10 bolles, una de les quals \'es negra. Sigui $Z$ el nombre
d'extraccions amb reposici\'o necess\`aries per treure la bolla negra. Quina
\'es la funci\'o de probabilitat per a $Z$? I la mitjana de $Z$?} 
\end{prob}

\begin{prob}
{Es llan\c{c}a una moneda a
l'aire fins que surt cara. Suposau que les tirades s\'on independents i que
la probabilitat que surti cara cada vegada \'es $p$.  
\begin{itemize}
\item[a)] {Demostrau
que la probabilitat que faci falta un nombre senar de llan\c{c}aments \'es
${p\over 1-q^2}$ on $q=1-p$.} 
\item[b)] {Trobau el valor de $p$ tal que la probabilitat
que faci falta un nombre senar de vegades sigui $0.6$.} 
\item[c)] {Es pot
trobar un valor de $p$ tal que la probabilitat que faci falta un nombre
senar de vegades sigui $0.5$?}
\end{itemize}
} 
\end{prob}

\begin{prob}
{S'ha observat que el tr\`ansit mitj\`a de
cotxes en un determinat punt d'un cam\'{\i} rural \'es de 3 cotxes/hora.
Suposau que els instants en qu\`e passen els cotxes s\'on independents. Sigui
$X$ el nombre de cotxes que passen per aquest punt en un interval de 20 minuts.
Trobau $\pp{X=0}$ i $\pp{X\geq 2}$.} 
\end{prob}

\begin{prob}
{Suposau que un de cada
10.000 nins neix cec. Si un hospital d'una ciutat gran va tenir 5.000
naixements el 1970, aproximau per la distribuci\'o de Poisson la probabilitat
que cap dels nascuts en aquest any fos cec en n\'eixer. Aproximau tamb\'e
la probabilitat que hagi nascut exactament 1 nin cec i la que hagin
nascut almenys 2 nins cecs.} 
\end{prob}

\begin{prob}
{Suposau que les vendes que fa un venedor de cotxes
usats es realitzen segons
un proc\'es de Poisson amb par\`ametre $\lambda =1$ cotxe per setmana. 
\begin{itemize}
\item[a)] {Quina
\'es la probabilitat que hi hagi exactament 3 vendes en un per\'{\i}ode de 2
setmanes? I com a m\'{\i}nim 3 vendes? I com a m\`axim 3 vendes?}
\item[b)] {Quina \'es la probabilitat que passin 3 per\'{\i}odes de 2
setmanes consecutives sense cap venda?}
\end{itemize}
} 
\end{prob}
 
\begin{prob}
{Trobau la funci\'o
generatriu factorial per a una variable aleat\`oria $X$ distribu\"{\i}da
uniformement en $(a,b)$.} 
\end{prob}

\begin{prob}
{Suposem que $X$ est\`a distribu\"{\i}da uniformement en
$(0,2)$ i $Y$ \'es una variable exponencial amb par\`ametre $\lambda$. Trobau
el valor de $\lambda$ tal que $\pp{X<1}=\pp{Y<1}$.} 
\end{prob}

\begin{prob}
{Suposem que el temps $X$ que fa falta perqu\`e un corredor
de fons recorri una milla \'es una variable normal amb par\`ametres $\mu =4$
minuts i 1 segon i $\sigma = 2$ segons. Quina \'es la probabilitat que
aquest atleta recorri la milla en menys de 4 minuts? I en m\'es de 3 minuts i
55 segons?} 
\end{prob}

\begin{prob}
{L'al\c{c}ada que salta un atleta de salt d'al\c{c}ada en cada intent \'es una
variable aleat\`oria normal amb mitjana de 2 metres i desviaci\'o est\`andard de
0.8
metres.
\begin{itemize}
\item[a)] {Quina \'es la m\`axima al\c{c}ada que pot saltar amb probabilitat
0.95?}
\item[b)] {Quina \'es l'al\c{c}ada que salta en nom\'es el 10\% dels intents?}
\end{itemize}
}
\end{prob}

\begin{prob}
{Una centraleta telef\`onica rep per terme mitj\`a 100 telefonades per hora.
\begin{itemize}
\item[a)] Trobau la probabilitat que passin almenys 25 minuts per rebre la
primera telefonada.
\item[b)] Trobau la probabilitat que passin almenys 30 minuts per rebre la
segona telefonada.
\item[c)] Si definim com a $p_k$ la probabilitat que passin almenys 30
minuts per rebre la $k-$\`essima telefonada, trobau la relaci\'o que hi ha entre
$p_k$ i $p_{k-1}$.

{\footnotesize
(Indicaci\'o: feu servir la distribuci\'o
gamma.)  Primer Parcial. Curs 92-93.}
\end{itemize}
}
\end{prob}

\begin{prob}
{
Sigui $X$ una variable aleat\`oria normal amb par\`ametres $\mu=1$ i
$\sigma^2=1$. Trobau el valor de~$b$ tal que $\pp{{(X-1)}^2\leq b}=0.1$.
\newline{\footnotesize Final. Juny 96.}
}
\end{prob}

\begin{prob}
{
Sigui $Z$ una variable aleat\`oria $N(0,1)$. Trobau
$\pp{\left(Z-\frac{1}{4}\right)^2 >\frac{1}{16}}$.
\newline{\footnotesize Final. Setembre 96.}
}
\end{prob}


\begin{prob}
{Dues persones juguen a cara o creu i han decidit continuar la partida fins
que s'hagin obtengut com a m\'{\i}nim 3 cares i 3 creus. Trobau la probabilitat
que el joc no s'acabi en 10 tirades.}
\end{prob}

\begin{prob}
{Arriben missatges a un ordinador a un ritme mitj\`a de 15 missatges per segon.
Sabem que el nombre de missatges que arriben en un segon \'es una variable
aleat\`oria de Poisson.
\begin{itemize}
\item[a)] Trobau la probabilitat que no arribi cap missatge en un segon.
\item[b)] Trobau la probabilitat que arribin m\'es de 10 missatges en un
segon.
\end{itemize}}
\end{prob}

\begin{prob}
{De dos xips, es planteja utilitzar-ne un en un cert sistema. El temps de
vida del xip 1 s'ha modelat segons una $N(20.000, 4.000^2)$ (la probabilitat d'un 
temps de vida negatiu \'es despreciable) i el del xip 2, segons una 
$N(22.000, 1.000^2)$. Quin xip s'hauria de triar si el temps de vida objectiu del 
sistema \'es de 20.000 hores? I si \'es de 24.000 hores?}
\end{prob}

\begin{prob}
{Sigui $X$ el nombre d'encerts en $n$ proves de Bernoulli independents amb
probabilitat $p$ d'encert.
\begin{itemize}
\item[a)] Si $k$ \'es el valor m\'es probable de $X$, provau que
$$
(n+1)p-1 \leq k \leq (n+1)p.
$$
\item[b)] Si llan\c{c}am 10 vegades un dau sense biaix, quin \'es el nombre 
m\'es probable de vegades que obtendrem un `2'?
\end{itemize}}
\end{prob}

