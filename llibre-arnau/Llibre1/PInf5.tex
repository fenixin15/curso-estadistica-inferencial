\chapter{Lleis dels grans nombres i teorema del l\'{\i}mit central}

\section{Resum te\`oric}

Considerem una successi\'o $\{ X_n \}_{n \geq 1}$ de variables aleat\`ories 
independents i id\`en\-ti\-ca\-ment distribu\"{\i}des. 
Les lleis dels grans nombres estudien el 
comportament asimpt\`otic de la successi\'o de mitjanes mostrals 
$\displaystyle \left\{ {S_n \over n} : n \geq 1 \right\},$ on $S_n = X_1 + \cdots + X_n.$ 
Com a conseq\"u\`encia de la llei forta s'obt\'e la coneguda propietat que 
diu que la freq\"u\`encia relativa d'un succ\'es convergeix cap a la seva
probabilitat quan el nombre de repeticions de l'experiment tendeix cap a infinit.

El teorema del l\'{\i}mit central afirma que, sota certes condicions, la distribuci\'o
de la suma normalitzada de $n$ variables aleat\`ories s'acosta quan $n$ creix cap
a la d'una variable aleat\`oria gaussiana. A difer\`encia de les lleis dels grans nombres, 
el teorema del l\'{\i}mit central no estudia l\'{\i}mits de successions de variables aleat\`ories, 
sin\'o l\'{\i}mits de les funcions de distribuci\'o; m\'es concretament, estudia la distribuci\'o l\'{\i}mit 
de la successi\'o de distribucions de les sumes parcials de les variables aleat\`ories donades.

\subsection{Lleis del grans nombres}

Sigui $X$ una variable aleat\`oria de la qual desconeixem la seva esperan\c ca $\EE X =
\mu$, que suposarem que \'es finita. Siguin $X_1, \ldots , X_n \ n$ c\`opies
repetides independentment de $X$; aix\`o vol dir que les $X_i$ s\'on $n$  {\bf variables
aleat\`ories i.i.d.} (independents i id\`enticament distribu\"{\i}des)
\index{variables!aleatories@aleat\| ories!i.i.d.} amb la mateixa
distribuci\'o que $X$. Per tal d'estimar $\EE X$ s'utilitza la {\bf mitjana mostral}\index{mitjana!mostral} de la
successi\'o:
$$M_n = {1 \over n} \sum_{i=1}^n X_i =\frac{S_n}{n}.$$

La {\bf llei feble dels grans nombres}\index{llei!feble dels grans nombres} afirma que

$$\forall \epsilon > 0, \ \lim_{n \to \infty} \pp{|M_n - \mu| < \epsilon} = 1.$$

Aix\`o ens diu que per a un valor de $n$ fixat prou gran, la mitjana mostral
usant $n$ mostres estar\`a pr\`oxima al valor mitj\`a real amb una probabilitat molt
alta. 

La {\bf llei forta dels grans nombres}
\index{llei!forta dels grans nombres} afirma que, donada una successi\'o de
variables aleat\`ories $X_1, X_2, \ldots $ i.i.d. amb mitjana comuna $\EE X = \mu$
finita i vari\`ancia comuna finita,
$$\pp{\lim_{n \to \infty} M_n = \mu} = 1,$$
\'es a dir que, amb probabilitat 1, la successi\'o de mitjanes mostrals,
$$ M_n = {X_1 + \cdots + X_n \over n},$$
tendeix cap al valor mitj\`a real $\mu$ quan augmentam el nombre $n$ de mostres.

En particular, donat un experiment aleatori, sigui $N_A(n)$ 
el nombre de vegades que apareix el succ\'es $A$ en $n$ repeticions de l'experiment. 
Aleshores la freq\"u\`encia relativa de $A$ en les $n$ repeticions \'es:
$$
f_A(n) = {N_A(n) \over n}.
$$
Observem que $N_A(n)$ es pot escriure tamb\'e com la suma
$$N_A(n) = X_1 + \cdots + X_n
$$
on les variables aleat\`ories $X_1, \ldots , X_n$ s\'on i.i.d. 
amb distribuci\'o comuna de Bernoulli amb par\`ametre 
$p = \pp{A}$ (\'es a dir, $\EE X_i =p\ \forall i=1,\ldots,n)$.

Per tant $f_A(n) = M_n$, la mitjana mostral, i aleshores
$$\pp{\lim_{n \to \infty} f_A(n) = \pp{A}} = 1,$$
\'es a dir que la freq\"u\`encia relativa del succ\'es $A$ convergeix cap a la seva probabilitat amb probabilitat 1.

\subsection{Teorema del l\'{\i}mit central}
\index{teorema!del limit central@del l\'{\i}mit central}

Sigui $X_1, X_2, \ldots $ una successi\'o de variables aleat\`ories i.i.d. amb
mitjana $\mu$ i vari\`ancia $\sigma^2$ finites. Considerem $S_n = X_1 + \cdots +
X_n$, i sigui
$$Z_n = {S_n - \EE S_n \over \sqrt{\Var S_n}} = {S_n - n \mu \over \sigma
\sqrt{n}}.$$

Aleshores
$$\lim_{n \to \infty} \pp{Z_n \leq z} = {1 \over \sqrt{2 \pi}} \int_{-\infty}^z
\e^{-x^2/2} \> dx,$$
\'es a dir que la funci\'o de distribuci\'o de $Z_n$ convergeix cap a la funci\'o de
distribuci\'o de la $N(0,1)$. Tamb\'e resulta que la funci\'o de distribuci\'o de $S_n$
convergeix cap a la de la $N(n \mu, n \sigma^2)$.

Com a cas particular tenim el {\bf teorema de De Moivre-Laplace:}
\index{teorema!de De Moivre-Laplace}

Sigui $X$ una variable aleat\`oria $B(n,p)$. Aleshores $X = X_1 + \cdots + X_n$,
on les $X_i$ s\'on $B(1,p)$ i.i.d. Aleshores la funci\'o de distribuci\'o de
$\displaystyle {X - np \over \sqrt{npq}}$ convergeix cap a la de la $N(0,1)$.
Tamb\'e resulta que la funci\'o de distribuci\'o de $X$ convergeix cap a la de la
$N(np,npq)$.

{\bf Notes.}

\begin{enumerate}

\item De fet, hist\`oricament, aquesta va ser la primera versi\'o del teorema del l\'{\i}mit central.

\item Com que $X$ \'es discreta i una normal \'es absolutament cont\'{\i}nua, per aproximar la
funci\'o de probabilitat de $X$ es fa la correcci\'o seg\"uent (d'altra manera
l'aproximaci\'o seria nu{\lgem}a, a causa de la continu\"{\i}tat 
de la distribuci\'o normal):
$$\pp{X = k} \simeq \pp{k - 1/2 \leq N(np,npq) \leq k + 1/2}.$$

Tamb\'e:
$$
\pp{k_1 \leq X \leq k_2} \simeq \pp{k_1 - 1/2 \leq N(np,npq) \leq k_2 + 1/2}.
$$
\end{enumerate}

\section{Problemes resolts}

\begin{probres}
{Si $X$ est\`a distribu\"{\i}da uniformement en l'interval $(0,1)$, comparau
\mbox{$\pp{\vert X-\mu_X\vert <k \sigma_X}$} amb els valors donats per la
desigualtat de Txebixef per a \mbox{$k={5\over 4},{3\over 2},{7\over 4}$ 
i $2$.}}
\end{probres}

\res{Recordem que si la variable $X$ est\`a distribu\"{\i}da uniformement en 
l'interval $(0,1)$, les funcions de densitat i de distribuci\'o s\'on:
$$
f_X (t)=
\left\{\begin{array}{ll}
1, & \text{si $t\in (0,1)$},\\ 
0, & \text{en cas contrari,}
\end{array}\right.
\quad 
F_X(t)=
\left\{\begin{array}{ll}
0, & \text{si $t<0$}, \\
t, & \text{si $0\leq t\leq 1$}, \\
1, & \text{si 
$t>1$.}
\end{array}\right.
$$
L'esperan\c{c}a i la vari\`ancia valdran, doncs:
\begin{eqnarray*}
	\mu_X & = & \int_0^1 x\, dx =\frac{1}{2},  \\
	\sigma_X^2 & = & \EE\left(X^2\right)-{(\EE X)}^2=\int_0^1 x^2\, dx 
	-\frac{1}{4}=\frac{1}{3}-\frac{1}{4}=\frac{1}{12}.
\end{eqnarray*}
Per tant el valor de \mbox{$\pp{\vert X-\mu_X\vert <k \sigma_X}$} ser\`a:
\begin{eqnarray}
	\pp{\vert X-\mu_X\vert <k \sigma_X} & = & \pp{\mu_X - k\sigma_X< X<\mu_X 
	+k\sigma_X}\nonumber \\ & = & 
	\pp{\frac{1}{2}-k\frac{\sqrt{3}}{6}<X<\frac{1}{2}+k
	\frac{\sqrt{3}}{6}} 
	  = 
	 \int_{\max\left\{0,\frac{1}{2}-k\frac{\sqrt{3}}{6}\right\}}^{\min\left\{1,
	 \frac{1}{2}+k\frac{\sqrt{3}}{6}\right\}} 1\, dx \nonumber \\ & = & \min\left\{1,
	 \frac{1}{2}+k\frac{\sqrt{3}}{6}\right\}-\max\left\{0,\frac{1}{2}-k\frac{\sqrt{3}}{6}\right\}.
	 \label{UNIFEXACTE}
\end{eqnarray}
A continuaci\'o exposam els resultats per als diferents $k$ indicant el 
valor exacte fent servir la f\'ormula~(\ref{UNIFEXACTE}).
$$
\renewcommand{\arraystretch}{1.5}
\begin{tabular}{|c|c|c|}
\hline
$k$&Valor exacte&Valor donat per la desigualtat de Txebixef ($1-\frac{1}{k^2}$) \\
\hline
$\frac{5}{4}$&$0.8608-0.1391=0.7216$&$1-\frac{16}{25}=0.36$\\
\hline
$\frac{3}{2}$&$0.9330-0.0669=0.8660$&$1-\frac{4}{9}=0.5555$\\
\hline
$\frac{7}{4}$&$1-0=1$&$1-\frac{16}{49}=0.6734$\\
\hline
$2$&$1-0=1$&$1-\frac{1}{4}=0.75$\\
\hline
\end{tabular}
$$}

\begin{probres}
{\begin{itemize}
\item[a)]{Considerem la funci\'o $g : \RR \to \RR$ donada per $g(p)=p(1-p) \ \forall  \> p \in [0,1|$. Demostrau que $g(p)$ t\'e un
m\`axim en $p={1\over 2}$. Dedu\"{\i}u que $g(p)\leq {1\over 4}$ per a tot $p\in
[0,1]$.}
\item[b)]{Si $X$ \'es una variable aleat\`oria binomial amb par\`ametres $n$ i~$p$,
aleshores \hbox{$\mu_X =n p$}, \hbox{$\sigma_X^2 =n p (1-p)$.} A m\'es, si
$Y={X\over n}$, tenim que \hbox{$\mu_Y =p$} i \hbox{$\sigma_Y^2 ={p(1-p)\over
n}$.} Demostrau que: $$\pp{\vert Y-p\vert <\delta }\geq 1-{p (1-p)\over
n\delta^2}.$$}
\item[c)]Per a la variable $Y$ definida en b), demostrau que:
$$\pp{\vert Y-p\vert <\delta }\geq 1-{1\over 4}\delta^2 .$$
\item[d)]{Suposem que volem trobar un valor de $n$ tal que:
$$\pp{\vert Y-p\vert <\delta }\geq 0.9,$$on $\delta$ \'es qualque
constant positiva. Demostrau que si $n\geq {2.5\over \delta^2}$, aleshores
se satisf\`a el que vol\'{\i}em.}
\end{itemize}}
\etiqueta{PROBLEMADOS}
\end{probres}

\res{
\begin{itemize}
	\item[a)] Per trobar els extrems relatius, hem de derivar i 
igualar a $0$. Per tant,
	$$
	g'(p)=1-2 p=0,
	$$
	d'on $g$ t\'e un \'unic extrem relatiu en $p=\frac{1}{2}$. 
	Per veure que \'es un m\`axim 
	relatiu, observem que $g''(p)=-2<0$.
	
	Per tant, podem dir:
    \begin{eqnarray*}
	\max_{p\in [0,1]} 
	g(p) & = & \max\left\{g(0),g(1),g\left(\frac{1}{2}\right)\right\} \\ 
	& = & \max\left\{g(0)=g(1)=0,
	g\left(\frac{1}{2}\right)=\frac{1}{4}\right\}=	
	g\left(\frac{1}{2}\right)=\frac{1}{4}.
	\end{eqnarray*}
	Aleshores $g$ t\'e un m\`axim absolut per a 
	$p\in [0,1]$ en $p=\frac{1}{2}$.
	
	Podem dir, per tant, que $g(p)\leq \frac{1}{4}$, $\forall p\in [0,1]$.
	
	\item[b)] Considerem la variable $Y=\frac{X}{n}$ on $X$ \'es una variable 
	aleat\`oria binomial amb par\`ametres $n$ i $p$. Hem de provar que:
	$$
	\pp{\vert Y-p\vert <\delta }\geq 1-{p (1-p)\over n\delta^2}.
   $$
   Observem que:
   \begin{eqnarray*}
   \EE Y & = & \mu_Y =\frac{1}{n} \EE X=\frac{1}{n} n p=p, \\
   \Var Y & = & \sigma_Y^2 =\frac{1}{n^2} \sigma_X^2 = 
   \frac{1}{n^2} n p (1-p)=\frac{p (1-p)}{n}.
   \end{eqnarray*}
	Aplicant la desigualtat de Txebixef, tenim que:
	\begin{eqnarray*}
		\pp{|Y- \mu_Y|<\delta} & \geq & 	 
		1-\frac{\Var  Y}{\delta^2},  \\
		\pp{|Y- p|<\delta} & \geq &  
1-\frac{\frac{p(1-p)}{n}}{\delta^2}=1-\frac{p(1-p)}{n\delta^2}\Box
	\end{eqnarray*}
	
	\item[c)] Fent servir l'apartat a) sabem que 
$p(1-p)\leq\frac{1}{4}$. 
	Per tant, podem escriure, fent servir l'apartat b):
	$$
	\pp{\vert Y-p\vert <\delta }\geq 1-{p (1-p)\over n\delta^2}\geq 
	1-\frac{1}{4 n\delta^2}\geq 1-\frac{1}{4\delta^2} 
\mbox{ ja que $n\geq 1$.}\Box
   $$
   
	\item[d)] Hem de veure que si 
	\begin{equation}
	n\geq\frac{2.5}{\delta^2},
	\label{CONDICIOD}
	\end{equation}
	aleshores:
	\begin{equation}
	\pp{\vert Y-p\vert <\delta }\geq 0.9
	\label{APARTATD}
   \end{equation}
   Sabem que:
   $$
	\pp{\vert Y-p\vert <\delta }\geq 1-\frac{1}{4 n\delta^2}.
   $$
   Considerem un valor $n$ tal que $1-\frac{1}{4 n\delta^2}\geq 0.9$. 
   Observem que per a aquest $n$ es complir\`a~(\ref{APARTATD}). Vegem que 
   la condici\'o que hem imposat per a $n$ \'es equivalent a~(\ref{CONDICIOD}):
   $$
   1-\frac{1}{4 n\delta^2}\geq 0.9\Leftrightarrow\frac{1}{4n\delta^2}\leq 
   0.1 \Leftrightarrow 4 n\delta^2\geq 10 \Leftrightarrow n\geq 
   \frac{10}{4\delta^2}=\frac{2.5}{\delta^2}\Box
   $$
\end{itemize}}

\begin{probres}
{En una taula de daus d'un casino es llancen un parell de daus 180
vegades per hora (aproximadament). Quina \'es la probabilitat (aproximada) 
que es llancin entre 700 i 750 sumes de 7 en 24 hores?}
\end{probres}

\res{Considerem la variable aleat\`oria $X:$``Nombre de sumes de $7$ en $24$ hores''. 

Podem considerar que $X$ \'es una variable aleat\`oria de Poisson amb par\`ametre 
$\lambda s$, on $\lambda$ \'es el nombre mitj\`a de vegades que surt $7$ en $1$ 
hora i $s=24$ hores. Trobem $\lambda$:
$$
\lambda =180\times\mbox{ probabilitat de sortir $7$ una vegada} = 
180\cdot\frac{6}{36}=30,
$$
ja que dels $36$ resultats possibles n'hi ha $6$ en qu\`e la suma 
dels dos daus val $7$:
$$
\{(1,6),(6,1),(2,5),(5,2),(3,4),(4,3)\}.
$$
Dedu\"{\i}m aix\'{\i} que $X$ \'es $Poiss(\lambda s=30\cdot 24=720)$.

Ens demanen calcular la probabilitat seg\"uent: $\pp{700\leq X\leq 750}$.

Com que els nombres que surten s\'on molt grans, la probabilitat anterior no es 
pot trobar aplicant directament la f\'ormula de la funci\'o 
de probabilitat d'una variable aleat\`oria de Poisson. 
Trobarem la probabilitat anterior aplicant el 
teorema del l\'{\i}mit central, o sigui, aproximarem $X$ per una variable aleat\`oria normal 
$X_N$
de par\`ametres \mbox{$\mu =\lambda s=720$} i \mbox{$\sigma^2 =\lambda s=720$.}
\begin{eqnarray*}
	\pp{700\leq X\leq 750} & \approx & \pp{699.5\leq X_N\leq 
	750.5} \\ 
	& = & \pp{\frac{699.5-720}{\sqrt{720}}\leq 
	Z=N(0,1)\leq\frac{750.5-720}{\sqrt{720}}} \\
	  & = &   \pp{-0.764\leq Z\leq 1.136}  
	   =  F_Z (1.136)-  F_Z(-0.764) \\ 
	 & = & F_Z(1.136)-1+F_Z(0.764)\approx 0.8708-1+0.7764=0.6472.
\end{eqnarray*}}

\begin{probres}
{Suposem que la probabilitat que t\'e un jugador de b\`asquet d'encertar
un tir lliure \'es una constant $p$ (desconeguda) i que els seus tirs s\'on
independents. Quants de tirs ha d'intentar si es vol que la proporci\'o de
b\`asquets estigui com a m\`axim a dist\`ancia $0.1$ de $p$ amb probabilitat m\'{\i}nima
de $0.8$?} 
\end{probres}

\res{Considerem la variable aleat\`oria $X_i$ que val $1$ si encerta en 
l'$i$-\`essim intent i $0$ en cas contrari. Tenim per tant que $X_i$ \'es una 
variable aleat\`oria de Bernoulli amb par\`ametre $p$.

Considerem la variable aleat\`oria $X=\sum\limits_{i=1}^n X_i$ que ens d\'ona 
els nombre de b\`asquets en $n$ intents. Fixau-vos que $X$ \'es una variable aleat\`oria
binomial amb par\`ametres $n$ i~$p$. 

La proporci\'o de b\`asquets la donar\`a la variable $Y=\frac{X}{n}$. 
Per tant, ens demanen trobar $n$ (nombre d'intents) tal que:
$$\pp{|Y-p|\leq 0.1}\geq 0.8$$
En general, si volem trobar $n$ tal que:
$$\pp{|Y-p|\leq \delta}\geq \gamma,$$
aproximam la variable $Y$ per una normal aplicant el teorema del 
l\'{\i}mit central i la probabilitat anterior es pot aproximar per:
\begin{eqnarray*}
	\pp{|Y-p|\leq \delta} & = & \pp{p-\delta\leq Y\leq p+\delta}=\pp{ 
	n(p-\delta)\leq X\leq n(p+\delta )}  \\
	 & \approx & \pp{n(p-\delta)-0.5\leq X_N = N(np,n p (1-p)\leq 
	 (p+\delta)n+0.5} \\ 
	 & = & \pp{\frac{-n\delta-0.5}{\sqrt{n p q}}\leq 
	 Z=N(0,1)\leq\frac{n\delta+0.5}{\sqrt{n p q}}} \\  
	 & = & F_Z\left(\frac{n\delta +0.5}{\sqrt{n p q}}\right)-F_Z\left(
	 \frac{-n\delta-0.5}{\sqrt{n p q}}\right)= 2 F_Z\left(\frac{n\delta 
	 +0.5}{\sqrt{n p q}}\right) -1,
\end{eqnarray*}
on $q=1-p$.

Per tant, volem trobar $n$ tal que:
$$
2 F_Z\left(\frac{n\delta  +0.5}{\sqrt{n p q}}\right) -1\geq \gamma,
	 $$
	 d'on dedu\"{\i}m que:
	 \begin{equation}
	 \frac{n\delta  +0.5}{\sqrt{n p q}}\geq z_{\frac{\gamma +1}{2}},
	 \label{CONDICION}
	\end{equation}
	 on $z_{\frac{\gamma +1}{2}}$ \'es el percentil $\frac{\gamma +1}{2}$\% de 
	 la variable $N(0,1)$.
	 
	 Fent c\`alculs en la condici\'o~(\ref{CONDICION}), tenim que $n$ ha de 
	 complir:
	 \begin{equation}
	 	n\delta +0.5\geq z_{\frac{\gamma +1}{2}}\sqrt{n p q}
	 	\label{FINALN}
	 \end{equation}
	 Per simplificar c\`alculs, prendrem $n$ tal que 
	  \begin{equation}
	 	n\delta \geq z_{\frac{\gamma +1}{2}}\sqrt{n p q}
	 	\label{FINALN2}
	 \end{equation}
	 Fixau-vos que si $n$ compleix~(\ref{FINALN2}) tamb\'e complir\`a~(\ref{FINALN}).
	 Finalment, simplificant~(\ref{FINALN2}) tenim que $n$ ha de complir:
	 \begin{equation}
	 	n\geq \frac{{z_{\frac{\gamma +1}{2}}}^2 p q}{\delta^2}
	 	\label{FINALN3}
	 \end{equation}
	 Fent servir el problema~\ref{PROBLEMADOS}, podem dir que \mbox{$p q\leq\frac{1}{4},\ 
	 \forall p\in [0,1]$}. Perqu\`e la condici\'o~(\ref{FINALN3}) no depengui de 
	 $p$ (par\`ametre desconegut) prendrem
$n$ que compleixi:
	 \begin{equation}
	 	n\geq \frac{{z_{\frac{\gamma +1}{2}}}^2 }{ 4 \delta^2}
	 	\label{FINALN4}
	 \end{equation}
	 En el nostre cas $\gamma =0.8$ i $\delta =0.1$ Mirant a les taules, 
	 resulta \mbox{$z_{\frac{\gamma +1}{2}}=z_{0.9}=1.28$.} Aix\'{\i} doncs, $n$ 
	 complir\`a:
	 $$
	 n\geq\frac{1.28^2}{4 {(0.1)}^2} \approx 41.
	 $$
	 S'hauran d'intentar com a m\'{\i}nim $41$ tirs per tenir la seguretat que la 
	 proporci\'o de b\`asquets estigui com a m\`axim a dist\`ancia $0.1$ de $p$ amb 
	 probabilitat m\'{\i}nima de $0.8$.}

\begin{probres}
{Sigui $X$ una variable aleat\`oria tal que $\EE X=0$ i $\Var X=1$.
Trobau el tant per cent m\'{\i}nim de valors de $X$ que estan en l'interval
$(-2,2)$?\newline{\footnotesize Final. Juny 1994.}}
\end{probres}

\res{Ens demanen la quantitat seg\"uent:
$$
\pp{-2<X<2}\times 100.
$$
Aplicant la desigualtat de Txebixef, tenim:
$$
\pp{|X-\mu_X|<2} =\pp{|X|< 2\sigma_X}=\pp{-2<X<2}\geq 1-\frac{1}{4}=0.75
$$
Hi ha com a m\'{\i}nim un $75\%$ de valors de la variable $X$ en l'interval 
$(-2,2)$.}

\begin{probres}
{Suposem que el 10\% dels votants d'una certa comunitat est\`a a favor 
d'una certa legislaci\'o. Es fa un enquesta entre la poblaci\'o i 
s'obt\'e una freq\"u\`encia relativa $f_A(n)$ com a estimaci\'o de la 
proporci\'o anterior ($n$ \'es el nombre d'enquestats). 
Determinau, aplicant la desigualtat de Txebixef, quants de votants 
s'haurien d'enquestar per tal que la probabilitat que $f_A(n)$ 
difereixi de 0.1 menys de 0.02 sigui d'almenys 0.95. Qu\`e podem dir 
si no coneixem la proporci\'o 0.1? (situaci\'o m\'es real)}
\end{probres}

\res{Considerem les variables aleat\`ories:
\[
I_i =
\left\{
\begin{array}{ll}
1, & \mbox{si la persona $i$ est\`a a favor de la lesgislaci\'o}, \\
0, & \mbox{en cas contrari},
\end{array}
\right.
\]
$\forall i=1,\ldots,n$. Aleshores les $I_i$ s\'on variables aleat\`ories i.i.d.
amb distribuci\'o comuna $B(1,p)$.

La freq\"u\`encia relativa ser\`a:
$$
f_A(n) = {\sum\limits_{i=1}^n I_i \over n}.
$$
Aleshores:
$$
\EE (f_A(n)) = {\sum\limits_{i=1}^n \EE I_i \over n} = 
{n \cdot p \over n} = p = 0.1,
$$
i
$$
\Var f_A(n) = {\sum\limits_{i=1}^n \Var I_i \over n^2} =
 {npq \over n^2} = {pq \over n} = {0.09 \over n}.
$$
Aleshores
$$
\pp{ |f_A(n) - 0.1 | < 0.02 } = 1 -
\pp{ |f_A(n) - 0.1 | \geq 0.02 } \geq 1 - {{0.09 \over n} \over 0.02^2} 
= 0.95,
$$
i d'aqu\'{\i} obtenim $n$:
$$
{0.09 \over n \cdot 0.02^2} = 0.05 \Longrightarrow n = {0.09 \over 0.05 \times 0.02^2} = 4 500.
$$

Si no coneixem $p = \EE (f_A(n))$, tampoc coneixerem $\Var f_A(n) = {pq \over n}$, per\`o de totes formes sabem que $p \cdot q \leq 1/4$, i aleshores
$$
\pp{ |f_A(n) - 0.1 | < 0.02 } \geq 1 - {pq \over n \times 0.02^2} \geq 1 - {1 \over 4 n \times 0.02^2} = 0.95,
$$
d'on obtenim
$$
{1 \over 4 n \times 0.02^2} = 0.05 \Longrightarrow n = {1 \over 4 \times 0.05 \times 0.02^2} = 12 500.
$$}

\begin{probres}
{Es llan\c{c}a a l'aire un dau regular 100 vegades. 
Aplicau la desigualtat de Txebixef per tal d'obtenir una cota 
de la probabilitat que el nombre total de punts obtenguts 
estigui entre 300 i 400. 
Quina probabilitat s'obt\'e aplicant el teorema del l\'{\i}mit central?}
\end{probres}

\res{Considerem les variables aleat\`ories:
$$
X_i : \mbox{ resultat en el llan\c{c}ament $i$-\`essim}, \ i = 1, \ldots , 100.
$$
Tenim:
$$
\EE X_i = \sum\limits_{k=1}^6 k \cdot {1 \over 6} = {21 \over 6} = {7 \over 2},
$$
i
$$
\Var X_i = \EE X_i^2 - \left( \EE X_i \right)^2 = \sum\limits_{k=1}^6 
k^2 \cdot {1 \over 6} - \left( {21 \over 6} \right)^2 = {91 \over 6} - 
{441 \over 36} = {35 \over 12}.
$$
El nombre total de punts obtenguts en els 100 llan\c{c}aments ser\`a:
$$
S_{100} = \sum\limits_{i=1}^{100} X_i.$$
Aleshores:
$$
\EE S_{100} = {2100 \over 6} = 350, \ \ \Var S_{100} = {3500 \over 12} = 
{875 \over 3}.$$

Per tant
\begin{eqnarray*}
\pp{300 < S_{100} < 400} & = & \pp{ | S_{100} - \EE S_{100} | < 50} = 1 - 
\pp{ | S_{100} - \EE S_{100} | \geq 50} \\ & \geq & 1 - 
{{875 \over 3} \over 50^2} = {53 \over 60} = 0.883.
\end{eqnarray*}

Si aplicam el teorema del l\'{\i}mit central, 
$\displaystyle {S_{100} - \EE S_{100} \over \sqrt{\Var S_{100}}}$ 
es pot aproximar per una $N(0,1)$. Aleshores,
\begin{eqnarray*}
\pp{300 < S_{100} < 400} &\simeq & \pp{{299.5 - 350 \over \sqrt{875/3}} 
< {S_{100} - \EE S_{100} \over \sqrt{\Var S_{100}}} < 
{400.5 - 350 \over \sqrt{875/3}}} \\ & \simeq & \pp{-2.95 < N(0,1) 
< 2.95} = 2 \times \pp{N(0,1) < 2.95} - 1 \\ &= & 2 \times 0.998411 - 1 =
 0.99682.
\end{eqnarray*}
}

\begin{probres} {El nombre d'errors d'impremta per p\`agina d'un llibre segueix
una distribuci\'o de Poisson, amb un nombre mitj\`a d'errors per p\`agina igual a
2. En un llibre de 300 p\`agines, quina \'es la probabilitat que en una o
m\'es p\`agines hi hagi m\'es de 5 errors? Calculau-la directament, aproximant
per una Poisson i aproximant per una normal.} \end{probres}

\res{Posem: $$ X : \mbox{nombre d'errors per p\`agina}. $$ Aleshores $X$ segueix
una distribuci\'o $Poiss(2)$. Per tant $$ \pp{ X > 5} = 1 - 
\pp{ X \leq 5} =1 - 0.9834 = 0.0166. $$ 
Si ara consideram: $$ Y : \mbox{nombre de p\`agines amb
m\'es de 5 errors}, $$ aleshores $Y$ segueix una distribuci\'o binomial amb
par\`ametres $n=300$ i $p=0.0166$. Per tant

\begin{eqnarray*} 
\pp{ Y \geq 1 } & = & 1 - \pp{ Y < 1} = 1 - \pp{ Y = 0 } \\
& = & 1 - {300 \choose 0} \times 0.0166^0 \times 0.9834^{300} = 1 - 0.0066 =
0.9934. \end{eqnarray*}

Tamb\'e es pot calcular aproximant la $B(n,p)$ per una $Poiss(np)$. En el nostre
cas, $$ \lambda = n \cdot p = 300 \times 0.0166 = 4.98. $$ Aleshores
\[
\pp{ Y \geq 1 }  =  1 - \pp{ Y = 0 }  = 1 - {4.98^0
\over 0!} \> \e^{-4.98} = 1 - \e^{-4.98} = 0.9931. 
\]

Si aplicam el teorema de De Moivre-Laplace,

\begin{eqnarray*} 
\pp{ Y \geq 1 } & = & 1 - \pp{ Y = 0 } \simeq 1 - \pp{ -0.5 <
Y \leq 0.5 }\\ & \simeq & 1 - p \left\{ {-0.5 - np \over \sqrt{npq}} < Z \leq
{0.5 - np \over \sqrt{npq}} \right\} \\ & = & 1 - p \left\{ Z < {0.5 - 4.98 \over
	\sqrt{4.897}} \right\} + p \left\{ Z < {-0.5 - 4.98 \over \sqrt{4.897}}
\right\} \\ & = & 1 - \pp{ Z < -2.02 } + \pp{ Z < -2.47 } = 
\pp{ Z < 2.02 } +\pp{ Z < -2.47} \\ & = & 0.97831 + 0.006756 = 0.985066, \end{eqnarray*}

on $Z$ t\'e la distribuci\'o $N(0,1)$.}


\begin{probres} {S'ha de computar la suma de 100 nombres reals. Suposem que els
nombres s'han aproximat per l'enter m\'es pr\`oxim de manera que cada nombre t\'e
un error uniformement distribu\"{\i}t en $(-1/2,1/2)$. Utilitzau el teorema
de l\'{\i}mit central per estimar la probabilitat que l'error total en la suma
dels 100 nombres superi 6.} \end{probres}

\res{Sigui $$ X_i : \mbox{error total del nombre $i$-\`essim}. $$ Aleshores $X_i$
t\'e una distribuci\'o uniforme en l'interval $(-1/2,1/2)$ i, per tant, \mbox{$\EE X_i =
0$,} $\Var X_i = \frac{1}{12}.$ L'error total en la suma dels 100 nombres el
donar\`a $$ S_{100} = X_1 + \cdots + X_{100}. $$ Aleshores $\EE S_{100} = 0
\mbox{ i } \Var S_{100} = \frac{100}{12}.$ Aix\'{\i} resultar\`a
\begin{eqnarray*} 
\pp{ S_{100} > 6 } & = & 1 - \pp{ S_{100} \leq 6}  = 1
- \pp{ Z_{100} \leq {6 \over \sqrt{{100 \over 12}}}} \\ 
& = & 1 - \pp{ Z_{100}\leq 2.08} \simeq  1 - 
\pp{ Z \leq 2.08} = 1 - 0.98124 = 0.01876,
\end{eqnarray*}
on $Z$ t\'e la distribuci\'o $N(0,1)$.}


\begin{probres} {La mitjana de bol\'{\i}grafs que es venen di\`ariament en una
papereria \'es 30 i la desviaci\'o t\'{\i}pica, 5. Aquests valors s\'on 20 i 4
per al nombre de quaderns venuts. Se sap, a m\'es, que el coeficient de
correlaci\'o entre les vendes d'ambd\'os productes \'es 0.7. Quina \'es la
probabilitat que el nombre total dels dos articles venuts durant un trimestre
estigui compr\`es entre 4.300 i 4.600 unitats?} 
\end{probres}

\res{Posem: 
\begin{eqnarray*} 
X & : & \mbox{nombre de bol\'{\i}grafs venuts en un dia}, \\ 
Y & : & \mbox{nombre de quaderns venuts en un dia}, \\ 
D & : & \mbox{nombre total d'aquests dos articles venuts en un dia } = X + Y.
\end{eqnarray*}

Sabem que $\EE X = 30, \EE Y = 20, \Var X = 5^2 = 25, \Var Y = 4^2 = 16, 
\mbox{ i } \rho_{XY} = 0.7$. Aleshores $\EE D = 50$, i
\[
 \rho_{XY}  = {\Cov (X,Y) \over \sigma_X \cdot \sigma_Y}\quad
\Longrightarrow 0.7  =  {\Cov (X,Y) \over 5 \times 4}
\quad \Longrightarrow  \Cov (X,Y) = 14. 
\]

Aleshores $\Var D = \Var X + \Var Y + 2 \> \Cov (X,Y) = 25 + 16 + 28 = 69.$ 
El nombre total dels dos articles venuts cada trimestre el d\'ona 
$$ S_{90} = D_1
+ \cdots + D_{90}, $$ on cada $D_i ( i = 1, \ldots , 90)$ t\'e la mateixa
distribuci\'o que $D$. Aleshores $\EE S_{90} = 90 \times \EE D = 4500 
\mbox{ i } \Var S_{90} = 90 \times \Var D = 6210$, d'on 
$\sigma_{S_{90}} = 78.8$. Finalment,
\begin{eqnarray*} 
\pp{ 4300 < S_{90} \leq 4600} & = & \pp{ S_{90} \leq 4600}
-\pp{ S_{90} \leq 4300} \\ 
& \simeq & \pp{ S_{90} \leq 4600.5} - \pp{ S_{90}\leq 4300.5} \\ 
& = & p \left\{ Z_{90} \leq {4600.5 - 4500 \over 78.8} \right\}
- p \left\{ Z_{90} 	\leq {4300.5-4500 \over 78.8} \right\} \\ 
& = & \pp{ Z_{90} \leq 1.28} - \pp{ Z_{90} \leq -2.53 } \\ 
& \simeq & \pp{ Z \leq 1.28 } - \pp{ Z \leq -2.53} \\ 
& = & 0.8997 - 0.0057 = 0.894, 
\end{eqnarray*}
on $Z$ t\'e la distribuci\'o $N(0,1)$.}

\section{Problemes proposats}

\begin{prob}
{Suposem que $X_1,X_2,X_3,\ldots,$ s\'on variables aleat\`ories
independents distribu\"{\i}des id\`enticament, cada una amb funci\'o de densitat:
$$f_{X}(x)=
\left\{\begin{array}{ll}
{1\over 2}, & \text{per a $x=-1,1$}, \\
0, & \text{per als altres casos.}
\end{array}\right.
$$
Definim la seq\"u\`encia de les mitjanes: $$\overline{X}_n ={1\over n}\sum_{i=1}^n
X_i.$$Demostrau que la funci\'o de densitat per a $\overline{X}_n$ \'es:
$$f_{\overline{X}_n} (x)=
\left\{\begin{array}{ll}
{{n\choose {n\over 2}(x+1)}\over 2^n}, & \text{per a
$x=\pm {1\over n},\pm {3\over n},\ldots,\pm {n-2\over n}, \pm 1$},\\ & \\
0, &
\text{en els altres casos.}
\end{array}\right.
$$(Suposau que $n$ \'es un nombre senar.)}
\end{prob}

\begin{prob}
{El nombre d'accidents en un tros de 10 Km de carretera de 2 carrils
\'es una variable aleat\`oria de Poisson amb mitjana de 2 accidents per setmana.
Quina \'es la probabilitat (aproximada) que hi hagi menys de 100 accidents
en aquest tros durant 1 any?}
\end{prob}

\begin{prob}
{La llargada que es pot estirar un fil de nil\'o \'es una variable aleat\`oria
exponencial amb mitjana de 1524 metres. Quina \'es la probabilitat (aproximada)
que la llargada mitjana de 100 fils estigui entre 1447.8 m i 1691.64 m?}
\end{prob}

\begin{prob}
{Les telefonades que es reben en un commutador
d'una ind\'ustria arriben com a successos de Poisson a ra\'o de 120 per hora. Quina
\'es la probabilitat que arribin entre 110 i 125 telefonades entre les 9.00 i les
10.00 del mat\'{\i} de qualsevol dia?}
\end{prob}

\begin{prob}
{Considerem la variable aleat\`oria $X=$``Nombre de cotxes arribats a un
supermercat en un determinat mes''. Suposem que l'arribada de cotxes al
supermercat \'es un proc\'es de Poisson i que per terme mitj\`a arriben 5.000 cotxes per mes.
Aplicant el teorema del l\'{\i}mit central, calculau la probabilitat $p$ que
arribin entre 4.900 i 5.200 cotxes, ambd\'os inclosos, al supermercat en un mes
escollit a l'atzar.\newline{\footnotesize Final. Setembre de 1994.}}
\end{prob}

\begin{prob}
{
La probabilitat que un jugador de b\`asquet encistelli \'es $p$.

Quants de llan\c{c}aments ha de fer com a m\'{\i}nim (aproximadament) 
per tal 
que la probabilitat que la mitjana de b\`asquets estigui a dist\`ancia 
$0.01$ de $p$ sigui de $0.99$?
\newline{\footnotesize Primer parcial. Febrer 95.}
}
\end{prob}

\begin{prob}
{
Sigui $X_1,\ldots,X_{n}$ amb $n=48$, una mostra aleat\`oria simple d'una 
variable aleat\`oria uniforme en l'interval $(0,a)$.
Aplicant el teorema del l\'{\i}mit central, trobau la probabilitat 
aproximada $\pp{\sum\limits_{i=1}^n X_i >a}$.
\newline{\footnotesize Final. Juny 96.}
}
\end{prob}

\begin{prob} {Un radiofar est\`a alimentat per una bateria amb un temps de vida
\'util $T$ governat per una distribuci\'o exponencial amb una esperan\c{c}a d'un
mes. Trobau el nombre m\'{\i}nim de bateries que s'han de subministrar al
radiofar per tal que sigui operatiu amb probabilitat 0.99 almenys un any.}
\end{prob}

\begin{prob} {Es llan\c{c}a l'aire una moneda sense biaix $n$ vegades. Calculau
$n$ de manera que la freq\"u\`encia relativa del `6' disti menys de 0.01 de la
probabilitat te\`orica 1/6?} \end{prob}

\begin{prob} {Se sap que en una poblaci\'o, la talla dels individus barons adults
\'es una variable aleat\`oria $X$ amb mitjana $\mu_X = 170$ cm i desviaci\'o
t\'{\i}pica $\sigma_X = 7$ cm. S'elegeixen 140 individus a l'atzar en condicions
independents. Calculau la probabilitat que la mitjana mostral $\bar{x}$
difereixi de $\mu_X$ menys de 1 cm.} \end{prob}

