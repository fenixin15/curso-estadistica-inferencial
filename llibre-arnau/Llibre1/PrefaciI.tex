\frontmatter
\chapter{Prefaci}

El primer dia de classe, els professors solem donar, a m\'es del
contingut de l'assignatura, una bibliografia bastant extensa 
en la qual els 
alumnes trobaran tot el que exposam durant el curs, i molt
m\'es. Sovint, aquests llibres recomanats s\'on cars i dif\'{\i}cils
de trobar, i quan s\'on a la biblioteca, normalment els t\'e en 
pr\'estec qualc\'u desconegut, moltes de vegades el professor.

D'altra banda, durant tot el curs inundam els alumnes amb llistes
i m\'es llistes de problemes dels quals, per motius estrictament
de temps, nom\'es  resolem una petita part. \'Es per tot aix\`o 
que els autors v\`arem decidir escriure un llibre que resolgu\'es  
aquests dos problemes. 
D'una banda, tenir un llibre que est\`as 
a l'abast dels alumnes i que cobr\'{\i}s el que s'explica a classe, 
i de l'altra proveir l'alumne d'una gran quantitat   
d'exercicis resolts que li permetin agafar confian\c{c}a amb  
l'assignatura. Aquest ha estat el nostre \`anim a l'hora    
d'escriure aquest llibre.

L'estructura del llibre est\`a dividida en tres parts:
\begin{itemize}
\item[a)] un resum te\`oric, on potser el nom no \'es del tot
adequat, ja que cada cap\'{\i}tol es desenvolupa en tota 
l'extensi\'o que creiem adequada per a la gent a qui es dirigeix,
l'\'unic que es troba a faltar s\'on les demostracions dels 
resultats que s'hi presenten. De totes formes, si hi apareguessin,
el llibre perdria el seu esperit inicial de llibre de problemes;
   
\item[b)] una \`amplia co{\lgem}ecci\'o de problemes resolts,   
seleccionats pel seu inter\`es did\`actic de les llistes de
problemes que any rere any s'han entregat als alumnes;

\item[c)] una bona co{\lgem}ecci\'o de problemes proposats
que, si s'han treballat els problemes resolts, s'haurien de
resoldre sense gaire inconvenient.
\end{itemize}

El material presentat en aquest llibre \'es adequat per a un
primer curs de Probabilitats en qualsevol carrera universit\`aria,
tenint sempre en compte que en l'\`ambit te\`oric no \'es ben b\'e
un llibre de text, sin\'o un llibre de consulta, i que la seva
utilitat en l'\`ambit pr\`actic s\'{\i} que \'es prou extensa.

L'obra es complementa amb el llibre ``Introducci\'o a la infer\`encia 
estad\'{\i}stica''
dels mateixos autors. Per aquest motiu, hem incl\`os
un cap\'{\i}tol dedicat a l'estad\'{\i}stica descriptiva, de manera que
els dos llibres formen
un material prou adequat per a qualsevol assignatura 
d'estad\'{\i}stica.
\cleardoublepage
\renewcommand{\contentsname}{Sumari}
\tableofcontents
\listoffigures
\mainmatter
