\chapter{Contrasts d'hip\`otesi}
\index{contrast!d'hipotesi@d'hip\`otesi}

\section{Resum te\`oric}

En molts de casos s'ha de decidir {\it a priori} entre dues opcions possibles 
sobre la distribuci\'o d'una variable aleat\`oria donada. 
Les dues possibilitats s'anomenen {\bf hip\`otesi nu{\lgem}a} 
\index{hipotesi@hip\`otesi!nul.la@nu{\lgem}a}$H_0$ i 
{\bf hip\`otesi alternativa} $H_1$.
\index{hipotesi@hip\`otesi!alternativa} Per exemple, si
$X$ \'es una variable aleat\`oria $N(\mu,4)$ amb $\mu$ desconeguda, 
$H_0: \mu = 2, \ H_1: \mu \not = 2$.

Un test o contrast d'hip\`otesi\index{contrast!d'hipotesi@d'hip\`otesi} 
\'es una regla que permet decidir quina de les dues
hip\`otesis \'es acceptada, partint d'una s\`erie d'experiments.

El m\`etode consisteix a determinar un succ\'es $E$ que pot oc\'orrer o no 
despr\'es de cada experiment i tal que, si $H_0$ \'es certa, 
la probabilitat que ocorri $E^c$ \'es quasi~$1$ i, si $H_1$ \'es certa, 
la probabilitat de $E$ \'es pr\`oxima a~$1$.
Aleshores es realitza l'experiment. Si ocorre $E^c$ es conclou que $H_0$ \'es
certa, i si ocorre $E$ es decideix que $H_1$ \'es certa.

Aix\'{\i} hem dividit l'espai mostral en dues parts, una part que fa 
acceptar $H_0$ i l'altra, anomenada {\bf regi\'o cr\'{\i}tica}, que fa rebutjar $H_0$.
\index{regio@regi\'o!critica@cr\'{\i}tica}

En tot contrast d'hip\`otesi, podem distingir dos possibles errors: 
l'{\bf error de tipus~I}\index{error!de tipus I} que es comet en rebutjar $H_0$ si \'es certa, i 
l'{\bf error de tipus~II}, com\`es en acceptar $H_0$ si \'es falsa.
\index{error!de tipus II}

Quan provam hip\`otesis respecte d'un par\`ametre~$\gamma$,
\index{parametre@par\`ametre} la probabilitat
d'acceptar $H_0$ \'es una funci\'o de $\gamma$, denotada per $C(\gamma)$, i
s'anomena la {\bf corba OC}\index{corba OC} (Operating Characteristic) de la prova. A $Q(\gamma)
= 1 - C(\gamma)$ se l'anomena la {\bf funci\'o de pot\`encia del test}.
\index{funcio@funci\'o!de potencia del test@de pot\`encia del test}

\subsection{Hip\`otesis simples}\index{hipotesi@hip\`otesi!simple}

Una hip\`otesi simple \'es la que especifica completament la funci\'o 
de probabilitat o densitat d'una variable aleat\`oria. 
Per exemple, si $f_X$ nom\'es dep\`en d'un par\`ametre $\gamma$, aleshores
\index{parametre@par\`ametre}
\begin{description}
\item $H_0: \gamma = \gamma_0$.
\item $H_1: \gamma = \gamma_1$.
\end{description}
\'es un contrast d'hip\`otesi simple davant d'alternativa simple.
\index{contrast!d'hipotesi@d'hip\`otesi!simple}

En aquest cas, les probabilitats d'error de tipus I i II estan ben definides i
es denoten per $\alpha$ i $\beta$, respectivament; \'es a dir,
\index{probabilitat!d'error tipus I}
\index{probabilitat!d'error tipus II}
$$\pp{\mbox{rebutjar } H_0 / H_0 \mbox{ certa}} = \alpha, \ 
\pp{\mbox{ acceptar } H_0 / H_1 \mbox{ certa}} = \beta.$$

Donada una regi\'o cr\'{\i}tica $R.C.$, s'anomena {\bf nivell de significaci\'o} 
\index{nivell!de significacio@de significaci\'o}
de la regi\'o a \break $\pp{X \in R.C. / H_0}$. 
Observem que el nivell de significaci\'o coincideix amb
$\alpha$, la probabilitat de l'error de tipus I.
\index{probabilitat!d'error tipus I}

Normalment no \'es complicat trobar una regi\'o cr\'{\i}tica.
\index{regio@regi\'o!critica@cr\'{\i}tica} De totes formes, sovint en
tenim m\'es d'una i volem determinar quina utilitzar. D'entre totes les regions
cr\'{\i}tiques $R.C.$ amb nivell de significaci\'o~$\alpha$, 
\index{nivell!de significacio@de significaci\'o}
volem determinar una regi\'o cr\'{\i}tica~${R.C.}^*$
\index{regio@regi\'o!critica@cr\'{\i}tica} amb error de tipus~II 
\index{error!de tipus II}
m\'{\i}nim. Si existeix ${R.C.}^*$, s'anomena {\bf regi\'o cr\'{\i}tica \`optima}.
\index{regio@regi\'o!critica@cr\'{\i}tica!optima@\`optima}

Existeix un resultat que permet obtenir regions cr\'{\i}tiques \`optimes.

{\bf Lema de Neyman-Pearson.}\index{lema de Neyman-Pearson}

Suposem que $X$ \'es una variable aleat\`oria amb funci\'o de distribuci\'o 
dependent d'un par\`ametre~$\gamma$.\index{parametre@par\`ametre} 
Sigui $X_1, \ldots , X_n$ una mostra aleat\`oria simple
de $X$. Sigui $L(\gamma)$ la funci\'o de versemblan\c ca de la mostra. Volem
\index{funcio@funci\'o!de versemblanca@de versemblan\c{c}a!de la mostra}
contrastar

\begin{description}
\item $H_0: \gamma = \gamma_0$.

\item $H_1: \gamma = \gamma_1$.
\end{description}

Aleshores
$$ R.C. = \left\{ (x = (x_1, \ldots , x_n) : {L(\gamma_1) \over
L(\gamma_0)} \geq k \right\}$$
\'es una regi\'o cr\'{\i}tica \`optima 
\index{regio@regi\'o!critica@cr\'{\i}tica!optima@\`optima}
amb nivell de significaci\'o~$\alpha$, on $k$
\index{nivell!de significacio@de significaci\'o}
s'elegeix de manera que $\pp{X \in R.C. /H_0} = \alpha$.

\begin{proposition}
Sigui $X_1, \ldots , X_n$ una mostra aleat\`oria simple d'una
variable alea\-t\`oria $N(\mu,\sigma^2)$ amb $\sigma^2$ coneguda. 
Aleshores la regi\'o cr\'{\i}tica \`optima amb nivell de significaci\'o $\alpha$ per al contrast $H_0: \mu =
\index{regio@regi\'o!critica@cr\'{\i}tica!optima@\`optima}
\index{nivell!de significacio@de significaci\'o}
\mu_0$ davant de $H_1: \mu = \mu_1$ s'obt\'e amb 
\begin{description}
\item $\bar{x} \geq \mu_0 + {\sigma \over \sqrt{n}} z_{1-\alpha} \ \ (\mu_1 >
\mu_0)$,

\item $\bar{x} \leq \mu_0 - {\sigma \over \sqrt{n}} z_{1-\alpha} \ \ (\mu_1 <
\mu_0)$,
\end{description}
on $z_{1-\alpha}$ \'es el $100(1-\alpha)$-\`essim percentil d'una $N(0,1)$.
\end{proposition}

\begin{proposition}
Sigui $X_1, \ldots , X_n$ una mostra aleat\`oria simple d'una
variable alea\-t\`oria $N(\mu,\sigma^2)$. Aleshores la regi\'o
cr\'{\i}tica \`optima amb nivell de significaci\'o $\alpha$ per al contrast $H_0: \sigma
= \sigma_0$ davant de $H_1: \sigma = \sigma_1$ s'obt\'e amb
\index{regio@regi\'o!critica@cr\'{\i}tica!optima@\`optima}
\index{nivell!de significacio@de significaci\'o}
\begin{description}
\item $\sum (x_i - \overline{x})^2 > \sigma_0^2 \chi_{1-\alpha}^2 \ \ (\sigma_1 >
\sigma_0)$,

\item $\sum (x_i - \overline{x})^2 < \sigma_0^2 \chi_\alpha^2 \ \ (\sigma_1 < \sigma_0)$,
\end{description}
on $\chi_{1-\alpha}^2$ i $\chi_\alpha^2$ s\'on, respectivament, el $100(1-\alpha)$
i el $100\alpha$-\`essim percentils d'una $\chi^2$ amb $n-1$ graus de llibertat.
\end{proposition}

Si la hip\`otesi alternativa \'es unilateral,
\index{hipotesi@hip\`otesi!alternativa!unilateral} 
per exemple $H_1: \mu > \mu_1$, aleshores la regi\'o cr\'{\i}tica de 
\index{regio@regi\'o!critica@cr\'{\i}tica!de Neyman-Pearson}
Neyman-Pearson \'es la que d\'ona la funci\'o de
pot\`encia m\'es alta possible.\index{funcio@funci\'o!de potencia@de pot\`encia}

La regi\'o cr\'{\i}tica de Neyman-Pearson d\'ona el valor m\'es petit 
possible per a $\beta$ d'entre totes les proves amb el mateix valor de $\alpha$.
\index{regio@regi\'o!critica@cr\'{\i}tica!de Neyman-Pearson}

Una altra caracter\'{\i}stica desitjable dels contrasts d'hip\`otesi \'es 
que en general siguin una funci\'o de l'estad\'{\i}stica suficient
\index{estadistica@estad\'{\i}stica!suficient} per al par\`ametre,
\index{parametre@par\`ametre} si existeix. El
resultat seg\"uent tracta aquest aspecte.

\begin{proposition}
Suposem que $X_1, \ldots , X_n$ \'es una mostra aleat\`oria simple
d'una variable aleat\`oria $X$ i que la funci\'o de distribuci\'o de $X$ dep\`en d'un
sol par\`ametre\index{parametre@par\`ametre} 
desconegut $\gamma$. Suposem que $Y = g(X_1, \ldots , X_n)$ \'es una
estad\'{\i}stica suficient
\index{estadistica@estad\'{\i}stica!suficient} per a~$\gamma$. 
Aleshores la regi\'o cr\'{\i}tica de Neyman-Pearson \'es funci\'o de $Y$.
\index{regio@regi\'o!critica@cr\'{\i}tica!de Neyman-Pearson}
\end{proposition}

\subsection{Hip\`otesis compostes}\index{hipotesi@hip\`otesi!composta}

Quan $H_0$ i/o $H_1$ s\'on compostes, aleshores no es disposa d'un resultat 
com el lema de Neyman-Pearson
\index{lema de Neyman-Pearson} que doni la regi\'o cr\'{\i}tica \`optima. 
\index{regio@regi\'o!critica@cr\'{\i}tica!optima@\`optima}De totes formes, hi ha
un procediment general semblant al donat pel lema que produeix un bon contrast.

Primer de tot observem que si $H_0$ \'es composta, aleshores no tenim un \'unic
valor per a la probabilitat de l'error de tipus~I. 
\index{error!de tipus I}En aquestes situacions
s'acostuma a indicar amb $\alpha$ el valor m\'es gran de la probabilitat de
l'error de tipus I per a tots els possibles valors del par\`ametre especificats
per $H_0$.

Aleshores, per a totes les proves possibles amb aquesta $\alpha$, se selecciona
la que t\'e la corba OC\index{corba OC} m\'es baixa per a tots els valors especificats per $H_1$, si
existeix.

El {\bf criteri de contrast de la ra\'o de versemblan\c ca}
\index{criteri de contrast!de la rao de versemblanca@de la ra\'o de versemblan\c{c}a} 
\'es un m\`etode que es pot utilitzar per obtenir una regi\'o cr\'{\i}tica
\index{regio@regi\'o!critica@cr\'{\i}tica} 
per provar $H_0$ davant de $H_1$, \`adhuc quan una, o les dues hip\`otesis, \'es composta. Vegem quin \'es.
\index{hipotesi@hip\`otesi!composta}

Suposem que $X_1, \ldots , X_n$ \'es una mostra aleat\`oria simple d'una 
variable aleat\`oria $X$ amb distribuci\'o dependent de $\gamma_1, \ldots , \gamma_r$. Per
provar $H_0$ davant de $H_1$, on una o les dues hip\`otesis poden ser 
compostes, sigui $L(\hat{\omega})$ el valor m\`axim de la funci\'o de 
versemblan\c ca de la mostra en qu\`e els par\`ametres
\index{funcio@funci\'o!de versemblanca@de versemblan\c{c}a!de la mostra}
\index{parametre@par\`ametre}
 $\gamma_1, \ldots , \gamma_r$ estan restringits als valors
especificats per $H_0$, i sigui $L(\hat{\Omega})$ el valor m\`axim de la funci\'o de
versemblan\c ca de la mostra en qu\`e els par\`ametres poden prendre qualsevol valor
especificat per la uni\'o de $H_0$ i $H_1$. Aleshores 
la regi\'o cr\'{\i}tica $R.C.$ est\`a
\index{regio@regi\'o!critica@cr\'{\i}tica}
formada pels resultats de la mostra tals que
$$\lambda = {L(\hat{\omega}) \over L(\hat{\Omega})} < A,$$
on $A$ es tria de manera que $P(\rm{error \> de \> tipus \> I}) = \alpha$.
\index{error!de tipus I}

L'ap\`endix A mostra una s\`erie de taules on s'especifiquen tots els contrasts
d'hip\`o\-tesi param\`etrics m\'es usuals. 
\index{contrast!d'hipotesi@d'hip\`otesi!parametric@param\`etric}
Es donen les condicions per a cada contrast,
l'estad\'{\i}s\-tica\index{estadistica@estad\'{\i}stica} 
a utilitzar en cada cas, la regi\'o cr\'{\i}tica 
\index{regio@regi\'o!critica@cr\'{\i}tica}i l'interval de confian\c{c}a 
\index{interval!de confianca@de confian\c{c}a}
corresponent a cada par\`ametre que surt en el contrast.

\newpage

\section{Problemes resolts}

\begin{probres}
{Suposem que $X$ \'es una variable aleat\`oria normal amb mitjana $\mu$ desconeguda i
vari\`ancia $9$. Si es pren una mostra aleat\`oria simple de 16 observacions de $X$ i es vol provar
$H_0:\mu=2$, demostrau que cada una de les seg\"uents regions cr\'{\i}tiques 
\'es de grand\`aria~$0.05$, o sigui, que l'error tipus I per a cada regi\'o cr\'{\i}tica \'es de
$0.05$:
$$\mbox{a) } \overline{X}>3.23,\quad \mbox{b) } \overline{X}<0.77,\quad
\mbox{c) } 1.952<\overline{X}<2.048.$$Quin tipus d'hip\`otesi alternativa $H_1$
seria apropiada per a cada una de les regions cr\'{\i}tiques anteriors?}
\end{probres}

\res{Tenint en compte que les regions cr\'{\i}tiques han d'estar
en conson\`ancia amb la hip\`otesi alternativa $H_1$ 
i que l'estimador millor per al par\`ametre $\mu$ \'es 
$\overline{X}$, les hip\`otesis
alternatives corresponents a les regions cr\'{\i}tiques anteriors s\'on:
\begin{itemize}
\item[a)] $H_1:\mu >2$

Trobem l'error tipus I $\alpha$:
\begin{eqnarray*}
\alpha & = & \pp{\mbox{Rebutjar $H_0$}/\mbox{$H_0$ certa}}=
\pp{R.C./\mu=2} \\ & = & 
\pp{\overline{X}>3.23/\mu=2} =
\pp{Z=N(0,1)=\frac{\overline{X}-\mu}{\frac{\sigma}{\sqrt{n}}}
>\frac{3.23 - 2}{\frac{3}{\sqrt{16}}}} \\ & = & 
\pp{ Z>1.64}=1-F_Z (1.64)=1-0.95 =0.05,
\end{eqnarray*}
tenint en compte que la variable $\overline{X}$ \'es normal 
$N\left(\mu,\frac{\sigma^2}{n}\right)$.
\item[b)] $H_1 :\mu <2$. Error tipus I:
\begin{eqnarray*}
\alpha  & = & \pp{ \overline{X}<0.77/\mu =2} =
\pp{ Z=N(0,1) <\frac{0.77 -2}{\frac{3}{\sqrt{16}}}} \\ & = &
\pp{ Z <-1.64} =\pp{ Z>1.64}=0.05.
\end{eqnarray*}
\item[c)] En aquest cas, no t\'e sentit escollir cap $H_1$ ja que 
la regi\'o cr\'{\i}tica escollida est\`a en contradicci\'o amb
la filosofia de realitzar un contrast d'hip\`otesi. Fixau-vos que
en aquest cas \mbox{$\mu =2\in\ R.C.$} Per tant, dir que $H_0$ \'es certa
seria equivalent a afirmar que \mbox{$\overline{X}\in\ R.C.$}

Comprovaci\'o que l'error tipus I val $0.05$:
\begin{eqnarray*}
\alpha & = & \pp{1.952<\overline{X}<2.048} =
\pp{\frac{1.952-2}{\frac{3}{\sqrt{16}}}<Z=N(0,1)<
\frac{2.048-2}{\frac{3}{\sqrt{16}}}} \\ & = & 
\pp{-0.064 < Z < 0.064}= 2 F_Z (0.064) -1 \approx 0.05.
\end{eqnarray*}
\end{itemize}
}

\begin{probres}
{Suposem que $X$ \'es una variable aleat\`oria uniforme en l'interval que va des de $0$
fins a $\theta$. S'observa un valor de $X$ i es vol provar $H_0:\theta=1$ en
funci\'o de $H_1:\theta>1$. Es decideix rebutjar $H_0$ si el valor de la mostra
\'es m\'es gran que $0.99$. Trobau l'error tipus I per a aquest contrast i feu un
dibuix de la funci\'o de pot\`encia.}
\etiqueta{EXGRAFFUNPOT}
\end{probres}

\res{La situaci\'o de forma esquem\`atica \'es la seg\"uent:

Tenim una variable aleat\`oria $X\ U(0,\theta)$.

Tenim una mostra aleat\`oria simple de $X$ de grand\`aria: $X_1$.

Volem fer el contrast seg\"uent:
\[
\left.
\begin{array}{ll}
H_0 :& \theta =1, \\
H_1 :& \theta >1.
\end{array}
\right\}
\]
Regi\'o cr\'{\i}tica escollida: $R.C.=\{X_1 > 0.99\}$.

Trobem l'error tipus I $\alpha$:
\begin{eqnarray*}
\alpha & = & \pp{\mbox{rebutjar $H_0$}/\mbox{$H_0$ certa}} =
\pp{ X_1 >0.99/\theta =1} \\ &=&\int_{0.99}^1 1\, dx =
1-0.99 =0.01,
\end{eqnarray*}
tenint en compte que la funci\'o de densitat de $X_1$ val, si
$H_0$ \'es certa:
\[
f_{X_1}(x)=f_X (x)=
\left\{\begin{array}{ll}
1, & \text{si $x\in (0,1)$}, \\
0, & \text{en cas contrari}.
\end{array}\right.
\]
Funci\'o de pot\`encia:
\begin{eqnarray*}
Q(\theta)&=&\pp{\mbox{rebutjar $H_0$}}=1-\pp{\mbox{acceptar $H_0$}}=
1-\pp{X_1\leq 0.99} \\ &=& 1-\int_0^{0.99}\frac{1}{\theta}\, dx =
1-\frac{0.99}{\theta}.
\end{eqnarray*}
El gr\`afic de la funci\'o de pot\`encia es pot veure en la figura
\ref{GRAFFUNPOT}.

\begin{figure}
\begin{center}
\setlength{\unitlength}{2.5cm}
\begin{picture}(4,2.25)(-0.5,-0.2)
\put (0,0){\line(1,0){4}}
\put (0,0){\line(0,1){2.25}}
\put (-0.3,0.02){\makebox(0,0){$0.01$}}
\put (-0.1,2){\makebox(0,0){$1$}}
\put (-0.25,1){\makebox(0,0){$0.5$}}
\put (0,0.02) {\line (1,0){0.1}}
\put (0,1) {\line (1,0){0.1}}
\put (0,2) {\line (1,0){0.1}}
\put (1,0) {\line (0,1){0.1}}
\put (2,0) {\line (0,1){0.1}}
\put (3,0) {\line (0,1){0.1}}
\put (4,0) {\line (0,1){0.1}}
\put (3.5,-0.2) {\makebox(0,0){$\theta$}}
\put (-0.4,1.5) {\makebox(0,0){$Q(\theta)$}}
\put (1,-0.2) {\makebox(0,0){$1$}}
\put (2,-0.2) {\makebox(0,0){$2$}}
\put (3,-0.2) {\makebox(0,0){$3$}}
\put (4,-0.2) {\makebox(0,0){$4$}}
\dashline[+30]{0.05}(0,2)(4,2)
\drawline
(1.0000,0.0200)
(1.0100,0.0396)
(1.0200,0.0588)
(1.0300,0.0777)
(1.0400,0.0962)
(1.0500,0.1143)
(1.0600,0.1321)
(1.0700,0.1495)
(1.0800,0.1667)
(1.0900,0.1835)
(1.1000,0.2000)
(1.1100,0.2162)
(1.1200,0.2321)
(1.1300,0.2478)
(1.1400,0.2632)
(1.1500,0.2783)
(1.1600,0.2931)
(1.1700,0.3077)
(1.1800,0.3220)
(1.1900,0.3361)
(1.2000,0.3500)
(1.2100,0.3636)
(1.2200,0.3770)
(1.2300,0.3902)
(1.2400,0.4032)
(1.2500,0.4160)
(1.2600,0.4286)
(1.2700,0.4409)
(1.2800,0.4531)
(1.2900,0.4651)
(1.3000,0.4769)
(1.3100,0.4885)
(1.3200,0.5000)
(1.3300,0.5113)
(1.3400,0.5224)
(1.3500,0.5333)
(1.3600,0.5441)
(1.3700,0.5547)
(1.3800,0.5652)
(1.3900,0.5755)
(1.4000,0.5857)
(1.4100,0.5957)
(1.4200,0.6056)
(1.4300,0.6154)
(1.4400,0.6250)
(1.4500,0.6345)
(1.4600,0.6438)
(1.4700,0.6531)
(1.4800,0.6622)
(1.4900,0.6711)
(1.5000,0.6800)
(1.5100,0.6887)
(1.5200,0.6974)
(1.5300,0.7059)
(1.5400,0.7143)
(1.5500,0.7226)
(1.5600,0.7308)
(1.5700,0.7389)
(1.5800,0.7468)
(1.5900,0.7547)
(1.6000,0.7625)
(1.6100,0.7702)
(1.6200,0.7778)
(1.6300,0.7853)
(1.6400,0.7927)
(1.6500,0.8000)
(1.6600,0.8072)
(1.6700,0.8144)
(1.6800,0.8214)
(1.6900,0.8284)
(1.7000,0.8353)
(1.7100,0.8421)
(1.7200,0.8488)
(1.7300,0.8555)
(1.7400,0.8621)
(1.7500,0.8686)
(1.7600,0.8750)
(1.7700,0.8814)
(1.7800,0.8876)
(1.7900,0.8939)
(1.8000,0.9000)
(1.8100,0.9061)
(1.8200,0.9121)
(1.8300,0.9180)
(1.8400,0.9239)
(1.8500,0.9297)
(1.8600,0.9355)
(1.8700,0.9412)
(1.8800,0.9468)
(1.8900,0.9524)
(1.9000,0.9579)
(1.9100,0.9634)
(1.9200,0.9688)
(1.9300,0.9741)
(1.9400,0.9794)
(1.9500,0.9846)
(1.9600,0.9898)
(1.9700,0.9949)
(1.9800,1.0000)
(1.9900,1.0050)
(2.0000,1.0100)
(2.0100,1.0149)
(2.0200,1.0198)
(2.0300,1.0246)
(2.0400,1.0294)
(2.0500,1.0341)
(2.0600,1.0388)
(2.0700,1.0435)
(2.0800,1.0481)
(2.0900,1.0526)
(2.1000,1.0571)
(2.1100,1.0616)
(2.1200,1.0660)
(2.1300,1.0704)
(2.1400,1.0748)
(2.1500,1.0791)
(2.1600,1.0833)
(2.1700,1.0876)
(2.1800,1.0917)
(2.1900,1.0959)
(2.2000,1.1000)
(2.2100,1.1041)
(2.2200,1.1081)
(2.2300,1.1121)
(2.2400,1.1161)
(2.2500,1.1200)
(2.2600,1.1239)
(2.2700,1.1278)
(2.2800,1.1316)
(2.2900,1.1354)
(2.3000,1.1391)
(2.3100,1.1429)
(2.3200,1.1466)
(2.3300,1.1502)
(2.3400,1.1538)
(2.3500,1.1574)
(2.3600,1.1610)
(2.3700,1.1646)
(2.3800,1.1681)
(2.3900,1.1715)
(2.4000,1.1750)
(2.4100,1.1784)
(2.4200,1.1818)
(2.4300,1.1852)
(2.4400,1.1885)
(2.4500,1.1918)
(2.4600,1.1951)
(2.4700,1.1984)
(2.4800,1.2016)
(2.4900,1.2048)
(2.5000,1.2080)
(2.5100,1.2112)
(2.5200,1.2143)
(2.5300,1.2174)
(2.5400,1.2205)
(2.5500,1.2235)
(2.5600,1.2266)
(2.5700,1.2296)
(2.5800,1.2326)
(2.5900,1.2355)
(2.6000,1.2385)
(2.6100,1.2414)
(2.6200,1.2443)
(2.6300,1.2471)
(2.6400,1.2500)
(2.6500,1.2528)
(2.6600,1.2556)
(2.6700,1.2584)
(2.6800,1.2612)
(2.6900,1.2639)
(2.7000,1.2667)
(2.7100,1.2694)
(2.7200,1.2721)
(2.7300,1.2747)
(2.7400,1.2774)
(2.7500,1.2800)
(2.7600,1.2826)
(2.7700,1.2852)
(2.7800,1.2878)
(2.7900,1.2903)
(2.8000,1.2929)
(2.8100,1.2954)
(2.8200,1.2979)
(2.8300,1.3004)
(2.8400,1.3028)
(2.8500,1.3053)
(2.8600,1.3077)
(2.8700,1.3101)
(2.8800,1.3125)
(2.8900,1.3149)
(2.9000,1.3172)
(2.9100,1.3196)
(2.9200,1.3219)
(2.9300,1.3242)
(2.9400,1.3265)
(2.9500,1.3288)
(2.9600,1.3311)
(2.9700,1.3333)
(2.9800,1.3356)
(2.9900,1.3378)
(3.0000,1.3400)
(3.0100,1.3422)
(3.0200,1.3444)
(3.0300,1.3465)
(3.0400,1.3487)
(3.0500,1.3508)
(3.0600,1.3529)
(3.0700,1.3550)
(3.0800,1.3571)
(3.0900,1.3592)
(3.1000,1.3613)
(3.1100,1.3633)
(3.1200,1.3654)
(3.1300,1.3674)
(3.1400,1.3694)
(3.1500,1.3714)
(3.1600,1.3734)
(3.1700,1.3754)
(3.1800,1.3774)
(3.1900,1.3793)
(3.2000,1.3813)
(3.2100,1.3832)
(3.2200,1.3851)
(3.2300,1.3870)
(3.2400,1.3889)
(3.2500,1.3908)
(3.2600,1.3926)
(3.2700,1.3945)
(3.2800,1.3963)
(3.2900,1.3982)
(3.3000,1.4000)
(3.3100,1.4018)
(3.3200,1.4036)
(3.3300,1.4054)
(3.3400,1.4072)
(3.3500,1.4090)
(3.3600,1.4107)
(3.3700,1.4125)
(3.3800,1.4142)
(3.3900,1.4159)
(3.4000,1.4176)
(3.4100,1.4194)
(3.4200,1.4211)
(3.4300,1.4227)
(3.4400,1.4244)
(3.4500,1.4261)
(3.4600,1.4277)
(3.4700,1.4294)
(3.4800,1.4310)
(3.4900,1.4327)
(3.5000,1.4343)
(3.5100,1.4359)
(3.5200,1.4375)
(3.5300,1.4391)
(3.5400,1.4407)
(3.5500,1.4423)
(3.5600,1.4438)
(3.5700,1.4454)
(3.5800,1.4469)
(3.5900,1.4485)
(3.6000,1.4500)
(3.6100,1.4515)
(3.6200,1.4530)
(3.6300,1.4545)
(3.6400,1.4560)
(3.6500,1.4575)
(3.6600,1.4590)
(3.6700,1.4605)
(3.6800,1.4620)
(3.6900,1.4634)
(3.7000,1.4649)
(3.7100,1.4663)
(3.7200,1.4677)
(3.7300,1.4692)
(3.7400,1.4706)
(3.7500,1.4720)
(3.7600,1.4734)
(3.7700,1.4748)
(3.7800,1.4762)
(3.7900,1.4776)
(3.8000,1.4789)
(3.8100,1.4803)
(3.8200,1.4817)
(3.8300,1.4830)
(3.8400,1.4844)
(3.8500,1.4857)
(3.8600,1.4870)
(3.8700,1.4884)
(3.8800,1.4897)
(3.8900,1.4910)
(3.9000,1.4923)
(3.9100,1.4936)
(3.9200,1.4949)
(3.9300,1.4962)
(3.9400,1.4975)
(3.9500,1.4987)
(3.9600,1.5000)
(3.9700,1.5013)
(3.9800,1.5025)
(3.9900,1.5038)
\end{picture}
\caption{Gr\`afic de la funci\'o de pot\`encia per a l'exercici 
\ref{EXGRAFFUNPOT}}
\label{GRAFFUNPOT}
\end{center}
\end{figure}
}

\begin{probres}
{Suposem que $X$ \'es una variable aleat\`oria de Bernoulli amb par\`ametre $p$. Es pren una
mostra aleat\`oria simple de 4 observacions de $X$ i es vol provar $H_0:p={1\over 4}$ contra
$H_1:p={3\over 4}$. Es rebutja $H_0$ si s'obtenen $4$ \`exits en la mostra.
Trobau els valors de $\alpha$ (error tipus I) i $\beta$ (error tipus II).}
\end{probres}

\res{Considerem $X_1,X_2,X_3,X_4$ una mostra aleat\`oria simple
d'una variable aleat\`oria $X$ Bernoulli $(p)$.

Es rebutja $H_0$ si hi ha $4$ \`exits en la mostra. Per tant, la regi\'o
cr\'{\i}tica ser\`a:
\[
R.C. =\{X_1=1\cap X_2=1\cap X_3 =1\cap X_4 =1\}=
\{\min\{ X_1,X_2,X_3,X_4 \}=X_{(1)}=1\},
\]
ja que considerar $4$ \`exits en la mostra \'es equivalent a dir 
que el m\'{\i}nim de les $X_i$ val $1$.

En el problema \ref{FUNDENMAXMINBERNOULLI} varem veure que la funci\'o
de probabilitat de $X_{(1)}$ era:
\begin{center}
\begin{tabular}{|c||c|c|}
\hline
$x$&$0$&$1$\\\hline
$f_{X_{(1)}}$&$1-p^4$&$p^4$\\\hline
\end{tabular}
\end{center}
Per tant, podem deduir que $\pp{X_{(1)}=1}=p^4$ i 
$\pp{X_{(1)}=0}=1-p^4$.

Trobem l'error tipus I ($\alpha$) i tipus II ($\beta$):
\begin{eqnarray*}
\alpha & = & 
\pp{\mbox{rebutjar $H_0$}/\mbox{$H_0$ certa}}
= \pp{ X_{(1)}=1/ p=\frac{1}{4}}={\left(\frac{1}{4}\right)}^4
\approx 0.003906,\\ 
\beta & = & 
\pp{\mbox{acceptar $H_0$}/\mbox{$H_0$ falsa}}
= \pp{ X_{(1)}=0/ p=\frac{3}{4}}=1-{\left(\frac{3}{4}\right)}^4
\approx 0.6836.
\end{eqnarray*}
}

\begin{probres}
{Si $X$ \'es una variable aleat\`oria normal amb mitjana $0$, trobau la prova uniforme m\'es
poderosa de $H_0:\sigma^2=2$ contra $H_1:\sigma^2 >2$ de grand\`aria $\alpha$,
donada una mostra aleat\`oria simple de $n$ observacions de $X$.}
\etiqueta{EXERCICIREGCRITPODEROSASIGMA}
\end{probres}

\res{Considerem $X_1,\ldots,X_n$ una mostra aleat\`oria simple de la variable
aleat\`oria \mbox{$X=N(\mu =0,\sigma^2)$.}

Sigui $\sigma_1$ un valor fix amb $\sigma_1^2 >2$. Considerem el 
contrast seg\"uent:
\begin{equation}
\left.
\begin{array}{ll}
H_0 :&  \sigma^2 =2, \\
H_1 :&  \sigma^2 =\sigma_1^2.
\end{array}
\right\}
\label{CONTRASTHIPSIMPLES}
\end{equation}
on l'error tipus I \'es $\alpha$.
 
Aplicant el teorema de Neyman-Pearson, la regi\'o cr\'{\i}tica m\'es
bona (la que t\'e l'error tipus II m\'es petit) per al contrast anterior
\'es:
\[
R.C.=\left\{
\frac{L\left(\sigma_1^2\right)}{L(2)}\geq k_{\alpha}
\right\},
\]
on $L\left(\sigma^2\right)$ \'es la funci\'o de versemblan\c{c}a de la   
mostra:
\[
L\left(\sigma^2\right)=\frac{1}{{(2\pi)}^{\frac{n}{2}}\sigma^n}
\e^{-\frac{1}{2\sigma^2}\sum\limits_{i=1}^n X_i^2}.
\]
Per tant, el quocient que apareix en la regi\'o cr\'{\i}tica anterior val:
\[
\frac{L\left(\sigma_1^2\right)}{L(2)}=\frac{2^{\frac{n}{2}}}{\sigma_1^n}
\e^{-\left(\frac{1}{2\sigma_1^2}-\frac{1}{4}\right)\sum\limits_{i=1}^n
X_i^2}.
\]
Per tant, escriure 
\(\frac{L\left(\sigma_1^2\right)}{L(2)}\geq k_{\alpha}\) \'es 
equivalent a:
\[
\sum_{i=1}^n X_i^2 \geq\frac{\ln\left(
\frac{k_{\alpha}\sigma_1^2}{2^{\frac{n}{2}}}\right)}
{\frac{1}{4}-\frac{1}{2\sigma_1^2}}:=C_{\alpha}.
\]
La regi\'o cr\'{\i}tica per al contrast \ref{CONTRASTHIPSIMPLES} ser\`a,
per tant:
\[
R.C.=\left\{\sum_{i=1}^n X_i^2 \geq C_{\alpha}\right\}.
\]
A continuaci\'o, trobem $C_{\alpha}$. 

Si $H_0$ \'es certa, aleshores
les variables $X_i$ s\'on normals $N(0,\sigma^2 =2)$. 
Per tant les variables $\frac{X_i}{\sqrt{2}}$
s\'on $N(0,1)$. D'aqu\'{\i} dedu\"{\i}m que la variable
\mbox{$\frac{\sum\limits_{i=1}^n X_i^2}{2}$} \'es una variable khi
quadrat amb $n$ graus de llibertat. 
Tenint en compte que l'error tipus I val $\alpha$, podem trobar el
valor de la constant $C_{\alpha}$:
\[
\alpha =\pp{\mbox{rebutjar $H_0$}/\mbox{$H_0$ certa}}=
\pp{\frac{\sum\limits_{i=1}^n X_i^2}{2}\geq\frac{C_{\alpha}}{2}
}=\pp{\chi_n^2\geq\frac{C_{\alpha}}{2}}.
\]
Per tant $C_{\alpha}=2\chi_{1-\alpha}^2$ on $\chi_{1-\alpha}^2$ \'es
el $100(1-\alpha)$-\`essim percentil per la distribuci\'o $\chi_n^2$.

Concloem que la millor regi\'o cr\'{\i}tica per al 
contrast~(\ref{CONTRASTHIPSIMPLES}) \'es:
\begin{equation}
R.C.=\left\{\sum_{i=1}^n X_i^2\geq 2\chi_{1-\alpha}^2\right\}.
\label{MILLORREGCRIHIPSIMPLES}
\end{equation}
Prenem com a regi\'o cr\'{\i}tica del contrast original
\begin{equation}
\left.
\begin{array}{ll}
H_0 :& \sigma^2 =2, \\
H_1 :&  \sigma^2 >2.
\end{array}
\right\}
\label{CONTRASTHIPCOMPOSTES}
\end{equation}
la regi\'o cr\'{\i}tica~(\ref{MILLORREGCRIHIPSIMPLES}) anterior.

Vegem que aquesta \'es la millor regi\'o cr\'{\i}tica per al
contrast~(\ref{CONTRASTHIPCOMPOSTES}).
\newpage
Trobem la funci\'o de pot\`encia:
\begin{eqnarray*}
Q\left(\sigma^2\right)& = & 1-\pp{\mbox{acceptar $H_0$}}
=\pp{\sum_{i=1}^n X_i^2\geq 2\chi_{1-\alpha}^2}\\ 
& = & \pp{\frac{\sum\limits_{i=1}^n X_i^2}{\sigma^2}=\chi_n^2
\geq \frac{2}{\sigma^2}\chi_{1-\alpha}^2}=
1-F_{\chi_n^2}\left(\frac{2}{\sigma^2}\chi_{1-\alpha}^2\right).
\end{eqnarray*}
La representaci\'o gr\`afica de la funci\'o de pot\`encia es pot 
veure en la figura \ref{GRAFFUNPOTCONHIPCOMPOSTES}.

\begin{figure}
\setlength{\unitlength}{2.5cm}
\begin{picture}(4,2.25)(-0.5,-0.2)
\put (0,0){\line(1,0){4}}
\put (0,0){\line(0,1){2.25}}
\put (-0.1,2){\makebox(0,0){$1$}}
\put (0,1) {\line (1,0){0.1}}
\put (0,2) {\line (1,0){0.1}}
\put (1,0) {\line (0,1){0.1}}
\put (2,0) {\line (0,1){0.1}}
\put (3,0) {\line (0,1){0.1}}
\put (4,0) {\line (0,1){0.1}}
\put (1,-0.2) {\makebox(0,0){$1$}}
\put (2,-0.2) {\makebox(0,0){$2$}}
\put (3,-0.2) {\makebox(0,0){$3$}}
\put (4,-0.2) {\makebox(0,0){$4$}}
\put (0,0.6){\line (1,0){0.1}}
\put (-0.1,0.6){\makebox(0,0){$\alpha$}}
\put (3.5,-0.2) {\makebox(0,0){$\sigma^2$}}
\put (-0.4,1.5) {\makebox(0,0){$Q(\sigma^2)$}}
\dashline[+30]{0.05}(0,2)(4,2)
\drawline
(2.0000,0.6001)
(2.0100,0.6083)
(2.0200,0.6164)
(2.0300,0.6246)
(2.0400,0.6327)
(2.0500,0.6409)
(2.0600,0.6490)
(2.0700,0.6571)
(2.0800,0.6652)
(2.0900,0.6733)
(2.1000,0.6814)
(2.1100,0.6895)
(2.1200,0.6975)
(2.1300,0.7056)
(2.1400,0.7136)
(2.1500,0.7216)
(2.1600,0.7296)
(2.1700,0.7375)
(2.1800,0.7454)
(2.1900,0.7534)
(2.2000,0.7613)
(2.2100,0.7691)
(2.2200,0.7770)
(2.2300,0.7848)
(2.2400,0.7926)
(2.2500,0.8004)
(2.2600,0.8081)
(2.2700,0.8158)
(2.2800,0.8235)
(2.2900,0.8312)
(2.3000,0.8388)
(2.3100,0.8464)
(2.3200,0.8540)
(2.3300,0.8615)
(2.3400,0.8690)
(2.3500,0.8765)
(2.3600,0.8840)
(2.3700,0.8914)
(2.3800,0.8988)
(2.3900,0.9061)
(2.4000,0.9134)
(2.4100,0.9207)
(2.4200,0.9280)
(2.4300,0.9352)
(2.4400,0.9424)
(2.4500,0.9495)
(2.4600,0.9567)
(2.4700,0.9637)
(2.4800,0.9708)
(2.4900,0.9778)
(2.5000,0.9848)
(2.5100,0.9917)
(2.5200,0.9986)
(2.5300,1.0055)
(2.5400,1.0123)
(2.5500,1.0191)
(2.5600,1.0259)
(2.5700,1.0326)
(2.5800,1.0393)
(2.5900,1.0459)
(2.6000,1.0525)
(2.6100,1.0591)
(2.6200,1.0657)
(2.6300,1.0722)
(2.6400,1.0786)
(2.6500,1.0850)
(2.6600,1.0914)
(2.6700,1.0978)
(2.6800,1.1041)
(2.6900,1.1104)
(2.7000,1.1166)
(2.7100,1.1228)
(2.7200,1.1290)
(2.7300,1.1351)
(2.7400,1.1412)
(2.7500,1.1472)
(2.7600,1.1532)
(2.7700,1.1592)
(2.7800,1.1651)
(2.7900,1.1710)
(2.8000,1.1769)
(2.8100,1.1827)
(2.8200,1.1885)
(2.8300,1.1942)
(2.8400,1.1999)
(2.8500,1.2056)
(2.8600,1.2113)
(2.8700,1.2168)
(2.8800,1.2224)
(2.8900,1.2279)
(2.9000,1.2334)
(2.9100,1.2389)
(2.9200,1.2443)
(2.9300,1.2497)
(2.9400,1.2550)
(2.9500,1.2603)
(2.9600,1.2656)
(2.9700,1.2708)
(2.9800,1.2760)
(2.9900,1.2812)
(3.0000,1.2863)
(3.0100,1.2914)
(3.0200,1.2965)
(3.0300,1.3015)
(3.0400,1.3065)
(3.0500,1.3114)
(3.0600,1.3164)
(3.0700,1.3212)
(3.0800,1.3261)
(3.0900,1.3309)
(3.1000,1.3357)
(3.1100,1.3404)
(3.1200,1.3451)
(3.1300,1.3498)
(3.1400,1.3545)
(3.1500,1.3591)
(3.1600,1.3637)
(3.1700,1.3682)
(3.1800,1.3727)
(3.1900,1.3772)
(3.2000,1.3817)
(3.2100,1.3861)
(3.2200,1.3905)
(3.2300,1.3948)
(3.2400,1.3992)
(3.2500,1.4035)
(3.2600,1.4077)
(3.2700,1.4119)
(3.2800,1.4161)
(3.2900,1.4203)
(3.3000,1.4244)
(3.3100,1.4286)
(3.3200,1.4326)
(3.3300,1.4367)
(3.3400,1.4407)
(3.3500,1.4447)
(3.3600,1.4486)
(3.3700,1.4526)
(3.3800,1.4565)
(3.3900,1.4603)
(3.4000,1.4642)
(3.4100,1.4680)
(3.4200,1.4718)
(3.4300,1.4755)
(3.4400,1.4793)
(3.4500,1.4830)
(3.4600,1.4866)
(3.4700,1.4903)
(3.4800,1.4939)
(3.4900,1.4975)
(3.5000,1.5011)
(3.5100,1.5046)
(3.5200,1.5081)
(3.5300,1.5116)
(3.5400,1.5151)
(3.5500,1.5185)
(3.5600,1.5219)
(3.5700,1.5253)
(3.5800,1.5286)
(3.5900,1.5320)
(3.6000,1.5353)
(3.6100,1.5386)
(3.6200,1.5418)
(3.6300,1.5450)
(3.6400,1.5483)
(3.6500,1.5514)
(3.6600,1.5546)
(3.6700,1.5577)
(3.6800,1.5608)
(3.6900,1.5639)
(3.7000,1.5670)
(3.7100,1.5700)
(3.7200,1.5730)
(3.7300,1.5760)
(3.7400,1.5790)
(3.7500,1.5820)
(3.7600,1.5849)
(3.7700,1.5878)
(3.7800,1.5907)
(3.7900,1.5935)
(3.8000,1.5964)
(3.8100,1.5992)
(3.8200,1.6020)
(3.8300,1.6048)
(3.8400,1.6075)
(3.8500,1.6103)
(3.8600,1.6130)
(3.8700,1.6157)
(3.8800,1.6183)
(3.8900,1.6210)
(3.9000,1.6236)
(3.9100,1.6262)
(3.9200,1.6288)
(3.9300,1.6314)
(3.9400,1.6339)
(3.9500,1.6365)
(3.9600,1.6390)
(3.9700,1.6415)
(3.9800,1.6439)
(3.9900,1.6464)
\end{picture}
\caption{Gr\`afic de la funci\'o de pot\`encia per a l'exercici 
\ref{EXERCICIREGCRITPODEROSASIGMA}}
\label{GRAFFUNPOTCONHIPCOMPOSTES}
\end{figure}

Suposem ara que tenim una altra regi\'o cr\'{\i}tica $R.C.'$ m\'es
bona que la donada per~(\ref{MILLORREGCRIHIPSIMPLES}). Aix\`o voldria
dir que la funci\'o de pot\`encia $Q'\left(\sigma^2\right)$ donada per
$R.C.'$ hauria d'estar per damunt de la funci\'o de pot\`encia
$Q\left(\sigma^2\right)$ donada per la regi\'o 
cr\'{\i}tica~(\ref{MILLORREGCRIHIPSIMPLES}). 
Per\`o aix\`o \'es absurd ja que sabem que,
per a $\sigma^2 =\sigma_1^2 >2$, la funci\'o de pot\`encia del 
contrast~(\ref{CONTRASTHIPSIMPLES}) pren el valor m\`axim en $Q\left(\sigma_1^2
\right)$ ja que la $R.C.$ donada per~(\ref{MILLORREGCRIHIPSIMPLES}) \'es
la millor regi\'o cr\'{\i}tica del contrast~(\ref{CONTRASTHIPSIMPLES}).
}

\newpage

\begin{probres}
{Suposem que la precipitaci\'o de pluja anual registrada en una
determinada estaci\'o \'es una variable aleat\`oria normal amb mitjana $\mu$ i desviaci\'o est\`andard
de $0.05$ metres. S'estudia una mostra aleat\`oria simple de grand\`aria~$5$. La pluja registrada (en
metres) en cada un dels 5 anys va ser: $$0.46,\ 0.52,\ 0.44,\ 0.38\mbox{ i
}0.57.$$Feu el contrast d'hip\`otesi seg\"uent:
$$\left.
\begin{array}{ll}
H_0 :&  \mu =0.5334,\\
H_1 :&  \mu<0.5334.
\end{array}
\right\}
$$

O sigui, trobau l'error tipus I m\`axim per poder acceptar $H_0$.}
\end{probres}

\res{Hem de fer un contrast respecte del par\`ametre $\mu$ d'una variable
aleat\`oria $X$ normal amb par\`ametres $\mu$ i $\sigma =0.05$ coneguda.
La regi\'o cr\'{\i}tica corresponent a aquest contrast (vegeu ap\`endix 
A.1 cas \setcounter{aux}{2}\Roman{aux}) \'es:
\[
R.C.=\{ Z\leq z_{\alpha}\},
\]
on $Z$ \'es una normal $N(0,1)$ i $z_{\alpha}$ \'es el percentil
$100\alpha\%$ de la mateixa variable $N(0,1)$.

L'estad\'{\i}stic a fer servir \'es (est\`a en el mateix ap\`endix):
\[
Z=\frac{\overline{X}-\mu_0}{\frac{\sigma}{\sqrt{n}}}=
\frac{0.474-0.5334}{\frac{0.05}{\sqrt{5}}}\approx -2.6564.
\]
L'error $\alpha_{\mbox{\footnotesize max}}$ per poder acceptar $H_0$ 
ser\`a:
\[
\alpha_{\mbox{\footnotesize max}}=\pp{Z\leq -2.6564}=1-
\pp{Z\leq 2.6564}\approx 1-0.996 =0.004,
\]
ja que si $\alpha< 0.004$, aleshores $Z=-2.6564 > z_{\alpha}$ i, per 
tant, el valor de l'estad\'{\i}stic $Z$ est\`a fora de la regi\'o 
cr\'{\i}tica.

Com que $\alpha_{\mbox{\footnotesize max}}$ \'es molt petit,
rebutjam~$H_0$ i concloem que $\mu <0.5334$.
}

\begin{probres}
{Fent servir la prova del criteri de la ra\'o de versemblan\c{c}a, trobau la
regi\'o cr\'{\i}tica per a la hip\`otesi 
\mbox{$H_0:\lambda \leq\lambda_0$} contra l'alternativa
\mbox{$H_1:\lambda >\lambda_0$}
on $\lambda$ \'es el par\`ametre d'una variable aleat\`oria exponencial.
Suposem que disposam d'una mostra aleat\`oria simple de $n$ observacions de 
$X$ i volem que l'error
tipus I sigui~$\alpha$.}
\etiqueta{CONHIPEXPLAMBDARAOVERSEM}
\end{probres}

\res{Les funcions de densitat i de versemblan\c{c}a corresponents
a la variable $X\ \mbox{Exp }(\lambda)$ i la mostra aleat\`oria simple,
respectivament, s\'on:
\begin{eqnarray*}
f_X (x) & = & \lambda \e^{-\lambda x},\ x>0,\\
L(\lambda) & = & \prod_{i=1}^n f_{X_i} (x_i) = \lambda^n \e^{
-\lambda \sum\limits_{i=1}^n x_i}, \ x_i>0.
\end{eqnarray*}
Segons el criteri de la ra\'o de versemblan\c{c}a, la millor regi\'o
cr\'{\i}tica \'es:
\begin{equation}
R.C.=\left\{\frac{L(\omega)}{L(\Omega)}< A\right\},
\label{REGCRITEXPLAMBDA}
\end{equation}
on 
\[ 
L(\omega)=\max\limits_{\lambda\in H_0} L(\lambda),\  
L(\Omega)=\max\limits_{\lambda\in H_0\cup H_1} L(\lambda),
\]
i $A$ s'ha d'escollir de tal forma que l'error tipus~I sigui~$\alpha$.

En el nostre cas, 
\[ 
L(\omega)=
\max\limits_{\lambda\leq \lambda_0} L(\lambda)
\mbox{ i } 
L(\Omega)=\max\limits_{\lambda\in \RR} L(\lambda)
\]
Quan estudi\`avem els estimadors de m\'axima versemblan\c{c}a 
(vegeu problema \ref{MAXVEREXPLAMBDA}), v\`arem veure que l'estimador
de m\`axima versemblan\c{c}a del par\`ametre $\lambda$ era 
$\frac{1}{\overline{X}}$. Per tant, podem afirmar que:
\begin{eqnarray*}
L(\Omega) & = & 
L\left(\frac{1}{\overline{X}}\right)=\frac{\e^{-n}}
{{\overline{X}}^n}, \\
L(\omega) & = & 
\max\limits_{\lambda\leq \lambda_0} L(\lambda) =
\left\{\begin{array}{ll}
L(\lambda_0)=\lambda_0^n \e^{-n\lambda_0 \overline{X}}, &
\text{si $\lambda_0<\frac{1}{\overline{X}}$},\\
L\left(\frac{1}{\overline{X}}\right)=\frac{\e^{-n}}
{{\overline{X}}^n}, & \text{si $\lambda_0\geq\frac{1}{\overline{X}}$}. 
\end{array}\right.
\end{eqnarray*}
El quocient $\frac{L(\omega)}{L(\Omega)}$ que surt en la regi\'o   
cr\'{\i}tica valdr\`a:
\[
\frac{L(\omega)}{L(\Omega)} =
\left\{\begin{array}{ll}
{(\lambda_0\overline{X})}^n e^{-n (\lambda_0\overline{X} -1)}, 
& \text{si $\lambda_0 <\frac{1}{\overline{X}}$}, \\
1, & \text{si $\lambda_0\geq \frac{1}{\overline{X}}$}.
\end{array}\right.
\]
El gr\`afic de la funci\'o $\frac{L(\omega)}{L(\Omega)}$  com 
a funci\'o de la variable $\frac{1}{\overline{X}}$ es pot veure en
la figura \ref{GRAFFUNVERSEMBLANCA}.
\begin{figure}
\setlength{\unitlength}{2.5cm}
\begin{picture}(4,2.25)(-0.5,-0.2)
\put (0,0){\line(1,0){4}}
\put (0,0){\line(0,1){2.25}}
\put (-0.1,2){\makebox(0,0){$1$}}
\put (0,1) {\line (1,0){0.1}}
\put (0,2) {\line (1,0){0.1}}
\put (1,0) {\line (0,1){0.1}}
%\put (2,0) {\line (0,1){0.1}}
%\put (3,0) {\line (0,1){0.1}}
%\put (4,0) {\line (0,1){0.1}}
\put (0,2) {\line (1,0){1}}
\put (-0.3,1.5){\makebox(0,0){$\frac{L(\omega)}{L(\Omega)}$}}
\put (3,-0,2) {\makebox(0,0){$\frac{1}{\overline{X}}$}}
\put (1,-0.2) {\makebox(0,0){$\lambda_0$}}
%\put (2,-0.2) {\makebox(0,0){$2$}}
%\put (3,-0.2) {\makebox(0,0){$3$}}
%\put (4,-0.2) {\makebox(0,0){$4$}}
%\put (0,1.0352){\line (1,0){0.1}}
%\put (-0.25,1.0352){\makebox(0,0){$1-\alpha$}}
\dashline[+30]{0.05}(0,1)(1.48,1)
\dashline[+30]{0.05}(1.48,0)(1.48,1)
\put (1.48,-0.2) {\makebox(0,0){$C'$}}
\put (-0.2,1) {\makebox(0,0){$A$}}
\drawline
(1.0000,2.0000)
(1.0100,1.9990)
(1.0200,1.9961)
(1.0300,1.9914)
(1.0400,1.9849)
(1.0500,1.9767)
(1.0600,1.9670)
(1.0700,1.9557)
(1.0800,1.9431)
(1.0900,1.9291)
(1.1000,1.9139)
(1.1100,1.8975)
(1.1200,1.8800)
(1.1300,1.8616)
(1.1400,1.8422)
(1.1500,1.8219)
(1.1600,1.8008)
(1.1700,1.7791)
(1.1800,1.7567)
(1.1900,1.7337)
(1.2000,1.7102)
(1.2100,1.6862)
(1.2200,1.6618)
(1.2300,1.6371)
(1.2400,1.6121)
(1.2500,1.5868)
(1.2600,1.5613)
(1.2700,1.5357)
(1.2800,1.5099)
(1.2900,1.4841)
(1.3000,1.4582)
(1.3100,1.4323)
(1.3200,1.4065)
(1.3300,1.3807)
(1.3400,1.3550)
(1.3500,1.3294)
(1.3600,1.3039)
(1.3700,1.2786)
(1.3800,1.2535)
(1.3900,1.2286)
(1.4000,1.2039)
(1.4100,1.1795)
(1.4200,1.1553)
(1.4300,1.1313)
(1.4400,1.1077)
(1.4500,1.0843)
(1.4600,1.0613)
(1.4700,1.0385)
(1.4800,1.0161)
(1.4900,0.9940)
(1.5000,0.9722)
(1.5100,0.9508)
(1.5200,0.9297)
(1.5300,0.9089)
(1.5400,0.8885)
(1.5500,0.8685)
(1.5600,0.8488)
(1.5700,0.8294)
(1.5800,0.8104)
(1.5900,0.7918)
(1.6000,0.7735)
(1.6100,0.7555)
(1.6200,0.7379)
(1.6300,0.7206)
(1.6400,0.7037)
(1.6500,0.6871)
(1.6600,0.6709)
(1.6700,0.6550)
(1.6800,0.6394)
(1.6900,0.6242)
(1.7000,0.6093)
(1.7100,0.5947)
(1.7200,0.5804)
(1.7300,0.5664)
(1.7400,0.5528)
(1.7500,0.5394)
(1.7600,0.5263)
(1.7700,0.5136)
(1.7800,0.5011)
(1.7900,0.4889)
(1.8000,0.4770)
(1.8100,0.4653)
(1.8200,0.4540)
(1.8300,0.4429)
(1.8400,0.4320)
(1.8500,0.4214)
(1.8600,0.4111)
(1.8700,0.4010)
(1.8800,0.3911)
(1.8900,0.3815)
(1.9000,0.3721)
(1.9100,0.3630)
(1.9200,0.3540)
(1.9300,0.3453)
(1.9400,0.3368)
(1.9500,0.3285)
(1.9600,0.3204)
(1.9700,0.3124)
(1.9800,0.3047)
(1.9900,0.2972)
(2.0000,0.2899)
(2.0100,0.2827)
(2.0200,0.2757)
(2.0300,0.2689)
(2.0400,0.2623)
(2.0500,0.2558)
(2.0600,0.2495)
(2.0700,0.2434)
(2.0800,0.2373)
(2.0900,0.2315)
(2.1000,0.2258)
(2.1100,0.2202)
(2.1200,0.2148)
(2.1300,0.2095)
(2.1400,0.2044)
(2.1500,0.1994)
(2.1600,0.1945)
(2.1700,0.1897)
(2.1800,0.1850)
(2.1900,0.1805)
(2.2000,0.1761)
(2.2100,0.1718)
(2.2200,0.1676)
(2.2300,0.1635)
(2.2400,0.1595)
(2.2500,0.1556)
(2.2600,0.1518)
(2.2700,0.1481)
(2.2800,0.1445)
(2.2900,0.1410)
(2.3000,0.1376)
(2.3100,0.1342)
(2.3200,0.1310)
(2.3300,0.1278)
(2.3400,0.1247)
(2.3500,0.1217)
(2.3600,0.1187)
(2.3700,0.1159)
(2.3800,0.1131)
(2.3900,0.1104)
(2.4000,0.1077)
(2.4100,0.1051)
(2.4200,0.1026)
(2.4300,0.1002)
(2.4400,0.0978)
(2.4500,0.0954)
(2.4600,0.0932)
(2.4700,0.0909)
(2.4800,0.0888)
(2.4900,0.0867)
(2.5000,0.0846)
(2.5100,0.0826)
(2.5200,0.0806)
(2.5300,0.0787)
(2.5400,0.0769)
(2.5500,0.0751)
(2.5600,0.0733)
(2.5700,0.0716)
(2.5800,0.0699)
(2.5900,0.0683)
(2.6000,0.0667)
(2.6100,0.0651)
(2.6200,0.0636)
(2.6300,0.0621)
(2.6400,0.0607)
(2.6500,0.0592)
(2.6600,0.0579)
(2.6700,0.0565)
(2.6800,0.0552)
(2.6900,0.0539)
(2.7000,0.0527)
(2.7100,0.0515)
(2.7200,0.0503)
(2.7300,0.0492)
(2.7400,0.0480)
(2.7500,0.0469)
(2.7600,0.0459)
(2.7700,0.0448)
(2.7800,0.0438)
(2.7900,0.0428)
(2.8000,0.0418)
(2.8100,0.0409)
(2.8200,0.0399)
(2.8300,0.0390)
(2.8400,0.0382)
(2.8500,0.0373)
(2.8600,0.0365)
(2.8700,0.0356)
(2.8800,0.0348)
(2.8900,0.0341)
(2.9000,0.0333)
(2.9100,0.0326)
(2.9200,0.0318)
(2.9300,0.0311)
(2.9400,0.0304)
(2.9500,0.0298)
(2.9600,0.0291)
(2.9700,0.0285)
(2.9800,0.0278)
(2.9900,0.0272)
(3.0000,0.0266)
(3.0100,0.0260)
(3.0200,0.0255)
(3.0300,0.0249)
(3.0400,0.0244)
(3.0500,0.0238)
(3.0600,0.0233)
(3.0700,0.0228)
(3.0800,0.0223)
(3.0900,0.0218)
(3.1000,0.0214)
(3.1100,0.0209)
(3.1200,0.0204)
(3.1300,0.0200)
(3.1400,0.0196)
(3.1500,0.0191)
(3.1600,0.0187)
(3.1700,0.0183)
(3.1800,0.0179)
(3.1900,0.0176)
(3.2000,0.0172)
(3.2100,0.0168)
(3.2200,0.0165)
(3.2300,0.0161)
(3.2400,0.0158)
(3.2500,0.0154)
(3.2600,0.0151)
(3.2700,0.0148)
(3.2800,0.0145)
(3.2900,0.0142)
(3.3000,0.0139)
(3.3100,0.0136)
(3.3200,0.0133)
(3.3300,0.0130)
(3.3400,0.0128)
(3.3500,0.0125)
(3.3600,0.0122)
(3.3700,0.0120)
(3.3800,0.0117)
(3.3900,0.0115)
(3.4000,0.0113)
(3.4100,0.0110)
(3.4200,0.0108)
(3.4300,0.0106)
(3.4400,0.0104)
(3.4500,0.0102)
(3.4600,0.0100)
(3.4700,0.0098)
(3.4800,0.0096)
(3.4900,0.0094)
(3.5000,0.0092)
(3.5100,0.0090)
(3.5200,0.0088)
(3.5300,0.0086)
(3.5400,0.0085)
(3.5500,0.0083)
(3.5600,0.0081)
(3.5700,0.0080)
(3.5800,0.0078)
(3.5900,0.0076)
(3.6000,0.0075)
(3.6100,0.0073)
(3.6200,0.0072)
(3.6300,0.0071)
(3.6400,0.0069)
(3.6500,0.0068)
(3.6600,0.0066)
(3.6700,0.0065)
(3.6800,0.0064)
(3.6900,0.0063)
(3.7000,0.0061)
(3.7100,0.0060)
(3.7200,0.0059)
(3.7300,0.0058)
(3.7400,0.0057)
(3.7500,0.0056)
(3.7600,0.0055)
(3.7700,0.0054)
(3.7800,0.0052)
(3.7900,0.0051)
(3.8000,0.0050)
(3.8100,0.0050)
(3.8200,0.0049)
(3.8300,0.0048)
(3.8400,0.0047)
(3.8500,0.0046)
(3.8600,0.0045)
(3.8700,0.0044)
(3.8800,0.0043)
(3.8900,0.0042)
(3.9000,0.0042)
(3.9100,0.0041)
(3.9200,0.0040)
(3.9300,0.0039)
(3.9400,0.0039)
(3.9500,0.0038)
(3.9600,0.0037)
(3.9700,0.0036)
(3.9800,0.0036)
(3.9900,0.0035)
(4.0000,0.0034)
\end{picture}
\caption{Gr\`afic de la funci\'o $\frac{L(\omega)}{L(\Omega)}$ per a l'exercici
\ref{CONHIPEXPLAMBDARAOVERSEM}}
\label{GRAFFUNVERSEMBLANCA}
\end{figure}
Fixau-vos que dir que $\frac{L(\omega)}{L(\Omega)}<A$ \'es equivalent
a dir que existeix una constant $C'$ tal que 
\mbox{$\frac{1}{\overline{X}}>C'$}. Per tant, la regi\'o cr\'{\i}tica
\ref{REGCRITEXPLAMBDA} es pot escriure com:
\[
R.C.=\left\{\frac{1}{\overline{X}}>C'\right\}=\left\{
\overline{X}<C\right\},
\]
on $C=\frac{1}{C'}$. Per acabar, diguem com trobar $C$.

Sabem que si $X_1,\ldots,X_n$ \'es una mostra aleat\`oria simple d'una
variable aleat\`oria $\mbox{Exp }(\lambda)$, aleshores la variable
\mbox{$Y=2\lambda\sum\limits_{i=1}^n X_i$} es distribueix
segons una distribuci\'o $\chi^2$ amb $2n$ graus de llibertat.

Per tant, tenint en compte que l'error tipus I \'es $\alpha$, trobem
el valor de $C$:
\[
\alpha =\pp{\overline{X}<C}=\pp{2\lambda_0 \sum_{i=1}^n
X_i < 2\lambda_0 n C}=\pp{\chi_{2n}^2 < 2\lambda_0 n C}.
\]
Per tant, tendrem que $2\lambda_0 n C=\chi_{2n,\alpha}^2$ (percentil
$100\alpha\%$ de la distribuci\'o $\chi^2$ amb $2n$ graus de llibertat).
D'on dedu\"{\i}m que el valor de $C$ \'es:
\[
C=\frac{\chi_{2n,\alpha}^2}{2\lambda_0 n}.
\]
La regi\'o cr\'{\i}tica obtenguda ser\`a:
\[
R.C.=\left\{\overline{X}<\frac{\chi_{2n,\alpha}^2}{2\lambda_0 n}\right\}.
\]
}

\begin{probres}
{Suposem que $X_1,X_2,\ldots,X_n$ \'es una mostra aleat\`oria 
simple d'una variable aleat\`oria normal amb
mitjana $\mu_X$ i vari\`ancia $\sigma^2$. 
Suposem tamb\'e que $Y_1,Y_2,\ldots,Y_n$ \'es
una mostra aleat\`oria simple independent d'una variable aleat\`oria normal amb mitjana $\mu_Y$ i vari\`ancia $\sigma^2$
(notau que les vari\`ancies s\'on iguals). Provau que si s'aplica la prova de la ra\'o
de versemblan\c{c}a al contrast
$$\left.
\begin{array}{ll}
H_0 :& \mu_X=\mu_Y,\\ H_1:& \mu_X\not
=\mu_Y,
\end{array}
\right\}$$
la regi\'o cr\'{\i}tica que surt es pot posar en funci\'o 
de la variable aleat\`oria 
$${(\overline{X}-\overline{Y})\sqrt{n}\over S\sqrt{2}},\mbox{ on
}S^2={\sum\limits_{i=1}^n {(X_i-\overline{X})}^2 +  \sum\limits_{i=1}^n
{(Y_i-\overline{Y})}^2\over 2(n-1)}.$$
Quina \'es la distribuci\'o de la variable aleat\`oria anterior
si $H_0$ \'es certa?}
\etiqueta{CONTRASTDUESNORMALSVARIGUALS}
\end{probres}

\res{Les funcions de densitat de les mostres corresponents a les variables
$X$ i $Y$ i la funci\'o de versemblan\c{c}a de la mostra
s\'on:
\begin{eqnarray*}
f_X (x;\mu_X,\sigma^2) & = & \frac{1}{\sigma\sqrt{2\pi}}  
\e^{-\frac{{(x-\mu_X)}^2}{2\sigma^2}}, \\
f_Y (y;\mu_Y,\sigma^2) & = & \frac{1}{\sigma\sqrt{2\pi}}  
\e^{-\frac{{(y-\mu_Y)}^2}{2\sigma^2}}, \\
L(\mu_X,\mu_Y,\sigma^2) & = & \frac{1}{{(2\pi\sigma^2)}^{n}}
\e^{-\frac{1}{2\sigma^2}\left(\sum\limits_{i=1}^n {(X_i-\mu_X)}^2
+\sum\limits_{i=1}^n {(Y_i -\mu_Y)}^2\right)}.
\end{eqnarray*}
La millor regi\'o cr\'{\i}tica fent servir el m\`etode de la ra\'o
de versemblan\c{c}a ser\`a:
\[
R.C=\left\{\frac{L(\omega)}{L(\Omega)}<A\right\}=
\left\{\frac{
\max\limits_{\mu_X =\mu_Y,\sigma^2} L(\mu_X,\mu_Y,\sigma^2)}
{\max\limits_{\mu_X,\mu_Y,\sigma^2} L(\mu_X,\mu_Y,\sigma^2)}<A\right\}. 
\]
Fent servir els problemes \ref{ESTMOMMAXVERDISTINTMUDISTINTSIGMA} i
\ref{ESTMOMMAXVERMATEIXMUMATEIXSIGMA} podem trobar els valors de $L(\omega)$
i $L(\Omega)$:
\begin{eqnarray*}
L(\omega) & = & 
\max\limits_{\mu_X =\mu_Y,\sigma^2} L(\mu_X,\mu_Y,\sigma^2) \\ &= &
L\left(\tilde{\mu}_{\mbox{\footnotesize max}}:= 
\frac{\overline{X}+\overline{Y}}{2},\tilde{S}_{\mbox{\footnotesize max}}^2:=
\frac{1}{2n}\left(
\sum_{i=1}^n X_i^2 +\sum_{i=1}^n Y_i^2\right)-{\left(      
\frac{\overline{X}+\overline{Y}}{2}\right)}^2\right)\\ 
& = & 
\frac{1}{{\left(\frac{\pi}{n}\left(\sum\limits_{i=1}^n 
{(X_i-\tilde{\mu}_{\mbox{\footnotesize max}})}^2 +
\sum\limits_{i=1}^n {(Y_i -\tilde{\mu}_{\mbox{\footnotesize max}})}^2\right)
\right)}^n} \e^{-n}, \\
L(\Omega) & = & \max\limits_{\mu_X,\mu_Y,\sigma^2} L(\mu_X,\mu_Y,\sigma^2) 
\\ & = & L\left(\overline{X},\overline{Y},\frac{\sum\limits_{i=1}^n
{(X_i -\overline{X})}^2+\sum\limits_{i=1}^n {(Y_i -\overline{Y})}^2}{2n}
\right) \\ & = & 
\frac{1}{{\left(\frac{\pi}{n}\left(\sum\limits_{i=1}^n 
{(X_i-\overline{X})}^2 +
\sum\limits_{i=1}^n {(Y_i -\overline{Y})}^2\right)\right)}^n} \e^{-n}. 
\end{eqnarray*}
Per tant, el quocient $\frac{L(\omega)}{L(\Omega)}$ val:
\[
\frac{L(\omega)}{L(\Omega)}=
{\left(\frac{
\sum\limits_{i=1}^n {(X_i -\overline{X})}^2 + 
\sum\limits_{i=1}^n {(Y_i -\overline{Y})}^2} 
{
\sum\limits_{i=1}^n {(X_i -\tilde{\mu})}^2 + 
\sum\limits_{i=1}^n {(Y_i -\tilde{\mu})}^2} 
\right)}^n, 
\]
on recordem que $\tilde{\mu}=\frac{\overline{X}+\overline{Y}}{2}$.

Fent servir que 
\begin{eqnarray*}
\sum_{i=1}^n {(X_i -\tilde{\mu}_{\mbox{\footnotesize max}})}^2 & = & \sum_{i=1}^n {(X_i -\overline{X})}^2
+\frac{n}{4} {(\overline{X}-\overline{Y})}^2 \\
\sum_{i=1}^n {(Y_i -\tilde{\mu}_{\mbox{\footnotesize max}})}^2 & = & \sum_{i=1}^n {(Y_i -\overline{Y})}^2
+\frac{n}{4} {(\overline{X}-\overline{Y})}^2,
\end{eqnarray*}
podem escriure el quocient anterior de la manera seg\"uent:
\begin{equation}
\begin{array}{rl}
\frac{L(\omega)}{L(\Omega)}= & 
{\left(\frac{
\sum\limits_{i=1}^n {(X_i -\overline{X})}^2 + 
\sum\limits_{i=1}^n {(Y_i -\overline{Y})}^2} 
{
\sum\limits_{i=1}^n {(X_i -\overline{X})}^2 + 
\sum\limits_{i=1}^n {(Y_i -\overline{Y})}^2 + (n/2)\cdot
{(\overline{X}-\overline{Y})}^2} 
\right)}^n  \\ = &
{\left(\frac{1}{1+\frac{(n/2)\cdot
{(\overline{X}-\overline{Y})}^2
}{
\sum\limits_{i=1}^n {(X_i -\overline{X})}^2 + 
\sum\limits_{i=1}^n {(Y_i -\overline{Y})}^2 
}}\right)}^n.
\end{array}
\label{QUOCIENTFUNVER}
\end{equation}
A continuaci\'o, relacionarem la variable que surt en el denominador
de la darrera f\'ormula amb alguna distribuci\'o coneguda.
Per fer-ho, hem de tenir en compte unes quantes consideracions.
\begin{itemize}
\item[$\bullet$]
Fent servir que les distribucions de les variables $\overline{X}$ i
$\overline{Y}$ s\'on normals $N\left(\mu_X,\frac{\sigma^2}{n}\right)$ i 
$N\left(\mu_Y,\frac{\sigma^2}{n}\right)$, si la hip\`otesi nu{\lgem}a $H_0$ \'es
certa, podem dir que la variable $\overline{X}-\overline{Y}$ \'es normal
$N\left(0,\frac{2\sigma^2}{n}\right)$. Per tant, la variable aleat\`oria \mbox{$\frac{\sqrt{n}
(\overline{X}-\overline{Y})}{\sqrt{2}\sigma}$} es distribuir\`a segons
una normal $N(0,1)$.
\item[$\bullet$]
Tenint en compte que les variables 
\mbox{$\frac{\sum\limits_{i=1}^n 
{(X_i -\overline{X})}^2}{\sigma^2}$} i 
\mbox{$\frac{\sum\limits_{i=1}^n 
{(Y_i -\overline{Y})}^2}{\sigma^2}$} es distribueixen segons una distribuci\'o
$\chi^2_{n-1}$, podem assegurar que 
la variable
\[
\frac{\sum\limits_{i=1}^n {(X_i -\overline{X})}^2 +
\sum\limits_{i=1}^n {(Y_i -\overline{Y})}^2}{\sigma^2}
\]
\'es $\chi^2_{2n-2}$,
ja que fent servir el problema~\ref{SUMACHIQUADRAT} 
sabem que la suma de dues variables
independents amb distribucions $\chi^2_{n_1}$ i $\chi^2_{n_2}$ es 
distribueix segons una distribuci\'o $\chi^2_{n_1 +n_2}$.
\item[$\bullet$]
Fent servir les dues consideracions anteriors i tenint en compte 
que la variable $t_m$ ($t$ de Student amb $m$ graus de llibertat)
es defineix com $t_m =\frac{Z}{\sqrt{\frac{V}{m}}}$ on $Z$ \'es $N(0,1)$ i
$V$ \'es $\chi^2_m$, podem assegurar que la variable seg\"uent es distribueix
segons una variable $t$ de Student amb $2(n-1)$ graus de llibertat:
\[
t:= \frac{
\frac{
\sqrt{n}(\overline{X}-\overline{Y})}{\sqrt{2}}}
{\sqrt{
\frac{\sum\limits_{i=1}^n {(X_i -\overline{X})}^2 +
\sum\limits_{i=1}^n {(Y_i -\overline{Y})}^2}{2 (n-1)}}} 
= \frac{\sqrt{n} (\overline{X}-\overline{Y})}{S \sqrt{2}}.
\]
\end{itemize}
Fent servir les considerancions anteriors,
el quocient \ref{QUOCIENTFUNVER} es pot posar com: 
\[
\frac{L(\omega)}{L(\Omega)}=
{\left(
\frac{1}{1+\frac{t^2}{2(n-1)}}\right)}^n.
\]
La millor regi\'o cr\'{\i}tica ens quedar\`a, doncs:
\begin{eqnarray*}
R.C. & = & \left\{\frac{L(\omega)}{L(\Omega)} < A\right\} =
\left\{
\frac{1}{1+\frac{t^2}{2(n-1)}}< A^{\frac{1}{n}}\right\} \\
& = & 
\{ t < -C\}\cup \{t> C\},
\end{eqnarray*}
on $C=\sqrt{2 (n-1) \left(A^{-\frac{1}{n}} -1\right)}$.

El nostre objectiu \'es trobar $C$. Fent servir que la 
distribuci\'o de $t$ \'es $t_{2(n-1)}$ i que l'error tipus I 
\'es $\alpha$, podem escriure que:
\[
\pp{t< -C}=\frac{\alpha}{2},
\]
D'aqu\'{\i} dedu\"{\i}m que $C=- t_{2(n-1),\frac{\alpha}{2}}$ (percentil 
$100\frac{\alpha}{2}\%$ de la distribuci\'o $t$ de Student de 
$2(n-1)$ graus de llibertat.)

Concloem que la millor regi\'o cr\'{\i}tica per realitzar aquest 
contrast \'es:
\[
\{ t < t_{2(n-1),\frac{\alpha}{2}}\}\cup \{t> t_{2(n-1),1-\frac{\alpha}{2}}\}.
\]
}

\begin{probres}
{Aplicaci\'o pr\`actica de l'exercici \ref{CONTRASTDUESNORMALSVARIGUALS}. 
Un determinat grup
de $20$ alumnes rep l'ensenyament de lectura 
pel m\`etode tradicional durant un
any. La suma de les qualificacions d'aquest grup i la dels seus quadrats 
s\'on $\sum x_i =1950$, $\sum x_i^2 =192861$. A un segon grup de $20$ estudiants
se'ls ensenya a llegir per un nou m\`etode, diferent del tradicional. Les
qualificacions d'aquest segon grup al final d'any s\'on $\sum y_i =2020$, $\sum
y_i^2 =205920$. Suposant que les mostres s\'on independents i de distribucions
normals amb la mateixa vari\`ancia, feu el contrast
$$\left.
\begin{array}{ll}
H_0 :& \mu_X=\mu_Y,\\ H_1 :& \mu_X\not
=\mu_Y.
\end{array}
\right\}$$
Trobau l'$\alpha_{\mbox{\footnotesize max}}$ per poder acceptar $H_0$.}
\end{probres}

\res{Fent servir l'exercici \ref{CONTRASTDUESNORMALSVARIGUALS} tenim que la
millor regi\'o cr\'{\i}tica per poder realitzar el contrast \'es:
\[
\{ t < t_{m,\frac{\alpha}{2}}\}\cup \{t> - t_{m,\frac{\alpha}{2}}\},
\]
on $m=2 (n-1) =2( 20-1) =38$ i l'estad\'{\i}stic $t$ val en aquest cas
\[
t=\frac{\sqrt{n}\left(\overline{X}-\overline{Y}\right)}{S\sqrt{2}}.
\]
A continuaci\'o, trobam els valors de $\overline{X}$, $\overline{Y}$ i $S$.
\begin{eqnarray*}
\overline{X} & = & \frac{1950}{20}=97.5,\quad \overline{Y}=\frac{2020}{20}
= 101, \\
S^2 & = & \frac{\sum\limits_{i=1}^n {(X_i -\overline{X})}^2 
+\sum\limits_{i=1}^n {(Y_i -\overline{Y})}^2}{2 (n-1)} =
\frac{\sum\limits_{i=1}^n X_i^2 -n {\overline{X}}^2 +
\sum\limits_{i=1}^n Y_i^2 -n {\overline{Y}}^2}{2\cdot 19} \\
& = & \frac{192861-20\cdot {97.5}^2 +205920-20\cdot {101}^2}{38}= 
\frac{4636}{38}=122.
\end{eqnarray*}
El valor de l'estad\'{\i}stic $t$ ser\`a:
\[
t=\frac{\sqrt{n}\left(\overline{X}-\overline{Y}\right)}{S\sqrt{2}}=
\frac{\sqrt{20}\cdot (97.5 -101)}{\sqrt{122\cdot 2}}=-1.00204708.
\]
L'error tipus I m\`axim que es podr\`a cometre en acceptar $H_0$ en 
les condicions del problema ser\`a:
\[
\frac{\alpha_{\mbox{\footnotesize max}}}{2}=\pp{
t_{38}\leq -1.00204}= 1-F_{t_{38}}(1.00204)\approx 1-0.85=0.15.
\]
Concloem, doncs, que l'error tipus I m\`axim val aproximadament $0.3$ i, 
per tant, \'es coherent acceptar que els dos m\`etodes d'ensenyament
tenen la mateixa efic\`acia.
}

\section{Problemes proposats}

\begin{prob}
{Suposem que $X$ \'es un variable aleat\`oria de Poisson 
amb par\`ametre $\lambda$. Es pren 
una mostra aleat\`oria simple $X_1,X_2,\ldots,X_n$ de $X$ i 
es vol provar $H_0:\lambda=2$ en funci\'o
de $H_1:\lambda >2$. Quina pot ser una regi\'o cr\'{\i}tica raonable?}
\end{prob}

\begin{prob}
{Suposem que el nombre de grams que carrega una m\`aquina embotelladora
de refrescs \'es una variable aleat\`oria normal amb mitjana $\mu$ i vari\`ancia $\sigma^2$. Si
$\sigma^2$ \'es massa gran, s'omplen massa botelles i la m\`aquina es desborda. Una
$\sigma^2$ tolerable seria aquella que no fos m\'es gran que $7$.
Suposem que es pren una mostra aleat\`oria simple de 20 botelles omplides per la m\`aquina i es vol
provar $H_0: \sigma^2=7$ en funci\'o de $H_1: \sigma^2 >7$. Es
decideix rebutjar $H_0$ si $$\sum_{i=1}^{20}{(X_i -\overline{X})}^2>213.64.$$
Quina \'es l'error tipus I en aquest contrast?}
\end{prob}

\begin{prob}
{Suposem que el temps de falla (en hores) d'un determinat tipus de
bombeta \'es una variable aleat\`oria exponencial de par\`ametre $\lambda$. Es fa una prova en $10$
bombetes i es vol provar la hip\`otesi $H_0:\lambda=0.001$ en funci\'o de
$H_1:\lambda<0.001$. Es decideix acceptar fins que falli la primera bombeta i
es rebutja $H_0$ si el temps de falla no \'es m\'es gran que $5.13$ hores. 
Provau que l'error tipus I per a aquest contrast \'es de $0.05$.}
\end{prob}

\begin{prob}
{Donada una variable aleat\`oria $X$ uniforme en l'interval $(0,\theta)$, es vol
provar $H_0:\theta =1$ en contra de $H_1:\theta =2$. Es pren una mostra aleat\`oria simple de 2
observacions de $X$ i es rebutja $H_0$ si $\overline{X}>0.99$. Calculau
$\alpha$ i $\beta$ per a aquest contrast.}
\end{prob}

\begin{prob}
{Suposem que la probabilitat que surti cara en una sola tirada d'una
moneda \'es $p$. Trobau la millor regi\'o cr\'{\i}tica (la que t\'e l'error tipus II m\'es
petit) per provar $H_0:p={1\over 2}$ contra $H_1: p={3\over 4}$, donada una
mostra aleat\`oria simple de $n$ tirades de la moneda.}
\end{prob}

\begin{prob}
{Suposem que $X$ \'es una variable aleat\`oria normal amb mitjana $\mu$ desconeguda i
vari\`ancia $\sigma^2$ coneguda. Trobau pel criteri de la ra\'o de versemblan\c{c}a,
la regi\'o cr\'{\i}tica dels contrasts seg\"uents:
$$\mbox{a) }\left.
\begin{array}{ll}
H_0:&  \mu =\mu_0,\\ H_1 :&  \mu\not
=\mu_0.
\end{array}
\right\}\quad\mbox{b) }\left.
\begin{array}{ll}
H_0 :&   \mu =\mu_0,\\ H_1 :& 
\mu >\mu_0.
\end{array}
\right\}
\quad\quad\mbox{c) }\left.
\begin{array}{ll}
H_0 :&   \mu =\mu_0,\\
H_1 :&  \mu <\mu_0.
\end{array}
\right\}$$}
\end{prob}

\begin{prob}
{Per decidir si un determinat tipus de planta \'es apropiat per formar
jardins, \'es important que cada planta tengui 
poca variabilitat en el creixement
per any (suposades les plantes de la mateixa edat). Suposem que els creixements
d'aquest determinat tipus de planta \'es una variable 
aleat\`oria normal amb mitjana $\mu$ i
vari\`ancia~$\sigma^2$. Aleshores, es decideix fer 
el seg\"uent contrast per veure si
la planta \'es adequada:
$$\left.
\begin{array}{ll}
H_0 :& \sigma^2 =0.08,\\ H_1 :& \sigma^2 <0.08.
\end{array}
\right\}
\mbox{ (amidament en metres).}$$
S'enregistra el creixement de $5$ plantes durant 1 any i aquests valen: 
$$0.58,\ 0.34,\ 0.82,\ 0.49\hbox{ i } 0.61\hbox{ metres.}$$
Hem d'acceptar $H_0$? O sigui, quin \'es l'error tipus I m\`axim per
poder acceptar $H_0$?}
\end{prob}

\begin{prob}
{Se sap que el temps en qu\`e pot fallar un determinat aparell
electr\`onic \'es una variable aleat\`oria exponencial 
amb par\`ametre $\lambda$. Es posen a prova
$1000$ aparells d'aquest tipus i es troba que la 
suma dels temps de falla \'es de
$109652$ hores. Quin \'es l'error tipus I m\`axim 
per poder acceptar $H_0:\lambda \leq 0.008$ en
contra de $H_1:\lambda >0.008$, donada aquella mostra aleat\`oria simple?} 
\end{prob}

\begin{prob}
{El nombre de
vegades que es pot encendre i apagar una interruptor \'es una variable aleat\`oria geom\`etrica $X$
amb par\`ametre $p$ (aleshores $X$ \'es el nombre de vegades que encenem 
l'interruptor fins
que falla). Donada una mostra aleat\`oria simple de grand\`aria~$10$, s'ha trobat que la suma de les
vegades que s'enc\'en i s'apaga l'interruptor ha estat de $15169$. 
Quin \'es l'error tipus I m\`axim per poder acceptar
$H_0:p=0.00005$ contra $H_1:p>0.00005$? {\footnotesize (Feu servir
el teorema l\'{\i}mit per a la prova de ra\'o de versemblan\c{c}a.)}}
\end{prob}

\begin{prob}
{Una empresa aliment\`aria indica en un paquet d'arr\`os que el seu pes mitj\`a
\'es de $900$ gr. En una inspecci\'o s'agafen $9$ paquets, que pesen
$$894,\ 890,\ 900,\ 892,\ 895,\ 896,\ 894,\ 903\hbox{ i }899.$$Est\`a d'acord la
mostra amb el pes indicat en el paquet? (Trobau l'error tipus I
m\`axim per poder acceptar que el pes mitj\`a del paquet \'es de $900$ gr.)
\newline{\footnotesize Final. Juny 91.}}
\end{prob}

\begin{prob}
{Sigui $X$ una variable aleat\`oria $N(\mu,\sigma^2)$. Sigui $X_1,X_2,\ldots,X_n$ una mostra aleat\`oria simple
de $X$. Considerem el contrast
$$\left.
\begin{array}{ll}
H_0 :& \mu=0,\\ H_1:&\mu\not =0.
\end{array}
\right\}$$
\begin{itemize}
\item[a)]{Trobau, raonant tots els passos, la millor regi\'o 
cr\'{\i}tica en el cas
que $\sigma=1$ coneguda i en el cas que $\sigma$ sigui desconeguda.}
\item[b)]{Aplicau l'apartat anterior a la mostra seg\"uent\newline
$1.12,\ -0.19,\ -0.32,\ -0.6,\ -1.1,\ 0.14,\ 0.19,\ 0.23,\ -1.12,\
-1.4.$}
\end{itemize}

{\footnotesize Final. Setembre 93.}}
\end{prob}

\newpage

\begin{prob}
{Sigui $X$ una variable aleat\`oria exponencial de par\`ametre
$\lambda$. Volem fer el contrast seg\"uent: 
$$\left.
\begin{array}{ll}
H_0 :& 
\lambda=\lambda_0,\\ H_1 :&  \lambda=\lambda_1.
\end{array}
\right\}$$
\begin{itemize}
\item[a)] {Digau quina \'es la millor regi\'o cr\'{\i}tica a 
un nivell de significaci\'o
$\alpha$. Digau tamb\'e com trobar-la.}
\item[b)] {Sigui $X$ una variable aleat\`oria exponencial 
de par\`ametre $\lambda$.
Feu el contrast:
$$\left.
\begin{array}{ll}
H_0 :&  \lambda=1,\\ H_1  :& 
\lambda=2,
\end{array}
\right\}$$
amb la seg\"uent mostra aleat\`oria simple de $X$:
$$0.735,\ 0.949,\ 0.739,\ 0.535,\ 0.910,\ 0.989,\ 0.851,\
0.786,\ 0.904,\ 0.932,$$
aplicant l'apartat a). Trobau l'error tipus I m\`axim per poder acceptar
$H_0$.
}
\end{itemize}

{\footnotesize Final. Juny 93.}}
\end{prob}

\begin{prob}
{Sigui la mostra aleat\`oria simple $1,2,3,4,5,6,7,8,9,10$ d'una
variable aleat\`oria $N(\mu,\sigma^2)$. Feim el contrast:
$$\left.
\begin{array}{ll}
H_0 :& \sigma =3, \\ H_1 :&  \sigma >3.
\end{array}
\right\}$$
Digau quina condici\'o \'es v\`alida per a $\alpha$ (error tipus
I) per tal de rebutjar $H_0$.
\newline{\footnotesize Final. Setembre 94.}}
\end{prob}

\begin{prob}
{
Sigui $X$ una variable aleat\`oria $Poiss(\lambda)$. Sigui 
\mbox{$X_1,\ldots,X_n$} una mostra aleat\`oria simple de $X$. Fent servir
el criteri de ra\'o de m\`axima versemblan\c{c}a, trobau la regi\'o cr\'{\i}tica
\`optima per fer el contrast:
$$
\left.
\begin{array}{ll}
H_0 :& \lambda =  \lambda_0,\\
H_1 :& \lambda >  \lambda_0.
\end{array}
\right\}
$$
Aplicaci\'o pr\`actica. Suposem que la mostra \'es:
$$
1,\ 2,\ 1.5,\ 1.7,\ 1.6,\ 0.9,\ 2,
$$
i $\lambda_0 =1$, digau a partir de quin $\alpha$ (error tipus I) 
acceptariem $H_0$.
\newline{\footnotesize Examen extraordinari de febrer 95.}
}
\end{prob}

\newpage

\begin{prob}
{
Sigui $X$ una variable aleat\`oria de Bernoulli amb par\`ametre $p$.
Prenem una mostra aleat\`oria simple de $n$ observacions \mbox{$X_1,
\ldots,X_n$}. Volem fer el contrast:
\[
\left.
\begin{array}{ll}
H_0 :&  p=p_0, \\
H_1 :&  p=p_1.
\end{array}
\right\}
\]
Rebutjam $H_0$ si $\overline{X}\leq\frac{p_0}{2}$. 
Trobau els valors de $\alpha$
(error tipus I) i $\beta$ (error tipus II).
\iffalse
($B(n,p)$ \'es la distribuci\'o binomial de par\`ametres $n$ i $p$ i $[x]$
\'es la part entera de $x$.)
\fi
\newline{\footnotesize Segon parcial. Juny 95.}
}
\end{prob}

\begin{prob}
{
Considerem el contrast d'hip\`otesi seg\"uent:
\[
\left.
\begin{array}{ll}
H_0 : & \sigma =\sigma_0, \\
H_1 : & \sigma\not =\sigma_0.
\end{array}
\right\}
\]
on $\sigma$ correspon a una variable aleat\`oria $N(\mu,\sigma^2)$ amb
$\mu$ desconeguda.

Suposant $\sigma_0=1$, $\tilde{S}=1.2$ i que tenim una mostra aleat\`oria
simple de $20$ observacions, trobau l'error $\alpha_{\mbox{\footnotesize max}}$
per poder acceptar $H_0$.
\newline{\footnotesize Final. Setembre 95.}
}
\end{prob}

\begin{prob}
{
Considerem les dues mostres seg\"uents de dues normals independents
de grand\`aria~$20$:
\begin{center}
\begin{tabular}{lllllll}
Mostra 1& 0.53009&2.65819&3.26338&2.54875&1.72988&2.76075 \\ &1.58425&
1.36660&0.56713&2.72981&3.34947&1.55879\\ &2.03271&1.70223&5.15895&
3.20326&3.11516&2.65967\\&3.34316&4.55594&&&&\\
Mostra 2&2.38721&1.48624&3.76097&0.53768&2.64521&0.54878\\&2.84931&
0.81125&3.07246&1.67631&1.05417&1.71277\\ &2.31605&1.90741&1.98674&
1.94790&0.84796&1.81512\\&2.99567&1.83909&&&&\\\hline
\end{tabular}
\end{center}
Trobau l'error tipus~I m\`axim ($\alpha_{\mbox{\footnotesize m\`ax}}$) per
poder acceptar~$H_0$ en el contrast seg\"uent:
\[
\left.
\begin{array}{ll}
H_0:& \mu_1  = \mu_2, \\
H_1:& \mu_1  \not =  \mu_2.
\end{array}
\right\}
\]
\noindent Indicaci\'o: heu de realitzar el contrast previ d'igualtat
de vari\`ancies.
\newline{\footnotesize Final. Febrer 96.}
}
\end{prob}

\begin{prob}
{
Sigui $X_1,\ldots,X_{40}$ una mostra aleat\`oria simple d'una variable
aleat\`oria~$N(\mu,\sigma^2)$ tal que $\sum X_i =100$ i $\sum X_i^2 =1000$.
Volem fer el contrast seg\"uent:
\[
\left.
\begin{array}{ll}
H_0:&\mu =3,\\
H_1:&\mu\not = 3.
\end{array}
\right\}
\]
Trobau l'error tipus~I m\`axim $\alpha_{\max}$ per poder acceptar la hip\`otesi
nu{\lgem}a~$H_0$.
\newline{\footnotesize Final. Juny 96.}
}
\end{prob}

\begin{prob}
{
Ens donen dues mostres de dues variables aleat\`ories normals
independents:

\noindent Mostra 1: 5,6,7,4,8,9,6,5,7,6.

\noindent Mostra 2: 2,3,4,3,4,4,3,1,5,4.

Trobau l'error tipus~I m\`axim $\alpha_{\max}$ per poder acceptar 
igualtat de vari\`ancies.
\newline{\footnotesize Final. Setembre 96.}
}
\end{prob}

