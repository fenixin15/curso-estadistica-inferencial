\chapter{Proves de la bondat d'ajustament i d'independ\`encia}
\index{prova!de la bondat d'ajustament}
\index{prova!d'independencia@d'independ\`encia}

\section{Resum te\`oric}

Els contrasts d'hip\`otesi vists en el tema anterior tractaven 
\index{contrast!d'hipotesi@d'hip\`otesi}de confrontar
afirmacions respecte de par\`ametres
\index{parametre@par\`ametre} desconeguts de les distribucions de 
poblacions d'inter\`es. Ara veurem els contrasts d'hip\`otesi on, 
en lloc de par\`ametres, el que desconeixem \'es qualque propietat de la 
forma de la funci\'o de distribuci\'o que se mostreja.
\index{funcio@funci\'o!de distribucio@de distribuci\'o} 
Endem\'es tamb\'e discutirem les proves d'independ\`encia 
\index{prova!d'independencia@d'independ\`encia!de dues variables}
de dues variables aleat\`ories en les quals l'evid\`encia mostral 
s'obt\'e mitjan\c cant la classificaci\'o de cada variable aleat\`oria 
en un cert nombre de categories.
\index{classificacio@classificaci\'o}\index{categoria}

Un m\`etode per provar les hip\`otesis referents a distribucions 
\'es la prova de la
bondat d'ajustament~$\chi^2$. Es basa en el resultat seg\"uent:
\index{prova!de bondat d'ajustament}

\begin{proposition}
Si $(X_1, \ldots , X_k)$ \'es un vector aleatori multinomial 
\index{vector aleatori!multinomial}amb par\`ametres $n,
p_1, \ldots , p_k$,\index{parametre@par\`ametre} aleshores la 
funci\'o de distribuci\'o de la variable aleat\`oria
$$U = \sum_{i=1}^k {(X_i - n p_i)^2 \over n p_i}$$
s'aproxima a la funci\'o de distribuci\'o $\chi^2$ amb $k-1$ graus 
de llibertat quan $n \to \infty$.
\end{proposition}

Aix\'{\i}, si volem provar la hip\`otesi que $(X_1, \ldots , X_k)$ 
\'es un vector aleatori multinomial amb par\`ametres especificats 
$n, p_1, \ldots , p_k$, aleshores es pren una mostra, es calcula
$$u = \sum_{i=1}^k {(x_i - n p_i)^2 \over n p_i},$$
i es rebutja la hip\`otesi si $u > \chi_{1-\alpha}^2$, 
el $100(1-\alpha)$-\`essim percentil de la distribuci\'o $\chi^2$ amb $k-1$ 
graus de llibertat.

Si la hip\`otesi \'es certa, aleshores $\alpha$ \'es 
la probabilitat de rebutjar-la, \'es a dir, \'es la 
probabilitat de l'error de tipus I.
\index{probabilitat!d'error tipus I}

Es pren com a cert que l'aproximaci\'o $\chi^2$ \'es bona per a $u$ 
sempre que \break $n p_i \geq 5, \ \forall i = 1, \ldots , k$. 
Si no \'es aix\'{\i}, es poden combinar algunes de
les classes perqu\`e se satisfaci aquest requisit.

Podem ara aplicar la prova multinomial 
\index{prova!multinomial}per provar la hip\`otesi que la funci\'o de
distribuci\'o $F_Y$ d'una variable aleat\`oria t\'e qualsevol 
forma especificada.

\begin{proposition}
Suposem que $Y$ t\'e una funci\'o de distribuci\'o $F_Y$, i que $Y_1, 
\ldots , Y_n$ \'es una mostra aleat\`oria simple de $Y$. Definim 
\begin{eqnarray*}
I_1 & = & \{ y : y \leq a_1 \}, \\
I_i & = & \{ y : a_{i-1} < y \leq a_i \}, \ i=2,\ldots,k-1,\\
I_k & = &  \{ y : a_{k-1} < y \},
\end{eqnarray*}
per a uns certs $a_1, \ldots , a_{k-1}$.
Sigui $X_i$ la variable aleat\`oria que d\'ona el nombre de valors 
mostrals que cauen dins $I_i, \ \forall i = 1, \ldots , k$.

Aleshores $(X_1, \ldots , X_k)$ \'es un vector aleatori multinomial 
\index{vector aleatori!multinomial}amb par\`ametres\break
\index{parametre@par\`ametre}$n, p_1, \ldots , p_k$, on $n$ \'es la 
grand\`aria de la mostra (de $Y$) i $p_i = \pp{Y \>
{\rm cau \> en \> } I_i},$\break $i = 1, \ldots , k$.
\end{proposition}

Segons aquest resultat, es pot provar que $Y$ t\'e funci\'o 
de distribuci\'o $F_Y$ provant la hip\`otesi que $(X_1, \ldots , X_k)$ 
t\'e els par\`ametres $n, p_1, \ldots , p_k$ sense importar la forma de $F_Y$.

Si es rebutja aquesta hip\`otesi, aleshores aparentment $F_Y$ 
no \'es apropiada comparada amb la mostra. Si, en canvi, s'accepta la 
hip\`otesi, aleshores pareix que $F_Y$ \'es consistent amb els valors de la mostra.

Quan la variable aleat\`oria \'es discreta, 
els intervals $I_i$ es determinen segons
els valors discrets que pot prendre la variable. En el cas continu, 
no hi ha cap regla general, nom\'es s'ha de tenir en compte que el 
nombre d'intervals ha de ser tan gran com sigui
possible, sempre verificant que $n p_i \geq 5 \ \forall i = 1, \ldots , k$.

Sovint interessa provar la hip\`otesi que la funci\'o de distribuci\'o 
\index{funcio@funci\'o!de distribucio@de distribuci\'o}t\'e una forma 
determinada sense haver d'especificar tamb\'e els valors dels 
par\`ametres de la funci\'o de distribuci\'o. Per exemple, 
podem voler provar la hip\`otesi que una variable aleat\`oria~$Y$ 
segueix una distribuci\'o de Poisson sense dir que el
\index{distribucio@distribuci\'o!de Poisson}
par\`ametre val, posem, $\lambda = 0.6$. En aquests casos nom\'es cal 
fer una petita adaptaci\'o del m\`etode. El seg\"uent resultat d\'ona la metodologia.

\begin{proposition}
Suposem que $Y$ t\'e una funci\'o de distribuci\'o 
\index{funcio@funci\'o!de distribucio@de distribuci\'o}
$F_Y$ que satisf\`a certes condicions de regularitat, 
\index{condicio de regularitat@condici\'o de regularitat}amb $r$ par\`ametres 
\index{parametre@par\`ametre}desconeguts $\gamma_1, \ldots , \gamma_r$.
Sigui $Y_1, \ldots , Y_n$ una mostra aleat\`oria simple de $Y$, i 
$\hat{\Gamma}_1,\ldots , \hat{\Gamma}_r$ els estimadors 
\index{estimador!de maxima versemblanca@de m\`axima versemblan\c{c}a}
de m\`axima versemblan\c{c}a per a $\gamma_1,
\ldots , \gamma_r$, respectivament. Definim $I_1, \ldots , I_k$ i $X_1, \ldots ,
X_k$ com a la proposici\'o anterior. 
Aleshores, si definim $\hat{P}_i = \pp{Y \> {\rm
cau \> en \> } I_i}, \ i = 1, \ldots , k$, on es reemplacen $\gamma_1, \ldots ,
\gamma_r$ en $F_Y$ per $\hat{\Gamma}_1, \ldots , \hat{\Gamma}_r$, la distribuci\'o
de l'estad\'{\i}stica\index{estadistica@estad\'{\i}stica}
$$V= \sum_{i=1}^k {(X_i - n \hat{P}_i)^2 \over n \hat{P}_i}$$
tendeix a la distribuci\'o d'una $\chi^2$ amb $k-r-1$ graus de llibertat quan $n
\to \infty$.
\end{proposition}

Per tant, si volem provar que $F_Y$ t\'e una forma determinada sense 
indicar res respecte dels seus par\`ametres,
\index{parametre@par\`ametre} basta utilitzar les dades per
estimar-los i llavors procedir com abans. \'Es a dir, es calcula el valor
observat de $V$ i si \'es major que el $100(1-\alpha)$-\`essim percentil de la
distribuci\'o $\chi^2$ amb $k-r-1$ graus de llibertat, es rebutja $H_0$. 
Observem que hem perdut un grau de llibertat per cada par\`ametre que 
s'ha hagut d'estimar.

Vegem finalment les proves d'independ\`encia.
\index{prova!d'independencia@d'independ\`encia} 
Sovint els elements d'una mostra es classifiquen segons dos o m\'es 
criteris i llavors es vol saber si els m\`etodes de
\index{criteri} classificaci\'o s\'on independents. 
\index{metode@m\`etode!de classificacio@de classificaci\'o}
Considerem el cas de dos m\`etodes de
classificaci\'o.

Suposem que tenim una mostra de grand\`aria $n$ i dues maneres de 
classificar els elements de la mostra. Suposem que el primer m\`etode de 
classificaci\'o t\'e $r$ nivells (o classes) i que el segon en t\'e $c$.
\index{nivell}\index{classe}

Sigui $x_{ij}$ el nombre observat d'elements de la mostra que cauen en el
$i$-\`essim nivell de classificaci\'o 1 i en el $j$-\`essim nivell de classificaci\'o 2,
\index{nivell!de classificacio@de classificaci\'o}
amb $i = 1, \ldots , r$ i $ j = 1, \ldots , c$. Anomenarem 
{\bf taula de conting\`encia}\index{taula de contingencia@taula de conting\`encia}
una representaci\'o tabular dels $x_{ij}$. 
\index{representacio tabular@representaci\'o tabular}Volem determinar si els dos m\`etodes de
classificaci\'o s\'on independents.
\index{metode@m\`etode!de classificacio@de classificaci\'o!independent}

La prova d'independ\`encia\index{prova!d'independencia@d'independ\`encia} 
en una taula de conting\`encia\index{taula de contingencia@taula de conting\`encia} 
\'es simplement un cas particular del resultat donat en la proposici\'o anterior.

Suposem que tenim una poblaci\'o infinita,
\index{poblacio@poblaci\'o!infinita} amb cada un dels seus elements en
exactament un nivell de classificaci\'o~$1$ i exactament un 
\index{nivell!de classificacio@de classificaci\'o}
nivell de classificaci\'o~$2$, essent $r$ i $c$ els nombres totals de nivells de 
classificaci\'o $1$ i $2$, respectivament. Sigui $p_{ij}$ la probabilitat que un 
element seleccionat a l'atzar\index{atzar} caigui en el $i$-\`essim nivell de 
classificaci\'o~$1$ i en el $j$-\`essim nivell
de classificaci\'o~$2$. Aleshores $\displaystyle \sum_i \sum_j p_{ij} = 1$.

Elegim una mostra\index{mostra} a l'atzar\index{atzar} de $n$ elements i definim $X_{ij} =$ nombre
d'elements de la mostra en el $i$-\`essim nivell de classificaci\'o 1 i en el
\index{nivell!de classificacio@de classificaci\'o}
$j$-\`essim nivell de classificaci\'o 2, amb $i = 1, \ldots , r$ i $ j = 1, \ldots ,
c$.

Aleshores, el vector aleatori\index{vector aleatori} 
format per les $r \cdot c$ variables aleat\`ories
$X_{ij}$ \'es multinomial\index{multinomial} amb par\`ametres 
\index{parametre@par\`ametre}$n, p_{ij} \ (i = 1, \ldots , r, \ j = 1,
\ldots , c$). Endem\'es, si els dos m\`etodes de classificaci\'o s\'on independents,
\index{metode@m\`etode!de classificacio@de classificaci\'o!independent}
$p_{ij} = w_i \cdot s_j, \ i = 1, \ldots , r, \ j = 1, \ldots , c$, on $w_i$
(resp. $s_j$) \'es la probabilitat que un element seleccionat a l'atzar caigui
en la classe~$i$ (resp. $j$).\index{classe}

Aleshores, si suposam independ\`encia dels dos m\`etodes de classificaci\'o,
\index{independencia@independ\`encia} l'estimador
de m\`axima versemblan\c ca de $w_i$ \'es
\index{estimador!de maxima versemblanca@de m\`axima versemblan\c{c}a}
$$\hat{W}_i = {X_{i\bullet} \over n} = {1 \over n} \sum_j X_{ij}, \ i = 1, \ldots ,
r,$$
ja que $(X_{1\bullet}, \ldots , X_{r\bullet})$ seria multinomial 
\index{multinomial}amb par\`ametres $n, w_1,
\ldots , w_r$.\index{parametre@par\`ametre} (Hem usat la notaci\'o $\displaystyle X_{i\bullet} = \sum_j X_{ij}$.)

An\`alogament, l'estimador de m\`axima versemblan\c{c}a de $s_j$ ser\`a
\index{estimador!de maxima versemblanca@de m\`axima versemblan\c{c}a}
$$\hat{S}_j = {X_{\bullet j} \over n} = {1 \over n} \sum_i X_{ij}, \ j = 1, \ldots , c.$$
(Hem usat la notaci\'o $\displaystyle X_{\bullet j} = \sum_i X_{ij}$.)

Aleshores,
$$V = \sum_i \sum_j {(X_{ij} - n \hat{W}_i \hat{S}_j)^2 \over n \hat{W}_i
\hat{S}_j}$$
\'es aproximadament una variable aleat\`oria $\chi^2$ amb 
$r c-(c-1)-(r-1)-1=(r-1)(c-1)$ graus de
llibertat per a $n$ gran.

Aquesta estad\'{\i}stica\index{estadistica@estad\'{\i}stica} s'utilitza de manera id\`entica a com s'ha comentat per a les
proves de la bondat d'ajustament.
\index{prova!de bondat d'ajust}

Aquestes taules de conting\`encia\index{taula de contingencia@taula de conting\`encia} 
es poden generalitzar a tres o m\'es nivells de
classificaci\'o
\index{nivell!de classificacio@de classificaci\'o}i el criteri 
\index{criteri}aplicat \'es b\`asicament el mateix.

\section{Problemes resolts}

\begin{probres}
{Un atleta universitari va fer $100$ llan\c{c}aments durant una setmana de
pr\`actica. En la seg\"uent taula surten les dist\`ancies a les quals va efectuar els
llan\c{c}aments (mesurades en metres)
\begin{center}
\begin{tabular}{|c|c|}
\hline
Dist\`ancia $y$& Freq\"u\`encia $x$ \\
\hline\hline
$y\leq 18.6$&$12$ \\
\hline
$18.6< y\leq
19.2$& $20$\\ \hline
$19.2< y\leq 19.8$&$40$
\\\hline
$19.8< y\leq 20.4$&$25$\\\hline
$y>20.4$&$\ 3$\\\hline
\end{tabular}
\end{center}

Trobau l'error tipus I m\`axim que es pot cometre en acceptar 
la hip\`otesi que
la dist\`ancia $Y$ a qu\`e pot efectuar un llan\c{c}ament 
\'es una variable aleat\`oria normal amb $\mu
= 19.2$ metres i $\sigma = 0.61$ metres.}
\end{probres}

\res{
Apliquem el test de la $\chi^2$ per resoldre el problema.

L'estad\'{\i}stic a trobar \'es:
\[
\chi^2 =\sum  \frac{{(Y_i -n p_i)}^2}{n p_i},
\]
on $Y_i$ s\'on les freq\"u\`encies emp\'{\i}riques i $n p_i$ les
freq\"u\`encies te\`oriques. $n$ representa la grand\`aria de la mostra; 
en el nostre cas, $n=100$.

A continuaci\'o, constru\"{\i}m la taula per trobar l'estad\'{\i}stic $\chi^2$:
\begin{center}
\renewcommand{\arraystretch}{1.5}
\begin{tabular}{|r@{,}l|c|c|c|c|c|}
\hline
\multicolumn{2}{|c|}{Intervals $I_i$}
&$Y_i$&$p_i$&$n p_i$&$|Y_i -n p_i|$&
$\frac{{(Y_i -n p_i)}^2}{n p_i}$\\\hline\hline
$(-\infty$&$18.6]$&$12$&$0.1635$&$16.35$&$\ 4.35$&$1.1573$\\\hline
$(18.6$&$19.2]$&$20$&$0.3365$&$33.65$&$13.65$&$5.5370$\\\hline
$(19.2$&$19.8]$&$40$&$0.3365$&$33.65$&$\ 6.35$&$1.1982$\\\hline
$(19.8$&$20.4]$&$25$&$0.1385$&$13.85$&$11.15$&\multirow{2}{1cm}
{$8.3011$}\\\cline{1-6}
$(20.4$&$\infty)$&$\ 3$&$0.0250$&$\ 2.50$&$\ 0.5$&
\\\hline\hline
\multicolumn{2}{|c|}{}&&&
&&$\chi^2 =16.1937$\\\hline
\end{tabular}
\end{center}
Fixau-vos que hem hagut de juntar els dos \'ultims intervals ja que
$n p_i =2.5$ \'es m\'es petit que $5$ en el darrer interval.

El c\`alcul de les probabilitats te\`oriques
 $p_i$ \'es el seg\"uent:
\begin{eqnarray*}
p_1 & = & \pp{X\leq 18.6}  = 
\pp{ Z=N(0,1)\leq\frac{18.6 -19.2}{0.61}}
\\ &=&P\{Z\leq -0.9836\} =0.1635, \\
p_2 &=&\pp{18.6\leq X\leq 19.2}  = \pp{ -0.9836\leq Z=N(0,1)\leq 0}
\\ &=&0.5 -
0.1635 =0.3365, \\
p_3 &=&\pp{19.2\leq X\leq 19.8}  = \pp{ 0\leq Z=N(0,1)\leq 0.9836}\\ &=&0.8365-
0 =0.3365, \\
p_4 & =&\pp{19.8\leq X\leq 20.4}  = \pp{ 0.9836\leq Z=N(0,1)\leq 1.9672}
\\ & =&0.975-
0.8365 =0.1385, \\
p_5 & = & \pp{X\geq 20.4}  = 1-\pp{ Z=N(0,1)\leq 1.9672}\\& =&1-0.975=0.025.
\end{eqnarray*}
En aquest cas la distribuci\'o de l'estad\'{\i}stic
 
$\chi^2$ s'aproxima a la distribuci\'o
$\chi^2$ amb $3$ graus de llibertat ja que el nombre 
d'intervals considerats
en la partici\'o \'es $4$ (recordem que hem juntat els dos darrers intervals).

La f\'ormula dels graus de llibertat \'es:

g.ll. = nombre d'intervals - par\`ametres estimats - 1.

La regi\'o cr\'{\i}tica per fer el contrast \'es:
\[
R.C.=\{ \chi^2 \geq \chi_{1-\alpha}^2\},
\]
on $\alpha$ \'es l'error tipus I.

L'error tipus I $\alpha_{\mbox{\footnotesize max}}$ m\`axim,
per sota del qual podem 
acceptar que la distribuci\'o $X$ \'es una normal $N(\mu=19.2,
\sigma =0.61)$ val:
\[
\alpha_{\mbox{\footnotesize max}} =\pp{\chi_3^2\geq 16.1937}\leq 
1-0.995 =0.005
\]
Com que l'error anterior $\alpha_{\mbox{\footnotesize max}}$ \'es molt petit,
rebutjam normalitat.
}

\newpage

\begin{probres}
{El nombre de cotxes que arriben a un supermercat \'es un proc\'es de
Poisson. Per tant, el temps entre les arribades successives 
\'es una variable aleat\`oria
exponencial. S'enregistraren les hores d'arribada per a tots els cotxes durant
dues hores i els temps entre arribades (en minuts) s\'on els que mostra la taula
adjunta.
\begin{center}
\begin{tabular}{|c|c|}
\hline
Temps $t$ entre arribades & Freq\"u\`encia \\\hline\hline
$t\leq 1$&$40$\\\hline
$1<t\leq 2$&$29$\\\hline
$2<t\leq
3$&$15$\\\hline
$t>3$&$\ 8$\\\hline
Total&$92$\\\hline
\end{tabular}
\end{center}

Provau la hip\`otesi que aquesta distribuci\'o de
temps \'es consistent amb una distribuci\'o exponencial.}
\end{probres} 

\res{
El primer que hem de fer \'es estimar el par\`ametre $\lambda$ del qual
dep\`en la distribuci\'o exponencial.

El millor estimador de $\lambda$ \'es $\frac{1}{\overline{X}}$. Trobem, 
doncs, $\overline{X}$:
\[
\overline{X}=\frac{0.5\cdot 40+1.5\cdot 29+2.5\cdot 15+3.5\cdot 8}{92}
\approx 1.4022.
\]
L'estimada de $\lambda$ valdr\`a: $\tilde{\lambda}=\frac{1}{\overline{X}}
\approx 0.7132$.

A continuaci\'o constru\"{\i}m la taula per trobar l'estad\'{\i}stic $\chi^2$:
\begin{center}
\renewcommand{\arraystretch}{1.5}
\begin{tabular}{|r@{,}l|c|c|c|c|c|}
\hline
\multicolumn{2}{|c|}{Intervals $I_i$}
&$Y_i$&$p_i$&$n p_i$&$|Y_i -n p_i|$&
$\frac{{(Y_i -n p_i)}^2}{n p_i}$\\\hline\hline
$(-\infty$&$1]$&$40$&$0.5099$&$46.9122$&$6.9122$&$1.0184$\\\hline
$(1$&$2]$&$29$&$0.2499$&$22.9909$&$6.0090$&$1.5705$\\\hline
$(2$&$3]$&$15$&$0.1225$&$11.2675$&$3.7324$&$1.2364$\\\hline
$(3$&$\infty)$&$\ 8$&$0.1177$&$10.8292$&$2.8292$&$0.7391$\\\hline\hline
\multicolumn{2}{|c|}{}&&&&&$\chi^2 =4.5646$\\\hline
\end{tabular}
\end{center}
Amb vista a donar els c\`alculs de les probabilitats te\`oriques $p_i$, 
recordem que la funci\'o de distribuci\'o d'una variable aleat\`oria 
$\mbox{Exp }(\lambda)$ \'es: $F_X (t)=1-e^{-\lambda t}$.

\begin{eqnarray*}
p_1 & = & \pp{X\leq 1}=1 - e^{-\tilde{\lambda}}\approx 0.5099, \\
p_2 & = & \pp{1\leq X\leq 2}=e^{-\tilde{\lambda}}-e^{-2\tilde{\lambda}}
\approx 0.2499, \\
p_3 & = & \pp{2\leq X\leq 3}=e^{-2 \tilde{\lambda}}-e^{-3\tilde{\lambda}}
\approx 0.1225, \\
p_4 & = & \pp{X\geq 3}=e^{-3 \tilde{\lambda}}\approx 0.1177.
\end{eqnarray*}
Tenint en compte que la distribuci\'o de l'estad\'{\i}stic $\chi^2$ 
s'aproxima a una distribuci\'o $\chi^2$ amb $2$ 
graus de llibertat (g.ll. $= 4-1-1$) i que la regi\'o cr\'{\i}tica \'es
\[
R.C.=\{ \chi^2 \geq \chi_{1-\alpha}^2\},
\]
on $\alpha$ \'es l'error tipus I, l'error tipus I m\`axim per sota
del qual podem acceptar que la distribuci\'o \'es
exponencial, compleix:
\[
0.1\leq \alpha_{\mbox{\footnotesize max}}=\pp{\chi_2^2\geq 4.5646}
\leq 0.15.
\]
En vista del valor de $\alpha_{\mbox{\footnotesize max}}$, podem acceptar
que la mostra correspon a una distribuci\'o exponencial.
}


\begin{probres}
{En un per\'{\i}ode de $72$ hores en els Estats Units varen tenir lloc en
total $290$ accidents de tr\`ansit. 
El nombre d'accidents per hora durant aquest
per\'{\i}ode va ser
\begin{center}
\begin{tabular}{|c|c|}
\hline
Nombre d'accidents per hora & Nombre d'hores \\\hline\hline
$0$ o
$1$&$\ 6$\\\hline
$2$&$11$\\\hline $3$&$15$
\\\hline $4$&$14$\\\hline 
$5$&$12$\\\hline $6$&$\ 8$\\\hline $7$
o m\'es&$\ 6$\\\hline
\end{tabular}
\end{center}
Provau la hip\`otesi que el nombre d'accidents
per hora durant un per\'{\i}ode semblant \'es una variable aleat\`oria de Poisson.}
\end{probres} 

\res{
El problema \'es verificar si la variable $X=$``Nombre d'accidents per hora''
segueix la distribuci\'o de Poisson amb par\`ametre $\lambda$.
Per tant, el primer que hem de fer \'es estimar el par\`ametre $\lambda$.
El millor estimador de $\lambda$ \'es $\overline{X}$. Per tant, el valor estimat
de $\lambda$ valdr\`a: 
\[
\tilde{\lambda}=\overline{X}=\frac{290}{72}\approx 4.0277,
\]
ja que hi ha hagut $290$ accidents en un total de $72$ hores.

La taula per trobar l'estad\'{\i}stic $\chi^2$ \'es:
\begin{center}
\renewcommand{\arraystretch}{1.5}
\begin{tabular}{|r@{,}l|c|c|c|c|c|}
\hline
\multicolumn{2}{|c|}{Intervals $I_i$}
&$Y_i$&$p_i$&$n p_i$&$|Y_i -n p_i|$&
$\frac{{(Y_i -n p_i)}^2}{n p_i}$\\\hline\hline
$(-\infty$&$1.5]$&$\ 6$&$0.0895$&$\ 6.4486$&$0.4486$&$0.0312$\\\hline
$(1.5$&$2.5]$&$11$&$0.1445$&$10.4037$&$0.5962$&$0.0341$\\\hline
$(2.5$&$3.5]$&$15$&$0.1940$&$13.9680$&$1.0319$&$0.0762$\\\hline
$(3.5$&$4.5]$&$14$&$0.1953$&$14.0650$&$0.0650$&$0.0003$\\\hline
$(4.5$&$5.5]$&$12$&$0.1573$&$11.3301$&$0.6698$&$0.0395$\\\hline
$(5.5$&$6.5]$&$\ 8$&$0.1056$&$\ 7.6059$&$0.3940$&$0.0204$\\\hline
$(6.5$&$\infty)$&$\ 6$&$0.1138$&$\ 8.1943$&$2.1943$&$0.5876$\\\hline\hline
\multicolumn{2}{|c|}{}&&&&&$\chi^2 =0.7895$\\\hline
\end{tabular}
\end{center}
Recordem que la funci\'o de probabilitat de la variable aleat\`oria de Poisson
\'es: 
\[
f_X (k)=\frac{\lambda^k}{k!} e^{-\lambda}.
\]
Fent servir la f\'ormula anterior, podem trobar les probabilitats te\`oriques
$p_i$:
\begin{eqnarray*}
p_1 & = & f_X (0)+f_X(1) = e^{-\tilde{\lambda}} (1+\tilde{\lambda})\approx 
0.0895, \\
p_2&  = & f_X (2) =e^{-\tilde{\lambda}} \frac{\tilde{\lambda}^2}{2!} \approx 
0.1444, \\
p_3 & = & f_X (3) =e^{-\tilde{\lambda}} \frac{\tilde{\lambda}^3}{3!} \approx 
0.1940, \\
p_4  & = & f_X (4) =e^{-\tilde{\lambda}} \frac{\tilde{\lambda}^4}{4!} \approx 
0.1953, \\
p_5 & = & f_X (5) =e^{-\tilde{\lambda}} \frac{\tilde{\lambda}^5}{5!} \approx 
0.1573, \\
p_6 & = & f_X (6) =e^{-\tilde{\lambda}} \frac{\tilde{\lambda}^6}{6!} \approx 
0.1573, \\
p_7  & = & \pp{X\geq 7}=1-\sum\limits_{i=0}^6 f_X(i) = 0.1138.
\end{eqnarray*}
Tenint en compte que la distribuci\'o de l'estad\'{\i}stic 
$\chi^2$ s'aproxima a la distribuci\'o $\chi_5^2$, l'error tipus I m\`axim 
per sota del qual podem acceptar
que $X$ segueix la distribuci\'o de Poisson
compleix:
\[
0.95\leq\alpha_{\mbox{\footnotesize max}}=\pp{\chi_5^2\geq 0.7895}
\leq 0.99.
\]
Concloem, doncs, que es pot acceptar que $X$ segueix la distribuci\'o de
Poisson.
}

\begin{probres}
{Es va examinar una mostra de $2000$ defuncions amb els seg\"uents resultats
\begin{center}
\begin{tabular}{|l||c|c||c|}
\hline &Mort per
c\`ancer&Mort per altres
causes&Totals\\\hline\hline Fumadors&$22$&$1178$&$1200$
\\\hline No Fumadors&$26$&$\ 774$&$\ \,800$
\\\hline Totals&$48$&$1952$&$2000$\\\hline 
\end{tabular}
\end{center}

Suposem que
aquests resultats s\'on el producte d'una mostra aleat\`oria 
simple d'una poblaci\'o determinada i
provau que les dues classificacions no s\'on independents,
 o sigui, trobau l'error
tipus~I m\`axim per poder acceptar independ\`encia i comprovau que aquest 
error \'es petit.}
\end{probres} 

\res{
Resoldrem aquest problema fent servir el test d'independ\`encia
de la~$\chi^2$.    

En aquest cas, l'estad\'{\i}stic a fer servir \'es:
\[
\chi^2 =\sum \frac{{(n_{ij}-\mbox{teor}_{ij})}^2}{\mbox{teor}_{ij}},
\]
on $n_{ij}$ s\'on les freq\"u\`encies emp\'{\i}riques (les donades per la
taula del problema) i $\mbox{teor}_{ij}$ s\'on les freq\"u\`encies
te\`oriques.
Recordem que 
la freq\"u\`encia te\`orica corresponent al nivell $i-j$ val:
\[
\mbox{teor}_{ij}=\frac{n_{i\bullet} n_{\bullet j}}{n},
\]
on $n_{i\bullet}$ \'es el nombre total d'individus que pertanyen al
nivell $i$ segons el criteri de classificaci\'o $1$ i $n_{\bullet j}$ \'es
el nombre total d'individus que pertanyen al nivell $j$ segons el criteri
de classificaci\'o $2$.

A continuaci\'o constru\"{\i}m la taula de freq\"u\`encies te\`oriques:
\begin{center}
\renewcommand{\arraystretch}{1.5}
\begin{tabular}{|c||c|c||c|}
\hline
&Mort per c\`ancer&Mort per altres causes&Totals\\\hline\hline
Fumadors&$\frac{1200\cdot 48}{2000}=28.8$&$\frac{1200\cdot 1952}{2000}
=1171.2$&$1200$\\\hline
No Fumadors&$\frac{\ 800\cdot 48}{2000}=19.2$&$\frac{\ 800\cdot 1952}{2000}
=\ 780.8$&$\ 800$\\\hline
\end{tabular}
\end{center}
El valor de l'estad\'{\i}stic $\chi^2$ valdr\`a:
\begin{eqnarray*}
\chi^2 & = & \frac{{(22-28.8)}^2}{28.8}+\frac{{(1178-1171.2)}^2}{1171.2}
+\frac{{(26-19.2)}^2}{19.2}+\frac{{(774-780.8)}^2}{780.8}\\
& = & 4.1125.
\end{eqnarray*}
La distribuci\'o de l'estad\'{\i}stic
 $\chi^2$ s'aproxima a la
 distribuci\'o
$\chi^2$ amb $(f-1)\cdot (c-1)$ graus de llibertat on $f$ \'es el nombre
de files (nombre de nivells segons el criteri de classificaci\'o $1$) i
$c$ \'es el nombre de columnes (nombre de nivells segons el criteri de 
classificaci\'o $2$). En el nostre cas, 
l'estad\'{\i}stic $\chi^2$  
s'aproximar\`a a una variable $\chi_1^2$.

Tenint en compte que la regi\'o cr\'{\i}tica per a aquest tipus de
contrast \'es:
\[
R.C.=\{\chi^2 \geq \chi_{1,1-\alpha}^2\},
\]
on $\alpha$ \'es l'error tipus I, l'error tipus I m\`axim 
per sota del qual acceptam
independ\`encia complir\`a:
\[
0.025\leq \alpha_{\mbox{\footnotesize max}}=\pp{\chi_1^2 \geq 4.1125}
\leq 0.05.
\]
Rebutjam, doncs, independ\`encia.
}


\begin{probres}
{Suposem que $X_{ij}$ amb $i=1,\ldots,r$, $j=1,\ldots,s$ \'es una
mostra aleat\`oria simple d'una variable aleat\`oria multinomial amb par\`ametres $n$ i $p_{ij}=w_i s_j$. ($w_i$ \'es la
probabilitat que $X_{ij}$ estigui en la classe $(i,1)$, $(i,2)$,...,
$(i,s)$ i $s_j$ \'es la
probabilitat que $X_{ij}$ estigui en la classe $(1,j)$, $(2,j)$,...,
$(r,j)$). Provau que els estimadors de m\`axima versemblan\c{c}a de $w_i$ i $s_j$
s\'on
$$\hat W_i ={X_{i\bullet}\over n}={\sum\limits_{j=1}^s X_{ij}\over n},
\quad\hat S_j ={X_{\bullet j}\over n}={\sum\limits_{i=1}^r X_{ij}\over n}.$$}
\end{probres} 

\res{
Recordem la funci\'o de probabilitat d'una variable aleat\`oria multinomial
amb par\`ametres $n,p_{11},\ldots,p_{rs}$ val:
\[
f(x_{11},\ldots x_{rs}) =\frac{n!}{x_{11}!\ldots x_{rs}!} p_{11}^{x_{11}}
\ldots p_{rs}^{x_{rs}}.
\]
A partir del m\`etode de la m\`axima versemblan\c{c}a, trobarem els 
estimadors dels par\`ametres $p_{ij}$.

El primer pas seria derivar la f\'ormula anterior respecte de cada una de les
variables $p_{ij}$. Fixau-vos que hi ha molts de productes; per tant, 
traguem logaritmes per simplificar el c\`alcul de les derivades.
\[
K:= \ln f(x_{11},\ldots x_{rs})=\ln\left(\frac{n!}{x_{11}!\ldots x_{rs}!}
\right) +\sum_{i=1}^r\sum_{j=1}^s x_{ij} \ln p_{ij}.
\]
Abans de comen\c{c}ar a derivar, hem de tenir en compte que les variables
$x_{ij}$ i els par\`ametres $p_{ij}$,
$i=1,\ldots,r$, $j=1,\ldots, s$ no s\'on independents ja que:
\begin{equation}
\begin{array}{rl}
x_{rs} = & n -\sum\limits_{(i,j)\not =(r,s)} x_{ij}, \\
p_{rs} = & 1 -\sum\limits_{(i,j)\not =(r,s)} p_{ij}.
\end{array}
\label{FORXIJPIJ}
\end{equation}
Per tant, les derivades de la funci\'o $K$ respecte del par\`ametre $p_{ij}$ valen:
\[
\frac{\partial K}{\partial p_{ij}}=\frac{x_{ij}}{p_{ij}}+\frac{x_{rs}}{p_{rs}}\cdot 
\frac{\partial p_{rs}}{\partial p_{ij}}=\frac{x_{ij}}{p_{ij}}-
\frac{x_{rs}}{p_{rs}},\  \forall (i,j)\not =(r,s).
\]
El sistema a resoldre de cara a trobar els estimadors de m\`axima versemblan\c{c}a 
$\tilde{p}_{ij}$ del par\`ametre $p_{ij}$ \'es
\[
\frac{\partial K}{\partial p_{ij}}=\frac{x_{ij}}{\tilde{p}_{ij}}-
\frac{x_{rs}}{\tilde{p}_{rs}}=0\  \forall (i,j)\not =(r,s).
\]
El sistema d'equacions anterior es pot posar de forma equivalent com: 
\begin{equation}
\frac{\tilde{p}_{ij}}{\tilde{p}_{rs}}=\frac{x_{ij}}{x_{rs}}\  \forall (i,j)\not =(r,s).
\label{EQPERPIJTILDE}
\end{equation}
Sumant totes les equacions anteriors per a tot $(i,j)\not =(r,s)$ i 
tenint en compte~(\ref{FORXIJPIJ}), arribam a la seg\"uent expressi\'o 
per a l'estimador $\tilde{p}_{rs}$:
\[
\frac{1-\tilde{p}_{rs}}{\tilde{p}_{rs}}=\frac{n-x_{rs}}{x_{rs}}.
\]
D'on podem trobar el valor de $\tilde{p}_{rs}$:
\[
\tilde{p}_{rs}=\frac{x_{rs}}{n}.
\]
Tenint en compte~(\ref{EQPERPIJTILDE}), podem trobar tots els valors dels altres 
par\`ametres $\tilde{p}_{ij}$:
\[
\tilde{p}_{ij}=\frac{x_{ij}}{x_{rs}}\tilde{p}_{rs} =\frac{x_{ij}}{n}.
\]
A continuaci\'o, trobam els estimadors $\tilde{w}_i$ i $\tilde{s}_j$ dels par\`ametres
$w_i$ i $s_j$.

Fent servir les seg\"uents relacions entre els par\`ametres ${w}_i$, ${s}_j$ i
$p_{ij}$
\[
w_i=\sum_{j=1}^s p_{ij}, \ s_j =\sum_{i=1}^r p_{ij},
\]
i tenint en compte les f\'ormules trobades per als estimadors 
$\tilde{p}_{ij}$ dels 
par\`ametres $p_{ij}$ arribam a la seg\"uent expressi\'o per als estimadors 
$\tilde{w}_i$ i $\tilde{s}_j$:
\[
\tilde{w}_i=\sum_{j=1}^s \tilde{p}_{ij}=\frac{\sum\limits_{j=1}^s x_{ij}}{n}, 
\ \tilde{s}_j =\sum_{i=1}^r \tilde{p}_{ij}=\frac{\sum\limits_{i=1}^r x_{ij}}{n}.
\]
}

\begin{probres}
{Sigui $X$ una variable aleat\`oria
 uniforme en l'interval $[0,1]$. 
Sigui $F(x)$ una
funci\'o real cont\'{\i}nua, 
estrictament creixent i 
derivable definida a tot $\RR$
tal que $F(x)\in [0,1]$ $\forall x\in\RR$ i que $\lim\limits_{x\to -\infty} 
F(x)=0,\ \lim\limits_{x\to\infty} F(x)=1$.
\begin{itemize}
\item[a)]{Provau que la variable aleat\`oria $Y=F^{-1}(X)$ t\'e com a funci\'o 
de distribuci\'o
$F(x)$.}
\item[b)]{Aplicant l'apartat a) i donada la seg\"uent mostra aleat\`oria simple d'una variable aleat\`oria uniforme
en $[0,1]$, contru\"{\i}u una mostra aleat\`oria simple d'una variable aleat\`oria $Y$ que tengui com a funci\'o de
densitat $$f_Y(y)={1\over \pi (1+y^2)},\quad y\in\RR .$$}
Mostra aleat\`oria simple de $X$:
\begin{center}
\begin{tabular}{cccccccc}
0.526,&0.973,&0.055,&0.826,&0.131,&0.285,&0.101,&0.165,\\
0.638,&0.586,&0.706,&0.487,&0.239,&0.362,&0.813,&0.497,\\
0.288,&0.270,&0.056,&0.692,&0.915,&0.706,&0.609,&0.717,\\
0.600,&0.138,&0.675,&0.340,&0.019,&0.022,&0.735,&0.844.\\
\end{tabular}
\end{center}
\item[c)]{Vegeu mitjan\c{c}ant el test de la $\chi^2$ si la
mostra anterior correspon a una mostra de la variable $Y$.\newline
{\footnotesize
Condici\'o per fer el problema: heu d'agafar com a m\'{\i}nim 4
intervals.}}
\end{itemize}
{\footnotesize Final. Juny 93.}
}
\etiqueta{PROBGENMOSTRES}
\end{probres} 

\res{
\begin{itemize}
\item[a)] Calculem la funci\'o de distribuci\'o de la variable aleat\`oria 
$Y$:
\[
F_Y (t)=\pp{Y\leq t}=\pp{ F^{-1}(X)\leq t}=\pp{X\leq F(t)}=F_X (F(t)).
\]
La pen\'ultima igualtat \'es certa ja que la funci\'o $F$ \'es estrictament 
creixent.

L'\'unic que hem de trobar ara \'es la funci\'o de distribuci\'o de $X$. 
Recordem que $X$ \'es $U[0,1]$ (uniforme en l'interval $[0,1]$). Per tant, la funci\'o
de distribuci\'o valdr\`a:
\[
F_X (s)=\left\{\begin{array}{ll}
0, & \text{si $s\leq 0$}, \\
s, & \text{si $0< s\leq 1$}, \\
1, & \text{si $s>1$}. 
\end{array}\right.
\]
Tenint en compte que $F(t)\in [0,1]$ per a tot $t\in\RR$, tenim que:
\[
F_Y (t)=F_X (F(t))=F(t).
\]
Concloem, doncs, que la funci\'o de distribuci\'o
 de $Y$ \'es $F$. 

\item[b)] Sigui ara una variable $Y$ amb funci\'o
 de densitat:
$f_Y (y)=\frac{1}{\pi (1+y^2)},\ y\in\RR$. Hem de construir una mostra
aleat\`oria simple de $Y$ a partir d'una mostra 
aleat\`oria simple
d'una variable uniforme $X$ en l'interval $[0,1]$. 

Per fer-ho, aplicarem
l'apartat anterior i trobarem la funci\'o de distribuci\'o
 de la
variable $Y$ $F_Y (y)$. Una vegada trobada la funci\'o
 de distribuci\'o,
trobarem la inversa $F_Y^{-1}$. L'\'unic que quedar\`a fer per trobar la 
mostra de $Y$ 
\'es aplicar la seg\"uent f\'ormula a tots els valors de la mostra
de $X$:
\[
Y= F_Y^{-1} (X).
\]
Fixau-vos que aquest \'es un m\`etode que val per trobar qualsevol mostra
aleat\`oria simple de qualsevol variable aleat\`oria $Y$. Aquest m\`etode
es diu {\bf m\`etode de la transformada inversa}.

Trobem la funci\'o de distribuci\'o $F_Y(t)$:
\[
F_Y (t)=\int_{-\infty}^y \frac{1}{\pi (1+y^2)}\, dy =\frac{1}{\pi}
{[\mbox{arctan } y]}_{-\infty}^y =\frac{1}{2}+\frac{\mbox{arctan } y}{\pi}.
\]
A continuaci\'o, trobem la seva inversa $F_Y^{-1}(t)$:
\[
\begin{array}{c}
F_Y (y)=t,\ \Longrightarrow \frac{1}{2}+\frac{1}{\pi}\mbox{arctan } y=t, \nonumber\\
\mbox{arctan } y = \left( t-\frac{1}{2}\right) \pi,\ \Longrightarrow y=F_Y^{-1} (t)=
\mbox{tan} \left(\pi \left(t-\frac{1}{2}\right)\right).
\end{array}
\]
La mostra aleat\`oria simple de la variable $Y$ aplicant la f\'ormula anterior \'es
\begin{center}
\begin{tabular}{r@{.}lr@{.}lr@{.}lr@{.}lr@{.}lr@{.}l}
0&0818,&11&7609,&-5&7297,&1&6434,&-2&2910,&-0&8011,\\-3&045,&-1&7531,&
0&4629,&0&2769,&0&7557,&-0&0408,\\-1&0716,&-0&4629,&1&5017,&-0&0094,&
-0&7857,&-0&8816,\\-5&6253,&0&6888,&3&6553,&0&7557,&0&3564,&0&8115,\\
0&3249,&-2&1602,&0&6128,&-0&5497,&-16&733,&-14&445,\\0&9099,&1&8744,&&&&&&&&
\end{tabular}
\end{center}

\item[c)] Constru\"{\i}m la taula per trobar l'estad\'{\i}stic del test de la $\chi^2$:
\begin{center}
\renewcommand{\arraystretch}{1.5}
\begin{tabular}{|r@{,}l|c|c|c|c|c|}
\hline
\multicolumn{2}{|c|}{Intervals $I_i$}
&$Y_i$&$p_i$&$n p_i$&$|Y_i -n p_i|$&
$\frac{{(Y_i -n p_i)}^2}{n p_i}$\\\hline\hline
$(-\infty$&$-1]$&$\ \,9$&$0.25$&$8$&$1$&$0.125$\\\hline
$(-1$&$0]$&$\ \,7$&$0.25$&$8$&$1$&$0.125$\\\hline
$(0$&$1]$&$11$&$0.25$&$8$&$3$&$1.125$\\\hline
$(1$&$\infty)$&$\ \,5$&$0.25$&$8$&$3$&$1.125$\\\hline
\multicolumn{2}{|c|}{}&&&&&$\chi^2 =2.5$\\\hline
\end{tabular}
\end{center}
C\`alcul de les probabilitats te\`oriques $p_i$:
\begin{eqnarray*}
p_1 & = & F_Y (-1) =\frac{1}{2}+\frac{\mbox{arctan }(-1)}{\pi}=0.25,\\
p_2 & = & F_Y (0)- F_Y(-1) =0.5 -0.25 =0.25, \\
p_3 & = & F_Y (1) - F_Y (0) = \frac{1}{2} +\frac{\mbox{arctan }1}{\pi} -0.5 =0.25, \\
p_4 & = & 1 - F_Y(1)= 1 -0.75 =0.25.
\end{eqnarray*}
L'estad\'{\i}stic $\chi^2$ s'aproxima a la distribuci\'o $\chi_3^2$. 

Regi\'o cr\'{\i}tica: $R.C.=\{\chi^2\geq \chi_{3,1-\alpha}^2\}$ 
on $\alpha$ \'es l'error tipus I.

L'error tipus I m\`axim per sota del qual acceptam que la mostra anterior segueix 
la mateixa distribuci\'o de $Y$ compleix:
\[
0.45\leq \alpha_{\mbox{\footnotesize max}}=\pp{\chi_3^2 \geq 2.5}\leq 0.5.
\]
Acceptam, doncs, que la funci\'o de distribuci\'o
 de la mostra anterior \'es $F_Y$.
\end{itemize}
}

\section{Problemes proposats}

\begin{prob}
{Un parell de daus es varen llan\c{c}ar $500$ vegades. 
En la taula seg\"uent es
mostren les sumes que varen tenir lloc. Provau la hip\`otesi que els daus no
estaven trucats, o sigui, 
comprovau que l'error tipus I m\`axim com\`es
en acceptar que els daus no estan trucats no \'es gaire petit.
\begin{center}
\begin{tabular}{|c|c|}
\hline
Suma & Freq\"u\`encia \\\hline\hline
$2,3$ o $4$&$\ 74$\\\hline $5$ o
$6$&$120$\\\hline $7$&$\ 83$\\\hline $8$ o
$9$&$135$\\\hline $10,11$ o $12$&$\ 88$\\\hline
\end{tabular}
\end{center}
}
\end{prob}

\begin{prob}
{El $1972$, l'informe oficial va donar la seg\"uent distribuci\'o del
nombre de dies que varen ser internats en l'hospital els malalts durant
1971.
\begin{center}
\begin{tabular}{|c|c|}
\hline 
Nombre de dies & Nombre de malalts
\\\hline\hline
$1$&$\ 89$\\\hline $2$&$152$\\\hline 
$3$&$105$
\\\hline $4-5$&$165$\\\hline 
$6-9$&$221$\\\hline $10-14$&$124$\\\hline $15-30$
&$106$\\\hline $31$ o m\'es&$\ 38$\\\hline 
\end{tabular}
\end{center}

Provau la hip\`otesi que aquestes dades es varen obtenir d'una
distribuci\'o $\chi^2$ amb $4$ graus de llibertat.}
\end{prob} 

\begin{prob}
{Es va seleccionar una mostra de $3000$ taronges de Val\`encia. Cada
taronja es va classificar segons el seu color (clar, mig i obscur) i es va
determinar el seu contengut de sucre (dol\c{c}a o no dol\c{c}a). Els resultats varen
ser
\begin{center}
\begin{tabular}{|c||c|c||c|}
\hline Color&Molt Dol\c{c}a&No
Dol\c{c}a&Totals\\\hline\hline Clar&$1300$&$\ 200$&$1500$
\\\hline Mig&$\ 500$&$\ 500$&$1000$\\\hline Obscur
&$\ 200$&$\ 300$&$\ 500$\\\hline Totals
&$2000$&$1000$&$3000$\\\hline 
\end{tabular}
\end{center}

Provau la hip\`otesi que la dol\c{c}or i el color s\'on independents.
}
\end{prob} 

\begin{prob}
{Es va fer una prova d'inte{\lgem}ig\`encia a $100$
estudiants. En la taula seg\"uent es mostren les qualificacions
obtengudes:
\begin{center}
\begin{tabular}{|c|c|}
\hline
Qualificaci\'o $x$&Freq\"u\`encia\\
\hline\hline
$70<x\leq 90$&$\ 8$\\\hline
$90<x\leq 110$&$38$\\\hline
$110<x\leq 130$&$45$\\\hline
$130<x\leq 150$&$\ 9$\\\hline 
\end{tabular}
\end{center}

Es pot suposar que aquestes qualificacions s\'on una
mostra aleat\`oria de les que tendrien totes les persones
possibles que fessin la prova. Provau la hip\`otesi que les
qualificacions obtengudes per la poblaci\'o (conceptualment
infinita) estarien distribu\"{\i}des normalment.
\newline
{\footnotesize
Final. Juny 91.}}
\end{prob} 

\begin{prob}
{Considerem la seg\"uent mostra
aleat\`oria simple d'una v.a. $X$ tal que\break $X(\Omega)=[0,1]$.
Provau, mitjan\c{c}ant el test $\chi^2$, que podem considerar que
$X$ segueix una distribuci\'o uniforme en $[0,1]$.(Preneu intervals
d'amplada $0.25$.)
\begin{center}
\begin{tabular}{ccccc}
 0.479, & 0.889, & 0.216, & 0.596, & 0.359\\
 0.347, & 0.646, & 0.359, & 0.991, & 0.227\\
 0.774, & 0.760, & 0.448, & 0.992, & 0.742\\
 0.402, & 0.049, & 0.213, & 0.296, & 0.711\\
 \end{tabular}
 \end{center}

{\footnotesize Final. Setembre 91.}}
\end{prob} 

\begin{prob}
{Sigui $X$ la variable aleat\`oria que t\'e
com a funci\'o de densitat: $$f(x)=
\left\{\begin{array}{ll}
x+1, & \text{si $x\in [-1,0],$}\\ 1-x, & \text{si
$x\in [0,1],$}\\
0, & \text{en cas contrari.} 
\end{array}\right.
$$
Comprovau mitjan\c{c}ant el test de la
$\chi^2$ que la seg\"uent mostra aleat\`oria simple 
segueix la mateixa distribuci\'o que $X$:
\begin{center}
\begin{tabular}{r@{.}lr@{.}lr@{.}lr@{.}lr@{.}lr@{.}lr@{.}lr@{.}l} 
0&183,&0&647,&0&148,&-0&143,&-0&625,&0&858,&-0&177,&0&350,\\-0&188,&-0&059,&
0&845,&0&031,&-0&156,&0&564,&-0&235,&0&237,\\0&294,&-0&257,&0&110,&0&478,&
0&647,&0&276,&-0&528,&-0&075,\\-0&498,&0&395,&-0&163,&-0&075,&-0&623,&0&053,&
-0&647,&0&348,\\-0&795,&-0&132,&-0&381,&-0&017,&-0&227,&0&277,&0&590,&-0&832\\
\end{tabular}
\end{center}
{\footnotesize Final. Setembre 93.}} 
\end{prob}

\begin{prob}
{Ens donen les notes de
certa assignatura de 3 grups d'alumnes $A$, $B$ i $C$:
\begin{center}
\begin{tabular}{|c||r@{.}lr@{.}lr@{.}lr@{.}lr@{.}lr@{.}lr@{.}lr@{.}l
r@{.}lr@{.}l|}
\hline
$A$&4&6&5&&5&1&5&6&4&6&5&&5&7&5&4&4&4&8&\\\hline
$B$&4&6&3&4&5&3&4&&3&5&4&&5&&4&7&3&6&4&1\\\hline
$C$&7&2&7&3&5&7&4&1&5&7&6&1&6&&7&8&7&&3&8\\\hline
\end{tabular}
\end{center}

Els classificam segons $2$ criteris: per grup i per nota tenint en compte
que:\newline {\bf Suspens} significa una nota m\'es petita que $5$
(nota $<5$).
\newline {\bf Aprovat} significa una nota entre $5$ i $6$ ($5\leq$ nota
$\leq$ 6).
\newline {\bf Notable} significa una nota m\'es gran que $6$
(nota $>\ 6$).\newline Vegeu a partir del test $\chi^2$ per a quins
valors de $\alpha$ (error tipus I) podem acceptar que els dos criteris s\'on
independents.\newline{\footnotesize Final. Juny 94.}}
\etiqueta{NOTESASIG}
\end{prob} 

\begin{prob}
{Llan\c{c}am un dau 100 vegades on:
$15$ vegades ha sortit la cara $1$,
$15$ vegades ha sortit la cara $2$,
$30$ vegades ha sortit la cara $3$,
$15$ vegades ha sortit la cara $4$,
$15$ vegades ha sortit la cara $5$,
$10$ vegades ha sortit la cara $6$.
Fent servir els test $\chi^2$, trobau el $\alpha$ m\`axim 
(error tipus I) per al qual acceptam que el dau no est\`a trucat.
\newline{\footnotesize Final. Juny 94.}}
\end{prob}

\begin{prob}
{Trobau l'error tipus I m\`axim per poder acceptar que la seg\"uent
mostra aleat\`oria simple segueix una distribuci\'o exponencial fent servir el
test de la $\chi^2$.
\begin{center}
\begin{tabular}{r@{.}lr@{.}lr@{.}lr@{.}lr@{.}lr@{.}lr@{.}lr@{.}lr@{.}lr@{.}l}
1&68,&
1&17,&0&6,&2&84,&1&55,&1&26,&1&24,&0&04,&0&28,&0&33,\\
0&38,&0&67,&0&18,&0&27,&0&95,&2&9,&1&93,&0&12,&0&14,&
0&
86,\\
1&71,&0&76,&0&7,&0&02,&0&53,&1&24,&1&27,&0&,&1&1,&1&9,
\\
1&02,&0&18,&0&63,&1&72,&1&9,&1&72,&0&12,&0&17,&2&76,&
1&19,\\
0&31,&0&45,&0&58,&1&97,&0&8,&0&35,&2&14,&0&29,&3&57,&
2&91,
\end{tabular}
\end{center}
Condici\'o per fer el problema: considerau la seg\"uent partici\'o de
$[0,\infty)$:
$$[0,\infty)=[0,0.25)\cup[0.25,0.5)\cup[0.5,0.75)\cup[0.75,1)\cup
[1,\infty).$$\newline{\footnotesize Final. Setembre 94.}}
\end{prob}

\begin{prob}
{
Es vol provar la hip\`otesi que el di\`ametre, en mm., d'una
pe\c{c}a \'es una variable aleat\`oria amb funci\'o 
de densitat:
\[
f(x)=\frac{3}{37}x^2,\ 3\leq x\leq 4.
\]
Per fer el contrast se selecciona una mostra aleat\`oria simple de grand\`aria 
$1000$ amb els resultats seg\"uents:
$$
\begin{tabular}{|c|c|c|}
\hline
Interval&Freq\"u\`encia\\\hline\hline
3.0-3.2&157\\\hline
3.2-3.4&174\\\hline
3.4-3.6&190\\\hline
3.6-3.8&226\\\hline
3.8-4.0&253\\\hline
\end{tabular}
$$
Fent servir el test $\chi^2$, trobau l'$\alpha_{\mbox{\footnotesize max}}$
(error tipus I m\`axim) per poder acceptar la hip\`otesi.
\newline{\footnotesize Segon parcial. Juny 95.}
}
\end{prob}

\begin{prob}
{
Classificam $N$ individus segons dos criteris. Cada criteri t\'e dos
nivells. La taula de conting\`encia \'es:
\begin{center}
\begin{tabular}{|c|r|r|}
\hline
$C2\backslash C1$&\multicolumn{1}{|c|}{$A_1$}&
\multicolumn{1}{|c|}{$A_2$}\\\hline\hline
$B_1$&10&5\\\hline
$B_2$&5&10\\\hline
\end{tabular}
\end{center}
Trobau l'error tipus~I m\`axim per poder acceptar que els dos criteris s\'on 
independents fent servir el test de la~$\chi^2$.
\newline{\footnotesize Final. Juny 96.}
}
\end{prob}

\begin{prob}
{
A partir de la seg\"uent mostra aleat\`oria
simple d'una variable aleat\`oria $X\ U(0,1)$ (uniforme en l'interval $(0,1)$),
generau una mostra aleat\`oria simple d'una variable aleat\`oria
$Y\ \textrm{Exp} (\lambda =1)$ i comprovau a partir del test de la 
bondat d'ajustament
que la mostra anterior correspon a la variable aleat\`oria~$Y$.
Preneu la seg\"uent partici\'o de $\RR$:
\[
\RR =(-\infty,0.5)\cap [0.5,1)\cap [1,\infty).
\]

\begin{center}
\begin{tabular}{ccccc}
0.033,&0.890,&0.548,&0.405,&0.281\\
0.858,&0.820,&0.819,&0.531,&0.198\\
0.038,&0.290,&0.917,&0.601,&0.912\\
0.333,&0.658,&0.428,&0.400,&0.208\\
0.369,&0.252,&0.154,&0.566,&0.575
\end{tabular}
\end{center}
{\footnotesize Final. Juny 96.}
}
\end{prob}

\begin{prob}
{
Llan\c{c}am un dau~$200$ vegades amb els resultats seg\"uents:
\begin{center}
\begin{tabular}{|l||c|c|c|c|c|c|}
\hline
Cara&1&2&3&4&5&6\\\hline\hline
Vegades&40&40&30&20&30&40\\\hline
\end{tabular}
\end{center}
Trobau l'error tipus~I m\`axim per poder acceptar que el dau no est\`a trucat
fent servir el test de la $\chi^2$.
\newline{\footnotesize Final. Setembre 96.}
}
\end{prob}

