\frontmatter
\chapter{Prefaci}

L'obra aqu\'{\i} presentada d'{\it Introducci\'o a 
l'estad\'{\i}stica inferencial}
\'es el resultat de m\'es de cinc anys de doc\`encia per
part dels autors en els estudis de les Enginyeries T\`ecniques
d'Inform\`atica de Gesti\'o i de Sistemes a la Universitat de les Illes
Balears.

Hem volgut escriure uns apunts d'estad\'{\i}stica molt pr\`actics. Per
aquest motiu, hem estructurat tots els cap\'{\i}tols en tres parts:
resum te\`oric, problemes resolts i problemes proposats.

L'obra es complementa amb el llibre {\it Introducci\'o a l'estad\'{\i}stica
descriptiva i a la teoria de les probabilitats}, dels mateixos autors.
De fet, es pot dir que aquests apunts s\'on la continuaci\'o de
l'esmentada obra.

Quan un pret\'en escriure uns apunts 
d'estad\'{\i}stica inferencial, ha
d'evitar un mal de molts llibres sobre el tema: explicar
l'estad\'{\i}stica com un receptari de cuina. En el nostre cas, la tasca
\'es doble: evitar escriure un receptari de cuina sense donar una
estad\'{\i}stica massa te\`orica que provocaria una p\`erdua
d'inter\`es per part de l'alumne.
Un tema on es manifesta de forma clara el fet anterior \'es el de
contrasts d'hip\`otesis en el qual donam nom\'es les regions cr\'{\i}tiques dels
contrasts m\'es importants sense explicar la regi\'o cr\'{\i}tica en el
cas general; o sigui, sense explicar el lema de Neyman-Pearson ni el
criteri de la ra\'o de versemblan\c{c}a.

Un altre canvi substancial respecte d'altres llibres d'estad\'{\i}stica
\'es el c\`alcul de l'error tipus~I m\`axim en els contrasts
d'hip\`otesis, ja que \'es el m\`etode habitual per realitzar un contrast
en molts paquests estad\'{\i}stics. La dificultat que hi ha quan es fan els
contrasts d'aquesta manera s\'on les taules estad\'{\i}stiques. Totes
les taules estad\'{\i}stiques que hem vist s\'on molt incompletes i, a
vegades, redundants. En aquests apunts, hem volgut donar unes taules el
m\'es completes possible.

Les pr\`actiques d'estad\'{\i}stica s\'on la clau per acabar d'entendre
tots els conceptes d'estad\'{\i}stica inferencial. Per aquest motiu, hem
volgut proposar una s\`erie de pr\`actiques que l'alumne pot
desenvolupar.

Un altre motiu que ens ha impulsat a escriure l'obra ha estat el pla
d'estudis. Despr\'es de la darrera revisi\'o dels plans d'estudis, quasi
totes les assignatures han vist retallades les hores de classe sense
retallar el temari. Si els alumnes tenen una s\`erie de problemes
resolts, podran seguir les classes m\'es f\`acilment ja que podran
fer un estudi previ de cada tema.

Els autors volem agrair la co{\lgem}aboraci\'o del professor Ricardo
Alberich en l'elaboraci\'o de la taula dels contrasts d'hip\`otesis
i, en general, la de tots els membres del Departament de Matem\`atiques i
Inform\`atica de la Universitat de les Illes Balears pel bon ambient de
feina.

\cleardoublepage
%\tableofcontents
%\listoffigures
%\mainmatter
