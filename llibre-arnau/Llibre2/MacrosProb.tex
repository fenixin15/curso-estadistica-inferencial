\input eplain.tex
\magnification=1200
\hoffset=0.46 true cm
%\voffset=-0.4 true cm
\hsize=15 true cm
\vsize=22 true cm
\newcount\tit
\newcount\subtit
\newcount\teorema
\newcount\proposicio
\newcount\lema
\newcount\corolari
\font\ftitol=cmbx12 scaled \magstep 4
\font\ftitolp=cmbx12 scaled \magstep 3
\font\fsubtitol=cmbx12 scaled \magstep 2
\font\ftitolpp =cmbx12
%%%%%%%%%
%%%%%%%%%OBRIM EL FIXER AUXILIAR JOBNAME.ARNAU
%%%%%%%%%
\newwrite\arnaufile\newif\ifarnaufileopened
\def\openarnaufile{\ifarnaufileopened\else
\arnaufileopenedtrue
\immediate\openout\arnaufile = \jobname.arnau
\fi}
\newif\ifrewritearnaufile \rewritearnaufiletrue
\def\readarnaufile{\testfileexistence{arnau}
\iffileexists
\input \jobname.arnau
\ifrewritearnaufile\openarnaufile\fi
\fi}
\readarnaufile
\openarnaufile
\def\ifDefinedCs#1{\expandafter
\ifx\csname#1\endcsname\relax}
%%%%%%%%%
%%%%%%%%%DEFINIM LA HEADLINE PER DUES LINIES
%%%%%%%%%
\advance\voffset by 2\baselineskip
\advance\vsize by -2\baselineskip
%%%%
%%%%DEFINIM LA CAPSALERA
%%%%
\def\capsalera#1{
\def\makeheadline{\vbox to 0
pt{\vss\noindent {
\ifodd \pageno
							{\hfill{\sevenrm	#1}\ \ {\bf\the\pageno}}
\else
							{\ifnum\tit=0
													{{\bf\the\pageno}\ \ {\sevenrm Problemes d'estad\'{\i}stica. Enginyeries T\`ecniques de
Inform\`atica de Gesti\'o i de Sistemes}\hfill}
						 \else
															{{\bf\the\pageno}\ \ {\sevenrm 
																						{\part}}\hfill}
							\fi}
\fi}
\break\vskip -0.3cm
\hrule}
\vskip\baselineskip
}
}
\def\capsaleraDos#1{
\def\makeheadline{\vbox to 0
pt{\vss\noindent {
							{\hfill{\sevenrm	#1}\ \ {\bf\the\pageno}}
}
\break\vskip -0.3cm
\hrule}
\vskip\baselineskip
}
}
%%ACABA LA DEFINICIO
\footline{}
\def\seccionrep#1{{\ftitolp {\uppercase\expandafter{\romannumeral
\the\tit}} .\ #1}}
%
\def\seccion#1#2{\seccio=0\global\subtit=0\global\advance\tit by
1\vfill\eject\ \vskip8cm\rightline{\ftitol  {\uppercase\expandafter{\romannumeral
\the\tit}} .\
#1}\vskip1cm\rightline{\ftitol #2}\def\part{#1#2}\def\makeheadline{}
\vfill\eject{\noindent\seccionrep{#1#2}}\capsalera{#1#2}}
\def\seccionDos#1#2{\seccio=0\global\subtit=0\global\advance\tit by
1\vfill\eject\ \vskip8cm\rightline{\ftitol
{\uppercase\expandafter{\romannumeral
\the\tit}} .\
#1}\vskip1cm\rightline{\ftitol #2}\def\makeheadline{}
\vfill\eject{\noindent\seccionrep{#1#2}}\capsaleraDos{#1#2}}
%%%%
%%%%
\def\sp#1{\vskip#1ex}    
\def\fsp#1{\vskip#1ex\noindent}
\def\R{I\!\!R}
\font\mitjana =cmbx10  
\font\gran=cmr12 scaled\magstep 2
\newcount\seccio
\newcount\subseccio
\newcount\subsubseccio
\newcount\subsubsubseccio
\newcount\noucont
\subsubsubseccio=0
\subsubseccio=0
\subseccio=0
\seccio = 0
 \def\pr
#1{\global\advance\seccio by
1\subseccio=0\subsubseccio=0\subsubsubseccio=0
\eqnumber=0\teorema=0\proposicio=0\lema=0\corolari=0\problema=0\sp3{ 
\bf\ftitolpp\the\seccio . \  #1}\sp3 
\capsalera{#1}
\writenumberedtocentry{section}{#1}{\the\seccio}
} 
 \def\prDos
#1{\global\advance\seccio by
1\subseccio=0\subsubseccio=0\subsubsubseccio=0
\eqnumber=0\teorema=0\proposicio=0\lema=0\corolari=0{ 
\bf\mitjana\the\seccio . \  #1} 
\capsaleraDos{#1}
\writenumberedtocentry{section}{#1}{\the\seccio}
} 
\def\eop{\vrule
height 4pt width 6pt depth2pt}
\def\subpr#1{\global\advance\subseccio by
1\subsubseccio=0\subsubsubseccio=0\sp2{\bf \the\seccio
.\the\subseccio\ #1}\sp2
\writenumberedtocentry{subsection}{#1}{\the\subseccio}} 
\def\subsubpr#1{\global\advance\subsubseccio by
1\subsubsubseccio=0\sp1{\bf \the\seccio .\the\subseccio
.\the\subsubseccio\ #1}\sp1} 
\def\subsubsubpr#1{\global\advance\subsubsubseccio by 1\sp1{\bf
\the\seccio .\the\subseccio .\the\subsubseccio .\the\subsubsubseccio\
#1}\sp1}  
\def\teor#1#2#3{\advance\teorema by 1 
\ifnum\seccio=0
\sp2{\bf #2 \the\teorema} {\it #3} \immediate\write\arnaufile{
\csname def\endcsname
\expandafter
\noexpand
\csname teor#1\endcsname
\string{\the\teorema\string}}\sp2
\else
\sp2{\bf #2 \the\seccio
.\the\teorema} {\it #3} \immediate\write\arnaufile{
\csname def\endcsname
\expandafter
\noexpand
\csname teor#1\endcsname
\string{\the\seccio .\the\teorema\string}
}\sp2
\fi}
\def\refteor#1{\ifDefinedCs{teor#1}teorema (?)\else teorema \csname
teor#1\endcsname\fi} 
\def\prop#1#2{\advance\proposicio by 1 
\ifnum\seccio=0
{\bf Proposici\'o \the\proposicio} {\it #2}
\immediate\write\arnaufile{
\csname def\endcsname
\expandafter
\noexpand
\csname prop#1\endcsname
\string{\the\proposicio\string}}
\else
{\bf Proposici\'o \the\seccio
.\the\proposicio} {\it #2}
\immediate\write\arnaufile{
\csname def\endcsname
\expandafter
\noexpand
\csname prop#1\endcsname
\string{\the\seccio .\the\proposicio\string}
}
\fi}
\def\refprop#1{\ifDefinedCs{prop#1}proposicio (?)\else proposicio
\csname prop#1\endcsname\fi}
 \def\lem#1#2{\advance\lema by 1 
\ifnum\seccio=0
{\bf Lema \the\lema} {\it #2}
\immediate\write\arnaufile{
\csname def\endcsname
\expandafter
\noexpand
\csname lem#1\endcsname
\string{\the\lema\string}}
\else
{\bf Lema
\the\seccio .\the\lema} {\it #2}
\immediate\write\arnaufile{
\csname def\endcsname
\expandafter
\noexpand
\csname lem#1\endcsname
\string{\the\seccio .\the\lema\string}}
\fi
}
\def\reflem#1{\ifDefinedCs{lem#1}lema (?)\else lema \csname
lem#1\endcsname\fi}
 \def\cor#1#2{\advance\corolari by 1
\ifnum\seccio=0
{\bf
Corol.lari \the\corolari} {\it #2}
\immediate\write\arnaufile{
\csname def\endcsname
\expandafter
\noexpand
\csname cor#1\endcsname
\string{\the\corolari\string}}
\else
{\bf
Corol.lari \the\seccio .\the\corolari} {\it #2}
\immediate\write\arnaufile{
\csname def\endcsname
\expandafter
\noexpand
\csname cor#1\endcsname
\string{\the\seccio .\the\corolari\string}}
\fi
}
\def\refcor#1{\ifDefinedCs{cor#1}corol.lari (?)\else corol.lari \csname
cor#1\endcsname\fi}
 \overfullrule=0pt 
\def\Q{\hbox{Q\kern -.64em 
{\raise .45ex \hbox{$\scriptscriptstyle |$}}
  \kern-.55em {\raise .53ex \hbox{$\scriptscriptstyle |$}} }}
\def\CC{\hbox{C\kern -.58em {\raise .54ex \hbox{$\scriptscriptstyle |$}}
  \kern-.55em {\raise .53ex \hbox{$\scriptscriptstyle |$}} }}
\def\md#1{\vert #1 \vert}
\def\mdg#1{\Bigl\vert #1\Bigr\vert}
\def\mdd#1{\parallel #1\parallel}
\def\mddg#1{\Bigl\parallel #1\Bigr\parallel}
\def\ssi#1{\sin\Bigl({#1\over\epsilon}\Bigr)}
\def\cco#1{\cos\Bigl({#1\over\epsilon}\Bigr)}
\def\lln#1{\ln\Bigl({#1\over\epsilon}\Bigr)}
\def\eqconstruct#1{\ifnum\seccio=0 #1\else\the\seccio.#1\fi}
\def\somb#1{{\buildrel {\sim}\over {#1}}}
%%%%
%%%%	DEFINICIONS PER CONSTRUIR EL INDEX
%%%%
\def\tocchapterentry#1#2#3{{\noindent{\bf
{\uppercase\expandafter{\romannumeral #2}}. #1}\dotfill #3\par}\tit=0}
\def\tocsectionentry#1#2#3{{\indent #2. #1\dotfill #3\par}\advance\tit by
1}  
\def\tocsubsectionentry#1#2#3{{\indent\indent \the\tit .#2. #1\dotfill
#3\par}} 
%%%%%%
%%%%%%
%%%%%%		DEFINICIONS PER FER LLISTES DE PROBLEMES
%%%%%%		FER SERVIR LA MACRO \prob
%%%%%%
%%%%%%
\newcount\problema
\problema=0
\def\prob#1#2{\advance\problema by 1 
\ifnum\seccio=0
{\bf Problema \the\problema .-} {#2}
\immediate\write\arnaufile{
\csname def\endcsname
\expandafter
\noexpand
\csname prob#1\endcsname
\string{\the\problema\string}}
\else
{\bf Problema
\the\seccio .\the\problema .-} {#2}
\immediate\write\arnaufile{
\csname def\endcsname
\expandafter
\noexpand
\csname prob#1\endcsname
\string{\the\seccio .\the\problema\string}}
\fi
}
\def\refprob#1{\ifDefinedCs{prob#1}problema (?)\else problema \csname
prob#1\endcsname\fi}
%%%%%%
%%%%%%MACROS PER CREAR LA TAULA DE GLOSARIS
%%%%%%
\newwrite\glosari
\immediate\openout\glosari=\jobname.glos
\def\g#1{\immediate\write\glosari{#1}\immediate\write\glosari{\the\pageno}}
