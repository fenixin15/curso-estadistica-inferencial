\chapter{Pr\`actiques d'estad\'{\i}stica proposades}

En aquest cap\'{\i}tol s'exposaran una s\`erie de pr\`actiques que 
l'alumne pot realitzar al llarg del curs.

Podem dir que una pr\`actica \'es un problema en qu\`e necessita l'ajut d'un
ordinador per resoldre'l.

La pr\`actica ha de constar dels fitxers seg\"uents:
\begin{enumerate}
\item Programa font. 
Fitxer en format ASCII 
on hi ha el programa font de 
la pr\`actica.

\item Dades. Fitxer en format ASCII 
on l'usuari ha de posar les dades del problema.

\item Executable. Fitxer 
binari on hi ha l'executable, o sigui, el
programa font una vegada compilat 
i enlla\c{c}at.

\item Ajut. Fitxer en format ASCII 
on hi haur\`a tota la informaci\'o de
com s'ha realitzat la pr\`actica. B\`asicament s'hi ha de trobar:
\begin{itemize}
\item[a)] Software que s'ha fet servir en la realitzaci\'o de la
pr\`actica.
\item[b)] Format del fitxer de dades per poder executar la pr\`actica. 
O sigui, com s'han d'entrar les dades perqu\`e el fitxer 
executable les entengui.
\item[c)] Resultats obtenguts amb algun exemple.
\end{itemize}
Aquest darrer fitxer s'ha d'entregar impr\`es.
\end{enumerate} 

La pr\`actica s'avaluar\`a executant el fitxer 
executable i vegent els resultats, tenint en compte els punts seg\"uents:
\begin{itemize}
\item Presentaci\'o.
\item Capacitat d'emmagatzematge de dades.
\item Resultats.
\end{itemize}

\section{Estad\'{\i}stica descriptiva
}

\subsection{Introducci\'o}

Sigui $X_1,\ldots,X_N$ un conjunt de dades.

\begin{itemize}
\item Feu la distribuci\'o 
de freq\"u\`encies amb 
intervals de classe amb l\'{\i}mits reals
 i l\'{\i}mits aparents
 d'amplada fixada $a$. 

Recordau que per trobar l'extrem de l'esquerra del primer 
interval, heu de trobar primer el m\'{\i}nim de 
la distribuci\'o.

\item Feu l'histograma de les freq\"u\`encies
 relatives i les 
freq\"u\`encies
relatives acumulades.

\item Siguin $X_{1,r},\ldots,X_{I,r}$ les marques 
de classe de la variable agrupada amb l\'{\i}mits 
reals i $X_{1,a},\ldots,X_{I,a}$
les marques de classe de la variable agrupada
 amb l\'{\i}mits 
aparents. Calculau:
\[
\begin{array}{rlcrl}
\overline{X}_r = & \frac{\sum\limits_{i=1}^I n_i X_{i,r}}{N}, 
& &\overline{X}_a = & \frac{\sum\limits_{i=1}^I n_i X_{i,a}}{N}, \\
S_r^2 = & \frac{\sum\limits_{i=1}^I n_i X_{i,r}^2}{N}- 
\overline{X}_r^2,
& & S_a^2 = & \frac{\sum\limits_{i=1}^I n_i X_{i,a}^2}{N}- 
\overline{X}_a^2.
\end{array}
\]
\end{itemize}

\subsection{Definici\'o de les variable aleat\`ories discretes
}

\subsection*{Variable aleat\`oria 
discreta per a l\'{\i}mits reals}

Considerem la variable aleat\`oria discreta
 $X_r^{(d)}$ tal que 
el seu rang \'es:
\[
X_r^{(d)} (\Omega) =\{L_1,\ldots,L_I\},
\]
on els valors~$L_j$ s\'on els extrems dels intervals a considerar.

La funci\'o de distribuci\'o de $X_r^{(d)}$ 
s'obt\'e a partir de les freq\"u\`encies
 relatives 
acumulades de la variable agrupada
en la secci\'o anterior.

Suposem que la distribuci\'o de 
freq\"u\`encies de la variable
agrupada es troba en la taula \ref{TAULAVARIABLEAGRUPADAREAL}.
\begin{table}[h]
\begin{center}
\begin{tabular}{|c|c|c|c|}
\hline
Intervals&$X_{i,r}$&$f_i$&$F_i$\\\hline\hline
$[L_1,L_2)$&$X_{1,r}$&$f_1$&$F_1$\\\hline
$[L_2,L_3)$&$X_{2,r}$&$f_2$&$F_2$\\\hline
$\vdots$&$\vdots$&$\vdots$&$\vdots$\\\hline
$[L_I,L_{I+1})$&$X_{I,r}$&$f_I$&$F_I$\\\hline
\end{tabular}
\end{center}
\caption{Taula de les freq\"u\`encies
 relatives de la variable 
agrupada amb l\'{\i}mits reals.}

\label{TAULAVARIABLEAGRUPADAREAL}
\end{table}

La funci\'o de distribuci\'o de la variable 
aleat\`oria $X_r^{(d)}$ ser\`a:
\[
F_{X_r^{(d)}}(x)=  
\left\{
\begin{array}{ll}
0, & \mbox{si $x< L_1$,} \\
F_1, & \mbox{si $L_1\leq x < L_2$,} \\
F_2, & \mbox{si $L_2\leq x < L_3$,} \\
\vdots & \vdots \\
1, & \mbox{si $x\geq L_{I+1}$.}
\end{array}
\right.
\]

Trobeu $\EE (X_r^{(d)})$ i $\mbox{Var }(X_r^{(d)})$.

\subsection*{Variable aleat\`oria discreta per a l\'{\i}mits aparents}

Feu el mateix que hem fet en la subsecci\'o anterior per\`o ara 
prenent l\'{\i}mits aparents.

O sigui, definim la variable aleat\`oria $X_a^{(d)}$ amb rang:
\[
X_a^{(d)} (\Omega) =\{L_1,\ldots,L_I\}.
\]
La funci\'o de distribuci\'o de $X_a^{(d)}$ 
es defineix de la mateixa manera que la funci\'o de distribuci\'o
 de $X_r^{(d)}$ per\`o
tenint en compte les freq\"u\`encies
 relatives 
acumulades de la variable agrupada amb l\'{\i}mits 
aparents.

\subsection*{Q\"uestions}
\begin{itemize}
\item[a)] Quins dels valors $\EE (X_r^{(d)})$ o $\EE (X_a^{(d)})$ s'aproximen
m\'es a $\overline{X}=\frac{\sum\limits_{i=1}^N X_i}{N}$?

\item[b)] Quins dels valors $\mbox{Var }(X_r^{(d)})$ o $\mbox{Var }
(X_a^{(d)})$ s'aproximen m\'es a \break
$S^2 =\frac{\sum\limits_{i=1}^N X_i^2}{N}-
\overline{X}^2$?

\item[c)] Feu els gr\`afics de les funcions de distribuci\'o
 de les variables
aleat\`ories 
$X_r^{(d)}$ i $X_a^{(d)}$.
\end{itemize}

\subsection{Definici\'o de les variables aleat\`ories
cont\'{\i}nues}

\subsection*{Variable aleat\`oria
cont\'{\i}nua per a l\'{\i}mits reals}

Ara definirem la variable aleat\`oria cont\'{\i}nua
 $X_r^{(c)}$.

Tenint en compte que la taula de les freq\"u\`encies relatives i 
acumulades es troba en la 
taula~\ref{TAULAVARIABLEAGRUPADAREAL}, la
funci\'o de distribuci\'o de 
$X_r^{(c)}$ est\`a definida de la 
manera seg\"uent:

\begin{itemize}
\item[] Si $x\leq L_1$, $F_{X_r^{(c)}}(x)=0$.

\item[] Si $L_1 < x\leq L_2$, $F_{X_r^{(c)}}(x)$ val la imatge que li 
correspon en la recta que uneix els punts $(L_1,0)$ i $(L_2,F_1)$.

\item[] Si $L_2 <x \leq L_3$, $F_{X_r^{(c)}}(x)$ val la imatge que li
correspon en la recta que uneix els punts $(L_2,F_1)$ i $(L_3,F_2)$.

\item[] $\vdots$

\item[] Si $x\geq L_{I+1}$, $F_{X_r^{(c)}}(x)$ val $1$.

\end{itemize}

O sigui, la funci\'o de distribuci\'o
 est\`a constru\"{\i}da com a interpolaci\'o
lineal de les 
freq\"u\`encies relatives acumulades com imatges dels intervals 
de classe. Vegeu el gr\`afic \ref{VARCONTREAL} per entendre
millor la construcci\'o.

\begin{figure}
%\setlength{\unitlength}{0.0125in}%
\setlength{\unitlength}{0.00825in}%
\begin{center}
\begin{picture}(500,350)(40,395)
\thinlines
\put(120,415){\line( 0, 1){  0}}
\put(120,415){\line( 0, 1){  5}}
\put(180,415){\line( 0, 1){  5}}
\put(240,415){\line( 0, 1){  5}}
\put(300,415){\line( 0, 1){  5}}
\put(355,415){\line( 0, 1){  5}}
\put( 80,745){\line( 0,-1){330}}
\put( 80,415){\line( 1, 0){460}}
\put( 80,440){\line( 1, 0){  5}}
\put( 80,560){\line( 1, 0){  5}}
\put( 80,650){\line( 1, 0){  5}}
\put(120,415){\line( 5, 2){ 60.345}}
\put(180,440){\line( 3, 4){ 60}}
\put(240,520){\line( 3, 2){ 60}}
\put(300,560){\line( 3, 5){ 50.735}}
\put(350,645){\line( 1, 0){190}}
\put( 80,520){\line( 1, 0){  5}}
\put(110,395){\makebox(0,0)[lb]{\raisebox{0pt}[0pt][0pt]{$L_1$}}}
\put(170,395){\makebox(0,0)[lb]{\raisebox{0pt}[0pt][0pt]{$L_2$}}}
\put(230,395){\makebox(0,0)[lb]{\raisebox{0pt}[0pt][0pt]{$L_3$}}}
\put(290,395){\makebox(0,0)[lb]{\raisebox{0pt}[0pt][0pt]{$L_4$}}}
\put(345,395){\makebox(0,0)[lb]{\raisebox{0pt}[0pt][0pt]{$L_5$}}}
\put( 50,435){\makebox(0,0)[lb]{\raisebox{0pt}[0pt][0pt]{$F_1$}}}
\put( 50,555){\makebox(0,0)[lb]{\raisebox{0pt}[0pt][0pt]{$F_3$}}}
\put( 50,515){\makebox(0,0)[lb]{\raisebox{0pt}[0pt][0pt]{$F_2$}}}
\put( 60,645){\makebox(0,0)[lb]{\raisebox{0pt}[0pt][0pt]{$1$}}}
\end{picture}
\end{center}
\caption{Gr\`afic de la funci\'o de distribuci\'o per a la
variable aleat\`oria $X_{r}^{(c)}$}
\label{VARCONTREAL}
\end{figure}

\subsection*{Variable aleat\`oria
cont\'{\i}nua per a l\'{\i}mits aparents}

Feu el mateix que hem fet en la subsecci\'o anterior per\`o ara 
prenent l\'{\i}mits aparents.

O sigui, definim la variable aleat\`oria $X_a^{(c)}$ on la
funci\'o de distribuci\'o de $X_a^{(c)}$ 
es defineix de la mateixa manera que la funci\'o de distribuci\'o
 de $X_r^{(c)}$ per\`o
tenint en compte les freq\"u\`encies
relatives acumulades de la variable agrupada 
amb l\'{\i}mits aparents.

\subsection*{Q\"uestions}
\begin{itemize}
\item[a)] Quins dels valors $\EE (X_r^{(c)})$ o $\EE (X_a^{(c)})$ s'aproximen
m\'es a $\overline{X}=\frac{\sum\limits_{i=1}^N X_i}{N}$?

\item[b)] Quins dels valors $\mbox{Var }(X_r^{(c)})$ o $\mbox{Var }
(X_a^{(c)})$ s'aproximen m\'es a \break $S^2 =\frac{\sum\limits_{i=1}^N 
X_i^2}{N}-\overline{X}^2$?
\enlargethispage*{1000pt}
\item[c)] Feu els gr\`afics de les funcions de distribuci\'o
 de les variables
aleat\`ories 
$X_r^{(c)}$ i $X_a^{(c)}$.
\end{itemize}

\newpage
\section{Teorema del l\'{\i}mit central}

\subsection{Generaci\'o de la mostra aleat\`oria simple}

Sigui $X$ una variable aleat\`oria  amb funci\'o de distribuci\'o coneguda. 
Per exemple $X$ pot ser:
\begin{description}
\item[$Poiss(\lambda)$] Variable aleat\`oria de Poisson 
amb par\`ametre
$\lambda$.

\item[$B(n,p)$] Variable aleat\`oria 
binomial amb par\`ametres 
$n$ i $p$.

\item[$\mbox{Exp }(\lambda)$] Variable aleat\`oria exponencial amb par\`ametre
$\lambda$.

\item[$N(\mu,\sigma^2)$] Variable aleat\`oria
 normal amb par\`ametres 
$\mu$ i $\sigma^2$.

\item[$\chi^2_n$] Variable aleat\`oria
 khi quadrat 
amb $n$ graus de llibertat.

\end{description}

Generau una mostra aleat\`oria 
simple de grand\`aria $N$ per a una variable aleat\`oria~$X$ amb qualsevol de 
les funcions de distribuci\'o descrites
anteriorment. Vegeu el problema~\ref{PROBGENMOSTRES} com a ajuda de com generar l'esmentada 
mostra.

\subsection{Comprovaci\'o que la mostra anterior correspon a la 
va\-ria\-ble aleat\`oria $X$}

Vegeu mitjan\c{c}ant el test $\chi^2$ i mitjan\c{c}ant el test
de Kolmogorov-Smirnov que la mostra anterior 
correspon a la variable aleat\`oria $X$.

O sigui, trobeu l'error tipus I m\`axim 
$\alpha_{\mbox{\footnotesize m\`ax}}$ per 
poder acceptar que la mostra anterior correspon a la
variable aleat\`oria $X$ fent servir els dos tests esmentats anteriorment.

\subsection{Definici\'o de la funci\'o de distribuci\'o
emp\'{\i}rica}

Sigui $X_1,\ldots,X_N$ la mostra aleat\`oria
 simple generada en
l'apartat anterior.

Definim una nova mostra aleat\`oria
 simple $Y_1,\ldots,Y_N$ a 
partir de la mostra anterior:
\[
Y_i = \frac{\overline{X}_i -\mu}{\frac{\sigma}{\sqrt{i}}}, 
\]
on
\[
\overline{X}_i =  \frac{\sum\limits_{j=1}^i X_j}{i},\quad 
\mu =  \EE (X), \quad
\sigma =  \sqrt{\mbox{Var }(X)}.
\]

Considerem la funci\'o de distribuci\'o
 emp\'{\i}rica 
a partir de la mostra anterior definida com:
\[
F_{Y}^{(N)}(t) = \frac{\#\{i\ |\ Y_i\leq t\}}{N}, \mbox{ on el 
s\'{\i}mbol $\#$ indica cardinal d'un conjunt.}
\]

Fixat $t$, trobeu:
\[  
\lim_{N\to\infty} F_{Y}^{(N)}(t) := L_t.
\]

Q\"uesti\'o:

\'Es cert que:
\[
L_t = \frac{1}{\sqrt{2\pi}}\int_{-\infty}^t e^{-\frac{t^2}{2}}\, dt =
0.5+\frac{1}{\sqrt{2\pi}}\int_{0}^t e^{-\frac{t^2}{2}}\, dt?
\]
\section{Estudi de la fiabilitat d'un examen tipus test}

\subsection{Introducci\'o}

Considerem un examen tipus test amb $N$ preguntes, realitzat per 
$M$ alumnes.

Suposem que cada pregunta t\'e en total $5$ respostes i que cada pregunta
t\'e la mateixa puntuaci\'o per a la nota final.

Si l'alumne contesta b\'e la pregunta, li donam $1$ punt damunt $N$ en la
nota final; si la contesta malament, li restam $0.25$ punts damunt $N$ i si
contesta en blanc, no li donam cap punt.

Considerem la matriu $A$ on tenim el resultat de l'examen:
\[  
A=\pmatrix{
a_{11}&\cdots&a_{1N}\cr
a_{21}&\cdots&a_{2N}\cr
\vdots&\ddots&\vdots\cr
a_{M1}&\cdots&a_{MN}},
\]
on $a_{ij}$ representa el que ha contestat l'alumne $i$ a la pregunta $j$.

Amb vista a fixar idees, suposeu que la resposta verdadera \'es sempre
la primera. O sigui, si l'examen d'un determinat alumne \'es:
\[
1,1,1,1\ldots,1,
\]
t\'e un $N$ damunt $N$ de nota final o un $10$ damunt $10$.

\subsection{Estudi de la independ\`encia
 de la variable {\it Resposta} i
de la variable {\it Nota}}

Per a cada pregunta $P_i$ ($i=1,\ldots,N$) considerem la taula seg\"uent:

\begin{center}
\begin{tabular}{|c||c|c|c|c|c|c|}
\hline
Nota$\backslash$Resposta&1&2&3&4&5&0\\\hline\hline
$[0,1)$&$b_{01}$&$b_{02}$&$b_{03}$&$b_{04}$&$b_{05}$&$b_{06}$\\\hline
$[1,2)$&$b_{11}$&$b_{12}$&$b_{13}$&$b_{14}$&$b_{15}$&$b_{16}$\\\hline
$[2,3)$&$b_{21}$&$b_{22}$&$b_{23}$&$b_{24}$&$b_{25}$&$b_{26}$\\\hline
$[3,4)$&$b_{31}$&$b_{32}$&$b_{33}$&$b_{34}$&$b_{35}$&$b_{36}$\\\hline
$[4,5)$&$b_{41}$&$b_{42}$&$b_{43}$&$b_{44}$&$b_{45}$&$b_{46}$\\\hline
$[5,6)$&$b_{51}$&$b_{52}$&$b_{53}$&$b_{54}$&$b_{55}$&$b_{56}$\\\hline
$[6,7)$&$b_{61}$&$b_{62}$&$b_{63}$&$b_{64}$&$b_{65}$&$b_{66}$\\\hline
$[7,8)$&$b_{71}$&$b_{72}$&$b_{73}$&$b_{74}$&$b_{75}$&$b_{76}$\\\hline
$[8,9)$&$b_{81}$&$b_{82}$&$b_{83}$&$b_{84}$&$b_{85}$&$b_{86}$\\\hline
$[9,10)$&$b_{91}$&$b_{92}$&$b_{93}$&$b_{94}$&$b_{95}$&$b_{96}$\\\hline
\end{tabular}
\end{center}
on $b_{ij}$ \'es la quantitat d'alumnes que han contestat la 
resposta $j$ i han tret una nota entre $i$ i $i+1$.

La partici\'o que hem fet de la variable 
{\it Nota} en la taula anterior \'es d'intervals d'amplada $1$. 
Per fer la pr\`actica, s'han de considerar altres particions 
per a amplades qualsevols.

Vegeu a partir del test $\chi^2$ si la variable {\it Resposta}
\'es independent de la variable {\it Nota}. O sigui, 
trobau l'error tipus~I m\`axim ($\alpha_{\mbox{\footnotesize m\`ax}}$) 
a partir del qual acceptam independ\`encia.


\subsection{Estudi de la dificultat i 
coher\`encia de cada pregunta}

Per a cada pregunta $P_i$ ($i=1,\ldots,N$), constru\"{\i}u la 
taula seg\"uent:

\begin{center}
\begin{tabular}{|c||c@{}c@{}c@{}c@{}c@{}c@{}c@{}c@{}c@{}c@{}|}
%\begin{tabular}{|c||cccccccccc|}
\hline
Nota&$[0,1)$&$[1,2)$&$[2,3)$&$[3,4)$&$[4,5)$&$[5,6)$&$[6,7)$&$[7,8)$&$[8,9)$&$[9,10)$\\\hline\hline
&$p_0^{(i)}\%$& $p_1^{(i)}\%$&$p_2^{(i)}\%$&$p_3^{(i)}\%$&$p_4^{(i)}\%$&$p_5^{(i)}\%$&$p_6^{(i)}\%$&$p_7^{(i)}\%$&$p_8^{(i)}\%$&$p_9^{(i)}\%$\\\hline
\end{tabular}
\end{center}
on $p_j^{(i)}\%$ \'es el tant per cent de gent que ha contestat b\'e
la pregunta $P_i$ i ha tengut una nota entre $j$ i $j+1$. O sigui, 
$p_j^{(i)}/100$ \'es el quocient entre la quantitat d'alumnes que han
contestat b\'e la pregunta~$P_i$ i han tengut una nota entre~$j$ i
$j+1$ i la quantitat d'alumnes que han contestat b\'e la
pregunta~$P_i$. S'ha de complir, doncs, que $\sum\limits_{j=0}^9 
p_j^{(i)} =100$.

Feu el gr\`afic de la variable $p_j^{(i)}\%$ com a funci\'o de la 
variable {\it Nota} per a cada pregunta $P_i$ i uniu de forma lineal
els punts que surten. Vegeu gr\`afic
\ref{PICENTNOTA} per entendre com ha de ser el gr\`afic demanat.
Agafeu les marques de classe dels 
intervals agrupats de la variable
{\it Nota} per representar el gr\`afic.

\begin{figure}
% GNUPLOT: LaTeX picture
\setlength{\unitlength}{0.240900pt}
\ifx\plotpoint\undefined\newsavebox{\plotpoint}\fi
\sbox{\plotpoint}{\rule[-0.200pt]{0.400pt}{0.400pt}}%
\begin{picture}(1500,900)(0,0)
\font\gnuplot=cmr10 at 10pt
\gnuplot
\sbox{\plotpoint}{\rule[-0.200pt]{0.400pt}{0.400pt}}%
\put(176.0,68.0){\rule[-0.200pt]{303.534pt}{0.400pt}}
\put(176.0,68.0){\rule[-0.200pt]{0.400pt}{194.888pt}}
\put(176.0,68.0){\rule[-0.200pt]{303.534pt}{0.400pt}}
\put(176.0,68.0){\rule[-0.200pt]{4.818pt}{0.400pt}}
\put(154,68){\makebox(0,0)[r]{0}}
\put(1416.0,68.0){\rule[-0.200pt]{4.818pt}{0.400pt}}
\put(176.0,108.0){\rule[-0.200pt]{303.534pt}{0.400pt}}
\put(176.0,108.0){\rule[-0.200pt]{4.818pt}{0.400pt}}
\put(154,108){\makebox(0,0)[r]{5}}
\put(1416.0,108.0){\rule[-0.200pt]{4.818pt}{0.400pt}}
\put(176.0,149.0){\rule[-0.200pt]{303.534pt}{0.400pt}}
\put(176.0,149.0){\rule[-0.200pt]{4.818pt}{0.400pt}}
\put(154,149){\makebox(0,0)[r]{10}}
\put(1416.0,149.0){\rule[-0.200pt]{4.818pt}{0.400pt}}
\put(176.0,189.0){\rule[-0.200pt]{303.534pt}{0.400pt}}
\put(176.0,189.0){\rule[-0.200pt]{4.818pt}{0.400pt}}
\put(154,189){\makebox(0,0)[r]{15}}
\put(1416.0,189.0){\rule[-0.200pt]{4.818pt}{0.400pt}}
\put(176.0,230.0){\rule[-0.200pt]{303.534pt}{0.400pt}}
\put(176.0,230.0){\rule[-0.200pt]{4.818pt}{0.400pt}}
\put(154,230){\makebox(0,0)[r]{20}}
\put(1416.0,230.0){\rule[-0.200pt]{4.818pt}{0.400pt}}
\put(176.0,270.0){\rule[-0.200pt]{303.534pt}{0.400pt}}
\put(176.0,270.0){\rule[-0.200pt]{4.818pt}{0.400pt}}
\put(154,270){\makebox(0,0)[r]{25}}
\put(1416.0,270.0){\rule[-0.200pt]{4.818pt}{0.400pt}}
\put(176.0,311.0){\rule[-0.200pt]{303.534pt}{0.400pt}}
\put(176.0,311.0){\rule[-0.200pt]{4.818pt}{0.400pt}}
\put(154,311){\makebox(0,0)[r]{30}}
\put(1416.0,311.0){\rule[-0.200pt]{4.818pt}{0.400pt}}
\put(176.0,351.0){\rule[-0.200pt]{303.534pt}{0.400pt}}
\put(176.0,351.0){\rule[-0.200pt]{4.818pt}{0.400pt}}
\put(154,351){\makebox(0,0)[r]{35}}
\put(1416.0,351.0){\rule[-0.200pt]{4.818pt}{0.400pt}}
\put(176.0,392.0){\rule[-0.200pt]{303.534pt}{0.400pt}}
\put(176.0,392.0){\rule[-0.200pt]{4.818pt}{0.400pt}}
\put(154,392){\makebox(0,0)[r]{40}}
\put(1416.0,392.0){\rule[-0.200pt]{4.818pt}{0.400pt}}
\put(176.0,432.0){\rule[-0.200pt]{303.534pt}{0.400pt}}
\put(176.0,432.0){\rule[-0.200pt]{4.818pt}{0.400pt}}
\put(154,432){\makebox(0,0)[r]{45}}
\put(1416.0,432.0){\rule[-0.200pt]{4.818pt}{0.400pt}}
\put(176.0,473.0){\rule[-0.200pt]{303.534pt}{0.400pt}}
\put(176.0,473.0){\rule[-0.200pt]{4.818pt}{0.400pt}}
\put(154,473){\makebox(0,0)[r]{50}}
\put(1416.0,473.0){\rule[-0.200pt]{4.818pt}{0.400pt}}
\put(176.0,513.0){\rule[-0.200pt]{303.534pt}{0.400pt}}
\put(176.0,513.0){\rule[-0.200pt]{4.818pt}{0.400pt}}
\put(154,513){\makebox(0,0)[r]{55}}
\put(1416.0,513.0){\rule[-0.200pt]{4.818pt}{0.400pt}}
\put(176.0,553.0){\rule[-0.200pt]{303.534pt}{0.400pt}}
\put(176.0,553.0){\rule[-0.200pt]{4.818pt}{0.400pt}}
\put(154,553){\makebox(0,0)[r]{60}}
\put(1416.0,553.0){\rule[-0.200pt]{4.818pt}{0.400pt}}
\put(176.0,594.0){\rule[-0.200pt]{303.534pt}{0.400pt}}
\put(176.0,594.0){\rule[-0.200pt]{4.818pt}{0.400pt}}
\put(154,594){\makebox(0,0)[r]{65}}
\put(1416.0,594.0){\rule[-0.200pt]{4.818pt}{0.400pt}}
\put(176.0,634.0){\rule[-0.200pt]{303.534pt}{0.400pt}}
\put(176.0,634.0){\rule[-0.200pt]{4.818pt}{0.400pt}}
\put(154,634){\makebox(0,0)[r]{70}}
\put(1416.0,634.0){\rule[-0.200pt]{4.818pt}{0.400pt}}
\put(176.0,675.0){\rule[-0.200pt]{303.534pt}{0.400pt}}
\put(176.0,675.0){\rule[-0.200pt]{4.818pt}{0.400pt}}
\put(154,675){\makebox(0,0)[r]{75}}
\put(1416.0,675.0){\rule[-0.200pt]{4.818pt}{0.400pt}}
\put(176.0,715.0){\rule[-0.200pt]{303.534pt}{0.400pt}}
\put(176.0,715.0){\rule[-0.200pt]{4.818pt}{0.400pt}}
\put(154,715){\makebox(0,0)[r]{80}}
\put(1416.0,715.0){\rule[-0.200pt]{4.818pt}{0.400pt}}
\put(176.0,756.0){\rule[-0.200pt]{303.534pt}{0.400pt}}
\put(176.0,756.0){\rule[-0.200pt]{4.818pt}{0.400pt}}
\put(154,756){\makebox(0,0)[r]{85}}
\put(1416.0,756.0){\rule[-0.200pt]{4.818pt}{0.400pt}}
\put(176.0,796.0){\rule[-0.200pt]{303.534pt}{0.400pt}}
\put(176.0,796.0){\rule[-0.200pt]{4.818pt}{0.400pt}}
\put(154,796){\makebox(0,0)[r]{90}}
\put(1416.0,796.0){\rule[-0.200pt]{4.818pt}{0.400pt}}
\put(176.0,837.0){\rule[-0.200pt]{303.534pt}{0.400pt}}
\put(176.0,837.0){\rule[-0.200pt]{4.818pt}{0.400pt}}
\put(154,837){\makebox(0,0)[r]{95}}
\put(1416.0,837.0){\rule[-0.200pt]{4.818pt}{0.400pt}}
\put(176.0,877.0){\rule[-0.200pt]{303.534pt}{0.400pt}}
\put(176.0,877.0){\rule[-0.200pt]{4.818pt}{0.400pt}}
\put(154,877){\makebox(0,0)[r]{100}}
\put(1416.0,877.0){\rule[-0.200pt]{4.818pt}{0.400pt}}
\put(176.0,68.0){\rule[-0.200pt]{0.400pt}{194.888pt}}
\put(176.0,68.0){\rule[-0.200pt]{0.400pt}{4.818pt}}
\put(176,23){\makebox(0,0){0}}
\put(176.0,857.0){\rule[-0.200pt]{0.400pt}{4.818pt}}
\put(239.0,68.0){\rule[-0.200pt]{0.400pt}{194.888pt}}
\put(239.0,68.0){\rule[-0.200pt]{0.400pt}{4.818pt}}
\put(239,23){\makebox(0,0){0.5}}
\put(239.0,857.0){\rule[-0.200pt]{0.400pt}{4.818pt}}
\put(302.0,68.0){\rule[-0.200pt]{0.400pt}{194.888pt}}
\put(302.0,68.0){\rule[-0.200pt]{0.400pt}{4.818pt}}
\put(302,23){\makebox(0,0){1}}
\put(302.0,857.0){\rule[-0.200pt]{0.400pt}{4.818pt}}
\put(365.0,68.0){\rule[-0.200pt]{0.400pt}{194.888pt}}
\put(365.0,68.0){\rule[-0.200pt]{0.400pt}{4.818pt}}
\put(365,23){\makebox(0,0){1.5}}
\put(365.0,857.0){\rule[-0.200pt]{0.400pt}{4.818pt}}
\put(428.0,68.0){\rule[-0.200pt]{0.400pt}{194.888pt}}
\put(428.0,68.0){\rule[-0.200pt]{0.400pt}{4.818pt}}
\put(428,23){\makebox(0,0){2}}
\put(428.0,857.0){\rule[-0.200pt]{0.400pt}{4.818pt}}
\put(491.0,68.0){\rule[-0.200pt]{0.400pt}{194.888pt}}
\put(491.0,68.0){\rule[-0.200pt]{0.400pt}{4.818pt}}
\put(491,23){\makebox(0,0){2.5}}
\put(491.0,857.0){\rule[-0.200pt]{0.400pt}{4.818pt}}
\put(554.0,68.0){\rule[-0.200pt]{0.400pt}{194.888pt}}
\put(554.0,68.0){\rule[-0.200pt]{0.400pt}{4.818pt}}
\put(554,23){\makebox(0,0){3}}
\put(554.0,857.0){\rule[-0.200pt]{0.400pt}{4.818pt}}
\put(617.0,68.0){\rule[-0.200pt]{0.400pt}{194.888pt}}
\put(617.0,68.0){\rule[-0.200pt]{0.400pt}{4.818pt}}
\put(617,23){\makebox(0,0){3.5}}
\put(617.0,857.0){\rule[-0.200pt]{0.400pt}{4.818pt}}
\put(680.0,68.0){\rule[-0.200pt]{0.400pt}{194.888pt}}
\put(680.0,68.0){\rule[-0.200pt]{0.400pt}{4.818pt}}
\put(680,23){\makebox(0,0){4}}
\put(680.0,857.0){\rule[-0.200pt]{0.400pt}{4.818pt}}
\put(743.0,68.0){\rule[-0.200pt]{0.400pt}{194.888pt}}
\put(743.0,68.0){\rule[-0.200pt]{0.400pt}{4.818pt}}
\put(743,23){\makebox(0,0){4.5}}
\put(743.0,857.0){\rule[-0.200pt]{0.400pt}{4.818pt}}
\put(806.0,68.0){\rule[-0.200pt]{0.400pt}{194.888pt}}
\put(806.0,68.0){\rule[-0.200pt]{0.400pt}{4.818pt}}
\put(806,23){\makebox(0,0){5}}
\put(806.0,857.0){\rule[-0.200pt]{0.400pt}{4.818pt}}
\put(869.0,68.0){\rule[-0.200pt]{0.400pt}{194.888pt}}
\put(869.0,68.0){\rule[-0.200pt]{0.400pt}{4.818pt}}
\put(869,23){\makebox(0,0){5.5}}
\put(869.0,857.0){\rule[-0.200pt]{0.400pt}{4.818pt}}
\put(932.0,68.0){\rule[-0.200pt]{0.400pt}{194.888pt}}
\put(932.0,68.0){\rule[-0.200pt]{0.400pt}{4.818pt}}
\put(932,23){\makebox(0,0){6}}
\put(932.0,857.0){\rule[-0.200pt]{0.400pt}{4.818pt}}
\put(995.0,68.0){\rule[-0.200pt]{0.400pt}{194.888pt}}
\put(995.0,68.0){\rule[-0.200pt]{0.400pt}{4.818pt}}
\put(995,23){\makebox(0,0){6.5}}
\put(995.0,857.0){\rule[-0.200pt]{0.400pt}{4.818pt}}
\put(1058.0,68.0){\rule[-0.200pt]{0.400pt}{194.888pt}}
\put(1058.0,68.0){\rule[-0.200pt]{0.400pt}{4.818pt}}
\put(1058,23){\makebox(0,0){7}}
\put(1058.0,857.0){\rule[-0.200pt]{0.400pt}{4.818pt}}
\put(1121.0,68.0){\rule[-0.200pt]{0.400pt}{194.888pt}}
\put(1121.0,68.0){\rule[-0.200pt]{0.400pt}{4.818pt}}
\put(1121,23){\makebox(0,0){7.5}}
\put(1121.0,857.0){\rule[-0.200pt]{0.400pt}{4.818pt}}
\put(1184.0,68.0){\rule[-0.200pt]{0.400pt}{194.888pt}}
\put(1184.0,68.0){\rule[-0.200pt]{0.400pt}{4.818pt}}
\put(1184,23){\makebox(0,0){8}}
\put(1184.0,857.0){\rule[-0.200pt]{0.400pt}{4.818pt}}
\put(1247.0,68.0){\rule[-0.200pt]{0.400pt}{194.888pt}}
\put(1247.0,68.0){\rule[-0.200pt]{0.400pt}{4.818pt}}
\put(1247,23){\makebox(0,0){8.5}}
\put(1247.0,857.0){\rule[-0.200pt]{0.400pt}{4.818pt}}
\put(1310.0,68.0){\rule[-0.200pt]{0.400pt}{194.888pt}}
\put(1310.0,68.0){\rule[-0.200pt]{0.400pt}{4.818pt}}
\put(1310,23){\makebox(0,0){9}}
\put(1310.0,857.0){\rule[-0.200pt]{0.400pt}{4.818pt}}
\put(1373.0,68.0){\rule[-0.200pt]{0.400pt}{194.888pt}}
\put(1373.0,68.0){\rule[-0.200pt]{0.400pt}{4.818pt}}
\put(1373,23){\makebox(0,0){9.5}}
\put(1373.0,857.0){\rule[-0.200pt]{0.400pt}{4.818pt}}
\put(1436.0,68.0){\rule[-0.200pt]{0.400pt}{194.888pt}}
\put(1436.0,68.0){\rule[-0.200pt]{0.400pt}{4.818pt}}
\put(1436,23){\makebox(0,0){10}}
\put(1436.0,857.0){\rule[-0.200pt]{0.400pt}{4.818pt}}
\put(176.0,68.0){\rule[-0.200pt]{303.534pt}{0.400pt}}
\put(1436.0,68.0){\rule[-0.200pt]{0.400pt}{194.888pt}}
\put(176.0,877.0){\rule[-0.200pt]{303.534pt}{0.400pt}}
\put(176.0,68.0){\rule[-0.200pt]{0.400pt}{194.888pt}}
\put(239,108){\usebox{\plotpoint}}
\multiput(365.00,108.58)(1.545,0.498){79}{\rule{1.329pt}{0.120pt}}
\multiput(365.00,107.17)(123.241,41.000){2}{\rule{0.665pt}{0.400pt}}
\multiput(491.00,149.58)(1.584,0.498){77}{\rule{1.360pt}{0.120pt}}
\multiput(491.00,148.17)(123.177,40.000){2}{\rule{0.680pt}{0.400pt}}
\multiput(617.00,187.92)(1.584,-0.498){77}{\rule{1.360pt}{0.120pt}}
\multiput(617.00,188.17)(123.177,-40.000){2}{\rule{0.680pt}{0.400pt}}
\multiput(743.00,147.92)(1.545,-0.498){79}{\rule{1.329pt}{0.120pt}}
\multiput(743.00,148.17)(123.241,-41.000){2}{\rule{0.665pt}{0.400pt}}
\put(239.0,108.0){\rule[-0.200pt]{30.353pt}{0.400pt}}
\multiput(995.00,108.58)(0.778,0.499){159}{\rule{0.722pt}{0.120pt}}
\multiput(995.00,107.17)(124.501,81.000){2}{\rule{0.361pt}{0.400pt}}
\multiput(1121.00,187.92)(1.584,-0.498){77}{\rule{1.360pt}{0.120pt}}
\multiput(1121.00,188.17)(123.177,-40.000){2}{\rule{0.680pt}{0.400pt}}
\multiput(1247.00,149.58)(0.778,0.499){159}{\rule{0.722pt}{0.120pt}}
\multiput(1247.00,148.17)(124.501,81.000){2}{\rule{0.361pt}{0.400pt}}
\put(239,108){\raisebox{-.8pt}{\makebox(0,0){$\bullet$}}}
\put(365,108){\raisebox{-.8pt}{\makebox(0,0){$\bullet$}}}
\put(491,149){\raisebox{-.8pt}{\makebox(0,0){$\bullet$}}}
\put(617,189){\raisebox{-.8pt}{\makebox(0,0){$\bullet$}}}
\put(743,149){\raisebox{-.8pt}{\makebox(0,0){$\bullet$}}}
\put(869,108){\raisebox{-.8pt}{\makebox(0,0){$\bullet$}}}
\put(995,108){\raisebox{-.8pt}{\makebox(0,0){$\bullet$}}}
\put(1121,189){\raisebox{-.8pt}{\makebox(0,0){$\bullet$}}}
\put(1247,149){\raisebox{-.8pt}{\makebox(0,0){$\bullet$}}}
\put(1373,230){\raisebox{-.8pt}{\makebox(0,0){$\bullet$}}}
\put(869.0,108.0){\rule[-0.200pt]{30.353pt}{0.400pt}}
\end{picture}
\caption{Gr\`afic de la variable $p_j^{(i)}\%$ com a funci\'o de la
 variable {\it Nota}}
\label{PICENTNOTA}
\end{figure}

Dibuxau en el mateix dibuix tots els gr\`afics corresponents a totes
les preguntes~$P_i$.

Interpretau els resultats.

Q\"uestions:

\begin{itemize}
\item Com es veu reflectida la dificultat d'una pregunta en el gr\`afic 
anterior?

\item Digau a partir del vostre gr\`afic quina seria segons el vostre
criteri la pregunta m\'es dif\'{\i}cil i la m\'es f\`acil.

\item Com es veu reflectida la coher\`encia
 d'una pregunta en el gr\`afic 
anterior? Per coher\`encia 
vull dir si la pregunta est\`a ben contestada pels alumnes bons i 
mal contestada pels dolents.

\item Digau, a partir del vostre gr\`afic, quina seria segons el vostre
criteri la pregunta m\'es coherent i la menys 
coherent.

\end{itemize}

\subsection{Estudi de la coher\`encia 
dels distractors}

Fixam una pregunta~$P_i$. Per a cada resposta err\`onia~$R_k$ ($k\not =1$),
constru\"{\i}u la taula:
\begin{center}
\begin{tabular}{|c||c@{}c@{}c@{}c@{}c@{}c@{}c@{}c@{}c@{}c@{}|}
\hline
Nota&$[0,1)$&$[1,2)$&$[2,3)$&$[3,4)$&$[4,5)$&$[5,6)$&$[6,7)$&$[7,8)$&$[8,9)$&$[9,10)$\\\hline\hline
&$p_0^{(i,k)}\%$& $p_1^{(i,k)}\%$&$p_2^{(i,k)}\%$&$p_3^{(i,k)}\%$&
$p_4^{(i,k)}\%$&$p_5^{(i,k)}\%$&$p_6^{(i,k)}\%$&$p_7^{(i,k)}\%$&
$p_8^{(i,k)}\%$&$p_9^{(i,k)}\%$\\\hline
\end{tabular}
\end{center}
on $p_j^{(i,k)}\%$ \'es el tant per cent de gent que ha contestat la 
resposta~$k$ i ha tret una nota entre~$j$ i \mbox{$j+1$.} O sigui, 
$p_j^{(i,k)}/100$ \'es el quocient entre la quantitat d'alumnes que han
contestat la resposta~$k$ en la pregunta~$i$ i han obtengut una nota
entre~$j$ i $j+1$ i la quantitat d'alumnes que han contestat la
resposta~$k$ en la pregunta~$i$. S'ha de complir, doncs, que
$\sum\limits_{j=0}^9 p_j^{(i,k)}=100$.

Feu el gr\`afic de la variable $p_j^{(i,k)}$ com a funci\'o de la variable
{\it Nota}. Vegeu gr\`afic~\ref{PICENTNOTA} per entendre millor com
ha de ser el gr\`afic demanat.

Intepretaci\'o dels resultats.

Q\"uesti\'o:

- Com es veu reflectida la coher\`encia del distractor~$R_k$ en els gr\`afics
anteriors? 

Un distractor~$R_k$ \'es coherent si hi ha hagut pocs alumnes bons que
l'han contestat i molts de dolents.

\section{Contrasts d'hip\`otesi corresponents als
par\`ametres $\mu$ i $\sigma$}


\subsection{Introducci\'o i generaci\'o de les mostres}

En aquesta pr\`actica veurem si els estimadors 
$\overline{X}$, $S^2$ i $\tilde{S}^2$ s\'on bons estimadors
 dels par\`ametres 
$\mu$ i $\sigma^2$ respectivament, fent servir la teoria del contrast
 d'hip\`otesi.

Generau dues mostres 
aleat\`ories simples: una d'una variable aleat\`oria 
$N(\mu,\sigma^2)$ on $\mu$ i $\sigma^2$ s\'on par\`ametres
 i l'altra d'una variable aleat\`oria $t_n$ 
($t$ de Student amb $n$ graus de llibertat) on 
$n$ tamb\'e \'es un altre par\`ametre. 
Els par\`ametres s'han de
fixar d'entrada. Vegeu problema~\ref{PROBGENMOSTRES} per entendre com podem generar
les esmentades mostres.

Comprovau a partir del test $\chi^2$ i el test de Kolmogorov-Smirnov
que les mostres anteriors corresponen a la variable 
aleat\`oria normal i $t_n$ de Student, respectivament.

\subsection{Contrast d'hip\`otesi per a la mitjana}

Sigui $X_1,\ldots,X_n$ la mostra generada en la secci\'o anterior.

Sigui $\mu_0$ un valor qualsevol fixat d'entrada.

Considerem el contrast d'hip\`otesi seg\"uent:
\[
\left.
\begin{array}{ll}
H_0:&\mu =\mu_0, \\
H_1:&\mu\not =\mu_0.
\end{array}
\right\}
\]
Trobau $\alpha_{\mbox{\footnotesize m\`ax}}(\mu_0)$: error tipus 
I m\`axim a partir del qual rebutjam la hip\`otesi\break nu{\lgem}a~$H_0$.

Representau gr\`aficament la funci\'o: 
$\alpha_{\mbox{\footnotesize m\`ax}}=\alpha_{\mbox{\footnotesize m\`ax}}(\mu_0)$.

Q\"uestions:
\begin{itemize}
\item \'Es cert que el gr\`afic anterior t\'e el m\`axim 
en $\overline{X}= \frac{\sum\limits_{i=1}^n X_i}{n}$?

\item Interpretau el gr\`afic anterior.
\end{itemize}

\subsection{Contrast d'hip\`otesi per a la vari\`ancia
}

Sigui $\sigma_0^2$ un valor qualsevol fixat d'entrada.

Considerem ara el contrast seg\"uent:

\[
\left.
\begin{array}{ll}
H_0:&\sigma^2 =\sigma_0^2, \\
H_1:&\sigma^2\not =\sigma_0^2.
\end{array}
\right\}
\]

Trobau $\alpha_{\mbox{\footnotesize m\`ax}}(\sigma_0^2)$: error tipus~I 
m\`axim a partir del qual rebutjam la hip\`otesi nu{\lgem}a~$H_0$.

Representau gr\`aficament la funci\'o: 
$\alpha_{\mbox{\footnotesize m\`ax}}=\alpha_{\mbox{\footnotesize m\`ax}}(\sigma_0^2)$.

\begin{itemize}
\item On t\'e el m\`axim el gr\`afic anterior?

\item De quin dels dos estimadors, $S^2$ o $\tilde{S}^2$ est\`a m\'es 
a prop el m\`axim anterior?

Recordem que les f\'ormules dels estimadors 
$S^2$ i $\tilde{S}^2$ s\'on:
\begin{eqnarray*}
S^2 & = & \frac{\sum\limits_{i=1}^n X_i^2}{n}-\overline{X}^2, \\
\tilde{S}^2 & = & \frac{1}{n-1}\sum_{i=1}^n {(X_i-\overline{X})}^2 = 
\frac{n}{n-1} S^2.
\end{eqnarray*}

\item Interpretau el gr\`afic anterior.
\end{itemize}

\section{Lleis dels grans nombres i test $\chi^2$}

\subsection{Introducci\'o}
\label{INTRODUCCIO}

Considerem $(X_1,\ldots,X_k)$ una variable aleat\`oria multinomial amb 
par\`ametres\break $p_1,\ldots,p_k,n$ on recordem que 
\[
\sum_{i=1}^k p_i =1,\quad \sum_{i=1}^k X_i =n.
\] 

Trobau una mostra aleat\`oria 
simple de la variable aleat\`oria anterior
de grand\`aria~$N$. Sigui
\[
\begin{array}{lcl}
(X_1^{(1)},&\ldots,&X_k^{(1)})\\
(X_1^{(2)},&\ldots,&X_k^{(2)})\\
&\vdots&\\
(X_1^{(N)},&\ldots,&X_k^{(N)}),
\end{array}
\]
l'esmentada mostra.

Per exemple, en el cas del llan\c{c}ament d'un dau, $k=6$ i
$p_i =\frac{1}{6},\ i=1,2,3,4,5,6$.


\subsection{Lleis dels grans nombres}

Trobau el l\'{\i}mit seg\"uent:
\[
\lim_{N\to\infty} \frac{\sum\limits_{i=1}^{N} X_j^{(i)}}{n N} := L_j.
\]

Comprovau la llei dels grans nombres, 
o sigui, vegeu que $L_j =p_j$, per a tot $j=1,\ldots,k$.

\subsection{Test $\chi^2$}

\subsection*{Estudi del test a partir del propi test $\chi^2$}

A partir de cada element de la mostra aleat\`oria simple generada 
en la secci\'o~\ref{INTRODUCCIO}, definim una altra mostra de la forma seg\"uent:
\[
Y^{(i)}=\sum_{j=1}^k \frac{{(X_j^{(i)}- n p_j)}^2}{n p_j},\ i=1,\ldots,N.
\]

Vegeu mitjan\c{c}ant el test de la $\chi^2$ si la mostra aleat\`oria
 simple
\[
Y^{(1)},\ldots,Y^{(N)},
\]
correspon a una variable aleat\`oria $\chi_{k-1}^2$ (variable khi 
quadrat amb $k-1$ graus de llibertat).

\subsection*{Estudi del test a partir de la funci\'o de 
distribuci\'o emp\'{\i}rica}

Considerem la funci\'o de distribuci\'o

emp\'{\i}rica de la mostra aleat\`oria
 simple 
\[
Y^{(1)},\ldots,Y^{(N)}:
\]
 
\[
F^{(N)} (t)= \frac{\#\{ i\ |\ Y^{(i)}\leq t\}}{N},
\]
on el s\'{\i}mbol $\#$ vol dir cardinal d'un conjunt. 

Fixam $t$. Trobau el l\'{\i}mit:
\[
\lim_{N\to\infty} F^{(N)} (t):= L_t.
\]

Q\"uesti\'o:

\'Es cert que el l\'{\i}mit anterior $L_t$ val $F_{\chi_{k-1}^2} (t)$ 
(funci\'o de distribuci\'o 
de la variable $\chi_{k-1}^2$ en el punt $t$)?

\section{Test ANOVA d'un factor}

\subsection{Generaci\'o de les mostres}

Siguin $X_1,\ldots,X_k$, $k$ variables aleat\`ories
 normals 
$X_i =N(\mu_i,\sigma^2)$ i independents.

Generau $k$ mostres aleat\`ories simples corresponents a les
variables aleat\`ories 
anteriors de grand\`aries respectives $n_1,\ldots,n_k$:
\[
\begin{array}{lcl}
Y_{1,1},&\ldots,&Y_{1,n_1},\\
Y_{2,1},&\ldots,&Y_{2,n_2},\\
&\vdots&\\
Y_{k,1},&\ldots,&Y_{k,n_k}.
\end{array}
\]

Vegeu el problema~\ref{PROBGENMOSTRES} per entendre millor com generar les 
mostres an\-te\-riors.

\subsection{Comprovaci\'o de la normalitat i la 
independ\`encia}

Vegeu mitjan\c{c}ant el test de la $\chi^2$ que les mostres anteriors
s\'on normals.

Fent servir el mateix test, comprovau que s\'on independents de
 dues en dues, o
sigui, vegeu que la mostra corresponent a la variable $Y_i$ \'es
independent de la mostra corresponent a la variable $Y_j$, per $i\not = j$.

\subsection{Comprovaci\'o de la igualtat de vari\`ancies}

Vegeu mitjan\c{c}ant el test de Bartlett si totes 
les mostres anteriors corresponen a $k$ variables 
aleat\`ories amb la mateixa 
vari\`ancia.


\subsection{Contrast ANOVA}

Realitzau el contrast:
\[
\left.
\begin{array}{rl}
H_0: & \mu_1 =\ldots =\mu_k \\
H_1: & \exists i,j\ |\ \mu_i\not =\mu_j
\end{array}
\right\}
\]
O sigui, trobau $\alpha_{\mbox{\footnotesize m\`ax}}$ (error tipus I
m\`axim) per poder acceptar $H_0$.

En el cas de rebuig de la hip\`otesi nu{\lgem}a $H_0$, vegeu quines mitjanes
s\'on diferents a partir del test de Scheff\'e.


\section{C\`alcul d'una integral definida 
per m\`etodes estad\'{\i}stics}

\subsection{Introducci\'o}

Sigui 
\(
f:[a,b] \longrightarrow  \RR 
\)
una funci\'o real de variable real, integrable de Riemann
 en l'interval~$[a,b]$.

L'objectiu d'aquesta pr\`actica \'es calcular 
\mbox{Int:= $\int\limits_a^b f(x)\, dx$.}

Suposem que la funci\'o~$f$ compleix dues condicions:
\begin{itemize}
\item $f(x)\geq 0$, per a tot $x\in [a,b]$.

\item $|f(x)|\leq M$ o $f(x)\in [0,M]$, per a tot $x\in [a,b]$, on
\mbox{$M=\max\limits_{x\in [a,b]} f(x):= f(x_M)$.}
\end{itemize}

Trobarem la integral de dues maneres diferents.

\subsection{C\`alcul de la integral a partir d'una variable aleat\`oria 
bidimensional}

\subsection*{Definicions de les variables aleat\`ories}
Considerem les variables aleat\`ories $X$ 
uniforme en l'interval $[a,b]$ i $Y$ uniforme en l'interval $[0,M]$.

Suposem que les variables aleat\`ories~$X$ i $Y$ s\'on 
independents.

La funci\'o de densitat conjunta 
de la variable aleat\`oria bidimensional
$(X,Y)$ ser\`a:
\[
f(x,y)=
\left\{
\begin{array}{ll}
\frac{1}{M(b-a)}, & \mbox{si $(x,y)\in [a,b]\times [0,M]$,} \\
0, & \mbox{en cas contrari.}
\end{array}
\right.
\]
Considerem la prova de Bernoulli seg\"uent:

Sigui $(X_1,Y_1)$ una mostra aleat\`oria
 simple de grand\`aria~$1$ de la 
variable aleat\`oria anterior $(X,Y)$.

Direm que ha sortit \`exit en la prova de Bernoulli si 
\mbox{$Y_1\leq f(X_1)$} i 
que ha sortit frac\`as si \mbox{$Y_1 > f(X_1)$.}

La probabilitat~$p$ d'\`exit valdr\`a, doncs:
\begin{eqnarray*}
p & = & \pp{Y\leq f(X)} = \int\!\int_{\{(x,y)\ |\ y\leq f(x)\}} 
\frac{1}{M(b-a)}\, dx\,dy \\ & = & \int_a^b
\int_0^{f(x)}\frac{1}{M(b-a)}\, dx\, dy =\frac{1}{M(b-a)} {\rm Int}.
\end{eqnarray*}

Sigui $Z$ la variable aleat\`oria de Bernoulli 
corresponent a la prova de Bernoulli definida anteriorment.

\subsection*{C\`alcul de la integral de forma aproximada}

Sigui $Z_1,\ldots,Z_n$ una mostra aleat\`oria simple de la variable
aleat\`oria~$Z$. Cada va\-ria\-ble $Z_i$ t\'e assignada una variable 
bidimensional $(X_i,Y_i)$ amb la mateixa 
distribuci\'o que la variable~$(X,Y)$.

Fent servir que l'estimador~$\overline{Z}$ del par\`ametre~$p$ \'es 
consistent, podem afirmar:
\[
\lim_{n\to\infty} \pp{|\overline{Z}-p| <\epsilon}= 1,\quad
\forall\epsilon >0,
\]
on $\overline{Z}=\frac{\sum\limits_{i=1}^n Z_i}{n}$.

Podem afirmar que $\overline{Z}$ \'es una bona aproximaci\'o del
par\`ametre~$p$.

Podem aproximar la integral~${\rm Int}$ de la forma seg\"uent:
\[
{\rm Int} = M(b-a) p\approx M(b-a)\overline{Z}.
\]

\subsection*{Control de l'error com\`es}

Donats $\epsilon >0$, $\delta >0$ fixats, trobeu~$n$ tal que
\[
\pp{|\overline{Z}-p| > \epsilon} \leq \delta,
\]
El que es demana, de fet, \'es fer una taula de valors per a 
diferents $\epsilon$, $\delta$ i~$n$. 

Indicaci\'o: feu servir l'aproximaci\'o binomial a la 
normal per poder trobar la probabilitat anterior.

\subsection{C\`alcul de la integral a partir d'una variable aleat\`oria 
unidimensional}

\subsection*{Definici\'o de les variables aleat\`ories}

Suposem que la funci\'o $f(x)$, a m\'es de complir les dues condiciones
esmentades anteriorment, \'es estrictament mon\`otona
 en l'interval
 $[a,b]$.

Sigui $X$ una variable aleat\`oria $U[a,b]$.
Definim una altra variable $Y$ de la forma seg\"uent:

Donat un element $\omega$ de l'espai mostral $\Omega$, 
definim $Y(\omega)$ com:
\[
Y(\omega)=
\left\{
\begin{array}{ll}
f\left(X(\omega)\right), & \mbox{si $X(\omega)\in [a,b]$,} \\
0, & \mbox{en cas contrari.}
\end{array}
\right.
\]

Vegeu que $\EE_Y (Y) =\EE_X (f(X))$, o sigui, comprovau la igualtat de les
f\'ormules seg\"uents:
\begin{equation}
\EE_Y (Y)\int_{\RR} y f_Y (y)\, dy = \int_a^b \frac{1}{b-a} f(x)\, dx.
\label{INTSEGFORMA}
\end{equation}

\subsection*{C\`alcul de la integral de forma aproximada}

Sigui $X_1,\ldots,X_n$ una mostra aleat\`oria
 simple de la v.a.~$X$.
Aleshores, podem afirmar que $Y_1:= f(X_1),\ldots,Y_n:= f(X_n)$ ser\`a una mostra
aleat\`oria simple de la v.a.~$Y$.

Fent servir l'expresi\'o~\ref{INTSEGFORMA}, podem afirmar: 
$\EE_Y (Y) =\frac{1}{b-a} {\rm Int}$.

Tenint en compte que $\overline{Y}=\overline{f(X)}=
\frac{\sum\limits_{i=1}^n f(X_i)}{n}$ \'es un bon estimador
 del par\`ametre $\EE_Y(Y)$, 
podem escriure:
\begin{equation}
\lim_{n\to\infty} \left\{\left|\overline{Y}-\frac{1}{b-a} {\rm Int}\right|
> \epsilon\right\} =0,\quad\forall\epsilon >0.
\label{ESTYCONSISTENT}
\end{equation}
Aix\'{\i} prenem com a aproximaci\'o de ${\rm Int}$:
\[
{\rm Int} = (b-a) E_Y (Y)\approx (b-a)\overline{Y}.
\]

Podem dir que la f\'ormula~\ref{ESTYCONSISTENT} continua essent certa 
per a funcions $f$ no mon\`otones? I per a funcions $f$ qualssevol integrables de
Riemann?


