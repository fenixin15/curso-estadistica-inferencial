\chapter{An\`alisi de la vari\`ancia}
\index{analisi@an\`alisi!de la variancia@de la vari\`ancia}

\section{Resum te\`oric}

\'Es un conjunt de t\`ecniques d'an\`alisi estad\'{\i}stica que permeten analitzar com
operen diversos factors simult\`aniament. Podem comprovar, per exemple, la
significaci\'o estad\'{\i}stica
\index{significacio estadistica@significaci\'o estad\'{\i}stica} 
entre les mitjanes\index{mitjana} de dues o m\'es mostres,\index{mostra} determinant si
les difer\`encies observades es poden assignar o no a fluctuacions del mostreig.
\index{mostreig}

Normalment, es vol estudiar com es diferencien els nivells
\index{nivell} d'un cert factor\index{factor!tractament}
subjecte a estudi (factor tractament), tenint en compte la incid\`encia d'altres
factors (ambientals),\index{factor!ambiental} la influ\`encia dels quals s'elimina 
mitjan\c cant una adequada descomposici\'o de la variabilitat.
\index{variabilitat}

\subsection{An\`alisi de la vari\`ancia d'un factor}
\index{analisi@an\`alisi!de la variancia@de la vari\`ancia!d'un factor}

Considerem el problema seg\"uent:

Siguin $Y_1, \ldots , Y_k$ $k$ variables aleat\`ories normals amb mitjanes
\index{variable!aleatoria@aleat\`oria!normal}\index{mitjana}
respectives\break $\mu_1, \ldots , \mu_k$ i totes amb la 
mateixa vari\`ancia
\index{variancia@vari\`ancia}
$\sigma^2$. Volem veure si totes tenen la mateixa mitjana, o sigui, volem fer el
contrast d'hip\`otesi seg\"uent:\index{contrast!d'hipotesi@d'hip\`otesi}

\begin{description}
\item $H_0 : \mu_1 = \cdots = \mu_k$.
\item $H_1$ : no totes les $\mu_i$ s\'on iguals.
\end{description}

Un exemple d'aplicaci\'o del problema anterior podria ser estudiar $k$ diferents
tipus d'ensenyament\index{tractament} i veure si tenen la mateixa efic\`acia.
\index{eficacia@efic\`acia} Per fer-ho, s'agafen
$k$ classes d'alumnes\index{classe!d'alumnes} i a cada classe li \'es aplicat
un m\`etode d'ensenyament\index{metode@m\`etode!d'ensenyament}
diferent.

En general tendrem:

$Y_{1j}, \ldots , Y_{n_j j}$ una mostra aleat\`oria simple de la variable
aleat\`oria $Y_j$.

Considerem les variables aleat\`ories $\varepsilon_{ij} = Y_{ij} - \mu_j, \ i = 1,
\ldots , n_j, \ j = 1 \ldots , k$. Totes s\'on $N(0,\sigma^2)$ i
independents.

Definim els par\`ametres\index{parametre@par\`ametre}
$$\mu = {\sum\limits_{j=1}^k n_j \mu_j \over n}, \quad \alpha_j = \mu_j - \mu,$$
on $n=\sum\limits_{j=1}^k n_j$.

\begin{enumerate}
\item {\bf Estimadors dels par\`ametres}\index{estimador!del parametre@del par\`ametre}
\begin{itemize}
\item[--] Estimador de $\mu_j$:

Com que $Y_{1j}, \ldots , Y_{n_j j}$ \'es una mostra aleat\`oria simple de la variable
aleat\`oria $Y_j$, l'estimador de m\`axima versemblan\c ca de $\mu_j$ val:
\index{estimador!de maxima versemblanca@de m\`axima versemblan\c{c}a}
$$ \hat{\mu}_j = {\sum\limits_{i=1}^{n_j} Y_{ij} \over n_j} := \bar{Y}_j.$$

\item[--] Estimador de $\mu$:

A partir de l'expressi\'o de $\mu$, obtenim el seu estimador de m\`axima
versemblan\c{c}a:
\index{estimador!de maxima versemblanca@de m\`axima versemblan\c{c}a}
$$ \hat{\mu} = {\sum\limits_{j=1}^k n_j \hat{\mu}_j \over n} =
{\sum\limits_{j=1}^k \sum\limits_{i=1}^{n_j} Y_{ij}\over n} := \bar{Y}.$$

\item[--] Estimador de $\alpha_j$:

Com que $\alpha_j = \mu_j - \mu$, l'estimador de m\`axima versemblan\c ca de
$\alpha_j$ \'es:
$$\hat{\alpha}_j = \hat{\mu}_j - \hat{\mu} = \bar{Y}_j - \bar{Y}.$$
\end{itemize}

\newpage

\item {\bf Variabilitat}\index{variabilitat}

Definim la variabilitat total de la mostra com:\index{variabilitat!total de la mostra}
$$VT = \sum\limits_{j=1}^k \sum\limits_{i=1}^{n_j} (Y_{ij} - \bar{Y})^2,$$
i la descomposam de la manera seg\"uent:
$$VT = \sum\limits_{j=1}^k \sum\limits_{i=1}^{n_j} (Y_{ij} - \bar{Y_j})^2 +
\sum\limits_{j=1}^k \sum\limits_{i=1}^{n_j} (\bar{Y}_j - \bar{Y})^2 + 2
\sum\limits_{j=1}^k \sum\limits_{i=1}^{n_j} (Y_{ij} - \bar{Y}_j)(\bar{Y}_j -
\bar{Y}).$$

El darrer sumand \'es $0$ i, per tant, resulta:
\begin{eqnarray*}
VT & = & \sum\limits_{j=1}^k \sum\limits_{i=1}^{n_j} (Y_{ij} - \bar{Y}_j)^2 +
\sum\limits_{j=1}^k \sum\limits_{i=1}^{n_j} (\bar{Y}_j - \bar{Y})^2 \\
& = &
\sum\limits_{j=1}^k \sum\limits_{i=1}^{n_j} (Y_{ij} - \bar{Y}_j)^2 +
\sum\limits_{j=1}^k n_j (\bar{Y}_j - \bar{Y})^2 := V_1 + V_2.
\end{eqnarray*}

$V_1$ \'es la variabilitat deguda a les difer\`encies entre els valors dins cada
grup, i $V_2$ \'es la variabilitat deguda a les difer\`encies entre els grups.
\index{variabilitat}

Definim ara les mitjanes quadr\`atiques seg\"uents:\index{mitjana!quadratica@quadr\`atica}
\begin{itemize}
\item[--] Mitjana quadr\`atica total: $MC_t = {VT \over n-1} = {\sum\limits_{j=1}^k
\sum\limits_{i=1}^{n_j} (Y_{ij} - \bar{Y})^2 \over n-1}$.
\index{mitjana!quadratica@quadr\`atica!total}

\item[--] Mitjana quadr\`atica intragrup: $MC_{tra} = {V_1 \over n-k} =
{\sum\limits_{j=1}^k \sum\limits_{i=1}^{n_j} (Y_{ij} - \bar{Y}_j)^2 \over n-k}$.
\index{mitjana!quadratica@quadr\`atica!intragrup}

\item[--] Mitjana quadr\`atica intergrup: $MC_{ter} = {V_2 \over k-1} =
{\sum\limits_{j=1}^k n_j (\bar{Y}_j - \bar{Y})^2 \over k-1}$.
\end{itemize}
\index{mitjana!quadratica@quadr\`atica!intergrup}

\item {\bf Estad\'{\i}stic de contrast}

Volem trobar un estad\'{\i}stic de tal forma que, si $H_0$ \'es certa, puguem dir quin
tipus de distribuci\'o segueix.\index{estadistic@estad\'{\i}stic}

Primerament, observem que
\begin{eqnarray*}
Q &:= & \sum\limits_{j=1}^k \sum\limits_{i=1}^{n_j} {(Y_{ij} - \mu_j)^2 \over
\sigma^2} \\ & = &  {\sum\limits_{j=1}^k \sum\limits_{i=1}^{n_j} (Y_{ij} - \bar{Y}_j)^2
\over \sigma^2} + {\sum\limits_{j=1}^k n_j (\bar{Y}_j - \bar{Y} - \alpha_j)^2
\over \sigma^2} + {n (\bar{Y} - \mu)^2 \over \sigma^2} \\ & := & Q_1 + Q_2 + Q_3.
\end{eqnarray*}

Aleshores:

\begin{itemize}
\item[--] La variable aleat\`oria $Q$ segueix la distribuci\'o $\chi_n^2$.

\item[--] La variable aleat\`oria $Q_1$ segueix la distribuci\'o $\chi_{n-k}^2$.

\item[--] La variable aleat\`oria $Q_2$ segueix la distribuci\'o $\chi_{k-1}^2$.

\item[--] La variable aleat\`oria $Q_3$ segueix la distribuci\'o $\chi_1^2$.
\end{itemize}

Si $H_0$ \'es certa, aleshores $\alpha_1 = \cdots = \alpha_k = 0$. Per tant
resulta:
$$Q_2 = {\sum\limits_{j=1}^k n_j (\bar{Y}_j - \bar{Y})^2 \over \sigma^2}.$$

D'aqu\'{\i} dedu\"{\i}m que
$$F := {Q_2 / (k-1) \over Q_1 / (n-k)} = {MC_{ter} \over MC_{tra}}$$
segueix la distribuci\'o $F$ de Fisher-Snedecor amb $k-1$ i $n-k$ graus de
llibertat.\index{distribucio@distribuci\'o!$F$ de Fisher-Snedecor}

La regi\'o cr\'{\i}tica ser\`a de la forma $\{ F > F_{k-1,n-k,1-\alpha} \}$.
\index{regio@regi\'o!critica@cr\'{\i}tica}

Les f\'ormules seg\"uents permeten calcular les diferents variabilitats de manera m\'es
senzilla:\index{variabilitat}

\begin{itemize}
\item[-] $VT = \sum\limits_{j=1}^k \sum\limits_{i=1}^{n_j} Y_{ij} - {\left(
\sum\limits_{j=1}^k \sum\limits_{i=1}^{n_j} Y_{ij}^2\right)^2 \over n}$.

\item[-] $V_1 = \sum\limits_{j=1}^k \sum\limits_{i=1}^{n_j} Y_{ij}^2 -
\sum\limits_{j=1}^k {\left( \sum\limits_{i=1}^{n_j} Y_{ij} \right)^2 \over n_j}$.

\item[-] $V_2 = VT - V_1 = \sum\limits_{j=1}^k {\left( \sum\limits_{i=1}^{n_j} Y_{ij}
\right)^2 \over n_j} - {\left( \sum\limits_{j=1}^k \sum\limits_{i=1}^{n_j}
Y_{ij}^2\right)^2 \over n}$.
\end{itemize}

Despr\'es, com ja hem dit, es calculen les mitjanes quadr\`atiques i, finalment,
l'estad\'{\i}stic $F$.\index{mitjana!quadratica@quadr\`atica}
\index{estadistic@estad\'{\i}stic}

\item {\bf Comprovaci\'o d'hip\`otesis pr\`evies}

Per tal de comprovar la igualtat de les vari\`ancies,\index{igualtat de variancies@igualtat de vari\`ancies}
$$\sigma_1^2 = \cdots = \sigma_k^2,$$
Bartlett va proposar l'estad\'{\i}stic de contrast seg\"uent:
\index{Bartlett}\index{estadistic@estad\'{\i}stic!de contrast}
$$B = {2.3026 \over A} \left[ \sum\limits_{j=1}^k (n_j - 1) \log_{10}
{\sum\limits_{j=1}^k (n_j - 1) \tilde{S}_j^2 \over \sum\limits_{j=1}^k (n_j - 1)}
- \sum\limits_{j=1}^k (n_j - 1) \log_{10} \tilde{S}_j^2 \right] = {C \over A},$$
on
$$\tilde{S_j}^2 = {\sum\limits_{i=1}^{n_j} (Y_{ij} - \bar{Y}_j)^2 \over n_j - 1},
\ \ A = 1 + {1 \over 3 (k-1)} \left[ \sum\limits_{j=1}^k {1 \over n_j - 1} - {1
\over \sum\limits_{j=1}^k n_j - 1} \right].$$

L'estad\'{\i}stic $B$ es distribueix segons una $\chi_{k-1}^2$ i, per tant, la regi\'o
cr\'{\i}tica vendr\`a donada per:\index{regio@regi\'o!critica@cr\'{\i}tica}
$$\{ B > \chi_{k-1,1-\alpha} \}.$$

\item {\bf Comprovaci\'o de la normalitat de les $Y_{ij}$}\index{normalitat}

S'aplica el test $\chi^2$ si les $n_j$ s\'on grans o el test de Kolmogorov-Smirnov,
en cas contrari.\index{test!khi quadrat}\index{test!de Kolmogorov-Smirnov}

\item {\bf Variables que tenen les mitjanes diferents}

Si hem rebutjat $H_0$, ens interessar\`a saber quines $Y_j$ tenen les mitjanes
diferents. La t\`ecnica m\'es utilitzada \'es el test de Scheff\'e, 
\index{test!de Scheffe@de Scheff\'e}que diu que, si hem
rebutjat $H_0$, acceptarem que $\mu_i \not = \mu_j$ al nivell de significaci\'o
$\alpha$ si\index{nivell!de significacio@de significaci\'o}
$${|\bar{Y_i} - \bar{Y_j}| \over \sqrt{MC_{tra} \cdot ({1 \over n_i} + {1 \over
n_j})}} \geq \sqrt{(k-1) F_{k-1,n-k,1-\alpha}}.$$
\end{enumerate}

\subsection{An\`alisi de la vari\`ancia de dos factors}
\index{analisi@an\`alisi!de la variancia@de la vari\`ancia!de dos factors}

En el cas anterior supos\`avem que la variaci\'o de les variables aleat\`ories $Y_i$
nom\'es era deguda a un factor: el m\`etode d'ensenyament, 
\index{metode@m\`etode!d'ensenyament}per exemple. Ara
suposarem, continuant amb l'exemple anterior, que l'efic\`acia
\index{eficacia@efic\`acia} dels alumnes dep\`en de
dos factors: el m\`etode d'ensenyament i el nivell social.\index{nivell!social}

Aix\'{\i}, doncs, sigui $Y_{rs}$ la variable aleat\`oria que ens indica el rendiment dels
\index{rendiment}
alumnes ensenyats pel m\`etode $r$ i que pertanyen al nivell social $s$ ($r = 1,
\ldots , f, \ s = 1, \ldots , c$). Suposem que les variables aleat\`ories $Y_{rs}$
s\'on $N(\mu_{rs},\sigma^2)$ i independents.

Sigui $Y_{1rs}, \ldots , Y_{nrs}$ una mostra aleat\`oria simple de la variable
aleat\`oria $Y_{rs}$\break $\forall r = 1, \ldots , f$, i $ \forall s = 1, \ldots , c$.
(Suposem que totes les mostres tenen la mateixa grand\`aria.)

Considerem les variables aleat\`ories $N(0,\sigma^2)$ $\varepsilon_{irs} = Y_{irs}
- \mu_{rs}$. Introdu\"{\i}m els par\`ametres seg\"uents:

\begin{itemize}
\item[--] Rendiment mitj\`a de totes les persones ensenyades pel m\`etode $r$:
\index{rendiment!mitja@mitj\`a}
$$\mu_{r\bullet} = {\sum\limits_{s=1}^c \mu_{rs} \over c}.$$

\item[--] Rendiment mitj\`a de totes les persones que pertanyen al nivell social $s$:
$$\mu_{\bullet s} = {\sum\limits_{r=1}^f \mu_{rs} \over f}.$$
\index{rendiment!mitja@mitj\`a}

\item[--] Rendiment mitj\`a de totes les persones:
\index{rendiment!mitja@mitj\`a}
$$\mu_{\bullet\bullet} = {\sum\limits_{s=1}^c \sum\limits_{r=1}^f \mu_{rs} \over f c} =
{\sum\limits_{r=1}^f \mu_{r\bullet} \over f} = {\sum\limits_{s=1}^c \mu_{\bullet s} \over c}.$$
\end{itemize}

Considerem els par\`ametres:\index{parametre@par\`ametre}
\begin{itemize}
\item[-] $\alpha_r = \mu_{r\bullet} - \mu_{\bullet\bullet}$ (difer\`encies degudes al m\`etode).
\index{diferencia@difer\`encia!deguda al metode@deguda al m\`etode}

\item[-] $\beta_s = \mu_{\bullet s} - \mu_{\bullet\bullet}$ (difer\`encies degudes al nivell social).
\index{diferencia@difer\`encia!deguda al nivell social}

\item[-] $\gamma_{rs} = \mu_{rs} - \mu_{r\bullet} - \mu_{\bullet s} + \mu_{\bullet\bullet}$ 
(interacci\'o).\index{interaccio@interacci\'o}
\end{itemize}

Aleshores
$$Y_{irs} = \mu_{rs} + \varepsilon_{irs} = \mu_{\bullet\bullet} + \alpha_r + \beta_s +
\gamma_{rs} + \varepsilon_{irs}.$$

Direm que hi ha interacci\'o entre els dos factors si $\gamma_{rs} \not = 0$ per a
algun $r$ i algun~$s$.\index{interaccio@interacci\'o}

De forma esquem\`atica, la situaci\'o \'es la seg\"uent:

\begin{center}
\begin{tabular}{c|c|cccc|c|}
\multicolumn{1}{c}{}&\multicolumn{6}{c}{{\bf FACTOR I: Nivell Social}}\\\cline{2-7}
\multirow{6}{3cm}{{\bf FACTOR II: Metodologia Docent}}&&$1$&$2$&$\ldots$&$c$&\\\cline{2-7}
&$1$&$\mu_{11}$&$\mu_{12}$&$\ldots$&$\mu_{1c}$&$\mu_{1\bullet}$\\
&$2$&$\mu_{21}$&$\mu_{22}$&$\ldots$&$\mu_{2c}$&$\mu_{2\bullet}$\\
&$\vdots$&$\vdots$&$\vdots$&$\ddots$&$\vdots$&$\vdots$\\
&$f$&$\mu_{f1}$&$\mu_{f2}$&$\ldots$&$\mu_{fc}$&$\mu_{f\bullet}$\\\cline{2-7}
&&$\mu_{\bullet 1}$&$\mu_{\bullet 2}$&$\ldots$&$\mu_{\bullet c}$&$\mu_{\bullet\bullet}$\\\cline{2-7}
\end{tabular}
\end{center}
\index{nivell!social}\index{metode@m\`etode!d'ensenyament}

\begin{enumerate}
\item {\bf Estimadors dels par\`ametres}\index{estimador!del parametre@del par\`ametre}

A partir de les definicions dels diferents par\`ametres i tenint en compte que
l'estimador de $\mu_{rs}$ \'es
$$\hat{\mu}_{rs} = {\sum\limits_{i=1}^n Y_{irs} \over n},$$
obtenim els estimadors seg\"uents:
\begin{itemize}

\item[-] $\hat{\mu}_{r\bullet} = {\sum\limits_{s=1}^c \sum\limits_{i=1}^n Y_{irs} \over n
c} := \hat{Y}_{r\bullet}$,

\item[-] $\hat{\mu}_{\bullet s} = {\sum\limits_{r=1}^f \sum\limits_{i=1}^n Y_{irs} \over n
f} := \hat{Y}_{\bullet s}$,

\item[-] $\hat{\mu}_{\bullet\bullet} = {\sum\limits_{s=1}^c \sum\limits_{r=1}^f
\sum\limits_{i=1}^n Y_{irs} \over n f c} := \hat{Y}_{\bullet\bullet}$,

\item[-] $\hat{\alpha}_r = \hat{Y}_{r\bullet} - \hat{Y}_{\bullet\bullet}$,

\item[-] $\hat{\beta}_s = \hat{Y}_{\bullet s} - \hat{Y}_{\bullet\bullet}$,

\item[-] $\hat{\gamma}_{rs} = \hat{Y}_{rs} - \hat{Y}_{r\bullet} - 
\hat{Y}_{\bullet s} + \hat{Y}_{\bullet\bullet}$.
\end{itemize}

\item {\bf Variabilitat}

Definim la variabilitat total de la mostra com:\index{variabilitat!total}
$$VT = \sum\limits_{s=1}^c \sum\limits_{r=1}^f \sum\limits_{i=1}^n (Y_{irs} -
\hat{Y}_{\bullet\bullet})^2,$$
i la descomponem de la manera seg\"uent:
\begin{eqnarray*}
VT & = & \sum\limits_{s=1}^c \sum\limits_{r=1}^f \sum\limits_{i=1}^n (Y_{irs} -
\hat{Y}_{rs})^2 + n c \sum\limits_{r=1}^f (\hat{Y}_{r\bullet} - 
\hat{Y}_{\bullet\bullet})^2 + n f \sum\limits_{s=1}^c (\hat{Y}_{\bullet s} - 
\hat{Y}_{\bullet\bullet})^2 \\ & & + n \sum\limits_{s=1}^c
\sum\limits_{r=1}^f (\hat{Y}_{rs} - \hat{Y}_{r\bullet} - \hat{Y}_{\bullet s} +
\hat{Y}_{\bullet\bullet})^2.
\end{eqnarray*}

Aix\'{\i} hem descompost $VT$ en
$$VT = VT_e + VT_f + VT_c + VT_{int},$$
on
\begin{itemize}
\item[--] $VT_e$ \'es la variaci\'o deguda a les difer\`encies entre els valors dins 
cada grup.

\item[--] $VT_f$ \'es la variaci\'o deguda a les difer\`encies entre files (Factor~II:
metodologia docent).
\index{diferencia@difer\`encia!entre files}

\item[--] $VT_c$ \'es la variaci\'o deguda a les difer\`encies entre columnes (Factor~I:
classe social).
\index{diferencia@difer\`encia!entre columnes}

\item[--] $VT_{int}$ \'es la variaci\'o deguda a la interacci\'o entre els dos factors.
\end{itemize}
\index{variacio@variaci\'o!deguda a la interaccio@deguda a la interacci\'o}

Definim ara les mitjanes quadr\`atiques seg\"uents:\index{mitjana!quadratica@quadr\`atica}
\begin{itemize}
\item[--] Mitjana quadr\`atica de l'error: $MC_e = {VT_e \over (n-1) f c}$.
\index{mitjana!quadratica@quadr\`atica!de l'error}

\item[--] Mitjana quadr\`atica per files: $MC_f = {VT_f \over f-1}$ (factor~II).
\index{mitjana!quadratica@quadr\`atica!per files}

\item[--] Mitjana quadr\`atica per columnes: $MC_c = {VT_c \over c-1}$ (factor~I).
\index{mitjana!quadratica@quadr\`atica!per columnes}

\item[--] Mitjana quadr\`atica d'interacci\'o: $MC_{int} = {VT_{int} \over (f-1)(c-1)}$.
\end{itemize}
\index{mitjana!quadratica@quadr\`atica!d'interaccio@d'interacci\'o}

\item {\bf Estad\'{\i}stics de contrast}\index{estadistic@estad\'{\i}stic!de contrast}

Quan feim una an\`alisi de la vari\`ancia de dos factors, tenim tres possibles contrasts
amb les seg\"uents hip\`otesis nu{\lgem}es:
\index{analisi@an\`alisi!de la variancia@de la vari\`ancia!de dos factors}

\begin{enumerate}
\item $H_{01} : \alpha_1 = \cdots = \alpha_f = 0$ o $\mu_{1\bullet} = \cdots = 
\mu_{f\bullet}
= \mu_{\bullet\bullet}$.

(No hi ha variaci\'o entre files (factor II)).\index{variacio@variaci\'o!entre files}

\item $H_{02} : \beta_1 = \cdots = \beta_c = 0$ o $\mu_{\bullet 1} = \cdots = 
\mu_{\bullet c} =
\mu_{\bullet\bullet}$

(No hi ha variaci\'o entre columnes (factor I)).\index{variacio@variaci\'o!entre columnes}

\item $H_{03} : \gamma_{rs} = 0 \ \forall r = 1, \ldots , f $ i $ \ s = 1, \ldots
,c$, o $\mu_{rs} - \mu_{\bullet\bullet} = (\mu_{r\bullet} - \mu_{\bullet\bullet}) + 
(\mu_{\bullet s} - 
\mu_{\bullet\bullet}) = \alpha_r + \beta_s \ \forall r = 1, \ldots , f $ i $ \ s = 1, \ldots ,c$

(No hi ha interacci\'o).\index{interaccio@interacci\'o}
\end{enumerate}

Haurem de donar un estad\'{\i}stic per a cada tipus de contrast. Observem que
\index{estadistic@estad\'{\i}stic!de contrast}

\begin{eqnarray*}
Q & := &  {\sum\limits_{s=1}^c \sum\limits_{r=1}^f \sum\limits_{i=1}^n (Y_{irs} -
\mu_{rs})^2 \over \sigma^2} = {\sum\limits_{s=1}^c \sum\limits_{r=1}^f
\sum\limits_{i=1}^n (Y_{irs} - \hat{Y}_{rs})^2 \over \sigma^2} \\ && + {n c
\sum\limits_{r=1}^f (\hat{Y}_{r\bullet} - \hat{Y}_{\bullet\bullet} - \alpha_r)^2 \over \sigma^2} +
{n f \sum\limits_{s=1}^c (\hat{Y}_{\bullet s} - \hat{Y}_{\bullet\bullet} - \beta_s)^2 \over
\sigma_2} \\ && + {n \sum\limits_{s=1}^c \sum\limits_{r=1}^f (\hat{Y}_{rs} -
\hat{Y}_{r\bullet} - \hat{Y}_{\bullet s} + \hat{Y}_{\bullet\bullet} - \gamma_{rs})^2 \over \sigma^2} + {n
f c (\hat{Y}_{\bullet\bullet} - \mu)^2 \over \sigma^2} \\ & := & Q_1 + Q_2 + Q_3 + Q_4 + Q_5.
\end{eqnarray*}

Aleshores:
\begin{itemize}
\item[--] La variable aleat\`oria $Q$ segueix la distribuci\'o $\chi_{n f c}^2$.

\item[--] La variable aleat\`oria $Q_1$ segueix la distribuci\'o $\chi_{(n-1) f c}^2$.

\item[--] La variable aleat\`oria $Q_2$ segueix la distribuci\'o $\chi_{f-1}^2$.

\item[--] La variable aleat\`oria $Q_3$ segueix la distribuci\'o $\chi_{c-1}^2$.

\item[--] La variable aleat\`oria $Q_4$ segueix la distribuci\'o $\chi_{(f-1)(c-1)}^2$.

\item[--] La variable aleat\`oria $Q_5$ segueix la distribuci\'o $\chi_1^2$.
\end{itemize}


A continuaci\'o donarem els diferents estad\'{\i}stics de contrast.
\index{estadistic@estad\'{\i}stic!de contrast}

\begin{enumerate}
\item Si $H_{01}$ \'es certa, aleshores $\alpha_1 = \cdots = \alpha_f = 0$. Per tant
resulta que
$$\bar{F}_f := {Q_2 / (f-1) \over Q_1 / ((n-1) f c)} = {MC_f \over MC_{e}}$$
segueix la distribuci\'o $F$ de Fisher-Snedecor amb $f-1$ i $(n-1) f c$ graus de
llibertat.\index{distribucio@distribuci\'o!$F$ de Fisher-Snedecor}

La regi\'o cr\'{\i}tica ser\`a de la forma $\{ \bar{F}_f > F_{f-1,(n-1) f c,1-\alpha} \}$.
\index{regio@regi\'o!critica@cr\'{\i}tica}

\item Si $H_{02}$ \'es certa, aleshores $\beta_1 = \cdots = \beta_c = 0$. Per tant
resulta que
$$\bar{F}_c := {Q_3 / (c-1) \over Q_1 / ((n-1) f c)} = {MC_c \over MC_{e}}$$
segueix la distribuci\'o $F$ de Fisher-Snedecor amb $c-1$ i $(n-1) f c$ graus de
llibertat.\index{distribucio@distribuci\'o!$F$ de Fisher-Snedecor}

La regi\'o cr\'{\i}tica ser\`a de la forma $\{ \bar{F}_c > F_{c-1,(n-1) f c,1-\alpha} \}$.
\index{regio@regi\'o!critica@cr\'{\i}tica}

\item Si $H_{03}$ \'es certa, aleshores $\gamma_{rs} = 0 \ \forall r = 1, \ldots , f
$ i $ \ s = 1, \ldots ,c$. Per tant resulta que
$$\bar{F}_{int} := {Q_4 / (f-1)(c-1) \over Q_1 / ((n-1) f c)} = {MC_{int} \over
MC_{e}}$$
segueix la distribuci\'o $F$ de Fisher-Snedecor amb $(f-1)(c-1)$ i $(n-1) f c$
graus de llibertat.\index{distribucio@distribuci\'o!$F$ de Fisher-Snedecor}

La regi\'o cr\'{\i}tica ser\`a de la forma $\{ \bar{F}_{int} > F_{(f-1)(c-1),(n-1) f
c,1-\alpha} \}$.\index{regio@regi\'o!critica@cr\'{\i}tica}
\end{enumerate}

Les f\'ormules seg\"uents permeten calcular les diferents variabilitats de manera m\'es
senzilla:\index{variabilitat}

\begin{itemize}
\item[-] $VT = \sum\limits_{r=1}^f \sum\limits_{s=1}^c \sum\limits_{i=1}^n Y_{irs}^2
- {\left( \sum\limits_{r=1}^f \sum\limits_{s=1}^c \sum\limits_{i=1}^n Y_{irs}
\right)^2 \over n f c}$.

\item[-] $VT_e = \sum\limits_{r=1}^f \sum\limits_{s=1}^c \sum\limits_{i=1}^n
Y_{irs}^2 - {\sum\limits_{r=1}^f \sum\limits_{s=1}^c \left( \sum\limits_{i=1}^n
Y_{irs} \right)^2 \over n}$.

\item[-] $VT_f = {\sum\limits_{r=1}^f \left( \sum\limits_{s=1}^c \sum\limits_{i=1}^n
Y_{irs} \right)^2 \over n c} - {\left( \sum\limits_{r=1}^f \sum\limits_{s=1}^c
\sum\limits_{i=1}^n Y_{irs} \right)^2 \over n f c}$.

\item[-] $VT_c = {\sum\limits_{s=1}^c \left( \sum\limits_{r=1}^f \sum\limits_{i=1}^n
Y_{irs} \right)^2 \over n f} - {\left( \sum\limits_{r=1}^f \sum\limits_{s=1}^c
\sum\limits_{i=1}^n Y_{irs} \right)^2 \over n f c}$.

\item[-] $ V_{int} = VT - VT_e - VT_f - VT_c$.
\end{itemize}

Despr\'es, com ja hem dit, es calculen les mitjanes quadr\`atiques i, finalment, els
estad\'{\i}stics $\bar{F}$.\index{mitjana!quadratica@quadr\`atica}
\index{estadistic@estad\'{\i}stic}

\item {\bf Variables que tenen les mitjanes diferents}

\begin{enumerate}
\item Si hem rebutjat $H_{01}$, seguint amb la mateixa filosofia que la part
d'an\`alisi de la vari\`ancia amb un sol factor, direm que $\mu_{r\bullet} \not = 
\mu_{r'\bullet}$
al nivell de significaci\'o $\alpha$ si:
$${|\bar{Y_{r\bullet}} - \bar{Y_{r'\bullet}}| \over \sqrt{MC_e \cdot ({1 \over 
n_{r\bullet}} + {1
\over n_{r'\bullet}})}} \geq \sqrt{(f-1) F_{f-1,(n-1) f c,1-\alpha}}.$$

En el nostre cas, $n_{r\bullet} = n_{r'\bullet} = n \cdot c$.

\item Si hem rebutjat $H_{02}$, direm que $\mu_{\bullet s} \not = \mu_{\bullet s'}$ al nivell
de significaci\'o $\alpha$ si:
$${|\bar{Y_{\bullet s}} - \bar{Y_{\bullet s'}}| \over \sqrt{MC_e \cdot ({1 
\over n_{\bullet s}} + {1
\over n_{\bullet s'}})}} \geq \sqrt{(c-1) F_{c-1,(n-1) f c,1-\alpha}}.$$

En el nostre cas, $n_{\bullet s} = n_{\bullet s'} = n \cdot f$.

\item Si hem rebutjat $H_{03}$, direm que $\mu_{rs} \not = \mu_{r's'}$ al nivell
de significaci\'o $\alpha$ si:
$${|\bar{Y}_{rs} - \bar{Y}_{r's'}| \over \sqrt{MC_e \cdot ({1 \over n} + {1 \over
n})}} \geq \sqrt{(f-1)(c-1) F_{(f-1)(c-1),(n-1) f c,1-\alpha}}.$$
\end{enumerate}
\end{enumerate}

\section{Problemes resolts}

\begin{probres}
{Desitjam con\`eixer els efectes de
l'alcohol en la realitzaci\'o de sumes. Per aix\`o, triam
a l'atzar $8$ persones que subdividim aleat\`oriament en tres
grups (amb $3$,$2$ i $3$ persones respectivament) als quals
aplicam, tamb\'e aleat\`oriament, tres tractaments distints:
$T_1$ (placebo), $T_2$ (baixa dosi d'alcohol), $T_3$
(alta dosi d'alcohol). Els resultats del quadre adjunt
representen el nombre d'errors comesos per cada una de
les $8$ persones:
\begin{center}
\begin{tabular}{|c|c|c|}
\hline
$T_1$&$T_2$&$T_3$\\\hline\hline
$2$&$7$&$10$\\\hline
$4$&$8$&$10$\\\hline
 &$9$&$\ \,6$\\\hline
\end{tabular}
\end{center}
\'Es compatible la hip\`otesi
$H_0:\mu_1=\mu_2=\mu_3$, igual mitjana d'errors en les $3$
poblacions, amb els resultats obtenguts? (trobau l'error tipus I m\`axim 
per sota del qual acceptam la igualtat de mitjanes d'errors.)}
\end{probres}

\res{
Constru\"{\i}m primer la taula que ens donar\`a les quantitats de qu\`e 
depenen les variabilitats.
\begin{center}
\begin{tabular}{|>{$}c<{$}|>{$}c<{$}|>{$}c<{$}|>{$}c<{$}|>{$}c<{$}|>{$}c<{$}|}
\hline
T_1&T_2&T_3&T_1^2&T_2^2&T_3^2\\\hline\hline
2&\ \,7&10&\ \,4&\ \,49&100\\\hline
4&\ \,8&10&16&\ \,64&100\\\hline
&\ \,9&\ \,6&&\ \,81&\ \,36\\\hline\hline
6&24&26&20&194&236\\\hline
\end{tabular}
\end{center}
Fixau-vos que, en el nostre cas, el nombre de grups val $k=3$, 
la grand\`aria de 
cada mostra corresponent a cada grup \'es $n_1 =2$, $n_2 =3$ i $n_3 =3$ i
la grand\`aria total de la mostra val $n=n_1+n_2+n_3 =8$.

Per tant, el valor de les quantitats anteriors \'es:
\begin{eqnarray*}
\sum_{j=1}^k \sum_{i=1}^{n_j} Y_{ij}^2 & = & 20+194+236 = 450, \\
\frac{{\left(\sum\limits_{j=1}^k \sum\limits_{i=1}^{n_j} Y_{ij}\right)}^2}{n} 
& = & \frac{{(6+24+26)}^2}{2+3+3}=\frac{3136}{8}=392, \\
\sum_{j=1}^k\frac{{\left(\sum\limits_{i=1}^{n_j} Y_{ij}\right)}^2}{n_j} & 
= &
\frac{6^2}{2}+\frac{{24}^2}{3}+\frac{{26}^2}{3}=435.33.
\end{eqnarray*} 
Els valors de les variabilitats totals ($VT$), intragrups ($V_{tra}$) 
i intergrups ($V_{ter}$) s\'on, respectivament:
\begin{eqnarray*}
VT & = & \sum_{j=1}^k \sum_{i=1}^{n_j} Y_{ij}^2 -
\frac{{\left(\sum\limits_{j=1}^k \sum\limits_{i=1}^{n_j} 
Y_{ij}\right)}^2}{n} = 450 - 392 = 58, \\
V_{tra} & = & \sum_{j=1}^k \sum_{i=1}^{n_j} Y_{ij}^2 -
\sum_{j=1}^k\frac{{\left(\sum\limits_{i=1}^{n_j} Y_{ij}\right)}^2}{n_j} =
450 - 435.33 = 14.66, \\
V_{ter} & = & VT - V_{tra} = 58 - 14.66 = 43.33.
\end{eqnarray*}
Finalment, constru\"{\i}m la taula per trobar l'estad\'{\i}stic de contrast 
$F$ on F.V. vol dir Font de Variaci\'o, g. ll. vol dir graus de llibertat i 
$MQ$, mitjana quadr\`atica:
\begin{center}
\renewcommand{\arraystretch}{1.5}
\begin{tabular}{|c||c|c|@{}c@{}|@{}c@{}|}
\hline
F.V.&$V$&g. ll.&$MQ=\frac{V}{\mbox{g. ll.}}$&$F=\frac{
MQ_{ter}}{MQ_{tra}}$\\\hline\hline
Inter&$43.33$&$k-1 
=3-1=2$&$MQ_{ter}=\frac{43.33}{2}=21.66$&\multirow{3}{3.4cm}
{$F=\frac{21.66}{2.93}=7.3864$}\\\cline{1-4}
Intra&$14.66$&$n-k=8-3=5$&$MQ_{tra}=\frac{14.66}{5}=2.93$&\\\cline{1-4}
Total&$58.00$&$n-1 =8-1=7$&&\\\hline
\end{tabular}
\end{center}
La distribuci\'o de l'estad\'{\i}stic $F$ \'es aproximadament la funci\'o 
$F$ de Fisher-Snedecor amb graus de llibertat $k-1=2$ i $n-k=5$.

Tenint en compte que la regi\'o cr\'{\i}tica per realitzar el contrast 
\'es:
\[
R.C.=\{ F>F_{2,5,1-\alpha}\},
\]
on $\alpha$ \'es l'error tipus I, l'error tipus I m\`axim per sota del 
qual acceptarem igualtat de mitjanes compleix:
\[
\alpha_{\mbox{\footnotesize max}}=\pp{F_{2,5}>7.3864}\leq 1-0.975 =0.025.
\]
En vista del resultat anterior, rebutjam $H_0$ i concloem que les 
mitjanes de tots els grups no s\'on iguals.
}

\begin{probres}
{Sigui una mostra aleat\`oria simple que
dividim aleat\`oriament en 3 submostres de grand\`aries $n_1=3$,
$n_2=4$ i $n_3=3$, respectivament. Desitjam comprovar si
la lluminositat ambiental modifica la percepci\'o visual.
Per aix\`o, sotmetem la primera mostra a una tasca de
reconeixement de lletres amb baixa lluminositat
ambiental. La segona realitza la tasca amb alta
lluminositat i la tercera la realitza sota un nivell
mitj\`a de lluminositat. Els resultats s\'on els
seg\"uents (en nombre de lletres reconegudes
correctament):
\begin{center}
\begin{tabular}{|c|c|c|}
\hline
Lluminositat
baixa&Lluminositat alta&Lluminositat
mitjana\\\hline\hline
$4$&$\ \,9$&$12$\\\hline
$5$&$\ \,9$&$15$\\\hline
 &$11$& \\\hline
\end{tabular}
\end{center}
Trobau l'error tipus I m\`axim per poder acceptar que el nombre de lletres
reconegudes sota cada un dels $3$ nivells de lluminositat \'es el mateix. 
Fent servir el m\`etode de Scheff\'e, trobau entre quins
tractaments hi ha difer\`encies.}
\end{probres}

\res{
En aquest problema, el nombre de grups val $k=3$, la grand\`aria de 
cada mostra corresponent a cada grup \'es $n_1 =2$, $n_2 =3$ i $n_3 =2$ i
la grand\`aria total de la mostra val $n=n_1+n_2+n_3 =7$.

Vegem primer que no podem acceptar que el nombre de lletres 
reconegudes sota cada un dels tres nivell de lluminositat \'es el mateix.
\begin{itemize}
	\item[a)] C\`alcul de les quantitats de qu\`e depenen les variabilitats:
	\begin{center}
		\begin{tabular}{|>{$}c<{$}|>{$}c<{$}|>{$}c<{$}|>{$}c<{$}|>{$}c<{$}|>{$}c<{$}|}
			\hline
			Y_1 & Y_2 & Y_3 & Y_1^2 & Y_2^2 & Y_3^2  \\
			\hline\hline
			4 & \ \,9 & 12 & 16 & \ \,81 & 144  \\
			\hline
			5 & \ \,9 & 15 & 25 & \ \,81 & 225  \\
			\hline
			 & 11 &  &  & 121 &   \\
			\hline\hline
			9 & 29 & 27 & 41 & 283 & 369  \\
			\hline
		\end{tabular}
	\end{center}
	Quantitats:
	\begin{eqnarray*}
\sum_{j=1}^k \sum_{i=1}^{n_j} Y_{ij}^2 & = & 41+283+369=693, \\
\frac{{\left(\sum\limits_{j=1}^k \sum\limits_{i=1}^{n_j} Y_{ij}\right)}^2}{n} 
& = & \frac{{(9+29+27)}^2}{2+3+2}=\frac{4225}{7}=603.5714, \\
\sum_{j=1}^k\frac{{\left(\sum\limits_{i=1}^{n_j} Y_{ij}\right)}^2}{n_j} & 
= &
\frac{9^2}{2}+\frac{{29}^2}{3}+\frac{{27}^2}{2}=685.3333.
\end{eqnarray*} 
	\item[b)] C\`alcul de les variabilitats:
	\begin{eqnarray*}
VT & = & \sum_{j=1}^k \sum_{i=1}^{n_j} Y_{ij}^2 -
\frac{{\left(\sum\limits_{j=1}^k \sum\limits_{i=1}^{n_j} 
Y_{ij}\right)}^2}{n} = 693 - 603.5714 = 89.4285, \\
V_{tra} & = & \sum_{j=1}^k \sum_{i=1}^{n_j} Y_{ij}^2 -
\sum_{j=1}^k\frac{{\left(\sum\limits_{i=1}^{n_j} Y_{ij}\right)}^2}{n_j} =
693 - 685.3333 = 7.6666, \\
V_{ter} & = & VT - V_{tra} = 89.4285 - 7.6666 = 81.7619.
\end{eqnarray*}
	
	\item[c)] Taula per trobar l'estad\'{\i}stic $F$:
	
Recordem que  F.V. vol dir Font de Variaci\'o, g. ll. vol dir graus de llibertat i 
$MQ$, mitjana quadr\`atica:
\begin{center}
\renewcommand{\arraystretch}{1.5}
\begin{tabular}{|c||c|c|@{}c@{}|@{}c@{}|}
\hline
F.V.&$V$&g. ll.&$MQ=\frac{V}{\mbox{g. ll.}}$&$F=\frac{
MQ_{ter}}{MQ_{tra}}$\\\hline\hline
Inter&$81.76$&$k-1 
=3-1=2$&$MQ_{ter}=\frac{81.76}{2}=40.80$&\multirow{3}{3.1cm}
{$F=\frac{40.88}{1.916}=21.33$}\\\cline{1-4}
Intra&$\ 7.66$&$n-k=7-3=4$&$MQ_{tra}=\frac{7.66}{4}=1.916$&\\\cline{1-4}
Total&$89.42$&$n-1 =7-1=6$&&\\\hline
\end{tabular}
\end{center}
	 La distribuci\'o de l'estad\'{\i}stic $F$ \'es aproximadament la 
	 distribuci\'o $F$ de Fisher-Snedecor
	 amb $2$ i $4$ graus de llibertat. 
	 
	 La regi\'o cr\'{\i}tica \'es:
	 \[
	 R.C.=\{F>F_{2,4,1-\alpha}\},
	 \]
	 on $\alpha$ \'es l'error tipus I. 
	 
	 L'error tipus I m\`axim, per sota del qual acceptam que la mitjana del 
	 nombre de lletres reconegudes \'es el mateix sota cada un dels tres 
	 nivells de lluminositat, compleix:
	 \[
	 0.005\leq \alpha_{\mbox{\footnotesize max}} =
	\pp{F_{2,4}>21.33}\leq 0.01.
	 \]
	 En vista del resultat anterior, rebutjam $H_0$ i concloem que els 
	 distints tipus de lluminositat modifiquen la percepci\'o visual
\end{itemize}
A continuaci\'o, fent servir el m\`etode de Scheff\'e, vegem entre quins 
tractaments hi ha difer\`encies.

Considerarem que hi ha difer\`encies entre el tractament $i$ i el $j$ al 
nivell de significaci\'o $\alpha$ si:
\begin{equation}
\frac{\overline{Y}_i 
-\overline{Y}_j}{\sqrt{MQ_{tra}\left(\frac{1}{n_1}+\frac{1}{n_2}\right)}} \geq
\sqrt{ (k-1) F_{k-1,n-k,1-\alpha}}.
\label{CONDICIOSCHEFFE}
\end{equation}
En el nostre cas, tenim que:
\[ 
k=3,\ n=7,\ \overline{Y}_1 =4.5,\ \overline{Y}_2 =9.66,\ \overline{Y}_3 
=13.5\ \mbox{i }MQ_{tra}=1.916.
\]
El que farem \'es trobar, en cada cas, l'error tipus I m\`axim per sota 
del qual acceptarem que no hi ha difer\`encies entre el tractament $i$ i 
el $j$.

Abans de comen\c{c}ar fixau-vos que la condici\'o~(\ref{CONDICIOSCHEFFE}) 
\'es equivalent a:
\[
\frac{{\left({\overline{Y}_i 
-\overline{Y}_j}
\right)}^2}{(k-1)\cdot 
{MQ_{tra}\left(\frac{1}{n_1}+\frac{1}{n_2}\right)}}
\geq F_{k-1,n-k,1-\alpha}.
\]
Aquest darrera condici\'o ser\`a la que farem servir.
\begin{itemize}
	\item  Tractaments $1$ i $2$.
	
	Fent c\`alculs, acceptarem que hi ha difer\`encies entre els tractaments 
	$1$ i $2$ si:
	\[
	\frac{{({\overline{Y}_1 
-\overline{Y}_2)}^2}}{2\cdot 
MQ_{tra}\left(\frac{1}{2}+\frac{1}{3}\right)}
=8.3565\geq  F_{2,4,1-\alpha}.
	\]
	Per tant, l'error tipus I m\`axim per sota del qual acceptarem que no hi 
	ha difer\`encies entre els tractaments $1$ i $2$ compleix:
	\[
	0.01\leq \alpha_{\mbox{\footnotesize max},1,2}=\pp{F_{2,4}> 8.35}\leq 
	0.05.
	\]
	Concloem, doncs, que hi ha difer\`encies entre els tractaments $1$ i $2$.
	\item  Tractaments $1$ i $3$.
	
	Acceptarem que hi ha difer\`encies entre els tractaments 
	$1$ i $3$ si:
		\[
	\frac{{({\overline{Y}_1 
-\overline{Y}_3)}^2}}{2\cdot 
MQ_{tra}\left(\frac{1}{2}+\frac{1}{2}\right)}
=21.1304\geq  F_{2,4,1-\alpha}.
	\]
	Per tant, l'error tipus I m\`axim per sota del qual acceptarem que no hi 
	ha difer\`encies entre els tractaments $1$ i $3$ compleix:
	\[
	0.005\leq \alpha_{\mbox{\footnotesize max},1,3}=\pp{F_{2,4}>21.13}\leq 
	0.01.
	\]
	Concloem, doncs, que hi ha difer\`encies entre els tractaments $1$ i $3$.
	\item   Tractaments $2$ i $3$.
	
	Acceptarem que hi ha difer\`encies entre els tractaments 
	$2$ i $3$ si:
		\[
	\frac{{({\overline{Y}_2 
-\overline{Y}_3)}^2}}{2\cdot 
MQ_{tra}\left(\frac{1}{3}+\frac{1}{2}\right)}
=4.6\geq  F_{2,4,1-\alpha}.
	\]
	Per tant, l'error tipus I m\`axim per sota del qual acceptarem que no hi 
	ha difer\`encies entre els tractaments $2$ i $3$ compleix:
	\[
	0.05\leq \alpha_{\mbox{\footnotesize max},2,3}=\pp{F_{2,4}>4.6}\leq 
	0.1.
	\]
	Podem acceptar, tenint en compte que el valor de 
	$\alpha_{\mbox{\footnotesize max},2,3}$ \'es m\'es gran que $0.05$, que no hi ha difer\`encies 
	entre els tractaments $2$ i $3$.
\end{itemize}
}

\begin{probres}
{Per estudiar les difer\`encies entre $4$
fertilitzants damunt la producci\'o de patates, es va
disposar de $5$ finques, cada una de les quals es va
dividir en $4$ parce{\lgem}es de la mateixa grand\`aria i tipus. Els
fertilitzants varen ser assignats a l'atzar en les
parce{\lgem}es de cada finca. El rendiment en tones va
ser:
\begin{center}
\begin{tabular}{c|c||c|c|c|c|c|}
\multicolumn{2}{c|}{}&\multicolumn{5}{c|}{Finca}\\\cline{3-7}
\multicolumn{2}{c|}{}&$1$&$2$&$3$&$4$&$5$\\\hline\hline
\multirow{4}{2.5cm}{Fertilitzant}&$1$&$2.1$&$2.2$&$1.8$&$2.0$&$1.9$\\\cline{2-7}
&$2$&$2.2$&$2.6$&$2.7$&$2.5$&$2.8$\\\cline{2-7}
&$3$&$1.8$&$1.9$&$1.6$&$2.0$&$1.9$\\\cline{2-7}
&$4$&$2.1$&$2.0$&$2.2$&$2.4$&$2.1$\\\hline
\end{tabular}
\end{center}
Es desitja saber si existeixen difer\`encies entre els
fertilitzants i entre les finques.}
\end{probres}

\res{
En aquest problema el nombre de nivells de variaci\'o per al factor {\bf 
Fertilitzant} val $f=4$, el nombre de nivells de variaci\'o per al factor {\bf 
Finca} val $c=5$ i la grand\`aria 
de la mostra corresponent al nivell $i$ per 
al factor {\bf Fertilitzant} i al nivell $j$ per al factor {\bf Finca} 
val $n=1$.

Com ja es pot veure, es tracta d'un contrast ANOVA de dos factors.

Per trobar els estad\'{\i}stics que ens donaran tota la informaci\'o per saber 
si hi ha difer\`encies entre els fertilitzants i entre les finques, primer hem 
de trobar les variabilitats. I abans d'aix\`o, hem de trobar les 
quantitats de qu\`e depenen les variabilitats o variacions:
\begin{eqnarray*}
	\sum_{r=1}^f\sum_{s=1}^c\sum_{i=1}^n Y_{irs}^2 & = & 93.52,  \\
	\frac{{\left(\sum\limits_{r=1}^f\sum\limits_{s=1}^c\sum\limits_{i=1}^n 
	Y_{irs}\right)}^2}{nfc} & = & \frac{{42.8}^2}{20}=91.592,  \\
	\frac{\sum\limits_{r=1}^f\sum\limits_{s=1}^c{\left(\sum\limits_{i=1}^n 
	Y_{irs}\right)}^2}{n} & = & 93.52,  \\
	\frac{\sum\limits_{r=1}^f {\left(\sum\limits_{s=1}^c \sum\limits_{i=1}^n 
	Y_{irs}\right)}^2}{nc} & = & 
	\frac{{10}^2}{5}+\frac{{12.8}^2}{5}+\frac{{9.2}^2}{5}+\frac{{10.8}^2}{5}=93.024,  \\
	\frac{\sum\limits_{s=1}^c {\left(\sum\limits_{r=1}^f \sum\limits_{i=1}^n 
	Y_{irs}\right)}^2}{nf} & = & 
	\frac{{8.2}^2}{4}+\frac{{8.7}^2}{4}+\frac{{8.3}^2}{4}+\frac{{8.9}^2}{4}+\frac{{8.7}^2}{4}=91.68.
\end{eqnarray*}

A continuaci\'o, calculem les variacions. Direm $VT$ a la 
variaci\'o total, $V_{int}$ a la variaci\'o  deguda a la interacci\'o entre 
els dos factors, $V_f$ a la variaci\'o deguda a la difer\`encia entre 
files (en el nostre cas, Fertilitzants), $V_c$ a la variaci\'o deguda a 
la difer\`encia entre les columnes (en el nostre cas, Finques) i $V_e$ a 
la variaci\'o no explicada per difer\`encies entre files, entre columnes 
ni entre interacci\'o.
\begin{eqnarray*}
	VT & = & \sum_{r=1}^f\sum_{s=1}^c\sum_{i=1}^n Y_{irs}^2 -
	\frac{{\left(\sum\limits_{r=1}^f\sum\limits_{s=1}^c\sum\limits_{i=1}^n 
	Y_{irs}\right)}^2}{nfc} = 93.52 - 91.592 =1.928, \\
	V_e & = & \sum_{r=1}^f\sum_{s=1}^c\sum_{i=1}^n Y_{irs}^2 -
	 \frac{\sum\limits_{r=1}^f\sum\limits_{s=1}^c{\left(\sum\limits_{i=1}^n 
	Y_{irs}\right)}^2}{n} = 93.52 - 93.52 =0, \\
	V_f & = & \frac{\sum\limits_{r=1}^f {\left(\sum\limits_{s=1}^c \sum\limits_{i=1}^n 
	Y_{irs}\right)}^2}{nc} -\frac{{\left(\sum\limits_{r=1}^f\sum\limits_{s=1}^c\sum\limits_{i=1}^n 
	Y_{irs}\right)}^2}{nfc} = 93.024 - 91.592 =1.432,
	 \\
	V_c & = & \frac{\sum\limits_{s=1}^c {\left(\sum\limits_{r=1}^f \sum\limits_{i=1}^n 
	Y_{irs}\right)}^2}{nf}- \frac{{\left(\sum\limits_{r=1}^f\sum\limits_{s=1}^c\sum\limits_{i=1}^n 
	Y_{irs}\right)}^2}{nfc} = 91.68 - 91.592 = 0.088,\\
	V_{int} & = & VT- V_e - V_f - V_c = 1.928 - 1.432 - 0.088 = 0.408.
\end{eqnarray*}
A continuaci\'o, obtenim la taula per trobar els estad\'{\i}stics $F$ per 
estudiar la variaci\'o entre files, entre columnes i interacci\'o 
fila-columna. 

Recordem que FV significa Font de Variaci\'o, g.ll., graus de 
llibertat i $MQ$, mitjana quadr\`atica. Introdu\"{\i}m el par\`ametre $N$:
nombre total d'elements. En el nostre cas $N=nfc=20$.

\begin{center}
\renewcommand{\arraystretch}{1.5}
	\begin{tabular}{|c|c|c|c|c|}
		\hline
		FV&$V$&g. ll.&$MQ=\frac{V}{\mbox{g. ll.}}$&$F=\frac{
MQ}{MQ_{e}}$\\\hline\hline
		Fertilitzant & $1.43$ & $f-1=4-1=3$ & $0.47$ & $F_f=\frac{0.47}{0}=??$  \\
		\hline
		Finca & $0.08$ & $c-1=5-1=4$ & $0.02$ & $F_c =\frac{0.02}{0}=??$  \\
		\hline
		Interacci\'o & $0.40$ & $(f-1)(c-1)=12$ & $0.03$ & 
		$F_{int}=\frac{0.03}{0}=??$  \\
		\hline
		Error & $0.00$ & $(n-1)fc=0$ & $0.00$ &   \\
		\hline
		Total & $1.92$ & $N - 1=19$ &  &   \\
		\hline
	\end{tabular}
\end{center}
Com podem observar, no podem trobar els estad\'{\i}stics $F_f$, $F_c$ i 
$F_{int}$ per estudiar difer\`encies entre files, columnes i interacci\'o 
fila-columna respectivament, ja que la mitjana quadr\`atica corresponent a 
l'error val zero. Aix\`o ens passar\`a sempre que $n=1$. El motiu \'es que 
no tenim prou informaci\'o per estudiar la interacci\'o entre els dos 
factors, ja que per estudiar la interacci\'o la 
grand\`aria de la mostra 
corresponent al nivell $i-j$ ha de valer com a m\'{\i}nim $2$.

Aix\'{\i}, en aquest cas (i en tots el casos on $n=1$), hem de 
descompondre la variabilitat total en nom\'es $3$ factors: el degut a la 
variaci\'o entre files, el degut a la variaci\'o entre columnes i el 
degut a la variaci\'o no explicada, anomenat error:
\[
VT=V_f + V_c + V_e.
\]
A continuaci\'o, constru\"{\i}m novament la taula per trobar els 
estad\'{\i}stics per estudiar la variaci\'o entre els fertilitzants i 
entre les finques:
\begin{center}
\renewcommand{\arraystretch}{1.5}
	\begin{tabular}{|c|c|c|c|c|}
		\hline
		FV&$V$&g. ll.&$MQ=\frac{V}{\mbox{g. ll.}}$&$F=\frac{
MQ}{MQ_{e}}$\\\hline\hline
Fertilitzant&$1.43$&$f-1=3$&$0.47$&$F_f =\frac{0.47}{0.03}\approx 
14.04$\\\hline
Finca&$0.08$&$c-1=4$&$0.02$&$F_c =\frac{0.02}{0.03}\approx 0.65$\\\hline
Error&$0.40$&$(f-1)(c-1)=12$&$0.03$&\\\hline
Total&$1.92$&$N-1=19$&&\\\hline
	\end{tabular}
\end{center}

A continuaci\'o, vegem si hi ha difer\`encies entre fertilitzants i entre 
finques:
\begin{itemize}
	\item[a)] Difer\`encies entre fertilitzants.
	
	En aquest cas, tenim que l'estad\'{\i}stic $F_f$ es distribueix 
	aproximadament segons la distribuci\'o 
	de Fisher-Snedecor amb graus de 
	llibertat $3$ i $12$. 
	
	La regi\'o cr\'{\i}tica val:
	\[
	R.C.=\{F_f > F_{3,12,1-\alpha}\},
	\]
	on $\alpha$ \'es l'error tipus I.
	
	Per tant, l'error tipus I m\`axim per sota del qual acceptarem que no hi 
	ha difer\`encies entre fertilitzants compleix:
	\[
	\alpha_{\mbox{\footnotesize max}}=\pp{F_{3,12}> 14.03}\leq 0.005.
	\]
	Concloem, doncs, que hi ha difer\`encies entre fertilitzants.
	\item[b)] Difer\`encies entre finques.
	
	En aquest cas, tenim que l'estad\'{\i}stic $F_c$ es distribueix 
	aproximadament segons la distribuci\'o de 
	Fisher-Snedecor amb graus de 
	llibertat $4$ i $12$. 
	
	La regi\'o cr\'{\i}tica val:
	\[
	R.C.=\{F_c > F_{4,12,1-\alpha}\},
	\]
	on $\alpha$ \'es l'error tipus I.
	
	Per tant, l'error tipus I m\`axim per sota del qual acceptarem que no hi 
	ha difer\`encies entre finques compleix:
	\begin{eqnarray*}
	0.6 &\leq & \alpha_{\mbox{\footnotesize max}}=\pp{F_{4,12}> 
	0.64}=1-\pp{F_{12,4}>\frac{1}{0.64}}\\ &=& 
	1-\pp{F_{12,4}>1.54}\leq 0.7.
	\end{eqnarray*}
	Concloem, doncs, que no hi ha difer\`encies entre finques.
\end{itemize}
}

\newpage

\begin{probres}
{S'han estudiat les taxes de
consum d'oxigen en dues esp\`ecies de mo{\lgem}uscs ({\it Acmaea
Scabra} i {\it Acmaea Digitalis}) en $3$ concentracions
d'aigua de mar. \newline La variable mesurada \'es 
$\mu l O_2 /$(mg. de  cos sec. minut) a
$22^0$ C.
\newline Els resultats obtenguts s\'on:
\begin{center}
\begin{tabular}{c|c|r@{.}l|r@{.}l|r@{.}l|}
\multicolumn{8}{r|}{CONCENTRACI\'O D'AIGUA DE MAR}\\\hline
&&\multicolumn{2}{c|}{$100\%$}&
\multicolumn{2}{c|}{$75\%$}&
\multicolumn{2}{c|}{$50\%$}\\\cline{3-8}
\multirow{16}{2cm}{ESP\`ECIE}&\multirow{6}{2.5cm}{{\it A. Scabra}}&
$7$&$16$&$5$&$20$&$11$&$11$\\\cline{3-8}
&&$6$&$78$&$5$&$20$&$9$&$74$\\\cline{3-8}
&&$13$&$60$&$7$&$18$&$18$&$8$\\\cline{3-8}
&&$8$&$93$&$6$&$37$&$9$&$74$\\\cline{3-8}
&&$8$&$26$&$13$&$20$&$10$&$50$\\\cline{3-8}
&&$14$&$00$&$8$&$39$&$14$&$60$\\\cline{3-8}
&&$16$&$10$&$10$&$40$&$11$&$10$\\\cline{3-8}
&&$9$&$66$&$7$&$18$&$11$&$80$\\\cline{2-8}
&\multirow{6}{2.5cm}{{\it A. Digitalis}}&
$6$&$14$&$4$&$47$&$9$&$63$\\\cline{3-8}
&&$3$&$86$&$9$&$90$&$6$&$38$\\\cline{3-8}
&&$10$&$40$&$5$&$75$&$13$&$40$\\\cline{3-8}
&&$5$&$49$&$11$&$8$&$14$&$50$\\\cline{3-8}
&&$6$&$14$&$4$&$95$&$14$&$50$\\\cline{3-8}
&&$10$&$00$&$6$&$49$&$10$&$20$\\\cline{3-8}
&&$11$&$60$&$5$&$44$&$17$&$70$\\\cline{3-8}
&&$5$&$80$&$9$&$90$&$12$&$30$\\\hline
\end{tabular}
\end{center}

Es desitja saber si hi ha difer\`encies entre les
concentracions d'aigua de mar i entre les diferents
esp\`ecies i si hi ha interacci\'o entre la
concentraci\'o d'aigua de mar i l'esp\`ecie.}
\end{probres}

\res{
Per resoldre el problema, hem de realitzar un contrast ANOVA de dos factors amb interacci\'o.

El nombre de nivells del factor fila o ESP\`ECIE \'es $f=2$, el nombre de 
nivells del factor columna o CONCENTRACI\'O D'AIGUA DE MAR \'es $c=3$ i 
la grand\`aria de la mostra corresponent al nivell fila $i$ i al nivell 
columna $j$ \'es $n=8$. La grand\`aria total de la mostra ser\`a doncs: 
$N=nfc=8\cdot 2\cdot 3=48$.
\begin{itemize}
	\item[a)]  C\`alcul de les quantitats de qu\`e depenen les variabilitats.
	\begin{eqnarray*}
	\sum_{r=1}^f\sum_{s=1}^c\sum_{i=1}^n Y_{irs}^2 & = & 5065.153,  \\
	\frac{{\left(\sum\limits_{r=1}^f\sum\limits_{s=1}^c\sum\limits_{i=1}^n 
	Y_{irs}\right)}^2}{nfc} & = & \frac{{461.74}^2}{48}=4441.7464,  \\
	\frac{\sum\limits_{r=1}^f\sum\limits_{s=1}^c{\left(\sum\limits_{i=1}^n 
	Y_{irs}\right)}^2}{n} & = & \frac{1}{8}({84.49}^2 + {59.43}^2 + 
	{63.12}^2 + {58.7}^2 + {97.39}^2 + {98.61}^2)\\ & = & 4663.6317,  \\
	\frac{\sum\limits_{r=1}^f {\left(\sum\limits_{s=1}^c \sum\limits_{i=1}^n 
	Y_{irs}\right)}^2}{nc} & = & 
	\frac{{245}^2}{8\cdot 3}+\frac{{216.74}^2}{8\cdot 3} =4458.3844,  \\
	\frac{\sum\limits_{s=1}^c {\left(\sum\limits_{r=1}^f \sum\limits_{i=1}^n 
	Y_{irs}\right)}^2}{nf} & = & 
	\frac{{143.92}^2}{8\cdot 2}+\frac{{121.82}^2}{8\cdot 
	2}+\frac{{196}^2}{8\cdot 2}=4623.0674.
    \end{eqnarray*}
	\item[b)] C\`alcul de les variabilitats, on recordem que:
	\begin{itemize} 
	\item $VT$ vol dir variaci\'o total.
	\item $V_f$ vol dir variaci\'o deguda a la difer\`encia 
	entre files (ESP\`ECIE).
	\item $V_c$ vol dir variaci\'o deguda a la 
	difer\`encia entre columnes (CONCENTRACI\'O D'AIGUA DE MAR). 
	\item $V_{int}$ vol dir variaci\'o deguda a la interacci\'o fila-columna.
	\item $V_e$ vol dir variaci\'o no explicada.
	\end{itemize}
	\begin{eqnarray*}
	VT & = & \sum_{r=1}^f\sum_{s=1}^c\sum_{i=1}^n Y_{irs}^2 -
	\frac{{\left(\sum\limits_{r=1}^f\sum\limits_{s=1}^c\sum\limits_{i=1}^n 
	Y_{irs}\right)}^2}{nfc} = 5063.153 - 4441.74 \\ & = & 623.4065, \\
	V_e & = & \sum_{r=1}^f\sum_{s=1}^c\sum_{i=1}^n Y_{irs}^2 -
	 \frac{\sum\limits_{r=1}^f\sum\limits_{s=1}^c{\left(\sum\limits_{i=1}^n 
	Y_{irs}\right)}^2}{n} = 5065.153 - 4663.6317 \\ & = & 401.5213, \\
	V_f & = & \frac{\sum\limits_{r=1}^f {\left(\sum\limits_{s=1}^c \sum\limits_{i=1}^n 
	Y_{irs}\right)}^2}{nc} -\frac{{\left(\sum\limits_{r=1}^f\sum\limits_{s=1}^c\sum\limits_{i=1}^n 
	Y_{irs}\right)}^2}{nfc} = 4458.3844 - 4441.74 \\ & = & 16.6380,
	 \\
	V_c & = & \frac{\sum\limits_{s=1}^c {\left(\sum\limits_{r=1}^f \sum\limits_{i=1}^n 
	Y_{irs}\right)}^2}{nf}- \frac{{\left(\sum\limits_{r=1}^f\sum\limits_{s=1}^c\sum\limits_{i=1}^n 
	Y_{irs}\right)}^2}{nfc} = 4623.0674 - 4441.74 \\ & = & 181.3210,\\
	V_{int} & = & VT- V_e - V_f - V_c = 623.4065 - 401.5213 - 16.6380 - 
	181.3210 \\ & = & 23.9262.
\end{eqnarray*}
	
	\item[c)] Taula per trobar els estad\'{\i}stics.
	\begin{center}
\renewcommand{\arraystretch}{1.5}
	\begin{tabular}{|c|c|c|@{}c@{}|@{}c@{}|}
		\hline
		FV&$V$&g. ll.&$MQ=\frac{V}{\mbox{g. ll.}}$&$F=\frac{
MQ}{MQ_{e}}$\\\hline\hline
		ESP\`ECIE & $\ \,16.63$ & $f-1=2-1=1$ & $16.63$ & 
		$F_f=\frac{16.63}{9.56}=1.74$  \\
		\hline
		CONCENT. & $181.32$ & $c-1=3-1=2$ & $90.66$ & $F_c 
		=\frac{90.66}{9.56}=9.48$  \\
		\hline
		Interacci\'o & $\ \,23.92$ & $(f-1)(c-1)=2$ & $11.96$ & 
		$F_{int}=\frac{11.96}{9.56}=1.25$  \\
		\hline
		Error & $401.52$ & $(n-1)fc=42$ & $\ 9.56$ &   \\
		\hline
		Total & $623.40$ & $N - 1=47$ &  &   \\
		\hline
	\end{tabular}
\end{center}
\end{itemize}
Finalment, analitzem si hi ha hagut difer\`encies entre files, entre 
columnes i interacci\'o fila-columna.
\begin{itemize}
	\item  Estudi de la variaci\'o entre files. (factor ESP\`ECIE)
	
	En aquest cas, l'estad\'{\i}stic $F_f$ s'aproxima a la distribuci\'o $F$ 
	de Fisher-Snedecor amb $1$ i $42$ graus de llibertat.
	
	La regi\'o cr\'{\i}tica val:
	\[
	R.C. =\{F_f > F_{1,42,1-\alpha}\},
	\]
	on $\alpha$ \'es l'error tipus I.
	
	L'error tipus I m\`axim per sota del qual acceptam que no hi ha 
	difer\`encies entre files compleix:
	\[
	0.1\leq \alpha_{\mbox{\footnotesize max}}=\pp{F_{1,42}>1.74}\leq 0.2.
	\]
	Acceptam, doncs, que no hi ha difer\`encies degudes a l'esp\`ecie.

	\item  Estudi de la variaci\'o entre columnes. (factor CONCENTRACI\'O 
	D'AIGUA DE MAR)
	
	En aquest cas, l'estad\'{\i}stic $F_c$ s'aproxima a la distribuci\'o $F$ 
	de Fisher-Snedecor amb $2$ i $42$ graus de llibertat.
	
	La regi\'o cr\'{\i}tica val:
	\[
	R.C. =\{F_c > F_{2,42,1-\alpha}\},
	\]
	on $\alpha$ \'es l'error tipus I.
	
	L'error tipus I m\`axim per sota del qual acceptam que no hi ha 
	difer\`encies entre columnes compleix:
	\[
	\alpha_{\mbox{\footnotesize max}}=\pp{F_{2,42}>9.48}\leq 0.005.
	\]
	Concloem que hi ha difer\`encies entre les distintes concentracions 
	d'aigua de mar.
	\item  Estudi de la interacci\'o fila columna. (ESP\`ECIE-CONCENTRACI\'O 
	D'AIGUA DE MAR)
	
	En aquest cas, l'estad\'{\i}stic $F_{int}$ s'aproxima a la distribuci\'o $F$ 
	de Fisher-Snedecor amb $2$ i $42$ graus de llibertat.
	
	La regi\'o cr\'{\i}tica val:
	\[
	R.C. =\{F_{int} > F_{2,42,1-\alpha}\},
	\]
	on $\alpha$ \'es l'error tipus I.
	
	L'error tipus I m\`axim per sota del qual acceptam que no hi ha 
	interacci\'o compleix:
	\[
	0.2\leq \alpha_{\mbox{\footnotesize max}}=\pp{F_{2,42}>1.25}\leq 0.3.
	\]
	Concloem, doncs, que no hi ha interacci\'o.
\end{itemize}
{\footnotesize Com que a les taules no surt la distribuci\'o $F_{1,42}$, 
hem agafat en el seu lloc $F_{1,40}$ ja que l'error que hem com\`es \'es 
	despreciable.}
}

\section{Problemes proposats}

\begin{prob}
{Dotze persones s\'on distribu\"{\i}des en $4$
grups de $3$ persones cada un. A cada grup, li \'es
assignat aleat\`oriament un temps distint d'entrenament
abans de realitzar una tasca. Els resultats en
l'esmentada tasca, amb el corresponent temps
d'entrenament, es donen a la taula seg\"uent:
\newpage
\begin{center}
\begin{tabular}{|c|c|c|c|}
\hline
$0.5$ hores&$1$ hora&$1.5$ hores&$2$ hores\\\hline\hline
$1$&$4$&$3$&$\ \,8$\\\hline
$3$&$6$&$5$&$10$\\\hline
$5$&$2$&$7$&$\ \,6$\\\hline
\end{tabular}
\end{center}
Vegeu si podem rebutjar la
hip\`otesi nu{\lgem}a $H_0:\mu_1=\mu_2=\mu_3=\mu_4.$}
\end{prob}

\begin{prob}
{Es varen registrar les
freq\"u\`encies dels dies del mes que va ploure a
diferents hores, durant els mesos de gener, mar\c{c}, maig
i juliol. Les dades obtengudes, durant un per\'{\i}ode de
10 anys, varen ser:
\begin{center}
\begin{tabular}{|c||c|c|c|c||c|}
\hline
Hora&gener&febrer&mar\c{c}&juliol&Total\\\hline\hline
$\ \,9$&$\ \,22$&$\ \,25$&$\ \,24$&$\ \,11$&$\ \,82$\\\hline
$10$&$\ \,21$&$\ \,19$&$\ \,18$&$\ \,16$&$\ \,74$\\\hline
$11$&$\ \,17$&$\ \,23$&$\ \,26$&$\ \,17$&$\ \,83$\\\hline
$12$&$\ \,20$&$\ \,31$&$\ \,25$&$\ \,24$&$100$\\\hline
$13$&$\ \,16$&$\ \,15$&$\ \,23$&$\ \,24$&$\ \,78$\\\hline
$14$&$\ \,21$&$\ \,35$&$\ \,23$&$\ \,20$&$\ \,99$\\\hline\hline
Total&$117$&$148$&$139$&$112$&$536$\\\hline
\end{tabular}
\end{center}
Estudiau la variabilitat entre mesos i entre hores.
}
\end{prob}

\begin{prob}
{Es realitz\`a un estudi per 
determinar el nivell d'aigua i el tipus de planta sobre
la llargada global del tronc de les plantes de p\`esols.
Es varen utilitzar $3$ nivells d'aigua i $2$ tipus de
plantes. Es disposa per a l'estudi de $18$ plantes sense
fulles. Es divideixen aleat\`oriament aquestes plantes en $3$
subgrups i despr\'es se'ls assigna els nivells d'aigua
aleat\`oriament. Se segueix un procediment
semblant amb $18$ plantes convencionals. Es varen obtenir
els resultats seg\"uents (la llargada del tronc es d\'ona
en cent\'{\i}metres):
\newpage
\begin{center}
\begin{tabular}{c|c|c|c|c|}
&\multicolumn{4}{c|}{FACTOR AIGUA}\\\hline
& &{baix}&{mitj\`a}&{alt}\\\cline{3-5}
\multirow{12}{1.75cm}{FACTOR PLANTA}&\multirow{6}{1cm}{Sense Fulles}&
$69.0$&$\ \,96.1$&$121.0$\\\cline{3-5}
&&$71.3$&$102.3$&$122.9$\\\cline{3-5}
&&$73.2$&$107.5$&$123.1$\\\cline{3-5}
&&$75.1$&$103.6$&$125.7$\\\cline{3-5}
&&$74.4$&$100.7$&$125.2$\\\cline{3-5}
&&$75.0$&$101.8$&$120.1$\\\cline{2-5}
&\multirow{6}{1cm}{Amb Fulles}&$71.1$&$\ \,81.0$&$101.1$\\\cline{3-5}
&&$69.2$&$\ \,85.8$&$103.2$\\\cline{3-5}
&&$70.4$&$\ \,86.0$&$106.1$\\\cline{3-5}
&&$73.2$&$\ \,87.5$&$109.7$\\\cline{3-5}
&&$71.2$&$\ \,88.1$&$109.0$\\\cline{3-5}
&&$70.9$&$\ \,87.6$&$106.9$\\\hline
\end{tabular}
\end{center} 

Es desitja saber si hi ha difer\`ecies entre els nivells
d'aigua i entre els diferents tipus de planta. Tamb\'e es
vol saber si hi ha interacci\'o entre els nivells d'aigua
i els tipus de plantes.}
\end{prob}

\begin{prob}
{Les variables aleat\`ories $X_i$ segueixen la distribuci\'o
$N(m_i,\sigma^2),\ i=1,2,3,4$. Considerem les seg\"uents
mostres de grand\`aries $n_i=7$ de les esmentades variables aleat\`ories:
\begin{center}
\begin{tabular}{cccccccc}
$X_1$&$20$&$26$&$26$&$24$&$23$&$26$&$21$\\
$X_2$&$24$&$22$&$20$&$21$&$21$&$22$&$20$\\
$X_3$&$16$&$18$&$20$&$21$&$24$&$15$&$17$\\
$X_4$&$19$&$15$&$13$&$16$&$12$&$11$&$14$\\
\end{tabular}
\end{center}

\begin{itemize}
\item[a)] {Comprovau si les vari\`ancies s\'on iguals.}
\item[b)] {Contrastau la igualtat de mitjanes.}
\end{itemize}
{\footnotesize Final. Juny 91.}
}
\end{prob}

\begin{prob}
{Vegeu problema~\ref{NOTESASIG} (problemes proposats). 
Suposant normalitat en cada grup i igualtat de
vari\`ancies, vegeu si podem acceptar que tots els grups tenen
la mateixa nota mitjana.
\newline{\footnotesize Final. Juny 94.}}
\end{prob}






