\documentclass[12pt,t]{beamer}
% \documentclass[t]{beamer}
\usepackage[utf8]{inputenc}
\usepackage[catalan]{babel}
\usepackage{verbatim}
\usepackage{hyperref}
\usepackage{amsfonts,amssymb,amsmath,amsthm, wasysym}
\usepackage{listings}
\usepackage[T1]{fontenc}        
\usepackage{pgf}
%\usepackage{epsdice}
\usepackage{pgfpages}
\usepackage{tikz}
%\usetikzlibrary{arrows,shapes,plotmarks,backgrounds,trees,positioning}
%\usetikzlibrary{decorations.pathmorphing,calc,snakes}
%\usepackage{marvosym}
%
\usetheme[hideothersubsections,left]{Marburg}
\usecolortheme{sidebartab}
\useinnertheme[shadow]{rounded}
% \useoutertheme[footline=empty,subsection=true,compress]{infolines}
% \useoutertheme[footline=empty,subsection=true,compress]{miniframes}
% \usefonttheme{serif}

\setbeamertemplate{caption}[numbered]
\setbeamertemplate{navigation symbols}{}


\newcommand{\red}[1]{\textcolor{red}{#1}}
\newcommand{\green}[1]{\textcolor{green}{#1}}
\newcommand{\blue}[1]{\textcolor{blue}{#1}}
\newcommand{\gray}[1]{\textcolor{gray}{#1}}
\renewcommand{\emph}[1]{{\color{red}#1}}

\setbeamertemplate{frametitle}
{\begin{centering}
\medskip
\color{blue}
\textbf{\insertframetitle}
\medskip
\end{centering}
}
\usecolortheme{rose}
\usecolortheme{dolphin}
\mode<presentation>


\newcommand{\CC}{\mathbb{C}}
\newcommand{\RR}{\mathbb{R}}
\newcommand{\ZZ}{\mathbb{Z}}
\newcommand{\NN}{\mathbb{N}}
\newcommand{\KK}{\mathbb{K}}
\newcommand{\MM}{\mathcal{M}}
%\newcommand{\dbinom}{\displaystyle\binom}

\newcommand{\limn}{{\displaystyle \lim_{n\to\infty}}}
\renewcommand{\leq}{\leqslant}
\renewcommand{\geq}{\geqslant}
\def\tendeix{{\displaystyle\mathop{\longrightarrow}_{\scriptscriptstyle
n\to\infty}}}

\newcommand{\matriu}[1]{\left(\begin{matrix} #1 \end{matrix}\right)}

% \newcommand{\qed}{\hbox{}\nobreak\hfill\vrule width 1.4mm height 1.4mm depth 0mm
%     \par \goodbreak \smallskip}
%
% %
\theoremstyle{plain}
\newtheorem{teorema}{Teorema}
\newtheorem{prop}{Proposició}
\newtheorem{cor}{Coro\l.lari}
\theoremstyle{definition}
\newtheorem{exemple}{Exemple}
\newtheorem{defin}{Definició}
\newtheorem{obs}{Observació}

\newcounter{seccions}
\newcommand{\seccio}[1]{\addtocounter{seccions}{1}
\medskip\par\noindent\emph{\theseccions.
#1}\smallskip\par }

\newcommand{\EM}{\Omega}
\newcommand{\PP}{\mathcal{P}}

\title[\red{Matemàtiques III}]{}
\author[]{}
\date{}



\usepackage{Sweave}
\begin{document}
\Sconcordance{concordance:Contrastes3.tex:Contrastes3.Rnw:%
1 1368 1 48 0}

\beamertemplatedotitem

\lstset{backgroundcolor=\color{green!50}}
\lstset{breaklines=true}
\lstset{basicstyle=\ttfamily}


\begin{frame}
\vfill
\begin{center}
\gray{\LARGE Contrastos d'hipòtesis:}\\[1ex]

\gray{\LARGE Dues mostres}
\end{center}
\vfill
\end{frame}




 
%\part{Inferencia estadística}
%\frame{\titlepage}

%\section[Índice]{Distribuciones en las muestras y descripción de datos.}

%%%%%
\part{Contrastos d'hipòtesis}
% 
% \frame{\partpage}

\section{Dues mostres independents}

\begin{frame}
\frametitle{Contrastos de  dues mostres}

Ara volem contrastar el valor d'un mateix paràmetre a dues poblacions
\medskip

Dos tipus:
\begin{itemize}
\item \emph{Mostres independents:} Les dues mostres s'han obtingut de manera independent
\smallskip

\blue{Exemple:} Provam un medicament sobre dues mostres de malalts de característiques diferents
\medskip

\item \emph{Mostres aparellades:} Les dues mostres corresponen als mateixos
individus, o a individus aparellats d'alguna manera
\smallskip

\blue{Exemple:} Provam dos medicaments sobre els mateixos malalts
\end{itemize}



\blue{Teniu una Taula de contrastos de dues mostres exhaustiva a Campus Extens}

\end{frame}


\begin{frame}
\frametitle{Mostres independents}

Tenim dues variables aleatòries  (que representen dues poblacions)
\medskip

\blue{Exemple:} Homes i Dones
\bigskip

Volem comparar el valor d'un paràmetre a  les dues poblacions
\medskip

\blue{Exemple:} Són, de mitjana, els homes més alts que les dones?
\bigskip

Ho farem a partir d'una m.a.s.\ de cada v.a., escollides a més de manera independent
\end{frame}

\subsection{Z-test}

\begin{frame}
\frametitle{Contrastos per a $\mu$}

Tenim dues v.a.\ $X_1$ i $X_2$, de mitjanes $\mu_1$ i $\mu_2$
\medskip

Prenem m.a.s.
$$
\begin{array}{l}
X_{1,1}, X_{1,2},\ldots, X_{1,n_1}\quad\mbox{ de }X_1\\
X_{2,1}, X_{2,2},\ldots, X_{2,n_2}\quad\mbox{ de }X_2
\end{array}
$$
Siguin $\overline{X}_1$ i $\overline{X}_2$  les seves mitjanes, respectivament
\medskip

Les hipòtesis
$$
H: \mu_1\left\{\hspace*{-1ex}\begin{array}{c} > \\ < \\ \neq\\ = \end{array}\hspace*{-1ex}\right\}\mu_2 \mbox{\quad equivalen a\quad }
H: \mu_1-\mu_2\left\{\hspace*{-1ex}\begin{array}{c} > \\ < \\ \neq\\ = \end{array}\hspace*{-1ex}\right\}0
$$
Empram un estadístic de contrast per a $\mu_1-\mu_2$
\end{frame}



\begin{frame}
\frametitle{C.H. per a $\mu$ de  poblacions normals o $n$ grans: $\sigma$ conegudes}

Suposem una de les dues situacions següents:
\begin{itemize}
\item $X_1$ i $X_2$ són normals, o

\item $n_1$ i $n_2$ són grans ($n_1,n_2\geq 30\mbox{ o } \red{\bf 40})$
\end{itemize}
\medskip

Suposem que coneixem a més les desv. típ. $\sigma_1$ i $\sigma_2$ de $X_1$ i $X_2$
\medskip

En aquest cas l'estadístic de contrast és
$$
Z=\frac{\overline{X}_1-\overline{X}_2}{\sqrt{\frac{\sigma_1^2}{n_1}+\frac{\sigma_2^2}{n_2}}}
$$
i els contrastos són com en el cas quan $H_0:\mu=0$
\end{frame}

\begin{frame}
\frametitle{Exemple}

Volem comparar els temps de realització d'un test entre estudiants de dos graus $G_1$ i $G_2$, i contrastar si és veritat que els estudiants de $G_1$ empren menys temps que els de $G_2$
\medskip

Suposem que les desviacions típiques són conegudes: $\sigma_1=1$ i $\sigma_2=2$
\medskip

Disposam de dues mostres independents de tests realitzats per estudiants de cada grau, $n_1=n_2=40$. Calculam les mitjanes dels temps emprats a cada mostra (en minuts):
$$
\overline{X}_1= 9.789,\quad  \overline{X}_2=11.385
$$

\end{frame}




\begin{frame}
\frametitle{Exemple}

\emph{Contrast}:
$$
\left\{\begin{array}{l}
H_0:\mu_1=\mu_2\\
H_1:\mu_1< \mu_2
\end{array}\right.
\Longleftrightarrow
\left\{\begin{array}{l}
H_0:\mu_1-\mu_2=0\\
H_1:\mu_1- \mu_2<0
\end{array}\right.
$$

\emph{Estadístic de contrast}: $Z=\dfrac{\overline{X}_1-\overline{X}_2}{\sqrt{\frac{\sigma_1^2}{n_1}+\frac{\sigma_2^2}{n_2}}}\sim N(0,1)$
\medskip

\emph{Valor que pren}: $z_0=\dfrac{9.789-11.385}{\sqrt{\frac{1}{40}+\frac{4}{40}}}=-4.514$
\medskip

\emph{$p$-valor}: $P(Z\leq -4.514)\approx 3\cdot 10^{-6}$ molt petit
\medskip

\emph{Decisió}: Rebutjam la hipòtesi que són iguals, en favor que a $G_1$ tarden menys que a $G_2$
\end{frame}


\begin{frame}
\frametitle{Exemple}
També podem calcular un \emph{interval de confiança} del 95\% per a la
diferència de mitjanes $\mu_1-\mu_2$ al contrast
$$
\left\{\begin{array}{l}
H_0:\mu_1-\mu_2=0\\
H_1:\mu_1- \mu_2<0
\end{array}\right.
$$

Segons la \blue{Taula de Contrastos}, aquest interval és
{\small $$
\red{\left] -\infty, \overline{X}_1 -\overline{X}_2
-z_{\alpha}\sqrt{\frac{\sigma_1^2}{n_1}+\frac{\sigma_2^2}{n_2}}\right[}
$$}
Al nostre cas, per a $\alpha=0.05$, dóna
$$
]-\infty,-1.0145[
$$
0 no hi pertany: \emph{Rebutjam que $\mu_1-\mu_2=0$}

\end{frame}

\subsection{T-test}

\begin{frame}
\frametitle{C.H. per a $\mu$ de normals o $n$ grans: $\sigma$ desconegudes}
\vspace*{-2ex}

Suposem un altre cop una de les dues situacions següents, i que ara no coneixem  $\sigma_1$ i $\sigma_2$:
\begin{itemize}
\item $X_1$ i $X_2$ són normals, o

\item $n_1$ i $n_2$ són grans ($n_1,n_2\geq 40)$
\end{itemize}
\medskip

En aquest cas, hem de distingir dos subcasos:
\begin{enumerate}
\item[(1)] Suposam que $\sigma_1=\sigma_2$
\item[(2)] Suposam que $\sigma_1\neq \sigma_2$
\end{enumerate}
\medskip
\pause

Com decidim en quin cas estam? Dues possibilitats:
\begin{itemize}
\item Fem els dos casos, i si donen el mateix, és el que contestam
\item (\red{Si són normals}) Fem un contrast d'igualtat de variàncies per decidir quin és el cas
\end{itemize}



\end{frame}


\begin{frame}
\frametitle{C.H. per a $\mu$ de normals o $n$ grans: $\sigma$ desconegudes}

Si suposam que $\sigma_1=\sigma_2$, l'estadístic de contrast és
$$
T=\frac{\overline{X}_1-\overline{X}_2}%
{\sqrt{(\frac{1}{n_1}+\frac{1}{n_2})\cdot 
\frac{(n_1-1)\widetilde{S}_1^2+(n_2-1)\widetilde{S}_2^2}%
{n_1+n_2-2}}}
$$
que, quan $\mu_1=\mu_2$, té distribució (aproximadament, en cas de mostres grans) $t_{n_1+n_2-2}$
\end{frame}


\begin{frame}
\frametitle{C.H. per a $\mu$ de normals o $n$ grans: $\sigma$ desconegudes}
\vspace*{-3ex}

Si suposam que $\sigma_1\neq \sigma_2$, l'estadístic de contrast és
$$
T=\frac{\overline{X}_1-\overline{X}_2}{\sqrt{\frac{\widetilde{S}_1^2}{n_1}+\frac{\widetilde{S}_2^2}{n_2}}}\sim t_f
$$
que, quan $\mu_1=\mu_2$, té distribució (aproximadament, en cas de mostres grans) $t_{f}$ amb
$$
f=\left\lfloor\frac{\displaystyle \left( \frac{\widetilde{S}_1^2}{n_1}+\frac{\widetilde{S}_2^2}{n_2}\right)^2}%
{\displaystyle \frac{1}{n_1-1}\left(\frac{\widetilde{S}_1^2}{n_1}\right)^2+\frac{1}{n_2-1}\left(\frac{\widetilde{S}_2^2}{n_2}\right)^2}\right\rfloor -2
$$



En els dos casos, els contrastos són com en el cas de $H_0:\mu=0$
\end{frame}

\begin{frame}
\frametitle{Exemple}

Volem comparar els temps de realització d'un test entre estudiants de dos graus $G_1$ i $G_2$, i determinar si és veritat que els estudiants de $G_1$ empren menys temps que els de $G_2$
\medskip

No coneixem $\sigma_1$ i $\sigma_2$
\medskip


Disposam de dues mostres independents  de tests realitzats per estudiants de cada grau, $n_1=n_2=40$. Calculam les mitjanes i les desviacions típiques
mostrals dels temps emprats a cada mostra:
$$
\begin{array}{ll}
\overline{X}_1= 9.789,&\quad  \overline{X}_2=11.385,\\[1ex]
\widetilde{S}_1=1.201,&\quad \widetilde{S}_2=1.579
\end{array}
$$
\end{frame}

\begin{frame}
\frametitle{Exemple}
\vspace*{-2ex}

\textbf{\blue{Cas 1: Suposam $\sigma_1=\sigma_2$}}
\medskip

\emph{Contrast}:
$$
\left\{\begin{array}{l}
H_0:\mu_1=\mu_2\\
H_1:\mu_1< \mu_2
\end{array}\right.
\Longleftrightarrow
\left\{\begin{array}{l}
H_0:\mu_1-\mu_2=0\\
H_1:\mu_1- \mu_2<0
\end{array}\right.
$$

\emph{Estadístic de contrast}: 
$$
T=\frac{\overline{X}_1-\overline{X}_2}%
{\sqrt{(\frac{1}{n_1}+\frac{1}{n_2})\cdot 
\frac{(n_1-1)\widetilde{S}_1^2+(n_2-1)\widetilde{S}_2^2}%
{n_1+n_2-2}}}\sim t_{40+40-2}
$$

\medskip

\emph{Valor que pren}: 
$$
t_0=\frac{9.789-11.385}{\sqrt{(\frac{1}{40}+\frac{1}{40})\frac{39\cdot 1.201^2+39\cdot 1.579^2}{78}}}=-5.0881
$$
\end{frame}

\begin{frame}
\frametitle{Exemple}
\vspace*{-2ex}

\textbf{\blue{Cas 1: Suposam $\sigma_1=\sigma_2$}}
\medskip

\emph{$p$-valor}: 
$$
P(t_{78}<-5.0881)= 1.2\cdot 10^{-6},
$$
molt petit
\medskip

\emph{Decisió}: Rebutjam la hipòtesi que són iguals, en favor que a $G_1$ tarden menys que a $G_2$
\end{frame}




\begin{frame}
\frametitle{Exemple}
\vspace*{-2ex}

\textbf{\blue{Cas 2: Suposam $\sigma_1\neq \sigma_2$}}
\medskip

\emph{Contrast}:
$$
\left\{\begin{array}{l}
H_0:\mu_1=\mu_2\\
H_1:\mu_1< \mu_2
\end{array}\right.
\Longleftrightarrow
\left\{\begin{array}{l}
H_0:\mu_1-\mu_2=0\\
H_1:\mu_1- \mu_2<0
\end{array}\right.
$$

\emph{Estadístic de contrast}: 
$$
T=\frac{\overline{X}_1-\overline{X}_2}{\sqrt{\frac{\widetilde{S}_1^2}{n_1}+\frac{\widetilde{S}_2^2}{n_2}}}\sim t_f
$$
on
{\small
$$
\hspace*{-2ex}f=\left\lfloor\frac{\displaystyle \left( \frac{1.201^2}{40}+\frac{1.579^2}{40}\right)^2}%
{\displaystyle \frac{1}{39}\left(\frac{1.201^2}{40}\right)^2+\frac{1}{39}\left(\frac{1.579^2}{40}\right)^2}\right\rfloor -2
=\lfloor 72.81\rfloor-2=70
$$
}
\end{frame}




\begin{frame}
\frametitle{Exemple}
\vspace*{-2ex}

\textbf{\blue{Cas 2: Suposam $\sigma_1\neq \sigma_2$}}
\medskip

\emph{Valor que pren}: 
$$
t_0=\frac{9.789-11.385}{\sqrt{\frac{1.201^2}{40}+\frac{1.579^2}{40}}}=-5.0881
$$
\medskip

\emph{$p$-valor}: 
$P(t_{70}\leq-5.0881)=1.5\cdot 10^{-6}$ molt petit
\medskip

\emph{Decisió}: Rebutjam la hipòtesi que són iguals, en favor que a $G_1$ tarden menys que a $G_2$
\pause\bigskip 

\emph{\bf Decisió final}: Els dos casos han donat el mateix, així que concloem que a $G_1$ tarden menys que a $G_2$


\end{frame}



\subsection{Fisher-test}

\begin{frame}
\frametitle{Contrast per a dues proporcions: Test de Fisher}

Tenim dues variables aleatòries $X_1$ i $X_2$ Bernoulli de proporcions  $p_1$ i $p_2$
\medskip

Prenem m.a.s. de cada una i obtenim la taula següent
\begin{center}
\begin{tabular}{c|cc|c|}
  & $X_1$ & $X_2$ &  Total\\\hline
Èxits & $n_{11}$ & $n_{12}$ &  $n_{1\bullet}$\\
Fracassos & $n_{21}$ & $n_{22}$ &  $n_{2\bullet}$ \\\hline 
Total & $n_{\bullet1}$ & $n_{\bullet2}$ & $n_{\bullet\bullet}$
\\\hline
\end{tabular}
\end{center}
\medskip

La hipòtesi nul·la que contrastam és $H_0: p_1=p_2$




\end{frame}

\begin{frame}
\frametitle{Contrast per a dues proporcions: Test de Fisher}
\vspace*{-2ex}

\begin{center}
\begin{tabular}{c|cc|c|}
  & $X_1$ & $X_2$ &  Total\\\hline
Èxits & $n_{11}$ & $n_{12}$ &  $n_{1\bullet}$\\
Fracassos & $n_{21}$ & $n_{22}$ &  $n_{2\bullet}$ \\\hline 
Total & $n_{\bullet1}$ & $n_{\bullet2}$ & $n_{\bullet\bullet}$
\\\hline
\end{tabular}
\end{center}
\medskip

Si $p_1=p_2$, la probabilitat d'obtenir $n_{11}$ èxits dins $X_1$ és la de:
\begin{quote}
\red{En una bossa hi tenim $n_{1\bullet}$ bolles E i $n_{2\bullet}$ bolles F. Probabilitat d'obtenir $n_{11}$ bolles E si en triam $n_{\bullet1}$ de cop.}
\end{quote}
És una hipergeomètrica $H(n_{1\bullet},n_{2\bullet},n_{\bullet1})$. L'empram per calcular els p-valors.

\end{frame}

\begin{frame}
\frametitle{Contrast per a dues proporcions: Test de Fisher}

\blue{Contrast:}
$
\left\{\begin{array}{l}
H_0:p_1=p_2\\
H_1:p_1> p_2
\end{array}\right.
$
\medskip

\red{p-valor}: $P(H(n_{1\bullet},n_{2\bullet},n_{\bullet1})\geq n_{11})$
\bigskip

\blue{Contrast:}
$
\left\{\begin{array}{l}
H_0:p_1=p_2\\
H_1:p_1< p_2
\end{array}\right.
$
\medskip

\red{p-valor}: $P(H(n_{1\bullet},n_{2\bullet},n_{\bullet1})\leq n_{11})$
\bigskip

\blue{Contrast:}
$
\left\{\begin{array}{l}
H_0:p_1=p_2\\
H_1:p_1\neq  p_2
\end{array}\right.
$
\medskip

\red{p-valor}: $\min\left\{2\min\{P(H\leq n_{11}), P(H\geq n_{11})\},1\right\}$.
\medskip
 
(\red{Controvertit}; per exemple, R ho fa d'una altra manera)

\bigskip





\end{frame}


\begin{frame}
\frametitle{Exemple}
\vspace*{-1ex}

Per determinar si la Síndrome de Mort Sobtada del Nadó (SIDS) té component genètic, es consideren els casos de SIDS en parelles de bessons monozigòtics i dizigòtics. Diguem: 
\begin{itemize}
\item $p_1$: proporció de parelles de bessons monozigòtics amb algun cas de SIDS on només un germà la sofrí

\item $p_2$: proporció de parelles de bessons dizigòtics amb algun cas de SIDS on només un germà la sofrí
\end{itemize}

Si la SIDS té component genètic, és d'esperar que $p_1<p_2$
\medskip

Hem de realitzar el contrast
$$
\left\{\begin{array}{l}
H_0:p_1=p_2\\
H_1:p_1< p_2
\end{array}\right.
$$
\end{frame}

\begin{frame}
\frametitle{Exemple}

En un estudi (Peterson et al, 1980), s'obtingueren les dades següents:
\begin{center}
\begin{tabular}{l|cc|c|}
\multicolumn{4}{c}{\textbf{\hphantom{ \textbf{Casos de }} Tipus de bessons}} \\ \hline
 \textbf{Casos de SIDS} & Monozigòtics & Dizigòtics & Total \\
 Un  & 23 & 35 & 58\\
 Dos & 1 & 2 & 3\\\hline
Total  & 24 & 37 & 61
\\\hline
\end{tabular}
\end{center}

\red{p-valor:}
$$
P(H(58,3,24)\leq 23) =\mbox{\texttt{phyper(23,58,3,24)}}=0.7841
$$

No podem rebutjar la hipòtesi nu\l.la

\end{frame}




%%%%%Continuar

\subsection{Prop-test}

\begin{frame}
\frametitle{Contrast per a dues proporcions: mostres grans}

Tenim dues variables aleatòries $X_1$ i $X_2$ Bernoulli de proporcions  $p_1$ i $p_2$
\medskip

Prenem m.a.s. \emph{grans} ($n_1,n_2\geq 50$ o \red{\bf 100})
$$
\begin{array}{l}
X_{1,1}, X_{1,2},\ldots, X_{1,n_1}\quad\mbox{ de }X_1\\
X_{2,1}, X_{2,2},\ldots, X_{2,n_2}\quad\mbox{ de }X_2
\end{array}
$$
Siguin $\widehat{p}_1$ i $\widehat{p}_2$  les seves proporcions mostrals
\medskip

Suposem que els nombres d'èxits i de fracasos a cada mostra són $\geq 5$ o \red{\bf 10})
\medskip


La hipòtesi nul·la que contrastam és $H_0: p_1=p_2$, que hem d'entendre $H_0: p_1-p_2=0$
\end{frame}


\begin{frame}
\frametitle{Contrast per a dues proporcions: mostres grans}

L'estadístic de contrast és
$$
Z=\frac{\widehat{p}_1 -\widehat{p}_2}{
\sqrt{\Big(\frac{n_1 \widehat{p}_1 +n_2 \widehat{p}_2}{n_1
+n_2}\Big)\Big(1-\frac{n_1 \widehat{p}_1 +n_2 \widehat{p}_2}{n_1
+n_2}\Big)\Big(\frac{1}{n_1}+\frac{1}{n_2}
\Big)}}$$
i té distribució aproximadament $N(0,1)$ si $p_1=p_2$.
\bigskip

Els contrastos són com en el cas de $H_0:p=0$

\end{frame}
\begin{frame}
\frametitle{Exemple}

Es prenen una mostra d'ADN de 100 individus amb almenys tres generacions
familiars a l'illa de Mallorca, i una altra de 50 individus amb almenys tres generacions
familiars a l'illa de  Menorca
\medskip

Es vol saber si un determinat al·lel d'un gen és present amb la mateixa proporció a les dues poblacions
\medskip

A la mostra mallorquina, 20 individus el tenen, i a la mostra menorquina, 12
\medskip

\blue{Contrastau la hipòtesi d'igualtat de proporcions al
nivell de significació $0.05$, i calculau l'interval de
confiança per a la diferència de proporcions per a aquest $\alpha$} 
\medskip

\red{Suposau que 50 és prou gran}



\end{frame}
\begin{frame}
\frametitle{Exemple}


\emph{Contrast}:
$$
\left\{\begin{array}{l}
H_0:p_1=p_2\\
H_1:p_1\neq p_2
\end{array}\right.
$$

\emph{Estadístic de contrast}: 
$$
Z=\frac{\widehat{p}_1 -\widehat{p}_2}{
\sqrt{\Big(\frac{n_1 \widehat{p}_1 +n_2 \widehat{p}_2}{n_1
+n_2}\Big)\Big(1-\frac{n_1 \widehat{p}_1 +n_2 \widehat{p}_2}{n_1
+n_2}\Big)\Big(\frac{1}{n_1}+\frac{1}{n_2}
\Big)}}\sim N(0,1)$$
\medskip

\emph{Valor que pren}: $\widehat{p}_1=0.2$, $\widehat{p}_2=0.24$, $n_1=100$, $n_2=50$
$$
z_0=\frac{0.2 -0.24}{
\sqrt{\Big(\frac{20 +12}{100+50}\Big)\Big(1-\frac{20 +12}{100+50}\Big)\Big(\frac{1}{100}+\frac{1}{50}
\Big)}}
=-0.5637$$

\end{frame}
\begin{frame}
\frametitle{Exemple}

\emph{$p$-valor}: $2\cdot P(Z\geq 0.5637)=0.573$
\medskip

\emph{Decisió}: Com que el $p$-valor és més gran que $\alpha=0.05$, acceptam la hipòtesi que les dues proporcions són la mateixa

\end{frame}
\begin{frame}
\frametitle{Exemple}

L'\emph{interval de confiança} per a  $p_1-p_2$
al nivell de confiança $(1-\alpha)\cdot 100\%$ en un contrast bilateral és
{\small $$
\begin{array}{l}
\left]\widehat{p}_1-\widehat{p}_2-z_{1-\frac{\alpha}{2}}\sqrt{\Big(\frac{n_1 \widehat{p}_1 +n_2 \widehat{p}_2}{n_1
+n_2}\Big)\Big(1-\frac{n_1 \widehat{p}_1 +n_2 \widehat{p}_2}{n_1
+n_2}\Big)\Big(\frac{1}{n_1}+\frac{1}{n_2}
\Big)},\right.\\[2ex]
\quad
\left.\widehat{p}_1-\widehat{p}_2+z_{1-\frac{\alpha}{2}}\sqrt{\Big(\frac{n_1 \widehat{p}_1 +n_2 \widehat{p}_2}{n_1
+n_2}\Big)\Big(1-\frac{n_1 \widehat{p}_1 +n_2 \widehat{p}_2}{n_1
+n_2}\Big)\Big(\frac{1}{n_1}+\frac{1}{n_2}
\Big)}
\right[
\end{array}
$$

}
\pause\bigskip

$$
\begin{array}{l}
](0.2 -0.24)-1.96\cdot 0.071,(0.2 -0.24)+1.96\cdot 0.071[\\
\qquad\qquad =]-0.179,0.099[
\end{array}
$$
Conté el 0, per tant no podem rebutjar que $p_1-p_2=0$
\end{frame}


\begin{frame}
\frametitle{Contrastos una mica més generals}

Fins ara hem pres $H_0:\mu_1=\mu_2$. Un tipus de contrastos lleugerament més generals serien
$$
\left\{\begin{array}{l}
H_0:\mu_1-\mu_2=\Delta\\
H_1:\mu_1-\mu_2<\Delta\mbox{ o }\mu_1-\mu_2>\Delta\mbox{ o }\mu_1-\mu_2\neq\Delta
\end{array}\right.
$$
amb $\Delta\in \RR$.
\medskip

Es fan igual, modificant lleugerament l'estadístic: substituïm als numeradors
\begin{center}
$\overline{X}_1-\overline{X}_2$ per
\red{$\overline{X}_1-\overline{X}_2-\Delta$}
\end{center}


\end{frame}

\begin{frame}
\frametitle{Exemple}
Tenim dos tractaments, A i B, d'una malaltia. Tractam 50 malalts amb A i 100 amb B. 20 malalts  tractats amb A i 25 tractats amb B manifesten haver sentit malestar general durant els 7 dies posteriors a iniciar el tractament.
\medskip

Podem concloure, a un nivell de significació del 5\%, que A produeix malestar general en una proporció dels malalts que és 5 punts percentuals superior  a la proporció dels malalts en què el produeix B?

\end{frame}

%%%%%%%%%

\begin{frame}
\frametitle{Exemple}

$p_1$: Fracció de malalts en què A produeix malestar general\\
$p_2$: Fracció de malalts en què B produeix malestar general
\medskip

\emph{Contrast}:
$$
\left\{\begin{array}{l}
H_0:p_1\leq p_2+0.05\\
H_1:p_1>p_2+0.05
\end{array}\right.
$$

\emph{Estadístic de contrast}:
$$
Z=\frac{\widehat{p}_1 -\widehat{p}_2-\Delta}{
\sqrt{\Big(\frac{n_1 \widehat{p}_1 +n_2 \widehat{p}_2}{n_1
+n_2}\Big)\Big(1-\frac{n_1 \widehat{p}_1 +n_2 \widehat{p}_2}{n_1
+n_2}\Big)\Big(\frac{1}{n_1}+\frac{1}{n_2}
\Big)}}\sim N(0,1)
$$
\bigskip

\end{frame}
\begin{frame}
\frametitle{Exemple}


\emph{Valor que pren}: $\widehat{p}_1=0.4$, $\widehat{p}_2=0.25$, $n_1=50$, $n_2=100$, $\Delta=0.05$
$$
z_0=\frac{0.4 -0.25-0.05}{
\sqrt{\Big(\frac{20+25}{50+100}\Big)\Big(1-\frac{20+25}{50+100}\Big)\Big(\frac{1}{50}+\frac{1}{100}
\Big)}}
=1.26
$$
\medskip

\emph{$p$-valor}: $P(Z\geq 1.26)= 0.104$
\medskip

\emph{Decisió}: Com que el $p$-valor és més gran que $\alpha=0.05$, no podem rebutjar la hipòtesi que $p_1-p_2$ és inferior a un $5\%$

\end{frame}
\begin{frame}
\frametitle{Exemple}

L'\emph{interval de confiança} per a  $p_1-p_2$
al nivell de confiança $(1-\alpha)\cdot 100\%$ en aquest contrast és

{\footnotesize $$
\left[\widehat{p}_1-\widehat{p}_2+z_{\alpha}\sqrt{\Big(\frac{n_1 \widehat{p}_1 +n_2 \widehat{p}_2}{n_1
+n_2}\Big)\Big(1-\frac{n_1 \widehat{p}_1 +n_2 \widehat{p}_2}{n_1
+n_2}\Big)\Big(\frac{1}{n_1}+\frac{1}{n_2}
\Big)},\infty
\right[
$$
}
Operant:
$$
[(0.4-0.25)-1.645\cdot 0.0794,\infty [=[0.0194,\infty[
$$
Conté $0.05$, per tant no podem rebutjar que $p_1\leq p_2+ 0.05$ (però en canvi, no conté 0 i per tant podríem rebutjar que $p_1=p_2$)
\end{frame}


\subsection{Var-test}
\begin{frame}
\frametitle{Contrast per a dues variàncies}

Necessitam decidir si les variàncies de les dues poblacions són iguals o diferents, per exemple en el marc d'una comparació de mitjanes de mostres independents
\medskip


Tenim dues variables aleatòries $X_1$ i $X_2$ \emph{normals} de desviacions típiques $\sigma_1$, $\sigma_2$ desconegudes
\medskip

Prenem m.a.s. 
$$
\begin{array}{l}
X_{1,1}, X_{1,2},\ldots, X_{1,n_1}\quad\mbox{ de }X_1\\
X_{2,1}, X_{2,2},\ldots, X_{2,n_2}\quad\mbox{ de }X_2
\end{array}
$$
Siguin $\widetilde{S}_1^2$ i $\widetilde{S}_2^2$  les seves variàncies mostrals
\end{frame}


\begin{frame}
\frametitle{Contrast per a dues variàncies}


El contrast té hipòtesi nul·la $H_0: \sigma_1=\sigma_2$, que correspon a
\red{$H_0:\dfrac{\sigma^2_1}{\sigma_2^2}=1$}
\medskip

S'empra l'estadístic de contrast
$$
F=\frac{\widetilde{S}_1^2}{\widetilde{S}_2^2}
$$
que, si les dues poblacions són normals, té distribució \emph{$F$ de Fisher} amb graus de llibertat $n_1-1$ i $n_2-1$
\end{frame}

\begin{frame}
\frametitle{La distribució $F$ de Fisher}

La distribució \red{$F_{n,m}$}, on $n,m$ són els \emph{graus de llibertat},
és la  d'una variable aleatòria
$$
{\chi_{n}^2}/{\chi_m^2}
$$
Té densitat
$$
f_{F_{n,m}}(x)=\frac{\Gamma(\frac{n+m}{2})\cdot(\frac{m}{n})^{m/2}x^{(m-2)/2}}%
{\Gamma(\frac{n}{2})\Gamma(\frac{m}{2})(1+\frac{m}{n}x)^{(m+n)/2}}
\quad\mbox{ si $x\geq 0$}
$$
on $\Gamma(x)=\int_{0}^{\infty} t^{x-1}e^{-t}\, dt$ si $x> 0$
\bigskip


La distribució està tabulada (\emph{Teniu les taules a Campus Extens}), i amb R és \texttt{f} 
\medskip

No és simètrica. \blue{Els $p$-valors es calculen com en el cas de la $\chi^2$.} Alerta en el cas bilateral!

\end{frame}


\begin{frame}
\frametitle{Exemple}
Recordau l'exemple on volíem comparar els temps de realització d'un test 
entre estudiants de dos graus $G_1$ i $G_2$. \emph{Suposem que aquests temps segueixen distribucions normals}.
\medskip

Disposam de dues mostres independents de tests realitzats per estudiants de cada grau, $n_1=n_2=40$. Calculam les desviacions típiques
mostrals dels temps emprats a cada mostra:
$$
\widetilde{S}_1=1.201,\quad \widetilde{S}_2=1.579
$$


\blue{Contrastau la hipòtesi d'igualtat de variàncies al
nivell de significació $0.05$} 

\end{frame}


\begin{frame}
\frametitle{Exemple}


\emph{Contrast}:
$$
\left\{\begin{array}{l}
H_0:\sigma_1=\sigma_2\\
H_1:\sigma_1\neq \sigma_2
\end{array}\right.
$$

\emph{Estadístic de contrast}: 
$$
F=\frac{\widetilde{S}_1^2}{\widetilde{S}_2^2}\sim F_{39,39}
$$
\medskip

\emph{Valor que pren}: $\widetilde{S}_1=1.201$, $\widetilde{S}_2=1.579$
$$
f_0=\frac{1.201^2}{1.579^2}=0.5785
$$

\end{frame}

\begin{frame}
\frametitle{Exemple}

\emph{$p$-valor}: No és simètrica
$$
\begin{array}{l}
2\cdot P(F_{39,39}\geq 0.5785)= 1.909\\
2\cdot P(F_{39,39}\leq 0.5785)=0.091
\end{array}
$$
El $p$-valor és $0.091$
\medskip

\emph{Decisió}: Com que el $p$-valor és més gran que $\alpha=0.05$, no podem rebutjar la hipòtesi que les dues variàncies són iguals. \medskip

\blue{Concloem que $\sigma_1= \sigma_2$}. Aquesta seria l'assumpció que hauríem de fer en el test de les $\mu$.

\end{frame}
%\begin{frame}
%\frametitle{Exemple}
%
%L'interval de confiança per a  $\sigma_1^2/\sigma_2^2$
%al nivell de confiança $(1-\alpha)\cdot 100\%$ és
%$$
%\left]\frac{\widetilde{S}_1^2}{\widetilde{S}_2^2}\cdot F_{n_1-1,n_2-1,\frac{\alpha}{2}},\frac{\widetilde{S}_1^2}{\widetilde{S}_2^2}\cdot F_{n_1-1,n_2-1,1-\frac{\alpha}{2}}\right[
%$$
%Dóna
%$$
%]0.5785\cdot 0.529, 0.5785\cdot 1.891[=]0.306,1.094[
%$$
%Conté el 1
%\end{frame}
%
%
%

\begin{frame}
\frametitle{Exemple}
Es desitja comparar l'activitat motora espontània d'un grup de 25 rates control i un altre de 36 rates desnodrides. Es va mesurar el nombre de vegades que passaven davant d'una cèl·lula fotoelèctrica durant 24 hores. Les dades obtingudes van ser les següents

\begin{center}
\begin{tabular}{|c|c|c|c|}
\hline
& $n$ & $\overline{X}$ & $\widetilde{S}$ \\ \hline
1. Control & 25 &  $869.8$ & $106.7$\\ \hline
2. Desnodrides & 36 & $665$ & $133.7$\\ \hline
\end{tabular}
\end{center}

\blue{S'observen diferències significatives entre el grup de control i el grup desnodrit? }\pause\medskip

\emph{Suposarem que aquests nombres de vegades segueixen  distribucions normals}
\end{frame}

\begin{frame}
\frametitle{Exemple}
\emph{Contrast}:
$$
\left\{\begin{array}{l}
H_0:\mu_1=\mu_2\\
H_1:\mu_1\neq \mu_2
\end{array}\right.
$$
\medskip

Per poder-lo efectuar, efectuarem primer el contrast
$$
\left\{\begin{array}{l}
H_0:\sigma_1=\sigma_2\\
H_1:\sigma_1\neq \sigma_2
\end{array}\right.
$$
per decidir quin test fer
\end{frame}

\begin{frame}
\frametitle{Exemple}


\emph{Contrast}:
$$
\left\{\begin{array}{l}
H_0:\sigma_1=\sigma_2\\
H_1:\sigma_1\neq \sigma_2
\end{array}\right.
$$

\emph{Estadístic de contrast}: 
$$
F=\frac{\widetilde{S}_1^2}{\widetilde{S}_2^2}\sim F_{24,35}
$$
\medskip

\emph{Valor que pren}: $\widetilde{S}_1=106.7$, $\widetilde{S}_2=133.7$
$$
f_0=\frac{106.7^2}{133.7^2}=0.637
$$

\end{frame}
\begin{frame}
\frametitle{Exemple}

\emph{$p$-valor}: 
$$
\begin{array}{l}
2\cdot P(F_{24,35}\leq 0.637)= 0.25\\
2\cdot P(F_{24,35}\geq 0.637)=1.75
\end{array}
$$
El $p$-valor és $0.25$, gran
\medskip

\emph{Decisió}:  $\sigma_1= \sigma_2$

\end{frame}

\begin{frame}
\frametitle{Exemple}
\emph{Contrast}:
$$
\left\{\begin{array}{l}
H_0:\mu_1=\mu_2\\
H_1:\mu_1\neq \mu_2
\end{array}\right.
$$

\emph{Estadístic de contrast}: Com que suposam que $\sigma_1= \sigma_2$
$$
T=\frac{\overline{X}_1-\overline{X}_2}%
{\sqrt{(\frac{1}{n_1}+\frac{1}{n_2})\cdot 
\frac{(n_1-1)\widetilde{S}_1^2+(n_2-1)\widetilde{S}_2^2}%
{n_1+n_2-2}}}\sim t_{25+36-2}
$$

\medskip

\emph{Valor que pren}: 
$$
t_0=\frac{869.8-665}%
{\sqrt{(\frac{1}{25}+\frac{1}{36})\cdot 
\frac{24\cdot 106.7^2+35\cdot 133.7^2}%
{25+36-2}}}=6.37
$$
\end{frame}

\begin{frame}
\frametitle{Exemple}

\emph{$p$-valor}: 
$2\cdot P(t_{59}\geq 6.37)\approx 0$
\medskip

\emph{Decisió}: Hi ha diferència (i com que $\overline{x}_2<\overline{x}_1$, concloem que les desnodrides es mouen menys)
\end{frame}






\section{Dues mostres aparellades}

\begin{frame}
\frametitle{Mostres aparellades}

Fins ara hem considerat que les mostres eren independents
\medskip

Un cas completament diferent és quan les dues mostres corresponen als mateixos
individus o a individus aparellats per algun factor 
\medskip

\blue{Exemples:}
\begin{itemize}
\item Mesuram l'estat d'una malaltia als mateixos individus abans i després d'un tractament

\item Mesuram la incidència de càncer en parelles de germans bessons
\end{itemize}
Es parla en aquest cas de \emph{mostres aparellades} (\textsl{paired})
\end{frame}

\begin{frame}
\frametitle{Mostres aparellades}

Per decidir si hi ha diferències, el contrast més comú consisteix a calcular les diferències
dels valors de cadascuna de les parelles de mostres i realitzar un
contrast per esbrinar si la mitjana de les diferències és  0

\pause \vspace*{1cm}



És important observar aquí que hi ha diferents maneres de
realitzar un disseny experimental per contrastar una hipòtesi, i que el disseny s'ha de fixar abans de la recollida de dades
\end{frame}

\subsection{Contrast  per a dues $\mu$}
\begin{frame}
\frametitle{Exemple: Contrast de dues mitjanes de mostres aparellades}

Disposam de dos algoritmes de plegament de proteïnes. Tots dos
produeixen resultats de la mateixa qualitat
\medskip

Estam interessats a saber quin dels dos és \emph{més eficient}, en el sentit que té la mitjana de temps d'execució
més petita. Suposam que aquests temps d'execució segueixen lleis normals.
\medskip

Prenem una mostra de proteïnes, li aplicam els dos algoritmes, i anotam els temps d'execució sobre cada proteïna
\end{frame}


\begin{frame}
\frametitle{Exemple: Contrast de dues mitjanes de mostres aparellades}
Els resultats obtinguts són:
\begin{table}
\centering
\scalebox{0.80}[0.8]{
\begin{tabular}{c|cccccccccc}
& 1 & 2 & 3 & 4 & 5 & 6 & 7 & 8 & 9 & 10\\
\hline
alg. 1  & 8.1 & 11.9 & 11.4 & 12.9 & 9.0 & 7.2 & 12.4 & 6.9 & 8.9 & 8.3\\
\hline
alg. 2 & 6.9 & 6.7 & 8.3 & 8.6 & 18.9 & 7.9 & 7.4 & 8.7 & 7.9 &
12.4\\
\hline
$d$ & 1.2 & 5.2 & 3.1 & 4.3 & -9.9 & -0.7 & 5.0 & -1.8 &
1.0 & -4.1
\end{tabular}

}
\end{table}
(La filera $d$ conté les diferències de temps entre el primer i el segon algoritme)
$$
\overline{d}=0.33,\ \widetilde{S}_d=4.72
$$
\smallskip

\blue{Volem contrastar la igualtat de mitjanes amb el test que correspongui. I si són diferents, decidir quin té major temps d'execució.}




\end{frame}

\begin{frame}
\frametitle{Exemple: Contrast de dues mitjanes de mostres aparellades}
\emph{Contrast}:
$$
\left\{\begin{array}{l}
H_0:\mu_1=\mu_2\\
H_1:\mu_1\neq \mu_2
\end{array}\right.
$$


Consultam la taula. L'estadístic de contrast és
$$
T=\frac{\overline{d}}{\widetilde{S}_d/\sqrt{n}}
$$
que té distribució $t_{n-1}=t_9$. Pren el valor
$$
t_0=\frac{0.33}{4.72/\sqrt{10}}=0.22
$$


\end{frame}

\begin{frame}
\frametitle{Exemple: Contrast de dues mitjanes de mostres aparellades}


El $p$-valor és
$$
2P(t_9\geq 0.22)=0.83
$$
molt gran, no podem rebutjar la hipòtesi nul·la que els temps mitjans són iguals. Per tant, no té sentit cercar quin té el temps d'execució més gran
\end{frame}




\subsection{Contrast per a dues proporcions}
\begin{frame}
\frametitle{Exemple: Contrast de dues proporcions de mostres aparellades}

Es pren una mostra de $1000$ persones afectades per migranya. Se'ls
facilita un fàrmac perquè n'alleugereixi els símptomes.
\medskip

Després de l'administració se'ls pregunta si han notat alleujament en
el dolor
\medskip

Al cap d'un temps es subministra als mateixos individus un placebo
i se'ls torna a preguntar si han notat o no milloria
\end{frame}

\begin{frame}
\frametitle{Exemple: Contrast de dues proporcions de mostres aparellades}

Els resultats són:
$$
\begin{tabular}{|c|c|cc|}
\cline{3-4}
\multicolumn{2}{c|}{}& \multicolumn{2}{|c|} {Placebo}\\\cline{3-4}
\multicolumn{2}{c|}{} & Sí & No \\\hline
Fàrmac & Sí & 300 & 62 \\
& No & 38 & 590
\\\hline
\end{tabular}
$$

\blue{És més efectiu el fàrmac que el placebo?}
\end{frame}


\begin{frame}
\frametitle{Exemple: Contrast de dues proporcions de mostres aparellades}

$p_1$: Proporció que troba milloria amb el fàrmac\\
$p_2$: Proporció que troba milloria amb el placebo
\medskip

\red{Contrast}:
$$
\left\{\begin{array}{l}
H_0:p_1=p_2\\
H_1:p_1> p_2
\end{array}\right.
$$
\end{frame}


\begin{frame}
\frametitle{Exemple: Contrast de dues proporcions de mostres aparellades}
\vspace*{-2ex}

Consultam la taula. L'estadístic de contrast és
$$
Z=\frac{\frac{b}{n}-\frac{d}{n}}{\sqrt{\frac{b+d}{n^2}}}\sim N(0,1)
$$
on
$$
\begin{tabular}{|c|c|cc|}
\cline{3-4}
\multicolumn{2}{c|}{}& \multicolumn{2}{|c|} {Placebo}\\\cline{3-4}
\multicolumn{2}{c|}{} & Sí & No \\\hline
Fàrmac & Sí & $a$ & $b$ \\
& No & $d$ & $c$
\\\hline
\end{tabular}
$$
Aquest contrast només és vàlid quan la mostra és gran i el nombre de \emph{casos discordants} $b+d$ és ``bastant gran'', posem $\geq 20$

\end{frame}


\begin{frame}
\frametitle{Exemple: Contrast de dues proporcions de mostres aparellades}

L'estadístic de contrast té el valor $z_0=2.4$
\medskip

El $p$-valor és
$$
P(Z>2.4)=0.0082,
$$
petit. Per tant, concloem que el fàrmac és més efectiu que el placebo.
\end{frame}




\end{document}
