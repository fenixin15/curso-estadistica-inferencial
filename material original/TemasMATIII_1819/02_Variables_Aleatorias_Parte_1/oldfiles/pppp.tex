\documentclass[handout]{beamer}
\begin{document}
\begin{frame}
Si denotamos por $F_X(x_0^{-})=\displaystyle{\lim_{x\to x_0^{-}}} F(x)$,
entonces se cumple que $P(X< x_0)=F_X(x_0^{-})$ y que $P(X=x_0)=F_X(x_0)-F_X(x_0^{-})$.

$\lim_{x\to x_0^{-}} F(x)$
\end{frame}



\begin{frame}

      Sea $F_{X}$ la función de distribución  de una  v.a. $X$ entonces:
\begin{enumerate}[a)]
\item  $0\leq F_{X}(x)\leq 1$.
\item La función $F_{X}$ es no decreciente.
\item La función $F_{X}$ es continua por la derecha.
\item Si denotamos por $F_X(x_0^{-})=\displaystyle{\lim_{x\to x_0^{-}}} F(x)$,
entonces se cumple que $P(X< x_0)=F_X(x_0^{-})$ y que $P(X=x_0)=F_X(x_0)-F_X(x_0^{-})$.
%            \item Se cumple que $\displaystyle\lim_{x\to\infty}F_{X}(x)=1$;
%            $\lim_{x\to-\infty}F_{X}(x)=0$.
%            \item  Toda función $F$ verificando las propiedades anteriores es función de
%            distribución de alguna v.a. $X$.
%            \item $P(X>x)=1-F_{X}(x)$
%            \item Dados $a,b\in\R$ con $a<b$ $$P(a<X\leq b)=F_{X}(b)-F_{X}(a)$$
\end{enumerate}
\end{frame}
\end{document}