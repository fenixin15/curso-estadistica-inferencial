 
 
 \documentclass[12pt,t]{beamer}
% \documentclass[t]{beamer}
\usepackage[utf8]{inputenc}
\usepackage[catalan]{babel}
\usepackage{verbatim}
\usepackage{hyperref}
\usepackage{amsfonts,amssymb,amsmath,amsthm, wasysym, multirow}
\usepackage{listings}
\usepackage[T1]{fontenc}        
\usepackage{pgf}
\usepackage{epsdice}
\usepackage{pgfpages}
\usepackage{tikz}
%\usetikzlibrary{arrows,shapes,plotmarks,backgrounds,trees,positioning}
%\usetikzlibrary{decorations.pathmorphing,calc,snakes}
%\usepackage{marvosym}
%
\usetheme[hideothersubsections,left]{Marburg}
\usecolortheme{sidebartab}
\useinnertheme[shadow]{rounded}
% \useoutertheme[footline=empty,subsection=true,compress]{infolines}
% \useoutertheme[footline=empty,subsection=true,compress]{miniframes}
% \usefonttheme{serif}

\setbeamertemplate{caption}[numbered]
\setbeamertemplate{navigation symbols}{}


\newcommand{\red}[1]{\textcolor{red}{#1}}
\newcommand{\green}[1]{\textcolor{green}{#1}}
\newcommand{\blue}[1]{\textcolor{blue}{#1}}
\newcommand{\gray}[1]{\textcolor{gray}{#1}}
\renewcommand{\emph}[1]{{\color{red}#1}}

\setbeamertemplate{frametitle}
{\begin{centering}
\medskip
\color{blue}
\textbf{\insertframetitle}
\medskip
\end{centering}
}
\usecolortheme{rose}
\usecolortheme{dolphin}
\mode<presentation>


\newcommand{\CC}{\mathbb{C}}
\newcommand{\RR}{\mathbb{R}}
\newcommand{\ZZ}{\mathbb{Z}}
\newcommand{\NN}{\mathbb{N}}
\newcommand{\KK}{\mathbb{K}}
\newcommand{\MM}{\mathcal{M}}
%\newcommand{\dbinom}{\displaystyle\binom}

\newcommand{\limn}{{\displaystyle \lim_{n\to\infty}}}
\renewcommand{\leq}{\leqslant}
\renewcommand{\geq}{\geqslant}
\def\tendeix{{\displaystyle\mathop{\longrightarrow}_{\scriptscriptstyle
n\to\infty}}}

\newcommand{\matriu}[1]{\left(\begin{matrix} #1 \end{matrix}\right)}

% \newcommand{\qed}{\hbox{}\nobreak\hfill\vrule width 1.4mm height 1.4mm depth 0mm
%     \par \goodbreak \smallskip}
%
% %
\theoremstyle{plain}
\newtheorem{teorema}{Teorema}
\newtheorem{prop}{Proposició}
\newtheorem{cor}{Coro\l.lari}
\theoremstyle{definition}
\newtheorem{exemple}{Exemple}
\newtheorem{defin}{Definició}
\newtheorem{obs}{Observació}

\newcounter{seccions}
\newcommand{\seccio}[1]{\addtocounter{seccions}{1}
\medskip\par\noindent\emph{\theseccions.
#1}\smallskip\par }

\newcommand{\EM}{\Omega}
\newcommand{\PP}{\mathcal{P}}

\title[\red{Matemàtiques III}]{}
\author[]{}
\date{}



\begin{document}
\beamertemplatedotitem

\lstset{backgroundcolor=\color{green!50}}
\lstset{breaklines=true}
\lstset{basicstyle=\ttfamily}


\begin{frame}
\vfill
\begin{center}
\gray{\LARGE ANOVA}
\end{center}
\vfill
\end{frame}




 
%\part{Inferencia estadística}
%\frame{\titlepage}

%\section[Índice]{Distribuciones en las muestras y descripción de datos.}

%%%%%
\part{ANOVA}
% 
% \frame{\partpage}

 
\section{Blocs complets aleatoris}
\subsection{Model bàsic}
\begin{frame}
\frametitle{Blocs complets aleatoris}

Generalitza el contrast de mitjanes  per a dues mostres aparellades, a $k$ mostres aparellades. L'objectiu és reduir l'efecte de variables estranyes, no contemplades en l'estudi.
\medskip

Suposem que tenim $k$ tractaments que volem comparar
\medskip

Escollim \emph{blocs}: conjunts de $k$ individus relacionats  (per exemple, $k$ còpies del mateix individu)
\medskip

Dins cada bloc, assignam aleatòriament a cada individu un tractament
\end{frame}



\begin{frame}
\frametitle{Blocs complets aleatoris}


En els experiments de \emph{Blocs complets aleatoris}:
\medskip

\begin{itemize}
\item S'han aparellat els individus en blocs  (\red{blocs})
\medskip

\item Els tractaments s'assignen de manera aleatòria dins els blocs  (\red{aleatoris})
\medskip

\item Cada tractament s'empra exactament una vegada dins cada bloc (\red{complets})
\medskip

\item Pel que fa als tractaments, és d'\emph{efectes fixats}  (la inferència serà vàlida només per als 
tractaments emprats)
\medskip

\item Pel que fa als blocs, pot ser d'\emph{efectes fixats}  (es trien tots els blocs adients) o \emph{aleatori}: en aquest cas el model és \emph{mixt}

\end{itemize}
\end{frame}


\begin{frame}
\frametitle{Blocs complets aleatoris}

Les dades es presenten en una taula:
{\small \begin{center}
\begin{tabular}{c|cccc}
\hline
\multicolumn{5}{c}{\hphantom{Blocs} Tractaments}\\\hline Bloc & Tractament $1$&Tractament
$2$&$\ldots$&Tractament $k$\\\hline
$1$&$X_{11}$&$X_{21}$&$\ldots$&$X_{k1}$\\
$2$&$X_{12}$&$X_{22}$&$\ldots$&$X_{k2}$\\
$\vdots$&$\vdots$&$\vdots$&$\vdots$&$\vdots$\\
$b$&$X_{1b}$&$X_{2b}$&$\ldots$&$X_{kb}$\\\hline
\end{tabular}
\end{center}

}
\medskip

\red{$X_{ij}$}: valor del tractament $i$-èsim en l'individu corresponent del bloc $j$-èsim 
\medskip

\red{\bf ALERTA!} A $X_{\red{i}\blue{j}}$, \red{$i$} hi indica la \red{columna}, tractament, i \blue{$j$} la \blue{filera}, bloc


\end{frame}

\begin{frame}
\frametitle{Blocs complets aleatoris}

El contrast que es vol realitzar és
$$
\left.
\begin{array}{l}
H_0 : \mu_{1\bullet} =\mu_{2\bullet} =\cdots =\mu_{k\bullet} \\
H_1 : \exists i,j \mid  \mu_{i\bullet} \not=\mu_{j\bullet}
\end{array}
\right\}
$$
on cada $\mu_{i\bullet}$ representa la mitjana del tractament $i$-èsim 
\end{frame}





\begin{frame}
\frametitle{Exemple 1}
Volem determinar si l'energia que es requereix per dur a terme tres activitats físiques (córrer, passejar i muntar amb bicicleta) és la mateixa o no. Per quantificar aquesta energia, mesuram el nombre de Kca consumides per Km
recorregut
\medskip

Les diferències metabòliques entre els individus poden afectar l'energia requerida per dur a terme una determinada
activitat
\medskip

Per tant, no és aconsellable triar tres grups d'individus i a cada un fer-li fer una de les tres activitats físiques: les diferències metabòliques entre els individus triats podrien  afectar els resultats i donar massa variació
\end{frame}



\begin{frame}
\frametitle{Exemple 1}

El que fem és seleccionar alguns individus (els \emph{blocs}), demanar a cadascun
que corri, camini i recorri amb bicicleta una distància fixada, i determinar per a cada individu el nombre
de Kca consumides per Km durant cada activitat
\medskip

Cada individu és utilitzat com un bloc.  Les activitats es realitzen en ordre aleatori, amb temps de recuperació entre
l'una i l'altra.
\medskip

En aparellar cada individu amb ell mateix, eliminam l'efecte de la variació individual
\medskip

\emph{Disseny de blocs complets aleatoris mixt}


\end{frame}

\begin{frame}
\frametitle{Exemple 1}
En la taula següent es mostren els resultats obtinguts per a 8
individus:

\begin{center}
\begin{tabular}{cccc}
\hline
&\multicolumn{3}{c}{Tractament}\\\hline
Bloc &1 (corrent) &2 (caminant)&3 (pedalejant)\\\hline
1&1.4&1.1&0.7\\
2&1.5&1.2&0.8\\
3&1.8&1.3&0.7\\
4&1.7&1.3&0.8\\
5&1.6&0.7&0.1\\
6&1.5&1.2&0.7\\
7&1.7&1.1&0.4\\
8&2.0&1.3&0.6\\\hline
\end{tabular}
\end{center}
\end{frame}


\begin{frame}
\frametitle{Exemple 1}

El contrast que volem realitzar és
$$
\left.
\begin{array}{l}
H_0 : \mu_{1\bullet} = \mu_{2\bullet} = \mu_{3\bullet} \\
H_1 : \exists i,j \mid  \mu_{i\bullet}
\not=\mu_{j\bullet}
\end{array}
\right\}
$$
on $\mu_{i\bullet}$, $i=1,2,3$ representa el nombre mitjà de Kca
consumides per Km mentre es corre, es passeja o es munta amb bicicleta,
respectivament
\end{frame}

%
%
%\begin{frame}
%\frametitle{Exemple 2}
%Es vol comparar en un assaig clínic els efectes de dos analgèsics i un placebo en el tractament de la cefalea. Per quantificar aquest efecte, mesuram
%el temps que triga a desaparèixer una cefalea amb el tractament.
%\medskip
%
%No totes les cefalees són de la mateixa intensitat i no tots els individus tenen la mateixa percepció del dolor
%\medskip
%
%Per tant, no és aconsellable triar tres grups de malalts i subministrar a cadascun d'ells un tractament diferent:  hi hauria una gran variació individual en les respostes que no seria  explicada pel tractament
%
%
% \end{frame}
%
%
%
%\begin{frame}
%\frametitle{Exemple 2}
%
%El que farem serà aplicar els tres tractaments als mateixos individus en diferents episodis de cefalea
%\medskip
%
%Cada individu és utilitzat com un bloc.  Els tractaments s'apliquen en ordre aleatori.
%\medskip
%
%En aparellar cada individu amb ell mateix, eliminam l'efecte de la variació individual
%\medskip
%
%Disseny de blocs complets aleatoris mixt
%
%\end{frame}
%
%
%
%
%\begin{frame}
%\frametitle{Exemple 2}
%La taula següent dóna els resultats obtinguts per a 5
%pacients:
%
%\begin{center}
%\begin{tabular}{cccc}
%\hline
%&\multicolumn{3}{c}{Tractament}\\\hline
%Bloc &1 (placebo) &2 (analgèsic A)&3 (analgèsic B)\\\hline
% 1 & 35 & 20 & 22\\
%2 & 40 & 35 & 42\\
%3 & 60 & 50 & 30\\
% 4 & 50 & 40 & 35\\
%5 & 50 & 30 & 22 \\\hline
%\end{tabular}
%\end{center}
%\end{frame}
%
%\begin{frame}
%\frametitle{Exemple 2}
%
%El contrast que volem realitzar és
%$$
%\left.
%\begin{array}{l}
%H_0 : \mu_{1\bullet} = \mu_{2\bullet} = \mu_{3\bullet} \\
%H_1 : \exists i,j \mid  \mu_{i\bullet}
%\not=\mu_{j\bullet}
%\end{array}
%\right\}
%$$
%on $\mu_{i\bullet}$, $i=1,2,3$ representa el temps mitjà de recuperació d'una cefalea amb el placebo, amb l'analgèsic A i amb l'analgèsic B, respectivament
%\end{frame}
%
%


\begin{frame}
\frametitle{Model}
\vspace*{-2ex}

\emph{Expressió matemàtica} del model considerat:
$$
\hspace*{-1ex} X_{ij}=\mu+ (\mu_{i\bullet}-\mu) +(\mu_{\bullet j}-\mu) + E_{ij}, \ \mbox{\footnotesize $i=1,\ldots,k,\ j=1,\ldots,b$}
$$
on:
\begin{itemize}
\item \red{$X_{ij}$}: valor del tractament $i$-èsim en el bloc $j$-èsim 
\smallskip

\item  \red{$\mu$}: mitjana global
\smallskip

\item  \red{$\mu_{i\bullet}$}: mitjana del  tractament $i$-èsim
\smallskip

\item  \red{$\mu_{\bullet j}$}:  mitjana del  bloc $j$-èsim
\smallskip

\item \red{$\mu_{i\bullet}-\mu$}: efecte del
tractament $i$-èsim (\emph{efecte tractament})
\smallskip

\item \red{$\mu_{\bullet j}-\mu$}: efecte de pertànyer al bloc
$j$-èsim (\emph{efecte bloc})
\smallskip

\item \red{$E_{ij}$}: error residual o aleatori
\end{itemize}
\end{frame}


\begin{frame}
\frametitle{El model}

Les suposicions del model són:
\begin{itemize}

\item Les $k\cdot b$ observacions constitueixen mostres aleatòries
independents, cadascuna de mida $1$, de $k\cdot b$ poblacions
\medskip

\item Aquestes $k\cdot b$ poblacions són totes normals i amb la mateixa
variància $\sigma^2$
\medskip

\item L'efecte dels blocs i els tractaments és \emph{additiu}: no hi ha \emph{interacció} entre els blocs i els tractaments:

\begin{itemize}
\item La diferència de les mitjanes poblacionals de cada parella concreta de blocs és la mateixa a cada tractament \smallskip

\item  La diferència de les mitjanes poblacionals de cada parella concreta de tractaments és la mateixa a cada bloc 
\end{itemize}
\end{itemize}
\end{frame}


\begin{frame}
\frametitle{No interacció}

Mesuram tres variables en homes i dones, i tenim les mitjanes poblacionals de cada variable dins cada grup
\medskip

\emph{No interacció:}
\begin{center}
\begin{tabular}{lccc}
\hline
&\multicolumn{3}{c}{Tractament}\\\hline
Bloc&$A$&$B$&$C$\\\hline
Homes&$\mu_{11}=4$&$\mu_{21}=5$&$\mu_{31}=7$\\
Dones&$\mu_{12}=3$&$\mu_{22}=4$&$\mu_{32}=6$\\\hline
\end{tabular}
\end{center}
\medskip

\emph{Sí interacció:}
\begin{center}
\begin{tabular}{lccc}
\hline
&\multicolumn{3}{c}{Tractament}\\\hline
Bloc&$A$&$B$&$C$\\\hline
Homes&$\mu_{11}=4$&$\mu_{21}=5$&$\mu_{31}=7$\\
Dones&$\mu_{12}=3$&$\mu_{22}=4$&$\mu_{32}=2$\\\hline
\end{tabular}
\end{center}

\end{frame}


\begin{frame}
\frametitle{Estadístics}
\begin{itemize}
\item $\red{T_{i\bullet}} = \sum\limits_{j=1}^b X_{ij}$, suma total del tractament $i$-èsim, $i=1,2,\ldots,k$
\medskip

\item $\red{\overline{X}_{i\bullet}} =\dfrac{T_{i\bullet}}{b}$, mitjana mostral del tractament $i$-èsim, $i=1,2,\ldots,k$
\medskip


\item $\red{T_{\bullet j}}=\sum\limits_{i=1}^k X_{ij}$, suma total del bloc $j$-èsim,  $j=1,2,\ldots,b$
\medskip

\item $\red{\overline{X}_{\bullet j}} =\dfrac{T_{\bullet j}}{k}$, mitjana mostral del bloc $j$-èsim,  $j=1,2,\ldots,b$

\end{itemize}
\end{frame}


\begin{frame}
\frametitle{Estadístics}
\begin{itemize}

\item $\red{T_{\bullet\bullet}}=\sum\limits_{i=1}^k\sum\limits_{j=1}^b X_{ij}=\sum\limits_{i=1}^k
T_{i\bullet}= \sum\limits_{j=1}^b T_{\bullet j}$, suma total
\medskip

\item $\red{\overline{X}_{\bullet\bullet}}=\dfrac{T_{\bullet\bullet}}{k\cdot b}$,
mitjana mostral global
\medskip

\item $\red{T^{(2)}_{\bullet\bullet}}=\sum\limits_{i=1}^k\sum\limits_{j=1}^b X_{ij}^2$, suma total de quadrats 
\end{itemize}
\end{frame}


\begin{frame}
\frametitle{Exemple 1}
Les dades:

\begin{center}
\begin{tabular}{ccccc}
\cline{1-4}
&\multicolumn{3}{c}{Tractament} & \\\cline{1-4}
Bloc &1 &2 &3  & \\\cline{1-4}
1&1.4&1.1&0.7 & $T_{\bullet1}$\\
2&1.5&1.2&0.8& $T_{\bullet2}$\\
3&1.8&1.3&0.7& $T_{\bullet3}$\\
4&1.7&1.3&0.8& $T_{\bullet4}$\\
5&1.6&0.7&0.1& $T_{\bullet5}$\\
6&1.5&1.2&0.7& $T_{\bullet6}$\\
7&1.7&1.1&0.4& $T_{\bullet7}$\\
8&2.0&1.3&0.6& $T_{\bullet8}$\\\cline{1-4}
 & $T_{1\bullet}$ & $T_{2\bullet}$ & $T_{3\bullet}$ & $T_{\bullet\bullet}$
\end{tabular}
\end{center}
\end{frame}


\begin{frame}[fragile]
\frametitle{Exemple 1}
Emmagatzemam les dades en una matriu {\tt kilocal}:
\begin{verbatim}
> kilocal = matrix(c(1.4,1.1,0.7,1.5,1.2,0.8,
 1.8,1.3,0.7,1.7,1.3,0.8,1.6,0.7,0.1,1.5,1.2,
 0.7,1.7,1.1,0.4,2.0,1.3,0.6),8,3,byrow=T)
> kilocal
[,1] [,2] [,3]
[1,] 1.4 1.1 0.7
[2,] 1.5 1.2 0.8
[3,] 1.8 1.3 0.7
[4,] 1.7 1.3 0.8
[5,] 1.6 0.7 0.1
[6,] 1.5 1.2 0.7
[7,] 1.7 1.1 0.4
[8,] 2.0 1.3 0.6
\end{verbatim}
\end{frame}


\begin{frame}[fragile]
\frametitle{Exemple 1}
Calculem estadístics
\begin{verbatim}
> sum.tracts=colSums(kilocal)
> sum.tracts
[1] 13.2  9.2  4.8
> mean.tracts=colMeans(kilocal)
> mean.tracts
[1] 1.65 1.15 0.60
> sum.blocs=rowSums(kilocal)
> sum.blocs
[1] 3.2 3.5 3.8 3.8 2.4 3.4 3.2 3.9
> mean.blocs=rowMeans(kilocal)
> mean.blocs
[1] 1.066667 1.166667 1.266667 1.266667 
[5] 0.800000 1.133333 1.066667 1.300000
\end{verbatim}
\end{frame}


\begin{frame}[fragile]
\frametitle{Exemple 1}
Calculem estadístics
\begin{verbatim}
> sum.tot=sum(kilocal)
> sum.tot
[1] 27.2
> mean.tot=mean(kilocal)
> mean.tot
[1] 1.133333
> sum.sq.tot=sum(kilocal^2)
> sum.sq.tot
[1] 36.18
\end{verbatim}
\end{frame}

\begin{frame}
\frametitle{Exemple 1}
\begin{itemize}
\item Sumes totals i mitjanes mostrals dels tractaments:
\begin{center}
\begin{tabular}{c|c|c}
${T_{1\bullet}}$ & ${T_{2\bullet}}$ & ${T_{3\bullet}}$  \\
\hline
13.2  & 9.2  & 4.8
\end{tabular}
\qquad 
\begin{tabular}{c|c|c}
${\overline{X}_{1\bullet}}$ & ${\overline{X}_{2\bullet}}$ & ${\overline{X}_{3\bullet}}$  \\
\hline
1.65 & 1.15 & 0.6
\end{tabular}
\end{center}
\smallskip

\item Sumes totals dels blocs:
\begin{center}
\begin{tabular}{c|c|c|c|c|c|c|c}
${T_{\bullet1}}$ & ${T_{\bullet2}}$ & ${T_{\bullet3}}$ & ${T_{\bullet4}}$ & ${T_{\bullet5}}$ & ${T_{\bullet6}}$ & ${T_{\bullet7}}$ & ${T_{\bullet8}}$  \\
\hline
3.2 & 3.5 &  3.8 &  3.8 &  2.4 &  3.4 &  3.2 &  3.9
\end{tabular}
\end{center}
\smallskip


\item Mitjanes mostrals  dels blocs:
\vspace*{-2ex}

\begin{center}
\hspace*{-5ex}\begin{tabular}{c|c|c|c|c|c|c|c}
${\overline{X}_{\bullet1}}$ & ${\overline{X}_{\bullet2}}$ & ${\overline{X}_{\bullet3}}$ & ${\overline{X}_{\bullet4}}$ & ${\overline{X}_{\bullet5}}$ & ${\overline{X}_{\bullet6}}$ & ${\overline{X}_{\bullet7}}$ & ${\overline{X}_{\bullet8}}$  \\
\hline
1.067  &  1.167  &  1.267  &  1.267 & 0.8 & 1.133 & 1.067 & 1.3
\end{tabular}
\end{center}
\smallskip


\item Suma total: $T_{\bullet\bullet}=27.2$
\smallskip



\item Mitjana mostral: $\overline{X}_{\bullet\bullet}=1.133$
\smallskip



\item Suma de quadrats: $T^{(2)}_{\bullet\bullet}=36.18$

\end{itemize}
\end{frame}

%\begin{frame}
%\frametitle{Exemple 2}
%Les dades:
%
%\begin{center}
%\begin{tabular}{cccc}
%\hline
%&\multicolumn{3}{c}{Tractament}\\\hline
%Bloc &1 (placebo) &2 (analgèsic A)&3 (analgèsic B)\\\hline
% 1 & 35 & 20 & 22\\
%2 & 40 & 35 & 42\\
%3 & 60 & 50 & 30\\
% 4 & 50 & 40 & 35\\
%5 & 50 & 30 & 22 \\\hline
%\end{tabular}
%\end{center}
%\end{frame}
%
%
%\begin{frame}
%\frametitle{Exemple 2}
%\begin{itemize}
%\item Sumes totals i mitjanes mostrals dels tractaments:
%\begin{center}
%\begin{tabular}{c|c|c}
%${T_{1\bullet}}$ & ${T_{2\bullet}}$ & ${T_{3\bullet}}$  \\
%\hline
% \quad  &  \quad  &  \quad
%\end{tabular}
%\qquad 
%\begin{tabular}{c|c|c}
%${\overline{X}_{1\bullet}}$ & ${\overline{X}_{2\bullet}}$ & ${\overline{X}_{3\bullet}}$  \\
%\hline
% \quad &  \quad &  \quad
%\end{tabular}
%\end{center}
%\smallskip
%
%\item Sumes totals dels blocs:
%\begin{center}
%\begin{tabular}{c|c|c|c|c}
%${T_{\bullet1}}$ & ${T_{\bullet2}}$ & ${T_{\bullet3}}$ & ${T_{\bullet4}}$ & ${T_{\bullet5}}$ \\
%\hline
% \qquad\qquad &  \qquad\qquad &   \qquad\qquad &   \qquad\qquad &   \qquad\qquad
%\end{tabular}
%\end{center}
%\smallskip
%
%
%\item Mitjanes mostrals  dels blocs:
%\begin{center}
%\begin{tabular}{c|c|c|c|c}
%${\overline{X}_{\bullet1}}$ & ${\overline{X}_{\bullet2}}$ & ${\overline{X}_{\bullet3}}$ & ${\overline{X}_{\bullet4}}$ & ${\overline{X}_{\bullet5}}$   \\
%\hline
% \qquad\qquad  &   \qquad\qquad  &   \qquad\qquad  &   \qquad\qquad &  \qquad\qquad
%\end{tabular}
%\end{center}
%\smallskip
%
%
%\item Suma total: $T_{\bullet\bullet}=$
%\smallskip
%
%
%
%\item Mitjana mostral: $\overline{X}_{\bullet\bullet}=$
%\smallskip
%
%
%
%\item Suma de quadrats: $T^{(2)}_{\bullet\bullet}=$
%
%\end{itemize}
%
%\end{frame}
%
%\begin{frame}
%\frametitle{Exemple 2}
%\begin{itemize}
%\item Sumes totals i mitjanes mostrals dels tractaments:
%\begin{center}
%\begin{tabular}{c|c|c}
%${T_{1\bullet}}$ & ${T_{2\bullet}}$ & ${T_{3\bullet}}$  \\
%\hline
%235 & 175 & 151
%\end{tabular}
%\qquad 
%\begin{tabular}{c|c|c}
%${\overline{X}_{1\bullet}}$ & ${\overline{X}_{2\bullet}}$ & ${\overline{X}_{3\bullet}}$  \\
%\hline
%47 & 35 & 30.2
%\end{tabular}
%\end{center}
%\smallskip
%
%\item Sumes totals dels blocs:
%\begin{center}
%\begin{tabular}{c|c|c|c|c}
%${T_{\bullet1}}$ & ${T_{\bullet2}}$ & ${T_{\bullet3}}$ & ${T_{\bullet4}}$ & ${T_{\bullet5}}$ \\
%\hline
%77 &117 &140 & 125 & 102
%\end{tabular}
%\end{center}
%\smallskip
%
%
%\item Mitjanes mostrals  dels blocs:
%\begin{center}
%\begin{tabular}{c|c|c|c|c}
%${\overline{X}_{\bullet1}}$ & ${\overline{X}_{\bullet2}}$ & ${\overline{X}_{\bullet3}}$ & ${\overline{X}_{\bullet4}}$ & ${\overline{X}_{\bullet5}}$   \\
%\hline
%25.667 & 39 & 46.667 & 41.667 & 34 
%\end{tabular}
%\end{center}
%\smallskip
%
%
%\item Suma total: $T_{\bullet\bullet}=561$
%\smallskip
%
%
%
%\item Mitjana mostral: $\overline{X}_{\bullet\bullet}=37.4$
%\smallskip
%
%
%
%\item Suma de quadrats: $T^{(2)}_{\bullet\bullet}=22907$
%
%\end{itemize}
%
%\end{frame}
%



%\subsection{Suma de quadrats}
\begin{frame}
\frametitle{Identitat de la suma de quadrats}
\begin{teorema}
{\small 
$$
\begin{array}{rl}
\sum\limits_{i=1}^k\sum\limits_{j=1}^b (X_{ij}- \overline{X}_{\bullet\bullet})^2
 & = b\sum\limits_{i=1}^k
(\overline{X}_{i\bullet}-\overline{X}_{\bullet\bullet})^2 \\ & \quad +
k\sum\limits_{j=1}^b (\overline{X}_{\bullet
j}-\overline{X}_{\bullet\bullet})^2 \\ &\quad + \sum\limits_{i=1}^k\sum\limits_{j=1}^b
(X_{ij} - \overline{X}_{i\bullet}- \overline{X}_{\bullet j}+\overline{X}_{\bullet\bullet})^2
\end{array}
$$}
\end{teorema}
\end{frame}



\begin{frame}
\frametitle{Identitat de la suma de quadrats}

\begin{itemize}
\item $\red{SS_{Total}} = \sum\limits_{i=1}^k\sum\limits_{j=1}^b (X_{ij}-
\overline{X}_{\bullet\bullet})^2$, variabilitat total  
\medskip

\item $\red{SS_{Tr}}=b\sum\limits_{i=1}^k
(\overline{X}_{i\bullet}-\overline{X}_{\bullet\bullet})^2$, variabilitat 
deguda als tractaments
\medskip

\item $\red{SS_{Blocs}}=k\sum\limits_{j=1}^b (\overline{X}_{\bullet
j}-\overline{X}_{\bullet\bullet})^2$, variabilitat  deguda als blocs
\medskip

\item $\red{SS_E}= \sum\limits_{i=1}^k\sum\limits_{j=1}^b (X_{ij} - \overline{X}_{i\bullet}-
\overline{X}_{\bullet j}+\overline{X}_{\bullet\bullet})^2$, variabilitat  
deguda a factors aleatoris
\end{itemize}
\end{frame}

\begin{frame}
\frametitle{Identitat de la suma de quadrats}
\begin{teorema}
$$
SS_{Total}=SS_{Tr}+SS_{Blocs}+SS_E
$$
\end{teorema}


\begin{itemize}
\item $\red{SS_{Total}} = \sum\limits_{i=1}^k\sum\limits_{j=1}^b (X_{ij}-
\overline{X}_{\bullet\bullet})^2$
\medskip

\item $\red{SS_{Tr}}=b \sum\limits_{i=1}^k
(X_{i\bullet}-\overline{X}_{\bullet\bullet})^2$\medskip

\item $\red{SS_{Blocs}}=k \sum\limits_{j=1}^b (X_{\bullet
j}-\overline{X}_{\bullet\bullet})^2 $
\medskip

\item $\red{SS_E}= \sum\limits_{i=1}^k\sum\limits_{j=1}^b (X_{ij} - X_{i\bullet}-
X_{\bullet j}+\overline{X}_{\bullet\bullet})^2$
\end{itemize}
\end{frame}


\begin{frame}
\frametitle{Identitat de la suma de quadrats}

Voldrem calcular $SS_{Tr}$, $SS_{Blocs}$, $SS_E$:
\medskip

\begin{itemize}
\item $\displaystyle SS_{Total}=T_{\bullet\bullet}^{(2)}-\dfrac{T_{\bullet\bullet}^2}{k\cdot b}$
\medskip

\item $\displaystyle SS_{Tr}= \sum\limits_{i=1}^k\dfrac{T_{i\bullet}^2}{b}-\dfrac{T_{\bullet\bullet}^2}{k\cdot b}$
\medskip


\item $\displaystyle SS_{Blocs}= \sum\limits_{j=1}^b \dfrac{T_{\bullet
j}^2}{k}-\dfrac{T_{\bullet\bullet}^2}{k\cdot b}$
\medskip

\item $SS_E=SS_{Total}-SS_{Tr}-SS_{Blocs}$
\end{itemize}

\end{frame}

\begin{frame}
\frametitle{Exemple 1}
\vspace*{-2ex}


\begin{center}
$k=3$, $b=8$
\bigskip

\begin{tabular}{c|c|c}
${T_{1\bullet}}$ & ${T_{2\bullet}}$ & ${T_{3\bullet}}$  \\
\hline
13.2  & 9.2  & 4.8
\end{tabular}
\bigskip

\begin{tabular}{c|c|c|c|c|c|c|c}
${T_{\bullet1}}$ & ${T_{\bullet2}}$ & ${T_{\bullet3}}$ & ${T_{\bullet4}}$ & ${T_{\bullet5}}$ & ${T_{\bullet6}}$ & ${T_{\bullet7}}$ & ${T_{\bullet8}}$  \\
\hline
3.2 & 3.5 &  3.8 &  3.8 &  2.4 &  3.4 &  3.2 &  3.9
\end{tabular}
\bigskip


$T_{\bullet\bullet}=27.2$,\quad  $T^{(2)}_{\bullet\bullet}=36.18$

\end{center}

$$
\begin{array}{ll}
SS_{Total}=\qquad\qquad & SS_{Tr}=\qquad\qquad {}\\[3ex]

SS_{Blocs}= & SS_E=
\end{array}
$$

\end{frame}



\begin{frame}[fragile]
\frametitle{Exemple 1}

\begin{itemize}
\item Variabilitat total:
\begin{verbatim}
> SS.Tot=sum((kilocal-mean.tot)^2)
> SS.Tot
[1] 5.353333
> sum.sq.tot-sum.tot^2/24
[1] 5.353333
\end{verbatim}
\medskip

\item Variabilitat deguda als tractaments:

\begin{verbatim}
> SS.Tr=8*sum((mean.tracts-mean.tot)^2)
> SS.Tr
[1] 4.413333
> sum(sum.tracts^2)/8-sum.tot^2/24
[1] 4.413333
\end{verbatim}
\end{itemize}

\end{frame}


\begin{frame}[fragile]
\frametitle{Exemple 1}

\begin{itemize}

\item Variabilitat deguda als blocs:
\begin{verbatim}
> SS.Bl=3*sum((mean.blocs-mean.tot)^2)
> SS.Bl
[1] 0.5533333
> sum(sum.blocs^2)/3-sum.tot^2/24
[1] 0.5533333
\end{verbatim}
\medskip

\item Variabilitat   deguda a factors aleatoris
\begin{verbatim}
> SSE=SS.Tot-SS.Tr-SS.Bl
> SSE
[1] 0.3866667
\end{verbatim}
\end{itemize}

\end{frame}

%
%\begin{frame}
%\frametitle{Exemple 2}
%\begin{center}
%$k=3$, $b=5$
%\bigskip
%
%\begin{tabular}{c|c|c}
%${T_{1\bullet}}$ & ${T_{2\bullet}}$ & ${T_{3\bullet}}$  \\
%\hline
%235 & 175 & 151
%\end{tabular}
%\qquad 
%\begin{tabular}{c|c|c|c|c}
%${T_{\bullet1}}$ & ${T_{\bullet2}}$ & ${T_{\bullet3}}$ & ${T_{\bullet4}}$ & ${T_{\bullet5}}$ \\
%\hline
%77 &117 &140 & 125 & 102
%\end{tabular}
%\medskip
%
%$T_{\bullet\bullet}=561$,  $T^{(2)}_{\bullet\bullet}=22907$
%\end{center}
%\vspace*{1cm}
%
%$$
%\begin{array}{ll}
%SS_{Total}=\qquad\qquad & SS_{Tr}=\qquad\qquad {}\\[3ex]
%
%SS_{Blocs}= & SS_E=
%\end{array}
%$$
%
%\end{frame}
%
%
%
%\begin{frame}
%\frametitle{Exemple 2}
%\begin{center}
%$k=3$, $b=5$
%\bigskip
%
%\begin{tabular}{c|c|c}
%${T_{1\bullet}}$ & ${T_{2\bullet}}$ & ${T_{3\bullet}}$  \\
%\hline
%235 & 175 & 151
%\end{tabular}
%\qquad 
%\begin{tabular}{c|c|c|c|c}
%${T_{\bullet1}}$ & ${T_{\bullet2}}$ & ${T_{\bullet3}}$ & ${T_{\bullet4}}$ & ${T_{\bullet5}}$ \\
%\hline
%77 &117 &140 & 125 & 102
%\end{tabular}
%\medskip
%
%$T_{\bullet\bullet}=561$,  $T^{(2)}_{\bullet\bullet}=22907$
%\end{center}
%\vspace*{1cm}
%
%$$
%\begin{array}{ll}
%SS_{Total}=1925.6\qquad & SS_{Tr}=748.8\qquad {}\\[3ex]
%
%SS_{Blocs}=767.6 & SS_E=409.2
%\end{array}
%$$
%
%\end{frame}
%

\begin{frame}
\frametitle{Contrast}

Per realitzar  el contrast 
$$
\left.
\begin{array}{l}
H_0 :  \mu_{1\bullet}=\cdots =\mu_{k\bullet} \\
H_1 : \exists i,j=1,\ldots ,k\mid  \mu_{i\bullet}
\neq \mu_{j\bullet}
\end{array}
\right\}
$$
emprarem els  estadístics següents:
\begin{itemize}
\item Quadrat mitjà dels tractaments: $\red{MS_{Tr}}=\dfrac{SS_{Tr}}{k-1}$
\medskip

\item Quadrat mitjà de l'error: $\red{MS_E} = \dfrac{SS_E}{(b-1) (k-1)}$
\medskip

\item (A més) Quadrat mitjà dels blocs:
$\red{MS_{Blocs}}=\dfrac{SS_{Blocs}}{b-1}$
\end{itemize}
\end{frame}



\begin{frame}
\frametitle{Contrast}
Es té que
$$
\begin{array}{l}
E(MS_{Tr})=\sigma^2 + \dfrac{b}{k-1}\sum\limits_{i=1}^k (\mu_{i\bullet}-\mu)^2 \\
E(MS_E)=\sigma^2
\end{array}
$$

Si $H_0:\mu_{1\bullet}=\cdots =\mu_{k\bullet}=\mu$ és certa,
$$
\sum\limits_{i=1}^k (\mu_{i\bullet}-\mu)^2 = 0,
$$
i si $H_0$ no és certa, aquesta quantitat seria $>0$
\end{frame}


\begin{frame}
\frametitle{Estadístics del contrast}

Considerarem com a \emph{estadístic de contrast} el quocient 
$$
\red{F=\frac{MS_{Tr}}{MS_E}}
$$
Si $H_0$ és certa:
\medskip

\begin{itemize}
\item la seva distribució és $F_{k-1,(k-1)(b-1)}$ (F de Fisher 
amb $k-1$ i $(k-1)(b-1)$ graus de llibertat)
\medskip

\item el seu valor serà proper a $1$
\end{itemize}
\medskip

i rebutjarem la hipòtesi nu\l.la si $F$ és prou més gran que 1



\end{frame}


\begin{frame}
\frametitle{Contrast}
\begin{itemize}
\item Calculam  el quocient 
$$
F=\frac{MS_{Tr}}{MS_E}
$$

\item Calculam el p-valor
$$
P(F_{k-1,(k-1)(b-1)}\geq F)
$$

\item Si el p-valor és més petit que el nivell de significació $\alpha$  rebutjam $H_0$ i concloem que no totes les mitjanes són iguals. En cas contrari, acceptam $H_0$.
\end{itemize}
\end{frame}





\begin{frame}
\frametitle{Exemple 1}
\vspace*{-3ex}

\begin{center}
\begin{tabular}{cccccc}
k & b & $SS_{Total}$ & $SS_{Tr}$ & $SS_{Blocs}$ & $SS_E$\\ \hline
3 & 8 & 5.3533  & 4.4133 & 0.5533  & 0.3867
\end{tabular}
\end{center}

\begin{itemize}
\item $MS_{Tr}=\dfrac{SS_{Tr}}{k-1}=2.2066$
\medskip

\item $MS_{Blocs}=\dfrac{SS_{Blocs}}{b-1}=0.079$
\medskip

\item $MS_E = \dfrac{SS_E}{(b-1) (k-1)}=0.0276$
\medskip

\item $F=\dfrac{MS_{Tr}}{MS_E}=79.949$
\medskip

\item p-valor: $P(F_{2,14}\geq 79.949)=2\cdot 10^{-8}$
\medskip

\item Conclusió: Rebutjam la hipòtesi nu\l.la i concloem que les tres mitjanes no són iguals

\end{itemize}
\end{frame}

%\begin{frame}
%\frametitle{Exemple 2}
%\vspace*{-2ex}
%
%\begin{center}
%\begin{tabular}{cccccc}
%k & b & $SS_{Total}$ & $SS_{Tr}$ & $SS_{Blocs}$ & $SS_E$\\ \hline
%3 & 5 & 1925.6  & 748.8 & 767.6 & 409.2
%\end{tabular}
%\end{center}
%
%\begin{itemize}
%\item $MS_{Tr}=$
%\medskip
%
%\item $MS_{Blocs}=$
%\medskip
%
%\item $MS_E = $
%\medskip
%
%\item $F=$
%\medskip
%
%\item p-valor: 
%\medskip
%
%\item Conclusió: 
%
%\end{itemize}
%\end{frame}
%
%
%\begin{frame}
%\frametitle{Exemple 2}
%\vspace*{-2ex}
%
%\begin{center}
%\begin{tabular}{cccccc}
%k & b & $SS_{Total}$ & $SS_{Tr}$ & $SS_{Blocs}$ & $SS_E$\\ \hline
%3 & 5 & 1925.6  & 748.8 & 767.6 & 409.2
%\end{tabular}
%\end{center}
%
%\begin{itemize}
%\item $MS_{Tr}=374.4$
%\medskip
%
%\item $MS_{Blocs}=191.9$
%\medskip
%
%\item $MS_E = 51.15$
%\medskip
%
%\item $F=7.3196$
%\medskip
%
%\item p-valor: $P(F_{2,8}\geq 7.3196)=0.0156$
%
%\medskip
%
%\item Conclusió: Rebutjam la hipòtesi nu\l.la i concloem que les tres mitjanes no són iguals
%\end{itemize}
%\end{frame}


\begin{frame}
\frametitle{Taula del contrast}

Per realitzar el contrast es construeix la taula següent:
{\small \begin{center}
\hspace*{-2ex}\begin{tabular}{|@{}l@{}|l@{}|l@{}|l@{}|l|l|}
\hline
Font de&Graus de&Suma de&Quadrats&Estadístic & p-valor\\
variació&llibertat&quadrats&mitjans& & \\\hline
Tract.&$k-1$ & $SS_{Tr}$&$MS_{Tr}$&$F=\dfrac{MS_{Tr}^{\ }}{MS_E}$ & p-valor\\[2ex]
Bloc&$b-1$&$SS_{Blocs}$&$MS_{Blocs}$& &\\[2ex]
Error&$(k-1)$&$SS_{E}$&$MS_E$& &\\
 & \ $\cdot (b-1)$ & & & & \\
\hline
\end{tabular}
\end{center}
}
\end{frame}



\begin{frame}
\frametitle{Exemple 1}

{\small \begin{center}
\hspace*{-2ex}\begin{tabular}{|@{}l@{}|l@{}|l@{}|l@{}|l|l|}
\hline
Font de&Graus de&Suma de&Quadrats&Estadístic & p-valor\\
variació&llibertat&quadrats&mitjans& & \\\hline
Tract.& 2 & $4.4133$&$2.2067$&$79.949$ & $2\cdot 10^{-8}$\\[2ex]
Bloc& 7 &$0.5533$&$0.079$& &\\[2ex]
Error& 14 &$0.3867$&$0.0276$& &\\
\hline
\end{tabular}
\end{center}
}
\end{frame}

%
%\begin{frame}
%\frametitle{Exemple 2}
%
%{\small \begin{center}
%\hspace*{-2ex}\begin{tabular}{|@{}l@{}|l@{}|l@{}|l@{}|l|l|}
%\hline
%Font de&Graus de&Suma de&Quadrats&Estadístic & p-valor\\
%variació&llibertat&quadrats&mitjans& & \\\hline
%Tract.& 2 & $748.8$&$374.4$&$7.3196$ & $0.0156$\\[2ex]
%Bloc& 4 &$767.6$&$191.9$& &\\[2ex]
%Error& 8 &$409.2$&$51.15$& &\\
%\hline
%Total&$14$&$1925.6$&& &\\\hline
%\end{tabular}
%\end{center}
%}
%\end{frame}



\begin{frame}[fragile]
\frametitle{Exemple 1 amb R}

Primer hem de construir un dataframe
\begin{verbatim}
> kilocal2=c(1.4,1.1,0.7,1.5,1.2,0.8,
  1.8,1.3,0.7,1.7,1.3,0.8,1.6,0.7,0.1,1.5,1.2,
  0.7,1.7,1.1,0.4,2.0,1.3,0.6)
> blocs=as.factor(rep(1:8,each=3))
> blocs
 [1] 1 1 1 2 2 2 3 3 3 4 4 4 5 5 5 6 6 6 7 7 7
[22] 8 8 8
Levels: 1 2 3 4 5 6 7 8
> tracts=as.factor(rep(1:3,times=8))
> tracts
 [1] 1 2 3 1 2 3 1 2 3 1 2 3 1 2 3 1 2 3 1 2 3 
[22] 1 2 3
Levels: 1 2 3
\end{verbatim}
\end{frame}



\begin{frame}[fragile]
\frametitle{Exemple 1 amb R}
\vspace*{-2ex}

\begin{verbatim}
> kilocal.df=data.frame(kilocal2,tracts,blocs)
> str(kilocal.df)
'data.frame':	24 obs. of  3 variables:
 $ kilocal2: num  1.4 1.1 0.7 1.5 1.2 0.8 1.8 
      1.3 0.7 1.7 ...
 $ tracts  : Factor w/ 3 levels "1","2","3": 1 
      2 3 1 2 3 1 2 3 1 ...
 $ blocs   : Factor w/ 8 levels "1","2","3",
     "4",..: 1 1 1 2 2 2 3 3 3 4 ...
> head(kilocal.df,4)
  kilocal2 tracts blocs
1      1.4      1     1
2      1.1      2     1
3      0.7      3     1
4      1.5      1     2
\end{verbatim}
\end{frame}


\begin{frame}[fragile]
\frametitle{Exemple 1 amb R}
Aplicam \texttt{aov} ``sumant els dos factors''
{\small 
\begin{verbatim}
> summary(aov(kilocal2~tracts+blocs,
    data=kilocal.df))
            Df Sum Sq Mean Sq F value  Pr(>F)    
tracts       2  4.413  2.2067  79.897 2.2e-08 ***
blocs        7  0.553  0.0790   2.862  0.0446 *  
Residuals   14  0.387  0.0276                    
\end{verbatim}
}
\end{frame}

%
%
%\begin{frame}[fragile]
%\frametitle{Exemple 2 amb R}
%
%Primer hem de construir un dataframe
%\begin{verbatim}
%> cefal=c(35,20,22,40,35,42,60,50,30, 50,
% 40,35,50,30,22)
%> blocs=as.factor(rep(1:5,each=3))
%> tracts=as.factor(rep(1:3,times=5))
%> cefal.df=data.frame(cefal,tracts,blocs)
%> str(cefal.df)
%'data.frame':	15 obs. of  3 variables:
% $ cefal : num  35 20 22 40 35 42 60 50 30 
%    50 ...
% $ tracts: Factor w/ 3 levels "1","2","3": 1 2 
%    3 1 2 3 1 2 3 1 ...
% $ blocs : Factor w/ 5 levels "1","2","3","4",
%    ..: 1 1 1 2 2 2 3 3 3 4 ...
%\end{verbatim}
%
%\end{frame}
%
%
%
%\begin{frame}[fragile]
%\frametitle{Exemple 2 amb R}
%
%Aplicam \texttt{aov}  
%\begin{verbatim}
%> summary(aov(cefal.df$cefal~
%   cefal.df$tracts+cefal.df$blocs))
%                Df Sum Sq Mean Sq
%cefal.df$tracts  2  748.8   374.4
%cefal.df$blocs   4  767.6   191.9
%Residuals        8  409.2    51.1
%                F value Pr(>F)  
%cefal.df$tracts   7.320 0.0156 *
%cefal.df$blocs    3.752 0.0528 .
%Residuals                       
%\end{verbatim}
%\end{frame}

\subsection{Efectivitat dels blocs}
\begin{frame}
\frametitle{Efectivitat en la construcció dels blocs}

Ha tingut èxit la construcció de blocs en el nostre experiment?
\medskip

En cas afirmatiu,   $SS_{Blocs}$ explicaria una part important de $SS_{Total}$ i faria reduir $SS_E$.
Això augmentaria el valor de $F$ i faria més difícil acceptar $H_0$, la qual cosa milloraria la potència del contrast.
\end{frame}

\begin{frame}
\frametitle{Efectivitat en la construcció dels blocs}
\begin{itemize}
\item En un ANOVA completament aleatori (CA) amb $k$ tractaments, els graus de llibertat de l'estadístic $F$ són $k-1$ i \red{$N-k$} ($N$ nombre total d'observacions)
\medskip

\item En un ANOVA de blocs complet  aleatori (BCA) amb $k$ tractaments i $b$ blocs, els graus de llibertat de l'estadístic $F$ són $k-1$ i $(k-1)(b-1)=$ $\red{N-k-(b-1)}$ (on $N=b\cdot k$)
\end{itemize}

Quan el $2^{\mathrm{on}}$ grau de llibertat de $F$ decreix, el 0.95-quantil de $F$ augmenta (mirau la taula), és més difícil rebutjar $H_0$, i la potència del contrast disminueix.
\medskip

%Per tant, per defecte, fer amb les mateixes dades el contrast CA, augmenta la potència que si el fem  BCA.

\end{frame}


\begin{frame}
\frametitle{Efectivitat en la construcció dels blocs}

L'efectivitat en la construcció dels blocs  s'estima amb l'\emph{eficiència relativa}, \red{$RE$}. S'interpreta com 
la relació entre el nombre d'observacions d'un experiment completament aleatori (CA) i el nombre d'observacions 
d'un experiment de blocs complet  aleatori (BCA) necessària per obtenir tests equivalents.
\medskip

\begin{itemize}
\item $RE=3$ significa que el disseny CA
requereix tres vegades tantes observacions com el disseny de BCA. En aquest
cas, ha valgut la pena l'ús de blocs. 

\item $RE=0.5$ significa que amb un disseny CA hagués bastat la meitat d'observacions que  al disseny 
BCA. En aquest
cas, no era aconsellable l'ús de blocs. 
\end{itemize}

\end{frame}

\begin{frame}
\frametitle{Efectivitat en la construcció dels blocs}

$RE$ s'estima amb
$$
\widehat{RE}=c+(1-c)\frac{MS_{Blocs}}{MS_E},
$$
on $c=\dfrac{b(k-1)}{(bk-1)}$
\bigskip

Per conveni, si  $\widehat{RE}>1.25$, s'entén que la
construcció dels blocs ha estat profitosa
\end{frame}

\begin{frame}[fragile]
\frametitle{Exemple}

A l'Exemple 1
$$
c=\frac{8\cdot 2}{24-1}=\frac{16}{23}
$$
i
$$
\widehat{RE}=\frac{16}{23}+\Big(1-\frac{16}{23}\Big)\cdot \frac{0.079}{0.0276}=1.5668
$$
%\medskip
%
%\only<2>{A l'exemple 2
%$$
%c=\hphantom{\frac{10}{14}}\vphantom{\frac{10}{14}}
%$$
%i
%$$
%\widehat{RE}=\hphantom{\frac{10}{14}+\Big(1-\frac{10}{14}\Big)\cdot \frac{191.9}{51.15}=1.7862}\vphantom{\frac{10}{14}+\Big(1-\frac{10}{14}\Big)\cdot \frac{191.9}{51.15}=1.7862}
%$$
%}
%
%\only<3>{A l'exemple 2
%$$
%c=\frac{10}{14}
%$$
%i
%$$
%\widehat{RE}=\frac{10}{14}+\Big(1-\frac{10}{14}\Big)\cdot \frac{191.9}{51.15}=1.7862
%$$
%Als dos casos, dóna $>1.25$: l'ús de blocs ha estat profitós}
\end{frame}

%
%\subsection{Comparacions}
%\begin{frame}
%\frametitle{Comparacions per parelles i múltiples}
%
%Com en el cas d'un factor i efectes fixats, si rebutjam la hipòtesi nul·la, podem demanar-nos quins són els nivells diferents
%\medskip
%
%Podem emprar el T-test, fent cada comparació \emph{per a mostres aparellades} i emprant una correcció del p-valor (Bonferroni, Holm)
%\end{frame}
%
%
%\begin{frame}[fragile]
%\frametitle{Exemple 1 amb R}
%
%A l'Exemple 1, hem conclòs que no totes les mitjanes són iguals.
%Emprem un contrast de Bonferroni per mirar quines són diferents.
%
%{\small \begin{verbatim}
%> pairwise.t.test(kilocal.df$kilocal2,
%  kilocal.df$tracts,paired=TRUE,
%  p.adjust.method="bonferroni")
%     Pairwise comparisons using paired t tests 
%data:  kilocal.df$kilocal2 and kilocal.df$tracts 
%  1       2      
%2 0.00108 -      
%3 0.00011 1.1e-05
%P value adjustment method: bonferroni 
%\end{verbatim}
%}
%Els p-valors són molt petits, per tant les mitjanes són dues a dues diferents
%\end{frame}
%
%
%%%%%%%%%%%%%%%%%%%%%%%%%%%%%%%%%%%%%%%%%%%%%%%


\section{ANOVA de dues vies}
\subsection{Introducció}
\begin{frame}
\frametitle{ANOVA de dues vies}

Les comparacions de mitjanes les podem portar a terme classificant la població mitjançant més d'un factor: s'en diu un \emph{experiment factorial}
\medskip

Aquí considerarem només el  cas més senzill: \emph{ANOVA de dues vies, disseny
completament aleatori amb efectes fixats}:
\begin{itemize}
\item Emprarem dos factors (\emph{dues vies})
\medskip

\item Emprarem tots els nivells de cada factor (\emph{efectes fixats})
\medskip

\item Prendrem mostres aleatòries independents \blue{de la mateixa mida} de cada combinació de nivells dels dos factors (\emph{completament aleatori })
\end{itemize}
\end{frame}



\begin{frame}
\frametitle{Format de les dades}

Tenim en compte dos factors, $A$ i $B$.  El factor $A$ té \red{$a$} nivells i el factor $B$, \red{$b$}
nivells.
\medskip

Fem \red{$n$} observacions  per a cada combinació de
tractaments. El nombre total d'observacions serà $n\cdot
a\cdot b$.
\medskip

La variable aleatòria \red{$X_{ijk}$}, $i=1,\ldots,a$, $j=1,\ldots,b$, $k=1,\ldots,n$,
ens dóna la resposta de la $k$-èsima unitat
experimental al nivell $i$-èsim del factor $A$ i el nivell $j$-èsim del
factor $B$

\end{frame}
\begin{frame}
\frametitle{Format de les dades}
Donarem les dades en una taula d'aquest estil:
{\small\begin{center}
\begin{tabular}{ccccc}
\hline
&\multicolumn{4}{c}{Factor $A$}\\\hline
Factor $B$&$1$&$2$&$\cdots$&$a$\\\hline
$1$&$X_{111}$&$X_{211}$&$\cdots$&$X_{a11}$\\
&$X_{112}$&$X_{212}$&$\cdots$&$X_{a12}$\\
&$\cdots$&$\cdots$&$\cdots$&$\cdots$\\
&$X_{11n}$&$X_{21n}$&$\cdots$&$X_{a1n}$\\\hline
$2$&$X_{121}$&$X_{221}$&$\cdots$&$X_{a21}$\\
&$X_{122}$&$X_{222}$&$\cdots$&$X_{a22}$\\
&$\cdots$&$\cdots$&$\cdots$&$\cdots$\\
&$X_{12n}$&$X_{22n}$&$\cdots$&$X_{a2n}$\\\hline
$\vdots$&$\vdots$&$\vdots$&$\vdots$&$\vdots$\\\hline
$b$&$X_{1b1}$&$X_{2b1}$&$\cdots$&$X_{ab1}$\\
&$X_{1b2}$&$X_{2b2}$&$\cdots$&$X_{ab2}$\\
&$\cdots$&$\cdots$&$\cdots$&$\cdots$\\
&$X_{1bn}$&$X_{2bn}$&$\cdots$&$X_{abn}$\\\hline
\end{tabular}
\end{center}}
\end{frame}


\begin{frame}
\frametitle{Exemple 2}

En un experiment per determinar l'efecte
de la llum i la temperatura sobre l'índex gonadosomàtic (GSI; una 
mesura de creixement de l'ovari) d'una espècie de peix, es van utilitzar dos fotoperíodes (14
hores de llum--10 hores de foscor, i 9 hores de llum--15 hores de
foscor) i dues  temperatures (16${}^\circ$ i 27 ${}^\circ$C)
\bigskip
 
L'experiment es va realitzar sobre $20$ femelles. Es van dividir
aleatòriament en $4$ subgrups de  $5$ exemplars cadascun. Cada grup va rebre una combinació diferent de llum i temperatura.
\bigskip

Als 3 mesos es mesuraren els GSI dels peixos

\end{frame}

\begin{frame}
\frametitle{Exemple 2}

Els resultats es mostren en
la taula següent:


\begin{center}
\begin{tabular}{ccc}
\hline
&\multicolumn{2}{c}{Factor $A$ (fotoperíode)}\\\hline
Factor $B$&$9$ hores&$14$ hores\\
(temperatura)&&\\\hline
$27^\circ$C&$0.90$&$0.83$\\
&$1.06$&$0.67$\\
&$0.98$&$0.57$\\
&$1.29$&$0.47$\\
&$1.12$&$0.66$\\\hline
$16^\circ$C&$1.30$&$1.01$\\
&$2.88$&$1.52$\\
&$2.42$&$1.02$\\
&$2.66$&$1.32$\\
&$2.94$&$1.63$\\\hline
\end{tabular}
\end{center}
\end{frame}

%
%\begin{frame}
%\frametitle{Exemple 2}
%Es vol provar l'eficàcia d'un somnífer estudiant possibles diferències degudes  al sexe dels subjectes
%\medskip
%
%Es trien a l'atzar dos grups de 5  homes insomnes i altres dos de 5 dones insomnes, i per a cada sexe a un grup se li subministra el somnífer i a l'altre un placebo. Es mesura a cada pacient  el temps, en minuts,  que triga a dormir-se.
%\end{frame}
%
%
%\begin{frame}
%\frametitle{Exemple 2}
%
%Els resultats es mostren en
%la taula següent:
%
%
%\begin{center}
%\begin{tabular}{ccc}
%\hline
%&\multicolumn{2}{c}{Factor $A$ (tractament)}\\\hline
%Factor $B$&Placebo &Somnífer\\
%(sexe)&&\\\hline
%Home &30&35\\
%&50& 32 \\
%&45 & 30 \\
%&47 & 25 \\
%& 38 & 30 \\\hline
%Dona & 50 & 42\\
%&35 & 30\\
%& 46 & 15 \\
%&25 & 18\\
%&32 & 23 \\\hline
%\end{tabular}
%\end{center}
%\end{frame}
%


\subsection{Model}


\begin{frame}
\frametitle{El model}

Les suposicions del model són:
\medskip

\begin{itemize}
\item Les observacions per a cada combinació de nivells constitueixen
mostres aleatòries simples independents, cadascuna de mida $n$, de $a\cdot b$
poblacions 
\medskip 

\item Cadascuna de les $a\cdot b$ poblacions és normal
\medskip 

\item Totes les $a\cdot b$ poblacions tenen la mateixa variància,
$\sigma^2$
\end{itemize}
\end{frame}

\begin{frame}
\frametitle{El model}
Els paràmetres que intervindran en el contrast són:
\medskip

\begin{itemize}
\item $\mu$: mitjana poblacional global
\medskip

\item $\mu_{i\bullet\bullet}$: mitjana poblacional del nivell $i$-èsim del factor $A$
\medskip

\item $\mu_{\bullet j\bullet}$: mitjana poblacional del nivell  $j$-èsim  del
factor $B$
\medskip

\item $\mu_{ij\bullet}$: mitjana poblacional de la combinació $(i,j)$ de nivells de $A$ i $B$
\end{itemize}

\end{frame}



\begin{frame}
\frametitle{El model}
\vspace*{-1ex}

\emph{Expressió matemàtica} del model a estudiar en aquest cas:
$$
\begin{array}{l}
X_{ijk} = \mu + \alpha_i + \beta_j + (\alpha\beta)_{ij}+E_{ijk},\\
\qquad\qquad i=1,\ldots,a,\ j=1,\ldots,b,\ k=1,\ldots,n
\end{array}
$$
on
\begin{itemize}
\item $\mu$: mitjana global
\medskip

\item $\alpha_i =\mu_{i\bullet\bullet}-\mu$: efecte per pertànyer al nivell  $i$-èsim del factor $A$
\medskip

\item $\beta_j =\mu_{\bullet j\bullet}-\mu$: efecte per pertànyer al nivell  $j$-èsim del factor $B$
\medskip

\item $(\alpha\beta)_{ij}=\mu_{ij\bullet}-\mu_{i\bullet\bullet}-\mu_{\bullet
j\bullet}+\mu$: efecte de la interacció entre el  nivell $i$-èsim del factor $A$ i
el  nivell  $j$-èsim del factor $B$
\medskip

\item $E_{ijk}=X_{ijk}-\mu_{ij\bullet}$: error residual o aleatori
\end{itemize}
\end{frame}

\begin{frame}
\frametitle{Sumes i mitjanes}
\vspace*{-2ex}

Emprarem les sumes i mitjanes següents:

\begin{itemize}
\item $\red{T_{ij\bullet}}=\sum\limits_{k=1}^n X_{ijk}$\qquad \qquad\qquad\hspace*{3.1ex} 
 $\red{\overline{X}_{ij\bullet}}=\dfrac{T_{ij\bullet}}{n}$
 \medskip

\item  $\red{T_{i\bullet\bullet}}=\sum\limits_{j=1}^b\sum\limits_{k=1}^n X_{ijk}=\sum\limits_{j=1}^b T_{ij\bullet}$\qquad    
$\red{\overline{X}_{i\bullet\bullet}}=\dfrac{T_{i\bullet\bullet}}{bn}$
\medskip

\item $\red{T_{\bullet j\bullet}}=\sum\limits_{i=1}^a\sum\limits_{k=1}^n X_{ijk}=\sum\limits_{i=1}^a T_{ij\bullet}$\qquad   
$\red{\overline{X}_{\bullet j\bullet}}=\dfrac{T_{\bullet j\bullet}}{an}$
\medskip

\item $\red{T_{\bullet\bullet\bullet}}=\sum\limits_{i=1}^{a}\sum\limits_{j=1}^b\sum\limits_{k=1}^n X_{ijk}=\sum\limits_{i=1}^{a}
T_{i\bullet\bullet}=\sum\limits_{j=1}^b T_{\bullet j\bullet}$\medskip

\item  $\red{\overline{X}_{\bullet\bullet\bullet}}=\dfrac{T_{\bullet\bullet\bullet}}{a b n}$\medskip


\item $\red{T^{(2)}_{\bullet\bullet\bullet}}=\sum\limits_{i=1}^{a}\sum\limits_{j=1}^b\sum\limits_{k=1}^n X_{ijk}^2$\medskip

\end{itemize}
\end{frame}

\begin{frame}
\frametitle{Exemple 2}
Les dades:

\begin{center}
\begin{tabular}{cccccc}
\cline{1-5}
&\multicolumn{4}{c}{Factor $A$}\\\cline{1-5}
Factor $B$&$9$ hores & &$14$ hores & & \\
\cline{1-5}
$27^\circ$C&$0.90$& & $0.83$& &\\ 
&$1.06$& & $0.67$& &\\
&$0.98$&\hspace*{-0.7cm} $T_{11\bullet}$ & $0.57$&\hspace*{-0.7cm} $T_{21\bullet}$ &$T_{\bullet1\bullet}$\\
&$1.29$& & $0.47$& &\\
&$1.12$& & $0.66$& &\\\cline{1-5}
$16^\circ$C&$1.30$& & $1.01$& &\\
&$2.88$& & $1.52$& &\\
&$2.42$&\hspace*{-0.7cm} $T_{12\bullet}$ & $1.02$&\hspace*{-0.7cm} $T_{22\bullet}$ &$T_{\bullet2\bullet}$\\
&$2.66$& & $1.32$& &\\
&$2.94$& & $1.63$& &\\\cline{1-5}
& $T_{1\bullet\bullet}$ & & $T_{2\bullet\bullet}$ & &
\end{tabular}
\end{center}
\end{frame}

\begin{frame}
\frametitle{Exemple 2}
Organitzarem les dades en un dataframe amb 3 variables:
\medskip

\begin{itemize}
\item {\tt GSI}, quantitativa, contendrà el $GSI$ de
cada peix
\medskip

\item {\tt llum}, factor,  contendrà el valor del nivell del  factor
$A$ per a cada peix
\medskip

\item {\tt temp}, factor,  contendrà el valor del nivell del  factor
$B$ per a cada peix
\end{itemize}
\end{frame}






\begin{frame}[fragile]
\frametitle{Exemple 2}

\begin{verbatim}
> GSI= c(0.90,0.83,1.06,0.67,0.98,0.57,
  1.29,0.47,1.12,0.66,1.30, 1.01,2.88,1.52,
  2.42,1.02,2.66,1.32,2.94,1.63)
> llum=as.factor(rep(c(9,14),10))
> llum
[1] 9 14 9 14 9 14 9 14 9 14 9 14 9 14 9
[16] 14 9 14 9 14
Levels: 9 14
> temp=as.factor(c(rep(27,10),rep(16,10)))
> temp
[1] 27 27 27 27 27 27 27 27 27 27 16 16 
[13] 16 16 16 16 16 16 16 16
Levels: 16 27
\end{verbatim}


\end{frame}



\begin{frame}[fragile]
\frametitle{Exemple 2}
\vspace*{-2ex}

\begin{verbatim}
> peixos=data.frame(GSI,llum,temp)
> str(peixos)
'data.frame':	20 obs. of  3 variables:
 $ GSI : num  0.9 0.83 1.06 0.67 0.98 0.57 
   1.29 0.47 1.12 0.66 ...
 $ llum: Factor w/ 2 levels "9","14": 1 2 
   1 2 1 2 1 2 1 2 ...
 $ temp: Factor w/ 2 levels "16","27": 2 2 
   2 2 2 2 2 2 2 2 ...
> head(peixos,5)
   GSI llum temp
1 0.90    9   27
2 0.83   14   27
3 1.06    9   27
4 0.67   14   27
5 0.98    9   27
\end{verbatim}
\end{frame}



\begin{frame}[fragile]
\frametitle{Exemple 2}
\vspace*{-2ex}

$T_{ij\bullet}$'s:
\vspace*{-1ex}

\begin{verbatim}
> n=5; a=2; b=2
> T.i.j.bullet=aggregate(GSI~llum+temp,
  data=peixos,sum)
> T.i.j.bullet
  llum temp   GSI
1    9   16 12.20
2   14   16  6.50
3    9   27  5.35
4   14   27  3.20
\end{verbatim}
\begin{center}
\begin{tabular}{c|cc}
$T_{ij\bullet}$ & 9 & 14\\ \hline
27 &  5.35 & 3.2\\
16 &12.20 & 6.5
\end{tabular}
\end{center}

\end{frame}

%
%\begin{frame}[fragile]
%\frametitle{Exemple 2}
%\vspace*{-2ex}
%
%$\overline{X}_{ij\bullet}$'s:
%\vspace*{-1ex}
%
%\begin{verbatim}
%> X.mitj.i.j.bullet=T.i.j.bullet
%> X.mitj.i.j.bullet$GSI=X.mitj.i.j.bullet$GSI/n
%> X.mitj.i.j.bullet
%  llum temp  GSI
%1    9   16 2.44
%2   14   16 1.30
%3    9   27 1.07
%4   14   27 0.64
%\end{verbatim}
%\begin{center}
%\begin{tabular}{c|cc}
%$\overline{X}_{ij\bullet}$ & 9 & 14\\ \hline
%27 &  1.07 & 0.64\\
%16 &2.44  & 1.30
%\end{tabular}
%\end{center}
%
%\end{frame}


\begin{frame}[fragile]
\frametitle{Exemple 2}
\vspace*{-2ex}

$T_{i\bullet\bullet}$'s i $T_{\bullet j\bullet}$'s:
\vspace*{-1ex}

\begin{verbatim}
> T.A=aggregate(GSI~llum,data=peixos,sum)
> T.A
  llum   GSI
1    9 17.55
2   14  9.70
> T.B=aggregate(GSI~temp,data=peixos,sum)
> T.B
  temp   GSI
1   16 18.70
2   27  8.55
\end{verbatim}
\begin{center}
\begin{tabular}{cc}
$T_{1\bullet\bullet}$ & $T_{2\bullet\bullet}$ \\ \hline
17.55 &  9.7
\end{tabular}
\qquad\qquad
\begin{tabular}{cc}
$T_{\bullet 1\bullet}$ & $T_{\bullet 2\bullet}$ \\ \hline
8.55 & 18.7
\end{tabular}
\end{center}

\end{frame}


\begin{frame}[fragile]
\frametitle{Exemple 2}
\vspace*{-2ex}

$T_{\bullet\bullet\bullet}$:
\vspace*{-1ex}

\begin{verbatim}
> sum(peixos$GSI)
[1] 27.25
\end{verbatim}

$\overline{X}_{\bullet\bullet\bullet}$:
\vspace*{-1ex}

\begin{verbatim}
> sum(peixos$GSI)/(n*a*b)
[1] 1.3625
\end{verbatim}

$T_{\bullet\bullet\bullet}^{(2)}$:
\vspace*{-1ex}

\begin{verbatim}
> sum(peixos$GSI^2)
[1] 48.2619
\end{verbatim}

\end{frame}

%%%%%
%
%\begin{frame}
%\frametitle{Exemple 2}
%Les dades:
%
%\begin{center}
%\begin{tabular}{ccc}
%\hline
%&\multicolumn{2}{c}{Factor $A$ (tractament)}\\\hline
%Factor $B$&Placebo &Somnífer\\
%(sexe)&&\\\hline
%Home &30&35\\
%&50& 32 \\
%&45 & 30 \\
%&47 & 25 \\
%& 38 & 30 \\\hline
%Dona & 50 & 42\\
%&35 & 30\\
%& 46 & 15 \\
%&25 & 18\\
%&32 & 23 \\\hline
%\end{tabular}
%\end{center}
%\end{frame}
%
%\begin{frame}
%\frametitle{Exemple 2}
%
%\begin{center}
%\begin{tabular}{c|cc}
%$T_{ij\bullet}$ & P & S\\ \hline
%H &  \qquad\qquad & \qquad\qquad\\
%D &\qquad\qquad & \qquad\qquad
%\end{tabular}
%\qquad\qquad\qquad
%\begin{tabular}{c|cc}
%$\overline{X}_{ij\bullet}$ & P & S\\ \hline
%H &  \qquad\qquad & \qquad\qquad\\
%D &\qquad\qquad & \qquad\qquad
%\end{tabular}
%\bigskip
%
%\begin{tabular}{cc}
%$T_{P\bullet\bullet}$ & $T_{T\bullet\bullet}$ \\ \hline
%\qquad\qquad &  \qquad\qquad
%\end{tabular}
%\qquad\qquad\qquad
%\begin{tabular}{cc}
%$\overline{X}_{P\bullet\bullet}$ &$\overline{X}_{T\bullet\bullet}$ \\ \hline
%\qquad\qquad &  \qquad\qquad
%\end{tabular}
%\bigskip
%
%\begin{tabular}{cc}
%$T_{\bullet H\bullet}$ & $T_{\bullet D\bullet}$ \\ \hline
%\qquad\qquad &  \qquad\qquad
%\end{tabular}
%\qquad\qquad\qquad
%\begin{tabular}{cc}
%$\overline{X}_{\bullet H\bullet}$ &$\overline{X}_{\bullet D\bullet}$ \\ \hline
%\qquad\qquad &  \qquad\qquad
%\end{tabular}
%\bigskip
%
%$T_{\bullet\bullet\bullet}=$\qquad\qquad $\overline{X}_{\bullet\bullet\bullet}=$\qquad\qquad $T^{(2)}_{\bullet\bullet\bullet}=$
%
%\end{center}
%
%\end{frame}
%
%
%
%\begin{frame}
%\frametitle{Exemple 2}
%
%\begin{center}
%\begin{tabular}{c|cc}
%$T_{ij\bullet}$ & P & S\\ \hline
%H &  210  &152\\
%D &188  & 128
%\end{tabular}
%\qquad\qquad\qquad
%\begin{tabular}{c|cc}
%$\overline{X}_{ij\bullet}$ & P & S\\ \hline
%H &  42.0 & 30.4\\
%D &37.6 & 25.6
%\end{tabular}
%\bigskip
%
%\begin{tabular}{cc}
%$T_{P\bullet\bullet}$ & $T_{T\bullet\bullet}$ \\ \hline
%398 &  280
%\end{tabular}
%\qquad\qquad\qquad
%\begin{tabular}{cc}
%$\overline{X}_{P\bullet\bullet}$ &$\overline{X}_{T\bullet\bullet}$ \\ \hline
%39.8 & 28
%\end{tabular}
%\bigskip
%
%\begin{tabular}{cc}
%$T_{\bullet H\bullet}$ & $T_{\bullet D\bullet}$ \\ \hline
%362 &  316
%\end{tabular}
%\qquad\qquad\qquad
%\begin{tabular}{cc}
%$\overline{X}_{\bullet H\bullet}$ &$\overline{X}_{\bullet D\bullet}$ \\ \hline
%36.2  & 31.6
%\end{tabular}
%\bigskip
%
%$T_{\bullet\bullet\bullet}=678$\qquad $\overline{X}_{\bullet\bullet\bullet}=33.9$\qquad $T^{(2)}_{\bullet\bullet\bullet}=24984$
%
%\end{center}
%
%\end{frame}
%

\begin{frame}
\frametitle{Identitats de sumes de quadrats}

Emprarem les sumes de quadrats següents:
\begin{itemize}
\item $\red{SS_{Total}} =
\sum\limits_{i=1}^a\sum\limits_{j=1}^b\sum\limits_{k=1}^n
(X_{ijk}-\overline{X}_{\bullet\bullet\bullet})^2$ (Variabilitat total)
\medskip

\item  $\red{SS_A} =b n\sum\limits_{i=1}^a
(\overline{X}_{i\bullet\bullet}-\overline{X}_{\bullet\bullet\bullet})^2$
(Variabilitat deguda al
factor $A$)
\medskip

\item  $\red{SS_B} =a n\sum\limits_{j=1}^b (\overline{X}_{\bullet
j\bullet}-\overline{X}_{\bullet\bullet\bullet})^2$ (Variabilitat deguda al
factor $B$)
\end{itemize}


\end{frame}


\begin{frame}
\frametitle{Identitats de sumes de quadrats}

Emprarem les sumes de quadrats següents:
\begin{itemize}

\item $\red{SS_{AB}}=n \sum\limits_{i=1}^a\sum\limits_{j=1}^b
(\overline{X}_{ij\bullet}-\overline{X}_{i\bullet\bullet}-\overline{X}_{\bullet
j\bullet}+\overline{X}_{\bullet\bullet\bullet})^2$ (Variabilitat deguda a la interacció dels factors $A$
i $B$)
\medskip


\item $\red{SS_{Tr}}=n\sum\limits_{i=1}^a\sum\limits_{j=1}^b
(\overline{X}_{ij\bullet}-\overline{X}_{\bullet\bullet\bullet})^2$ (Variabilitat com si empràssim un sol factor)
\medskip


\item $\red{SS_E} =
\sum\limits_{i=1}^a\sum\limits_{j=1}^b\sum\limits_{k=1}^n
(X_{ijk}-\overline{X}_{ij\bullet})^2$ (Variabilitat deguda a l'error aleatori)
\end{itemize}

\end{frame}


\begin{frame}
\frametitle{Identitats de sumes de quadrats}

\begin{teorema}
$$
\begin{array}{l}
SS_{Total} = SS_{Tr}+SS_E\\[1ex]
SS_{Tr} = SS_A+SS_B+SS_{AB}
\end{array}
$$
\end{teorema}
\end{frame}


\begin{frame}
\frametitle{Càlcul de les sumes de quadrats}
\begin{itemize}

\item $SS_{Total} = T_{\bullet\bullet\bullet}^{(2)}
-\dfrac{T_{\bullet\bullet\bullet}^2}{abn}$
\medskip

\item $SS_A = \displaystyle \dfrac{1}{bn}\sum\limits_{i=1}^a
T_{i\bullet\bullet}^2-\dfrac{T_{\bullet\bullet\bullet}^2}{abn}$
\medskip

\item $SS_B =\displaystyle\dfrac{1}{an}\sum\limits_{j=1}^b T_{\bullet
j\bullet}^2-\dfrac{T_{\bullet\bullet\bullet}^2}{abn}$
\medskip

\item $SS_{Tr}=\displaystyle\dfrac{1}{n}\sum\limits_{i=1}^a\sum\limits_{j=1}^b T_{ij\bullet}^2 -
\dfrac{T_{\bullet\bullet\bullet}^2}{abn}$
\medskip

\item $SS_{AB} = SS_{Tr} -SS_A-SS_B$
\medskip

\item $SS_E = SS_{Total}-SS_{Tr}$
\end{itemize}
\end{frame}





\begin{frame}
\frametitle{Exemple 2}
\begin{center}
\begin{tabular}{c|cc}
$T_{ij\bullet}$ & 9 & 14\\ \hline
27 &  5.35 & 3.2\\
16 &12.20 & 6.5
\end{tabular}
\medskip

\begin{tabular}{cc}
$T_{1\bullet\bullet}$ & $T_{2\bullet\bullet}$ \\ \hline
17.55 &  9.7
\end{tabular}
\quad
\begin{tabular}{cc}
$T_{\bullet 1\bullet}$ & $T_{\bullet 2\bullet}$ \\ \hline
8.55 & 18.7
\end{tabular}
\quad
\begin{tabular}{cc}
$T_{\bullet\bullet\bullet}$ & $T_{\bullet\bullet\bullet}^{(2)}$ \\ \hline
27.25 & 48.2619
\end{tabular}
\bigskip

\only<1>{\begin{tabular}{cccccc}
$SS_{Total}$ & $SS_A$ & $SS_B$ & $SS_{Tr}$ &  $SS_{AB}$ &  $SS_E$\\ \hline
\hphantom{11.1338}  & \hphantom{3.0811}  &  \hphantom{5.1511}  &  \hphantom{8.8624}  &  \hphantom{0.6301}  &  \hphantom{2.2714}
\end{tabular}}
\only<2>{\begin{tabular}{cccccc}
$SS_{Total}$ & $SS_A$ & $SS_B$ & $SS_{Tr}$ &  $SS_{AB}$ &  $SS_E$\\ \hline
11.1338  & 3.0811  &  5.1511  &  8.8624  &  0.6301  &  2.2714
\end{tabular}}
\end{center}


\end{frame}

%
%\begin{frame}
%\frametitle{Exemple 2}
%
%\begin{center}
%\begin{tabular}{c|cc}
%$T_{ij\bullet}$ & P & S\\ \hline
%H &  210  &152\\
%D &188  & 128
%\end{tabular}
%\medskip
%
%\begin{tabular}{cc}
%$T_{P\bullet\bullet}$ & $T_{T\bullet\bullet}$ \\ \hline
%398 &  280
%\end{tabular}
%\qquad 
%\begin{tabular}{cc}
%$T_{\bullet H\bullet}$ & $T_{\bullet D\bullet}$ \\ \hline
%362 &  316
%\end{tabular}
%\medskip
%
%$T_{\bullet\bullet\bullet}=678$\qquad $T^{(2)}_{\bullet\bullet\bullet}=24984$
%\vspace*{1cm}
%
%\only<1>{\begin{tabular}{cccccc}
%$SS_{Total}$ & $SS_A$ & $SS_B$ & $SS_{Tr}$ &  $SS_{AB}$ &  $SS_E$\\ \hline
%   &    &     &     &     &   
%   \end{tabular}}
%   
%\only<2>{\begin{tabular}{cccccc}
%$SS_{Total}$ & $SS_A$ & $SS_B$ & $SS_{Tr}$ &  $SS_{AB}$ &  $SS_E$\\ \hline
%1999.8 & 696.2 & 105.8 & 802.2   & 0.2 &1197.6\end{tabular}}
%
%\end{center}
%
%\end{frame}
%


\begin{frame}
\frametitle{Quadrats mitjans}
Emprarem els quadrats mitjans següents:
\begin{itemize}
\item  $MS_A =\dfrac{SS_A}{a-1}$ 
\medskip

\item  $MS_B =\dfrac{SS_B}{b-1}$
\medskip

\item $MS_{AB}=\dfrac{SS_{AB}}{(a-1)(b-1)}$
\medskip

\item $MS_{Tr}=\dfrac{SS_{Tr}}{ab-1}$
\medskip

\item $MS_E=\dfrac{SS_E}{ab (n-1)}$

\end{itemize}
\end{frame}



\begin{frame} 
\frametitle{Exemple 1}

$n=5$, $a=b=2$
\medskip

\begin{center}
\begin{tabular}{cccccc}
$SS_{Total}$ & $SS_A$ & $SS_B$ &  $SS_{AB}$ & $SS_{Tr}$ &  $SS_E$\\ \hline
11.1338  & 3.0811  &  5.1511 &  0.6301 &  8.8624    &  2.2714
\end{tabular}
\bigskip

\only<1>{\begin{tabular}{ccccc}
$MS_A$ & $MS_B$ & $MS_{AB}$ & $MS_{Tr}$ & $MS_E$\\ \hline
\hphantom{3.0811}  &  \hphantom{5.1511} & \hphantom{0.6301} & \hphantom{2.9541} & \hphantom{0.142}
\end{tabular}}
\only<2>{\begin{tabular}{ccccc}
$MS_A$ & $MS_B$ & $MS_{AB}$ & $MS_{Tr}$ & $MS_E$\\ \hline
3.0811  &  5.1511 & 0.6301 & 2.9541 & 0.142
\end{tabular}}

\end{center}
%\pause\bigskip
%
%\blue{\bf Exemple 2}
%\begin{center}
%\begin{tabular}{cccccc}
%$SS_{Total}$ & $SS_A$ & $SS_B$ & $SS_{AB}$ & $SS_{Tr}$ &    $SS_E$\\ \hline
%1999.8 & 696.2 & 105.8 & 0.2 & 802.2   & 1197.6 
%\end{tabular}
%\medskip
%
%\begin{tabular}{ccccc}
%$MS_A$ & $MS_B$ & $MS_{AB}$ & $MS_{Tr}$ & $MS_E$\\ \hline\pause
%696.2 & 105.8 & 0.2 & 267.4 & 74.85
%\end{tabular}
%\end{center}

\end{frame}









\subsection{Contrastos}



\begin{frame}
\frametitle{Contrastos a realitzar}

En una ANOVA de dues vies, ens poden interessar 
els quatre contrastos següents:
\bigskip

 \emph{Contrast de mitjanes del factor $A$}: Contrastam si  hi ha diferències entre els nivells del factor $A$:
$$
\left\{
\begin{array}{l}
H_0 : \mu_{1\bullet\bullet}=\mu_{2\bullet\bullet}=\cdots
=\mu_{a\bullet\bullet} \\
H_1 :  \exists i,i'\mid  \mu_{i\bullet\bullet}
\not = \mu_{i'\bullet\bullet}
\end{array}
\right.
$$
\medskip

L'estadístic de contrast és
$$
F=\red{\frac{MS_A}{MS_E}},
$$
el qual, si $H_0$ és certa, té
distribució $F$ de Fisher amb $a-1$ i $ab(n-1)$ graus de llibertat i valor proper a 1

\end{frame}


\begin{frame}
\frametitle{Contrastos a realitzar}

En una ANOVA de dues vies, ens poden interessar 
els quatre contrastos següents:
\bigskip

 \emph{Contrast de mitjanes del factor $B$}: Contrastam si  hi ha diferències entre els nivells del factor $B$:
$$
\left\{
\begin{array}{l}
H_0 : \mu_{\bullet 1\bullet}=\mu_{\bullet 2\bullet}=\cdots =\mu_{ \bullet
b\bullet} \\
H_1 :  \exists j,j'\mid  \mu_{\bullet j\bullet}
\not = \mu_{\bullet j'\bullet}
\end{array}
\right.
$$
\medskip

L'estadístic de contrast és
$$
F=\red{\frac{MS_B}{MS_E}},
$$
el qual, si $H_0$ és certa, té
distribució $F$ de Fisher amb $b-1$ i $ab(n-1)$ graus de llibertat i valor proper a 1

\end{frame}



\begin{frame}
\frametitle{Contrastos a realitzar}

En una ANOVA de dues vies, ens poden interessar 
els quatre contrastos següents:
\bigskip

 \emph{Contrast dels tractaments}: Contrastam si  hi ha diferències entre les parelles (nivell de  $A$, nivell de  $B$):
$$
\left\{
\begin{array}{l}
H_0 : \forall i,j,i',j'\mid \mu_{ij\bullet}=\mu_{i'j'\bullet} \\
H_1 :  \exists i,j,i',j'\mid \mu_{ij\bullet}\neq \mu_{i'j'\bullet}
\end{array}
\right.
$$
\medskip

L'estadístic de contrast és
$$
F=\red{\frac{MS_{Tr}}{MS_E}},
$$
el qual, si $H_0$ és certa, té
distribució $F$ de Fisher amb $ab-1$ i $ab(n-1)$ graus de llibertat i valor proper a 1

\end{frame}

\begin{frame}
\frametitle{Contrastos a realitzar}

En una ANOVA de dues vies, ens poden interessar 
els quatre contrastos següents:
\bigskip

\emph{Contrast de no interacció}: Contrastam si  hi ha interacció entre
els factors $A$ i $B$:
$$
\left\{
\begin{array}{l}
H_0 :  \forall i,j\mid (\alpha\beta)_{ij} =0 \\
H_1 :  \exists i,j\mid (\alpha\beta)_{ij}
\not = 0
\end{array}
\right.
$$
\medskip

L'estadístic de contrast és
$$
F = \red{\frac{MS_{AB}}{MS_E}},
$$
el qual, si $H_0$ és certa, té
distribució $F$ de Fisher amb $(a-1)(b-1)$ i $ab(n-1)$ graus de
llibertat  i valor proper a 1

\end{frame}


\begin{frame}
\frametitle{Contrastos a realitzar}

En els quatre casos, el p-valor és
$$
P(F_{x,y}\geq \mbox{valor de l'estadístic})
$$
on $F_{x,y}$ representa la distribució $F$ de Fisher amb els graus de llibertat que pertoquin.

\end{frame}


\begin{frame}
\frametitle{Taula ANOVA}
Els contrastos anteriors es resumeixen en la taula  següent:

{\small \begin{center}
\begin{tabular}{|@{}c@{}|@{}c@{}|@{}c@{}|@{}c@{}|@{}c@{}|@{}c@{}|}
\hline
Variació\hspace*{1ex}&\ Graus de\hspace*{1ex}&\ Suma de\hspace*{1ex}&\ Quadrats\hspace*{1ex}&$F$ &\ p-valors\hspace*{1ex} \\
&llibertat&quadrats&mitjans& &\\\hline
Tract.&$ab-1$&$SS_{Tr}$&$MS_{Tr}$&\ $\dfrac{MS_{Tr}}{MS_E}$ &\ p-valor \\[2ex]
$A$&$a-1$&$SS_A$&$MS_{A}$&\ $\dfrac{MS_{A}}{MS_E}$& \ p-valor\\[2ex]
$B$&$b-1$&$SS_B$&$MS_{B}$&\ $\dfrac{MS_{B}}{MS_E}$& \ p-valor\\[2ex]
$AB$&\ {\footnotesize $(a-1)(b-1)$}\hspace*{1ex}&$SS_{AB}$&$MS_{AB}$&\ $\dfrac{MS_{AB}}{MS_E}$& \ p-valor\\
Error&$ab(n-1)$&$SS_E$&$MS_{E}$&& \\\hline
\end{tabular}
\end{center}
}
\end{frame}


\begin{frame}
\frametitle{Exemple 2}
$n=5$, $a=b=2$
\medskip

\begin{center}
\begin{tabular}{ccccc}
$MS_A$ & $MS_B$ & $MS_{AB}$ & $MS_{Tr}$ & $MS_E$\\ \hline
3.0811  &  5.1511 & 0.6301 & 2.9541 & 0.142
\end{tabular}
\end{center}
\medskip

$\dfrac{MS_{A}}{MS_E}=21.7$, p-valor$=P(F_{1,16}\geq 21.7)=0.0003$
\bigskip

Al contrast del factor $A$, rebutjam la hipòtesi nu\l.la i concloem que hi ha diferències entre els dos nivells


\end{frame}
\begin{frame}
\frametitle{Exemple 2}
$n=5$, $a=b=2$
\medskip

\begin{center}
\begin{tabular}{ccccc}
$MS_A$ & $MS_B$ & $MS_{AB}$ & $MS_{Tr}$ & $MS_E$\\ \hline
3.0811  &  5.1511 & 0.6301 & 2.9541 & 0.142
\end{tabular}
\end{center}
\medskip

$\dfrac{MS_{B}}{MS_E}=36.3$, p-valor$=P(F_{1,16}\geq 36.3)=2\cdot 10^{-5}$
\bigskip


Al contrast del factor $B$, rebutjam la hipòtesi nu\l.la i concloem que hi ha diferències entre els dos nivells

\end{frame}
\begin{frame}
\frametitle{Exemple 2}
$n=5$, $a=b=2$
\medskip

\begin{center}
\begin{tabular}{ccccc}
$MS_A$ & $MS_B$ & $MS_{AB}$ & $MS_{Tr}$ & $MS_E$\\ \hline
3.0811  &  5.1511 & 0.6301 & 2.9541 & 0.142
\end{tabular}
\end{center}
\bigskip

$\dfrac{MS_{Tr}}{MS_E}=20.8$, p-valor$=P(F_{3,16}\geq 20.8)=9\cdot 10^{-6}$
\medskip


Al contrast dels tractaments, rebutjam la hipòtesi nu\l.la i concloem que hi ha diferències entre les parelles (nivell de $A$, nivell de $B$)

\end{frame}



\begin{frame}
\frametitle{Exemple 2}
$n=5$, $a=b=2$
\medskip

\begin{center}
\begin{tabular}{ccccc}
$MS_A$ & $MS_B$ & $MS_{AB}$ & $MS_{Tr}$ & $MS_E$\\ \hline
3.0811  &  5.1511 & 0.6301 & 2.9541 & 0.142
\end{tabular}
\end{center}
\bigskip


$\dfrac{MS_{AB}}{MS_E}=4.437$,  p-valor$=P(F_{1,16}\geq 4.437)=0.051$
\medskip

Al contrast de no interacció, estam en zona de penombra;  amb nivell de significació $0.05$, no podem rebutjar la hipòtesi nu\l.la i hem de concloure que no hi ha interacció entre els factors

\end{frame}




\begin{frame}
\frametitle{Exemple 2}
{\small
\begin{center}
\begin{tabular}{|c|c|c|c|c|c|}
\hline
Variació&Graus de&Suma de&Quadrats& $F$ & p-valors\\
&llibertat&quadrats&mitjans& &\\\hline
Tract.&$3$&$8.862$&$2.954$&$20.8$ & $9\cdot 10^{-6}$ \\
$A$&$1$&$3.081$&$3.081$&$21.7$ & $0.0003$\\
$B$&$1$&$5.151$&$5.151$&$36.3$ & $2\cdot 10^{-5}$\\
$AB$&$1$&$0.630$&$0.630$&$4.44$ & $0.051$\\
Error&$16$&$2.271$&$0.142$& &\\\hline
Total&$19$&$11.134$&& &\\\hline
\end{tabular}
\end{center}
}
\end{frame}


\begin{frame}[fragile]
\frametitle{Exemple 2 amb R}

{\small \begin{verbatim}
> summary(aov(GSI~llum*temp,data=peixos))
            Df Sum Sq Mean Sq F value   Pr(>F)    
llum         1  3.081   3.081  21.704 0.000262 ***
temp         1  5.151   5.151  36.285 1.77e-05 ***
llum:temp    1  0.630   0.630   4.439 0.051268 .  
Residuals   16  2.271   0.142                     
\end{verbatim}
}

\end{frame}


\begin{frame}[fragile]
\frametitle{Exemple 2 amb R}

Hi falta la filera dels Tractaments. Aquesta s'obté amb un ANOVA d'un (nou) factor que tengui com a  nivells les parelles de nivells de A i B. Es pot realitzar amb
{\small \begin{verbatim}
> summary(aov(GSI~llum:temp,data=peixos))
            Df Sum Sq Mean Sq F value   Pr(>F)    
llum:temp    3  8.862   2.954   20.81 9.06e-06 ***
Residuals   16  2.271   0.142                     
\end{verbatim}
}


\end{frame}

%
%\begin{frame}
%\frametitle{Exemple 2}
%\vspace*{-3ex}
%
%\begin{center}
%\begin{tabular}{ccccc}
%$MS_A$ & $MS_B$ & $MS_{AB}$ & $MS_{Tr}$ & $MS_E$\\ \hline
%696.2 & 105.8 & 0.2 & 267.4 & 74.85
%\end{tabular}
%\end{center}
%\begin{itemize}
%\item  $\dfrac{MS_{A}}{MS_E}=$\qquad, p-valor$=$\qquad \\
%Conclusió:
%\bigskip
%
%\item  $\dfrac{MS_{B}}{MS_E}=$\qquad, p-valor$=$\qquad \\
%Conclusió:
%\bigskip
%
%\item  $\dfrac{MS_{Tr}}{MS_E}=$\qquad, p-valor$=$\qquad \\
%Conclusió:
%\bigskip
%
%\item  $\dfrac{MS_{AB}}{MS_E}=$\qquad, p-valor$=$\qquad \\
%Conclusió:
%\bigskip
%
%\end{itemize}
%\end{frame}
%
%
%\begin{frame}
%\frametitle{Exemple 2}
%\vspace*{-3ex}
%
%\begin{center}
%\begin{tabular}{ccccc}
%$MS_A$ & $MS_B$ & $MS_{AB}$ & $MS_{Tr}$ & $MS_E$\\ \hline
%696.2 & 105.8 & 0.2 & 267.4 & 74.85
%\end{tabular}
%\end{center}
%\begin{itemize}
%\item  $\dfrac{MS_{A}}{MS_E}=9.3$, p-valor$=P(F_{1,16}\geq 9.3)=0.0076$ \\
%Conclusió: Hi ha diferències entre els tractaments
%
%\item  $\dfrac{MS_{B}}{MS_E}=1.41$, p-valor$=P(F_{1,16}\geq 1.41)=0.25$ \\
%Conclusió: No hi ha diferències entre els sexes
%
%\item  $\dfrac{MS_{Tr}}{MS_E}=3.57$, p-valor$=P(F_{3,16}\geq 3.57)=0.038$ \\
%Conclusió: Hi ha diferències entre les parelles (sexe, tractament)
%
%\item  $\dfrac{MS_{AB}}{MS_E}=0.003$, p-valor$=P(F_{1,16}\geq 0.003)=0.96$ \\
%Conclusió: No hi ha interacció entre els factors
%
%\end{itemize}
%\end{frame}
%
%
%\begin{frame}
%\frametitle{Exemple 2}
%{\small
%\begin{center}
%\begin{tabular}{|c|c|c|c|c|c|}
%\hline
%Variació&Graus de&Suma de&Quadrats& $F$ & p-valors\\
%&llibertat&quadrats&mitjans& &\\\hline
%Tract.& & & &\qquad\qquad\qquad   &  \\
%$A$&& & & &  \\
%$B$&& & & &  \\
%$AB$&& & & & \\
%Error&& & & &  \\\hline
%Total&& & & &  \\\hline
%\end{tabular}
%\end{center}
%}
%\end{frame}
%
%\begin{frame}
%\frametitle{Exemple 2}
%{\small
%\begin{center}
%\begin{tabular}{|c|c|c|c|c|c|}
%\hline
%Variació&Graus de&Suma de&Quadrats& $F$ & p-valors\\
%&llibertat&quadrats&mitjans& &\\\hline
%Tract.&$3$&$802.2$&$267.4$&$3.57$ & $0.038$ \\
%$A$&$1$&$696.2$&$696.2$&$9.3$ & $0.0076$\\
%$B$&$1$&$105.8$&$105.8$&$1.41$ & $0.25$\\
%$AB$&$1$&$0.2$&$0.2$&$0$ & $0.96$\\
%Error&$16$&$1197.6$&$74.85$& &\\\hline
%Total&$19$&$1999.8$&& &\\\hline
%\end{tabular}
%\end{center}
%}
%\end{frame}
%
%
%\begin{frame}[fragile]
%\frametitle{Exemple 2 amb R}
%
%{\small \begin{verbatim}
%> summary(aov(Temps~Tract*Sexe,data=df.temps))
%            Df Sum Sq Mean Sq F value  Pr(>F)   
%Tract        1  696.2   696.2   9.301 0.00764 **
%Sexe         1  105.8   105.8   1.413 0.25182   
%Tract:Sexe   1    0.2     0.2   0.003 0.95941   
%Residuals   16 1197.6    74.9                   
%\end{verbatim}
%}
%\end{frame}
%
%
%\subsection{Comparacions}
%\begin{frame}
%\frametitle{Comparacions amb Duncan}
%
%Com en altres casos, si rebutjam una hipòtesi nu\l.la (o les dues)
%de igualtat de mitjanes entre tots els nivells del factor $A$ o del factor $B$,
%podem investigar per a quines parelles nivells hi ha diferències
%\medskip
%
%Es pot emprar el contrast per parelles amb ajustaments, \texttt{pairwise.test}
%\medskip
%
%També es pot emprar el test de Duncan, com als casos anteriors, \texttt{duncan.test}
%\end{frame}
%
%
%\begin{frame}
%\frametitle{Comparacions amb Duncan}
%
%Per trobar diferències entre mitjanes de nivells del factor $A$, empram
%$$
%SSR_p = r_p \sqrt{\frac{MS_E}{bn}}
%$$
%\medskip
%
%Per trobar diferències entre mitjanes de nivells del factor $B$, empram
%$$
%SSR_p = r_p \sqrt{\frac{MS_E}{an}}
%$$
%\medskip
%
%\emph{Als dos casos, amb $MS_E$ calculat com en un ANOVA d'un factor i efectes fixats}
%\end{frame}



\end{document}

