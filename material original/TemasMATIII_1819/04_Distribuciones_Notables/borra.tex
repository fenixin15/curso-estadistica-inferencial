\documentclass{article}
\begin{document}

\begin{itemize}
\item Supongamos que sólo estamos interesados en el número de fracasos antes de obtener el
primer éxito. 
\item Si definimos $Y$= número de fracasos antes del primer éxito. Entonces $Y$ toma
valores en $\{0,1,2,\ldots\}$ y su función de probabilidad es:
$$P_Y(y)=\left\{\begin{array}{ll}
 q^k p & \mbox{ si } k=0,1,2,\ldots\\
 0 &\mbox{ en otro caso}
    \end{array}\right..$$
\item Notemos que $Y=X-1$.
\end{itemize}

\end{document}