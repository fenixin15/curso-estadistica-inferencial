\documentclass[12pt]{article}
\usepackage{enumerate}
\usepackage[utf8]{inputenc}
 %\input{8bitdefs}
% \textwidth 15cm
% \textheight 22cm
 \setlength{\textwidth}{16.5cm}
\setlength{\textheight}{24cm}
 \setlength{\oddsidemargin}{-0.3cm}
 \setlength{\evensidemargin}{1cm} \addtolength{\headheight}{\baselineskip}
\addtolength{\topmargin}{-3cm}
\newcommand{\RR}{\mbox{I\kern-.2em\hbox{R}}}
\def\N{I\!\!N}
\def\R{I\!\!R}
\def\Z{Z\!\!\!Z}
\def\Q{O\!\!\!\!Q}
\def\C{I\!\!\!\!C}

% \def\Q{{\mathchoice {\setbox0=\hbox{$\displaystyle\rm
% Q$}\hbox{\raise
% 0.15\ht0\hbox to0pt{\kern0.4\wd0\vrule height0.8\ht0\hss}\box0}}
% {\setbox0=\hbox{$\textstyle\rm Q$}\hbox{\raise
% 0.15\ht0\hbox to0pt{\kern0.4\wd0\vrule height0.8\ht0\hss}\box0}}
% {\setbox0=\hbox{$\scriptstyle\rm Q$}\hbox{\raise
% 0.15\ht0\hbox to0pt{\kern0.4\wd0\vrule height0.7\ht0\hss}\box0}}
% {\setbox0=\hbox{$\scriptscriptstyle\rm Q$}\hbox{\raise
% 0.15\ht0\hbox to0pt{\kern0.4\wd0\vrule height0.7\ht0\hss}\box0}}}}

\begin{document}

%\pagestyle{empty}
\font\sc=cmcsc10
\parskip=1ex
%%\newcount\problemes
\newcounter{problemes}
\setcounter{problemes}{0}

%%%\newcount\problemes
%%%\problemes=0

%%%\newenvironment{prob}{\vskip 0.25cm\addtocounter{problemes}{1}
%%%\noindent{\textbf{\thebloque.\theproblemes.- }}}
\newenvironment{prob}{\vskip 0.25cm\addtocounter{problemes}{1}
\noindent{\textbf{\theproblemes.- }}}
%\newenvironment{resolucion}{\hfill $\bullet$}


\newcommand{\sol}[1]{{\textbf{\footnotetext[\theproblemes]{Sol.: #1} }}}

\newcommand{\probl}{\vskip 0.25cm\addtocounter{problemes}{1}
\noindent{\textbf{\theproblemes.- }}}

%\newcommand{\sol}[1]{{\textbf{\footnotetext[\the\problemes]{Sol.: #1} }}}
%%%
%%%\newcommand{\probl}{\addtocounter{problemes}{1} \vskip 2ex \noindent
%%%{\bf \the\problemes)}}
%%%
%%%\def\probl{\advance\problemes by 1
%%%\vskip 1mm\noindent{\bf \the\problemes) }}
%%%\newcounter{pepe}



%%%%%PROBABILIDAD

%%%\newcount\problemes
%%%\problemes=0

%\chapter*{\large \textbf{BLOQUE 1 ESTAD?STICA DESCRIPTIVA (resumen)}}



\begin{centerline}
{{\bf PROBABILIDAD}}
\end{centerline}


\newcommand{\pr}[1]{P(#1)}

%\newcounter{problema}
%\newcommand{\prb}{\addtocounter{problema}{1}
%\noindent\vskip 2mm {\textbf{\theproblemes  }}


%\setcounter{problema}{105}
 %
 %  %tema 7
%%%fulla 4

\probl  En una carrera en la que participan diez caballos ¿de
cuántas maneras diferentes se pueden dar los cuatro primeros
lugares? \sol{\bf 5040}

\probl  Una empresa de reciente creación encarga a un diseñador gráfico la elaboración
del su logotipo, indicando que ha de seleccionar exactamente tres colores de una lista de
seis. ¿Cuántos grupos tienen para elegir el diseñador? \sol{\bf 20}

\probl  ¿Cuántas palabras diferentes, de cuatro letras, se pueden
formar con la palabra {\bf byte}? \sol{\bf 24}

\probl  ¿De cuantas maneras diferentes se pueden elegir el director y el subdirector de
un departamento formado por 50 miembros? \sol{\bf 2450}

\probl  Con once empleados ¿cuántos comités de  empresa de cinco
personas se pueden formar? \sol{\bf  462}

\probl  ¿Cuántas maneras distintas hay de colocar quince libros diferentes en una
estantería si queremos  que el  de Probabilidades esté el  primero y  el de Estadística
en el tercero? \sol{\bf 6227020800}

\probl  ¿cuántos caracteres diferentes podemos formar utilizando a lo sumo  a tres
símbolos  de los utilizados en el alfabeto Morse? \sol{\bf 8+4+2=14}

\probl  Un supermercado  organiza una rifa con un premio de  una botella de cava para
todas las papeletas que tengan las dos últimas cifras iguales a las correspondientes dos
últimas cifras del número premiado en el sorteo de Navidad. Supongamos que todos los
décimos tienen cuatro cifras y que existe un único décimo de cada numeración ¿Cuántas
botellas repartirá el supermercado? \sol{\bf 100}

\probl  ¿Cuántas  palabras diferentes podemos formar con todas las letras de la palabra
estadística? \sol{\bf 2494800}

\probl  En una tienda de regalos hay relojes de  arena con cubetas
de colores, y no hay diferencia alguna entre las  dos cubetas que
forman cada reloj. Si hay cuatro  colores posibles y el color de
los dos recipientes puede coincidir ¿cuántos  modelos de reloj de
arena puede ofrecer el establecimiento? \sol{\bf 6}

\probl  En una partida de parchís gana aquel jugador que consigue alcanzar antes  con sus
cuatro fichas la llegada. Si hay  cuatro jugadores y la partida continua hasta que todos
han completado el recorrido ¿cuántos  órdenes de llegadas hay para la dieciséis fichas?
\sol{\bf 63063000}

\probl  Se han de repartir cinco becas entre diez españoles  y seis extranjeros, de
manera que se den tres a españoles y dos a extranjeros ¿De Cuántas maneras se puede hacer
el reparto? \sol{\bf 1800}

\probl  ¿Cuantas  fichas tiene un dominó? \sol{\bf 28}




\probl  Calcular la probabilidad de que al lanzar a la vez $5$
dados se obtenga:
\begin{enumerate}[a)]
\item repóker (5 resultados iguales);
\item póker (4 resultados iguales);
\item full (3 resultados iguales y los otros distintos pero iguales entre si);
\item trío (3  resultados iguales y los otros dos  diferentes);
\item doble pareja (2 resultados  iguales, otros 2 iguales y diferentes de los anteriores y el restante diferente
);
\item pareja (exactamente 2 resultados iguales);
\item nada (5 resultados distintos).
\end{enumerate}
\sol{ a) $\bf 6/6^5$; b) $\bf 150/6^5$;c) $\bf 300/6^5$;  d) $\bf
1200/6^5$; e) $\bf 1800/6^5$; f) $\bf 3600/6^5$; g) $\bf 720/6^5$}

\probl  Tenemos 12 radios de las   que  5 son defectuosas. Elegimos  3 radios al azar.
¿Cuál  es la probabilidad de que solo una de las 3 sea defectuosa? \sol{{\bf 21/44}}

\probl  Lanzamos al aire 6 dados.
\begin{enumerate}[a)]
\item ¿Cuál es la probabilidad de que todos ellos den resultados distintos?
\item ¿Cuál es la probabilidad de obtener 3 parejas?
\end{enumerate}
\sol{ a)$\mathbf{120/6^5}$; b)  $\mathbf{300/6^5}$ }

\probl  Supongamos que en una empresa de fabricación de componentes electrónicos se sabe
que en un lote de 550 almacenados el 2\% son   defectuosos ¿Cuál es la probabilidad de
encontrar 2 de defectuosos si cogemos de forma equiprobable 25? \sol{$\mathbf{0.074}$}

\probl  Si mezclamos   suficientemente una baraja de 52 cartas ¿Cuál es la probabilidad
de que los 4 ases queden colocados consecutivamente? \sol{{\bf 24/132600}}

\probl  Una forma de incrementar la  fiabilidad de un sistema es la introducción de una
copia de los componentes en una configuración paralela. Supongamos  que la N.A.S.A.
quiere un vuelo  con una probabilidad no inferior a $0.99999$ de  que el transbordador
espacial entre en órbita alrededor de la Tierra con éxito. ¿cuántos   motores se han de
montar en  paralelo para que se  alcance esta fiabilidad, si se sabe que la probabilidad
de que cada uno de los motores funcione adecuadamente es $0.95$? Suponer que los motores
funcionan de manera independiente entre si. \sol{{\bf 4}}

\probl  ¿Cuál es la probabilidad de  que  entre
 $n$ personas, que no han nacido el 29 de
febrero,  haya  como a mínimo dos que hayan nacido el mismo día del año? (no
necesariamente del mismo año). Calcular la probabilidad para los siguientes valores de
$n: 10, 15, 22, 23, 30, 40, 50, 55$. \sol{$\mathbf 0.12; 0.25; 0.48; 0.51; 0.71; 0.89;
0.97; 0.99$}

\probl  Cuatro cartas numeradas de $1$ a $4$ están colocadas  boca abajo sobre una  mesa.
Una persona, supuestamente clarividente, irá adivinando  los valores de las $4$ cartas
una a una. Si suponemos que es un farsante y que lo que hace  es decir los cuatro números
al azar ¿Cuál es la probabilidad de   que acierte como mínimo una? (Obviamente, no
repite ningún número) \sol{\bf 15/24}

\probl  En una lotería hay $500$ billetes y $5$ premios. Si una persona compra $10$
billetes ¿Cuál es la probabilidad de obtener?:
\begin{enumerate}[a)]
\item el primer premio?
\item como  mínimo un premio?
\item exactamente un premio?
\end{enumerate} \sol{{\bf 0.02; 0.096; 0.093}}

\probl  Se elige de forma equiprobable un número del $1$ al
$6000$. Calcular la probabilidad de que sea múltiplo de $2$ o de
$3$ o de $4$ o de $5$. \sol{\bf 0.73}


\probl  Si elegimos un número de entre los 120 primeros enteros
  positivos  ¿Cuál es la probabilidad de  que  sea
múltiplo de $3$, no sea divisible por 5, y sea divisible por 4 o
por 6? \sol{\bf 2/15}

\probl  Una cuarta parte de la población ha sido  vacunada contra
una enfermedad contagiosa. Durante una epidemia, se observa que de
  uno de entre cada cuatro enfermos ha sido vacunado.
\begin{enumerate}[a)]
\item ¿Ha tenido alguna eficacia la vacuna?
\item Por otra parte, se sabe que hay un enfermo  entre cada $12$ personas
vacunadas ¿Cuál es la probabilidad de que  esté enferma una
persona que no se ha vacunado?\sol{a) {\bf No}; b) {\bf 11/108}}
\end{enumerate}

\probl La probabilidad de  que un estudiante  acabe una carrera determinada es $0.4$.
Dado un grupo de $5$ estudiantes de esta carrera, calcular la probabilidad de que:

\begin{enumerate}[a)]
\item  ninguno acabe la carrera,
\item solo uno acabe la carrera,
\item al menos dos acaben la carrera;
\item todos la acaben.
\end{enumerate}
\sol{\bf a) 0.07776; b) 0.2592; c) 0.66304; d) 0.01024}

\probl  Un  mensaje se ha codificado con un alfabeto de dos símbolos $A$ y $B$ para poder
transmitirse a través de un canal de comunicación. La codificación es tal que $A$ aparece
el doble de veces que $B$ en el mensaje codificado. El ruido del canal es tal que cuando
$A$ se transmite, se recibe  $A$ con una probabilidad de $0.8$ y  $B$ con una
probabilidad de  $0.2$; cuando se transmite $B$ se recibe  $B$ con una probabilidad de
$0.7$ y se recibe $A$ con probabilidad $0.3$.
\begin{enumerate}[a)]
\item ¿Cuál es la frecuencia relativa de $A$ en el mensaje recibido?
\item Si última letra del mensaje recibido es una $A$ ¿Cuál es la probabilidad de que
 se haya enviado una $A$?
\end{enumerate}\sol{{\bf a) 0.633; b) 0.84}}

\probl  En una ciudad se publican 3 diarios $A$, $B$ y $C$. El $30
\%$ de la población lee $A$, el $20 \%$ lee $B$ y el $15 \%$ lee
C; el $12 \%$ lee $A$ y $B$, el $9 \%$ lee $A$ y $C$, y el $6 \%$
lee $B$ y $C$; finalmente, el $3 \%$ lee $A$, $B$ y $C$.
Calcular:\
\begin{enumerate}[a)]
\item El porcentaje de gente que lee al menos uno de los tres diarios.
\item El porcentaje de gente que solo lee $A$.
\item  El porcentaje de gente que lee $B$ o $C$, pero no $A$.
\item  El porcentaje de gente que lee $A$ o no lee ni $B$ ni $C$.
\sol{ \bf{ a) 0.41 ; b) 0.12; c) 0.11; d) 0.89}}
\end{enumerate}


 \probl  Supongamos que en un dado la probabilidad de cada una de sus seis
 caras es proporcional al número inscrito en ella. Calcular la probabilidad
de obtener un número par. \sol{\bf 4/7}

\probl  En una reunión, $n$ personas ($n \geq 3$) lazan una moneda
al  aire. Si una de ellas da diferente de todas las otras, su
propietario paga una ronda ¿Cuál es la probabilidad de que pase
esto? \sol{$\mathbf{(n \cdot \left( {1 \over 2} \right)^{n-1})}$}



%%%%fulla 3


\probl  Un matrimonio planifica  su descendencia  considerando los
siguientes esquemas (se supone que tener una varón o una hembra es
equiprobable):
\begin{description}
\item[Esq. A)] Tener $3$ hijos (de cualquier sexo:varón o hembra).
\item[Esq. B)] Tener varones hasta que nazca la primera hembra, o ya tengan tres
hijos (lo que pase primero).
\item[Esq. C)] Tener hijos hasta que tengan una pareja de ambos sexos,
 o ya tengan tres hijos (lo que pase primero).
\end{description}
Sea $B_i$ el suceso  han nacido $i$ varones ($i=1, 2, 3$) y $C$ el suceso tener más varones
que hembras.
\begin{enumerate}[1)]
\item Calcular $p(B_1)$ y $p(C)$ en cada uno de  los tres esquemas.
\item Calcular $p(B_2)$ y $p(B_3)$ en cada uno de los tres esquemas.
\item Sea $E$ el suceso que el total de hijos  tenga igual número de
varones que de hembras. Encontrar $p(E)$ en cada uno de los tres
esquemas.

\sol{a) $\bf(p(B_1)=3/8,1/4,5/8; p(C)=1/2,1/2,1/4)$; b) $\bf
(p(B_2)=3/8,1/8,1/8; p(B_3)=1/8,1/8,1/8)$; c)
$\mathbf{(0,1/4,1/2)}$.}
\end{enumerate}

\probl  Un comerciante ha de viajar en avión de Bangkok a Bagdad. Preocupado, pide a la
compañía aérea ¿Cuál es la probabilidad de que haya como mínimo una bomba en el avión y le
dicen que es de $0.1$. Más preocupado aún pide cuál es la probabilidad de que haya como
mínimo dos bombas y le dicen que es $0.01$. Más tranquilo, decide llevar una bomba en su
equipaje. Haciendo las suposiciones adicionales oportunas ¿qué  valoración estadística
podemos hacer de su decisión? \sol{\bf Decisión absurda por sentido común y
estadísticamente. }

\probl  Dos sistemas con cuatro componentes independientes con
fiabilidades respectivas $p_1, p_2, p_3$ y $p_4$ se configuran de
las dos maneras siguientes: En el sistema $A$, la combinación en
serie de los componentes $1$  y $2$ se configura en paralelo con la
combinación en serie de los componentes $3$ y $4$; en el sistema
$B$, la combinación en paralelo de $1$ y $3$ se configura en serie
con la combinación en paralelo de $2$ y $4$. Determinar el sistema
más fiable. \sol{\bf B}

\probl  Si un sistema que consiste en tres componentes independientes con la misma
fiabilidad ($p_1=p_2=p_3$) tiene una fiabilidad de $0.8$, determinar $p_1$ en los
siguientes casos:

\begin{enumerate}[a)]
\item el componente 3 está configurado en serie con la combinación en paralelo
$1$ y $2$.
 \item el componente 3 está configurado en paralelo
con la combinación en serie de $1$ y $2$.
\end{enumerate}
 \sol{\bf a) 0.825; b) 0.652}
%%
%%\probl  N'\`Oscar diu la veritat nou vegades de cada deu i n'Ivan
%%set de cada nou. S'extreu a l'atzar una bolla d'una bossa on hi
%%havia 5 bolles blanques i 20 negres. Tots dos observen el color de
%%la bolla extreta i llavors diuen de manera independent que la
%%bolla extreta ?s blanca. Quina ?s la probabilitat que aix\`o sigui
%%cert? \sol{\bf 0.89}



%%%%%%%%%%%%%tema 8
%
\newpage

%\problemes=0
\begin{centerline}
{{\bf VARIABLES ALEATORIAS}}
\end{centerline}


\probl{En los ocho problemas siguientes determinar la función de probabilidad y la de
distribución de las variables aleatorias que aparezcan}
\begin{enumerate}[a)]
\item Consideremos el experimento  consistente en lanzar simultáneamente dos dados; repetimos
el experimento dos veces. Sea $X$ la variable aleatoria que da  el número de
lanzamientos en que los dos dados han mostrado un número par. Sea $Y$ la variable aleatoria
que nos da el número de lanzamientos en que la suma de los dos dados es par.

\item  Supongamos que tenemos almacenadas $10$ piezas, de las que sabemos que hay $8$ del
tipo I y $2$ del tipo II; se toman dos al azar de forma equiprobable. Sea $X$ la variable
aleatoria que da el número de piezas de tipo I que hemos cogido.

\item  Supongamos  que un estudiante realiza el tipo de examen siguiente: El profesor le va
formulando preguntas hasta que el estudiante falla una (no os preguntéis como se evalúa ni
yo lo sé). La probabilidad  de que el estudiante acierte  una pregunta cualquiera es $0.9$
(examen fácil). Sea $X$ la variable aleatoria que nos da el número de preguntas formuladas
por el profesor ¿Cuál es el número más probable de preguntas formuladas?

\item   Consideremos  dos cañones que van disparando alternativamente hacia el mismo
objetivo. El primer cañón tiene una probabilidad de acertar el objetivo de $0.3$ mientas
que en el segundo es de $0.7$. El primer cañón comienza la serie de lanzamientos y no se
detienen hasta que uno de los cañones desintegre el blanco (es suficiente darle una vez).
Sea $X$ la variable aleatoria que nos da el número de proyectiles lanzados por el primer
cañón e $Y$ la que nos da los proyectiles lanzados por el segundo cañón.

\item  En la misma situación que en el ítem anterior,
considerar la variable aleatoria $X$ que da el número de proyectiles lanzados por el primer
cañón condicionado a que gana  y sea  $Y$ la variable aleatoria que cuenta el número de
proyectiles lanzados por segundo cañón cuando gana.

\item  Supongamos que se hace una tirada de $100000$ ejemplares de un determinado libro
. La probabilidad de que una encuadernación sea incorrecta es  $0.0001$ ¿Cuál es la
probabilidad de que haya $5$ libros de la tirada mal encuadernados?

\item  Dos compañeros de estudios  se encuentran en un conocido bar de copas de Palma
y deciden jugar a dardos de una manera especial: Lanzarán consecutivamente un dardo cada
hasta que uno de los dos acierte el triple $10$ (centro de la diana).  El que lanza en
primer lugar tiene una probabilidad de $0.7$ de acertar  y el que lo hace en segundo lugar
$0.8$. Sea $X$ la variable aleatoria que da el número total de lanzamientos de dardos
hechos por los dos compañeros.

\item  Un examen tipo test  consta de $5$ preguntas, cada una  con tres opciones de
respuesta, solo hay una opción correcta. Un estudiante contesta al azar  a las $5$
cuestiones. Sea $X$ la variable aleatoria que da el número de puntos obtenidos por el
alumno.
\begin{enumerate}[i)]
\item Si les respuestas erróneas  no restan puntos.
\item Si cada respuesta errónea resta 1 punto.
\end{enumerate}
\end{enumerate}



\probl  Un coche tiene que  pasar por cuatro semáforos. En cada uno de ellos el coche tiene
la misma probabilidad de seguir su marcha que de detenerse. Hallar la función de
distribución del número de semáforos que pasa el coche sin detenerse.


\probl Un individuo quiere invertir un capital de medio millón de euros en un negocio que
tiene una rentabilidad del $50\%$, pero con el riesgo de perder toda la inversión. Su
asesor financiero le informa que este negocio tiene una probabilidad de ser rentable del
$0.8$ ¿Cuál es el beneficio esperado? $\mathbf{(100000)}$

\probl Un juego se dice justo si la ganancia esperada de cada jugador es $0$. Dos jugadores
A y  B tiran un dado por turnos, y gana el primero que obtiene un $5$. Cada jugador apuesta
una cantidad $ c_j \ (j=1,2),  $  y el total se lo queda el ganador. Si suponemos que
comienza  a jugar A ¿qué relación tienen que verificar $c_1$ y $c_2$ para que el juego sea
justo?


\probl  (\textbf{Opcional}) \textbf{Problema de la ruina del jugador} Supongamos que jugamos a la ruleta en un casino. Sea $p<1/2$ la probabilidad de que
salga un número rojo. Supongamos que apostamos a la par (lo que quiere decir, que si
apostamos $k$ dólares, cuando sale rojo nos dan $k$ dólares más los que hemos entregado al
apostar y perdemos los $k$ dólares que hemos apostado si no sale rojo). La primera apuesta es de 1 dólar. Si
ganamos, nos retiramos. Si perdemos, hacemos una segunda apuesta de $2$ dólares. Si
ganamos, nos retiramos. Si perdemos, apostamos $2^2$ dólares,  y así sucesivamente. La
$n$-ésima apuesta ser? de $2^{n-1}$ dólares.

\begin{enumerate}[a)]
\item Probar  que con este sistema es seguro que ganamos 1 dólar.
\item Calcular el importe esperado de la apuesta ganadora. $\mathbf{(\infty)}$
\end{enumerate}


Supongamos ahora que la casa tiene un límite de $2^L$ dólares (el máximo que está permitido
apostar), de manera que, si no hemos ganado antes, esa ser? nuestra apuesta final.

\begin{enumerate}[a)]
\item  ¿Cuál es la ganancia esperado cuando paramos de jugar?
$\mathbf{(1-(2(1-p))^{L+1})}$
\item ¿Qué? es mejor un límite alto o bajo?  \textbf{(bajo)}
\end{enumerate}

\probl Se venden 50000 números de lotería a 10 euros cada uno, para  un sorteo con un
premio de  300000 euros. ¿Cuál es la ganancia (pérdida) esperada de una persona que compra
tres billetes? $\mathbf{(-12\quad \textrm{euros})}$


\probl (\textbf{Opcional}) Un sistema de transmisión emite los dígitos -1, 0, 1. Cuando se transmite el símbolo
{\it i}, se recibe el símbolo {\it j} con las probabilidades siguientes: $ \pr{r_1 / t_1} =
1, \ \pr{r_{-1} / t_{-1}} = 1, \ \pr{r_1 / t_0} = 0.1, \ \pr{r_{-1} / t_0} = 0.1, \ \pr{r_0
/ t_0} = 0.8. $ Se dice que en este caso se ha producido una distorsión $ \displaystyle
(i-j)^2. $ ¿Cuál es el valor medio de la distorsión? $\mathbf{(1/15)}$

%%%%va la 13
\probl (\textbf{Opcional}) Una fuente  binaria emite de manera equiprobable e independiente un bloque  de 3
dígitos (0 ó 1) cada segundo. De cada bloque se envía a una canal de transmisión un 0 si en
el bloque hay más ceros que unos y un 1 en caso contrario. El canal transmite el dígito con
una probabilidad de error $p$. El receptor reconstruye la terna de dígitos repitiendo tres
veces el dígito que ha recibido ¿Cuál es el número medio de bits erróneos por bloque?
$\mathbf{(\frac{3}{4}+\frac{3\cdot p}{2})}$ ¿Cuál tendría que ser la probabilidad $p$ para
que este valor medio no fuera más grande que 1?  $\mathbf{(p \leq 1/6)}$

\probl (\textbf{Opcional})  Dos personas juegan a cara o cruz, y han decidido continuar la partida hasta que se
obtengan como mínimo $3$ caras  y $3$ cruces. Hallar la probabilidad de que el juego no se
acabe en $10$ tiradas y el número esperado de tiradas.
\sol{$\mathbf{\frac{7}{64}=0.109375}$; $\mathbf{E(X)=\frac{63}{8}}$.}

%
%%%\probl  Calcular la esperanza y la varianza del número de puntos obtenidos en el
%%%lanzamiento de un dado. \sol{$\mathbf{(7/2,\quad 35/12)}$}



\probl (\textbf{Opcional}) Es un buen ejercicio calcular la esperanza y la varianza de todas las
variables que aparecen en el problema 35 (Algunas son difíciles).

%%%%%ggggggggg
%\headline{\hfill \bf Fulla 6}

% {\bf PROBABILITATS I PROCESSOS ALEATORIS. 2on TELEM\`ATICA}
% \vskip 1 cm
%
% \magnification \magstep 1

\begin{center}
\textbf{Variables aleatorias continuas}
\end{center}

\probl  Sea  $X$ una variable aleatoria continua con función de densidad $f(x)$ dada por:
$$f(x) = \cases{ k \cdot (1+x^2) & si $x \in (0,3)$\cr 0 & si $ x \not \in (0,3)$\cr}$$
\begin{enumerate}[a)]
\item Calcular la constante $k$  y la función de distribución de $X$.
\item Calcular la probabilidad de que $X$ esté comprendida entre $1$ y $2$
\item Calcular la probabilidad de  que $X $ sea menor que $1.$
\item Sabiendo que $X$ es mayor que $1$, calcular la probabilidad de
que sea menor que $2$. \sol{a) $\mathbf{k=1/12}$; b) $\mathbf{5/18}$; c) $\mathbf{1/9}$; e) $\mathbf{5/16}$}
\end{enumerate}

\probl  La función de densidad de una variable aleatoria continua es: $$f(x) = \cases{ a
\cdot x^2 + b & si $ x \in (0,3)$\cr 0 & si $ x \not \in (0,3)$\cr}.$$

 Determinar $a$ y $b$, sabiendo que
$P(1 < X \leq 2) = 2/3$.  \sol{$\mathbf{a=-1/2, b=11/6}$}

\probl  (\textbf{Opcional, es una integral rara.})
La duración en minutos de unas ciertas comunicaciones telefónicas es una variable aleatoria con función de distribución:
$$F(x) = \cases{ 0 & si $ x \leq 0$\cr 1-{1 \over 2} e^{-x/3} - {1 \over 2} e^{-R[x/3]} &
si $ x > 0$\cr}$$ donde $ R[x] $ es la parte entera de $x.$ Calcular la probabilidad de la
comunicación  dure:
\begin{enumerate}[a)]
\item Más de $6$ minutos.
\item Menos de $4$ minutos.
\item Exactamente $3$ minutos.
\item Menos de $9$ minutos, sabiendo que ha durado más de $5$.
\item Más de $5$ minutos, sabiendo que ha durado menos de $9$.\sol{a)$\mathbf{e^{-2}}$;
b)$\mathbf{1-{1 \over 2} e^{-4/3} -{1 \over 2} e^{-1}}$  ; c)
$\mathbf{{1 \over 2}(1-e^{-1})}$ }
\end{enumerate}

\probl  Sea $X$ una variable aleatoria continua con densidad:

$$f(x) = \cases{1-|x| & si $ |x| \leq 1$\cr 0 & en caso
contrario\cr}$$
\begin{enumerate}[a)]
\item  Encontrar  la función de distribución de $ X.$
\item Calcular $ \pr{X \geq 0}$  y $\pr{|X| < 1/2}.$  \sol{b) $\mathbf{1/2,
3/4}$}
\end{enumerate}




% \headline{\hfill \bf Fulla 7}
%
% {\bf PROBABILITATS I PROCESSOS ALEATORIS. 2on TELEM\`ATICA}
%
% \vskip 1 cm
% \magnification \magstep 1


\probl  Se llama {\bf distribución triangular} a cualquier distribución continua tal que su densidad es cero salvo en un cierto intervalo  $(a,b), $ en el que su gráfica tiene forma de triángulo isósceles. Hallar la función de densidad y de distribución de una distribución
triangular.



%%%%%%%%distribuciones notables

%%%%%%pppppppp


\probl Sea $X$ una variable aleatoria  continua con densidad (Laplaciana)

$$f(x) = \frac{1}{2} \cdot e^{-|x|} \mbox{ si } x\in\RR$$

\begin{enumerate}[a)]
\item Calcular $ \pr{|X|> 2}.$  \sol{$\mathbf{e^{-2}}$}
\item Calcular $E(X)$ y $Var(X)$. \sol{$E(X)=0$, $Var(X)=2.$}
\end{enumerate}

\probl  Consideremos $f:{\R}\to {\R}$ dada por

$$f(x) = \cases{ 0 & si $ x \leq 0$\cr a (1+x) & si $ 0 < x \leq
1$\cr 2/3 & si $ 1 < x \leq 2$\cr 0 & si $ x > 2$\cr}$$
\begin{enumerate}[a)]
\item  Determinar el valor de $a$ para que $f$  sea una densidad.
\item  En este caso, si $X$ es una variable aleatoria continua con densidad $f$, calcular $ \pr{1/2 < X\leq 3/2}$.
\item Calcular, para el valor de $a$ encontrado $E(X)$ y $Var(X)$.
\sol{a)$\mathbf{a=2/9}$ ; b) $\mathbf{19/36}$; c) $E(X)=\frac{32}{27}$, $Var(X)=\frac{409}{1458}$.}
\end{enumerate}

% \nopagenumbers
%
% \parindent = 0 pt
%
% \headline{\hfill \bf Full 8}
%
% {\bf PROBABILITATS I PROCESSOS ALEATORIS. 2on TELEM\`ATICA}
%
% \vskip 1 cm
% \magnification \magstep 1

\newpage

\begin{center}
\textbf{Transformación de variables aleatorias}
\end{center}

\probl (\textbf{No es difícil!!!!, haced los casos.})Sea $X$ la variable que nos da la puntuación obtenida al lanzar un dado. Calcular la
distribución de las variables $Y=X^2$, $Z=X^2-6\cdot X+6$. Calcular las esperanzas y las
varianzas de las variables $Y$ y $Z$.

\probl (\textbf{Opcional}) Conocida  la función de distribución de una variable aleatoria continua $X$, hallar
la función de densidad de $\displaystyle Y = X^2$ y de $ Z = e^X.$

\probl  La función de distribución de una variable aleatoria  $X$ es: $$F_{X}(x) = \cases{
0 & si $ x \leq 0$\cr 1-e^{-x^2} & si $ x
> 0$\cr}$$ Encontrar la función de densidad de la variable aleatoria
$Y = \ln(X+1).$


\probl (\textbf{Tiene algo de dificultad.}) La función de densidad de una variable aleatoria $X$ es:

$$f_{X}(x) = \cases{ x+1 & si $ x \in (-1,0]$\cr -x+1 & si $ x \in
(0,1]$\cr 0 & si $x \in (-\infty,-1] \cup (1,\infty)$\cr}$$

Definimos la variable aleatoria $Y=g(X)$,  donde $g$ es la función

$$g(x) = \cases{ 1 & si $ x \in (1/2,\infty)$\cr 0 & si $ x \in
(-1/2,1/2]$\cr -1 & si $x \in (-\infty,-1/2]$\cr}$$

Determinar la función  de probabilidad y la de distribución de $Y.$ Calcular las
esperanzas y varianzas de $X$ e $Y$.




\probl (\textbf{Opcional, difícil!!!!!}) El precio por estacionar un vehículo en un aparcamiento es de $75$ céntimos de euro por  la
primera hora o fracción, y de $60$ ceńtimos más a partir de la segunda hora o fracción. Supongamos
que el tiempo, en horas, que un vehículo cualquiera permanece en el aparcamiento se
modeliza según la siguiente función de densidad

$$f_{X}(x) = \cases{e^{-x} & si $ x \geq 0$\cr 0 & si $ x <
0$\cr}.$$

Calcular  el ingreso medio por vehículo. $\mathbf{(109.919)}$







%%%\probl Calculau l'esperan\c{c}a i la vari\`ancia del nombre de punts
%%%obtingut en el llan\c{c}ament d'un dau. $\mathbf{(21/6,\quad 35/12)}$





\probl Calcular $E(X)$ y $Var(X)$ para una v.a. $X$ que tiene por  función de densidad $f_X$
dada por:
$$f_X(x) = \cases{{1\over 2\sqrt{x}} & si $ x \in (0,1)$\cr 0 & si $ x \not \in
(0,1)$\cr}$$ $\mathbf{(1/3,\quad 4/45)}$


\probl Consideremos una variable aleatoria $X$ con función de densidad:
$$f_X(x) = \cases{1/2 & si $ 0<x<2$\cr 0 & en caso contrario\cr}$$ determinar
$E(Y)$, donde $Y = \ln( X).$ $\mathbf{(-0.3069)}$


\probl Sea $X$ una variable aleatoria con función de densidad:

$$f_{X}(x) = \cases{2 x & si $x \in (0,1)$\cr 0 & si $x \not \in
(0,1)$\cr}$$
\begin{enumerate}[a)]
\item Determinar $E(\sqrt{X})$ a partir de $f_{\sqrt{X}}.$
$\mathbf{(4/5)}$
\item Hacer lo mismo a partir de $f_{X}.$
\end{enumerate}


%%%\probl Un sistema de transmissi? emet els d?gits -1, 0, 1. Quan se
%%%transmet el s?mbol {\it i}, se rep el s?mbol {\it j} amb les
%%%seg?ents probabilitats: $ \pr{r_1 / t_1} = 1, \ \pr{r_{-1} /
%%%t_{-1}} = 1, \ \pr{r_1 / t_0} = 0.1, \ \pr{r_{-1} / t_0} = 0.1, \
%%%\pr{r_0 / t_0} = 0.8.  $ Se diu en aquest cas que s'ha produ?t una
%%%distorsi? $ \displaystyle (i-j)^2.  $ Quin ?s el valor mitj\`a de la
%%%distorsi?? $\mathbf{(1/15)}$

%%%%va la 13
%%%\probl Una font bin\`aria emet de manera equiprobable i
%%%independentment un bloc de 3 d?gits (0 o 1) cada segon. De cada
%%%bloc envia a un canal de transmissi? un 0 si al bloc hi ha m?s 0's
%%%que 1's, i un 1 en cas contrari. El canal transmet el d?git amb
%%%una probabilitat d'error {\it p}. El receptor reconstrueix la
%%%terna de d?gits repetint tres vegades el d?git que ha rebut. Quin
%%%?s el nombre mitj\`a de bits erronis per bloc?
%%%$\mathbf{(\frac{3}{4}+\frac{3\cdot p}{2})}$ Quina hauria de ser la
%%%probabilitat {\it p} per tal que aquest valor mitj\`a no fos m?s
%%%gran que 1? $\mathbf{(p \leq 1/6)}$




\begin{centerline}
{{\bf DISTRIBUCIONES NOTABLES}}
\end{centerline}
 %\textbf{Distribuciones notables}

%  
% \begin{resolucion}
%  
%  Sea $X$ el número de caras esn $n$ lanzamientos indpendientes de una moneda conprobabilidad de cara $p$.
%  La v.a. sigue una ley $B(n,\frac{1}{2})$. Sabemos que $E(X)=\frac{n}{2}$ y que $Var(X)=\frac{n}{4}$.
%  La frecuencia realtiva de caras en los $n$ lanzamientos  ser? $\frac{X}{n}$
% Nos piden que calculemos mediante la desigualdad de Chebychef el menor valor de $n$ tal que:

% $$P(0.45\leq \frac{X}{n}\leq 0.55)>0.9$$


% $P(0.45\leq \frac{X}{n}\leq 0.55)= P(n\cdot 0.45\leq X \leq n\cdot 0.55)=
% P(n\cdot 0.45- \frac{n}{2}\leq X - \frac{n}{2} \leq n\cdot 0.55- \frac{n}{2})= 
% P(n\cdot 0.45- \frac{n}{2}\leq X - \frac{n}{2} \leq n\cdot 0.55- \frac{n}{2})= $
%\end{resolucion}


%%%
%%%\end{document}
%%%% {\bf PROBABILITATS I PROCESSOS ALEATORIS. 2on TELEM\`ATICA}
%%%\newpage
%%%%\problemes=0
%%%\begin{centerline}
%%%{{\bf MOMENTS D'UNA V.A.}}
%%%\end{centerline}


%%%\probl  Un individu vol invertir un capital de mig mili? de pts.
%%%en un negoci que t? una rendibilitat del $50\%$, per\`o amb el
%%%possible risc de perdre tota la inversi?. El seu assessor financer
%%%l'ha informat que aquest negoci t? una probabilitat 0.8 de ser
%%%rendible. Quin ?s el benefici esperat? \sol{$\mathbf{(100.000)}$}
%%%
%%%\probl  Un joc se diu que ?s just si el guany esperat de cada
%%%jugador ?s 0. Dos jugadors A i B tiren un dau per torns, i guanya
%%%el primer que obt? un 5. Cada jugador aposta una quantitat $ c_j \
%%%(j=1,2),  $ i el total se'l quedar\`a el guanyador. Si suposam que
%%%comen\c{c}a a jugar A, quina condici? han de verificar $ c_1$ i $c_2 $
%%%perqu\`e el joc sigui just? \sol{$\mathbf{(6 c_2 = 5 c_1)}$}
%%%
%%% \probl   El preu per estacionament en un aparcament ?s
%%%de 75 pts. per a la primera hora o fracci?, i de 60 pts. a partir
%%%de la segona hora o fracci?. Suposem que el temps, en hores, que
%%%un vehicle qualsevol roman a l'aparcament se modelitza segons la
%%%funci? de densitat
%%%
%%%$$f_{X}(x) = \cases{e^{-x} & si $ x \geq 0$\cr 0 & si $ x <
%%%0$\cr}$$
%%%
%%%Calculau l'ingr?s mitj\`a per vehicle. \sol{$\mathbf{(109.8)}$}
%%%
%%%\probl   Sigui $ X  $ una variable aleat\`oria amb funci? de
%%%densitat
%%%
%%%$$f(x) = \cases{0 & si $ x \leq 0$\cr 2(1+x)/9 & si $ 0 < x \leq
%%%1$\cr 2/3 & si $ 1 < x \leq 2$\cr 0 & si $ x > 2$\cr}$$
%%%
%%%Calculau $\mathrm{E}(X )$ i  $\mathrm{Var}(X)$.
%%%\sol{$\mathbf{(32/27,\quad 0.28)}$}
%%%
%%%\probl   Suposem que jugam a la ruleta. Sigui $p<1/2$ la
%%%probabilitat que surti un número vermell. Suposem que apostam a la
%%%par (vol dir que, si apostam $k$ d\`ol.lars, mos donen $k$ d\`ol.lars
%%%m?s els que hem apostat si surt vermell i perdem els $k$ d\`ol.lars
%%%que hem apostat si surt negre). La primera aposta ?s de 1 d\`ol.lar.
%%%Si guanyam, mos retiram. Si perdem, feim una segona aposta de 2
%%%d\`ol.lars. Si guanyam, mos retiram. Si perdem, apostam $2^2$
%%%d\`ol.lars, i aix? successivament. La $n$-\`essima aposta ser\`a de
%%%$2^{n-1}$ d\`ol.lars.
%%%
%%%\begin{enumerate}[a)]
%%%\item Provau que amb aquest sistema ?s segur que guanyarem 1 d\`ol.lar.
%%%\item Trobau l'import esperat de l'aposta guanyadora.
%%%\end{enumerate}
%%%
%%%
%%%Suposem ara que la casa t? un l?mit de $2^L$ d\`ol.lars (el m\`axim
%%%que permet apostar), de manera que, si no hem guanyat abans,
%%%aquesta ser\`a la nostra aposta final.
%%%
%%%\begin{enumerate}[a)]
%%%\item  Quin ?s el guany esperat quan mos aturem de jugar?
%%%
%%%\item Qu\`e ?s millor un l?mit alt o baix?  \sol{1b)
%%%$\mathbf{(\infty)}$; 2a)$\mathbf{(1-(2(1-p))^{L+1})}$; 2b)baix }
%%%\end{enumerate}
%%%%%%%%va la 11
%%%
%%%\probl  El temps de vida, en anys, d'un cert component d'una
%%%m\`aquina se modelitza mitjan\c{c}ant la seg\"{u}ent funci? de densitat
%%%
%%%$$f_{X}(x) = \cases{e^{-x} & si $ x \geq 0$\cr 0 & si $ x <
%%%0$\cr}$$
%%%
%%% Si el cost $ Y  $ de funcionament del component, en milions de
%%%pts., ?s funci? del temps de vida, $ Y = 2 X^2 + 1,  $ calculau la
%%%quantitat que espera gastar l'empresa en concepte de manteniment.
%%%\sol{$\mathbf{(5)}$}
%%%
%%%%%%%%%%%\probl  Calculau l'esperan\c{c}a i la vari\`ancia de les variables
%%%%%%%%%%%aleat\`ories estudiades en el tema 2.
%%%
%%%\probl  Una dona a qui robaren la seva bossa va descriure el seu
%%%assaltant com un homo de 2m 10cm d'alt, amb un capell color
%%%taronja, camisa vermella, cal\c{c}ons verds i sabates grogues. Al cap
%%%d'una estona, una persona que responia a aquesta descripci? va
%%%esser detinguda a uns quants blocs de dist\`ancia, i se la va acusar
%%%del robatori. Creis que hi ha prou proves perqu\`e sigui declarat
%%%culpable? Per qu\`e?\sol{No}

\probl   Sea $X$ el número de éxitos en $n$ repeticiones independientes de un experimento
con probabilidad de éxito $p$.
\begin{enumerate}[a)]
\item  Si $k$ es el valor más probable de $X$, probar que
 $$(n+1)p - 1 \leq k \leq (n+1)p$$
\item  Si lanzamos $10$ veces un dado bien balanceado, ¿Cuál es el número de veces más probable
en el que obtendremos un $2$.\sol{b) $\mathbf{(1)}$}
\end{enumerate}

\probl  Una cadena de producción da salida a $10000$ unidades diarias,  el número medio de
unidades incorrectas es $200$. Una vez al día, se inspecciona un lote de $100$ unidades.
Determinar la probabilidad de que el lote contenga más de $3$ unidades incorrectas
\begin{enumerate}[a)]
\item Utilizando la distribución binomial.
\item Utilizando la aproximación de Poisson.
\sol{a) $\mathbf{(0.1410)}$ ; b)  $\mathbf{(0.1429)}$}
\end{enumerate}

\probl  En una planta de fabricación de circuitos integrados, la proporción de circuitos
defectuosos es $p$. Supongamos que la incidencia de circuitos defectuosos es completamente
aleatoria.
\begin{enumerate}[a)]
\item  Determinar la distribución del número $X$ de circuitos
aceptables producidos antes del primer circuito defectuoso.
\item  ¿Cuál es la longitud media de una cadena de producción exitosa? si
$p = 0.05.$\sol{b) $\mathbf{(19)}$}
\end{enumerate}

\probl Un servidor de mensajería esta en funcionamiento. Los clientes acceden a él de forma
independiente. La probabilidad de que el servidor caiga cuando accede el cliente es $p$.
Calcular la distribución de probabilidad del número de clientes a los que se dará servicio
antes de que el servidor caiga.

\begin{enumerate}[a)]
\item Calcular el valor esperado y la varianza de esta variable.
\item  ¿Cuál es la probabilidad de que atienda a más de $1000$ clientes sin que se
caiga el servidor?
\end{enumerate}
\sol{a) $\mathbf{(\frac{q}{p};\, \frac{q}p^2)}$ b) $\mathbf{(q^{1001})}$}

\probl Un sistema informático  dispone de un sistema de seguridad compuesto por tres claves
de $3$ dígitos (del $0$ al $9$) cada una. Para entrar en el sistema hay que averiguar la
primera clave, luego la segunda y por último la tercera. Un pirata informático intenta
entrar ilegalmente en el sistema, para ello va introduciendo al azar distintas claves de
forma independiente, olvidando las que ha introducido antes. Calcular el valor esperado y
la varianza del número de intentos antes de romper el sistema.

Comparar el resultado anterior cuando se ataca el sistema de forma similar pero cuando el
sistema de seguridad solo consta de una clave de $9$ dígitos. ¿Cuál es el sistema más
seguro, desde el punto de vista del número de intentos necesarios para violarlo?




\probl De un grupo de $10$ personas se eligen $5$ de forma equiprobable y se les pregunta
si están a favor de una cierta ley. Sea $X$ la variable aleatoria que cuenta el número de
personas a favor de la ley entre las $5$. Supongamos que hay una persona a favor por cada
nueve en contra.

\begin{enumerate}[a)]
\item Calcular la función de probabilidad de $X$.
\item Calcular la esperanza y la varianza de $X$.
\item Supongamos ahora que la población es de tamaño $10000$ ¿podemos suponer un modelo
binomial para $X$?, en este caso ¿varía mucho la esperanza y la varianza con respecto al
modelo hipergeométrico?
\end{enumerate}




\probl (\textbf{Pensad bien y os saldrá, es fácil}) Los taxis llegan aleatoriamente (según un proceso Poisson) a la terminal de un
aeropuerto con un ritmo medio de un taxi cada $3$ minutos. ¿Cuál es la probabilidad de que
el último pasajero de una cola de $4$ tenga que esperar un taxi más de un cuarto de hora?
\sol{$\mathbf{(0.265)}$}


%%%%%%%%%%va la 12



\probl  La variable aleatoria $X$ sigue una ley  $\displaystyle N(\mu, \sigma^2). $ Sabemos
que   $ \mu = 5  \sigma, $ y  que $ \pr{X < 6} = 0.8413.$

\begin{enumerate}[a)]
\item Determinar la esperanza y la varianza de $X$.
\item ¿Cuál es la función de distribución de $ Y = 3
- X^2  $ y su esperanza? \sol{a)$\mathbf{(5,\quad 1)}$ ; b) $\mathbf{E(Y) = -23}$}
\end{enumerate}

%%%\probl  Sigui $ X  $ una variable aleat\`oria amb funci? de
%%%densitat:
%%%
%%%$$f_{X}(x) = \cases{2x & si $x \in (0,1)$\cr 0 & si $x \not \in
%%%(0,1)$\cr}$$
%%%\begin{enumerate}[a)]
%%%\item Determinau $\mathrm{E}(\sqrt{X})$ a partir de $f_{\sqrt{X}}.$
%%%\sol{a) $\mathbf{(4/5)}$}
%%%\item El mateix a partir de $f_{X}.$
%%%\end{enumerate}




\probl  Consideremos una variable aleatoria $X$ con función de densidad $f_X$ dada por:
$$f_X(x) = \left\{\begin{array}{ll}
1/2 & \mbox{si }  0<x<2\\ 0 & \mbox{en otro caso}\end{array}\right.$$
 determinar
$\mathrm{E}( Y)$, donde $Y= \ln X.$ \sol{$\mathbf{(-0.3069)}$}

\probl  (\textbf{Opcional}) Sea $X$ una variable aleatoria continua con distribución uniforme sobre el
intervalo  $(0,1).$
\begin{enumerate}[a)]
\item Encontrar las funciones de densidad de las variables aleatorias
 $Y
= g(X)$ y $Z = h(X),$ donde $g(x) = 8 x^3$ y $h(x) = (x-1/2)^2.$
\item  Calcular la esperanza de las variables aleatorias $Y$ y
$Z$ como esperanzas de v.a. y como esperanzas de funciones de la v.a. $X$. \sol{b)
$\mathbf{(2,\quad 1/12)}$}
 \end{enumerate}





\probl  El cuantil $0.9$ de una variable aleatoria $X$ es el valor $x_{0.9}$ para el que $
\displaystyle F_{X}(x_{0.9}) = \pr{X \leq x_{0.9}} = 0.9. $ De manera similar, el percentil
$0.5$ es el valor $x_{0.5}$ que satisface $ \pr{X \leq x_{0.5}} = 0.5 $ que recibe el nombre
de  {\bf mediana} (poblacional). Determinar estos dos valores para  una variable aleatoria
exponencial de valor medio $10$. \sol{$\mathbf{(23.0585,\quad 6.93147)}$}



\end{document}
