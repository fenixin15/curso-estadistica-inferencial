\documentclass[12pt]{article}
\usepackage{enumerate}
\usepackage[T1]{fontenc}
 %\input{8bitdefs}
% \textwidth 15cm
% \textheight 22cm
 \setlength{\textwidth}{16.5cm}
\setlength{\textheight}{24cm}
 \setlength{\oddsidemargin}{-0.3cm}
 \setlength{\evensidemargin}{1cm} \addtolength{\headheight}{\baselineskip}
\addtolength{\topmargin}{-3cm}
\newcommand{\RR}{\mbox{I\kern-.2em\hbox{R}}}
\def\N{I\!\!N}
\def\R{I\!\!R}
\def\Z{Z\!\!\!Z}
\def\Q{O\!\!\!\!Q}
\def\C{I\!\!\!\!C}

% \def\Q{{\mathchoice {\setbox0=\hbox{$\displaystyle\rm
% Q$}\hbox{\raise
% 0.15\ht0\hbox to0pt{\kern0.4\wd0\vrule height0.8\ht0\hss}\box0}}
% {\setbox0=\hbox{$\textstyle\rm Q$}\hbox{\raise
% 0.15\ht0\hbox to0pt{\kern0.4\wd0\vrule height0.8\ht0\hss}\box0}}
% {\setbox0=\hbox{$\scriptstyle\rm Q$}\hbox{\raise
% 0.15\ht0\hbox to0pt{\kern0.4\wd0\vrule height0.7\ht0\hss}\box0}}
% {\setbox0=\hbox{$\scriptscriptstyle\rm Q$}\hbox{\raise
% 0.15\ht0\hbox to0pt{\kern0.4\wd0\vrule height0.7\ht0\hss}\box0}}}}

\begin{document}

%\pagestyle{empty}
\font\sc=cmcsc10
\parskip=1ex
%%\newcount\problemes
\newcounter{problemes}
\setcounter{problemes}{0}

%%%\newcount\problemes
%%%\problemes=0

%%%\newenvironment{prob}{\vskip 0.25cm\addtocounter{problemes}{1}
%%%\noindent{\textbf{\thebloque.\theproblemes.- }}}
\newenvironment{prob}{\vskip 0.25cm\addtocounter{problemes}{1}
\noindent{\textbf{\theproblemes.- }}}
%\newenvironment{resolucion}{\hfill $\bullet$}


\newcommand{\sol}[1]{{\textbf{\footnotetext[\theproblemes]{Sol.: #1} }}}

\newcommand{\probl}{\vskip 0.25cm\addtocounter{problemes}{1}
\noindent{\textbf{\theproblemes.- }}}

%\newcommand{\sol}[1]{{\textbf{\footnotetext[\the\problemes]{Sol.: #1} }}}
%%%
%%%\newcommand{\probl}{\addtocounter{problemes}{1} \vskip 2ex \noindent
%%%{\bf \the\problemes)}}
%%%
%%%\def\probl{\advance\problemes by 1
%%%\vskip 1mm\noindent{\bf \the\problemes) }}
%%%\newcounter{pepe}



%%%%%PROBABILIDAD
\newpage
\vskip 0.5cm
%%%\newcount\problemes
%%%\problemes=0

%\chapter*{\large \textbf{BLOQUE 1 ESTADÍSTICA DESCRIPTIVA (resumen)}}
\begin{centerline}
{{\bf ESTAD\'ISTICA DESCRIPTIVA UNIVARIANTE}}
\end{centerline}


\begin{prob}
Una cadena de franquicias de restaurantes de 1991 a 1995 tuvo 4, 8, 16, 26 y 82
restaurantes  respectivamente. Un modelo estadístico estima que el número de
restaurantes entre 1991 y 1997 es 4, 8, 16, 33, 140, 280, y 586. Representar los datos en
un diagrama de barras combinado.
\end{prob}


\begin{prob}
Determinar la media (aritmética) la mediana y la moda del siguiente conjunto de datos:
\begin{center}
\begin{tabular}{rrrrrrrrrr}
3,&5,&7,&8,&8,&8,&10,&11,&12,&12\\ 13,&14,&14,&15,&16,&18,&19,&21,&23,&25
\end{tabular}
\end{center}
\end{prob}

\begin{prob}
Estimar la media la mediana y la moda de los siguientes datos continuos agrupados en
intervalos:

\begin{center}
\begin{tabular}{l|c}
Intervalo de clase & Frecuencia\\ \hline de 0 a 9 & 50 \\ de 10 a 19 & 150\\ de 20 a 29 &
100\\ de 30 a 40 & 50\\ \hline
\end{tabular}
\end{center}
\end{prob}

\begin{prob}
\'Idem que en el anterior para la tabla:


\begin{center}
\begin{tabular}{l|c}
Intervalo de clase & Frecuencia\\ \hline $\left[ -0.5,9.5 \right)$ & 50 \\ $\left[
9.5,19.5 \right)$ & 150\\ $\left[ 19.5,29.5 \right)$ & 100\\ $\left[ 29.5,40.5 \right)$ &
50\\ \hline
\end{tabular}
\end{center}
\end{prob}


\begin{prob}
?`Son diferentes los resultados de los dos ejercicios anteriores, por qué?
\end{prob}


\begin{prob}
Discutir el significado de las distintas medidas de centralización en cada uno de los
casos siguientes:
\begin{enumerate}[a)]
\item Datos sobre el número de empleados anuales de una
compañía fundada en $1920$.
\item Datos de localidades preferidas por las grandes empresas para
celebrar sus reuniones anuales.
\item Datos sobre el coste medio de una pensión de jubilación.
\item Datos de una muestra de $100$ batidoras domésticas de una
población de $10000$ batidoras.
\end{enumerate}
\end{prob}

\begin{prob}
Una editorial tiene $4000$ títulos en catálogo. Podemos clasificar los distintos libros
en novelas, biografías  y otros tipos de libros de venta más preferente en librerías. De
los siguientes datos de número de copias vendidas, estimar las ventas medias por título.

\begin{center}
\begin{tabular}{l|c}
{\parbox{2cm} {Intervalo de unidades vendidas}} & Frecuencia \\ \hline 0-999 & 500 \\
1000-4999 & 800 \\ 5000-24999 & 700 \\ 25000- 49999 & 1500 \\ 50000 o más & 500\\
\hline
\end{tabular}
\end{center}
\end{prob}

\begin{prob}
Con los datos del ejercicio anterior. Si el título de mayores ventas
 vendió 1000000 de copias en el año en que se recogieron los
datos ?`cuál es la desviación estándar estimada para las ventas por título?
\end{prob}


\begin{prob}A 50 aspirantes a un determinado puesto de trabajo se
les sometió a una prueba. Las puntuaciones obtenidas fueron:
\begin{center}
\begin{tabular}{cccccccccc}
4 & 4 & 2 & 10 & 1 & 9 & 5 & 3 & 4 & 5 \\ 6 & 6 & 7 & 6 & 8 & 7 & 6 & 8 & 7 & 6 \\ 5 & 4
& 4 & 4 & 5 & 6 & 6 & 7 & 5 & 6 \\ 6 & 7 & 5 & 6 & 6 & 7 & 5 & 6 & 4 & 3 \\ 2 & 6 & 6 & 7
& 7 & 8 & 8 & 9 & 8 & 7
\end{tabular}
\end{center}
\begin{enumerate}[a)]
\item {Construir la tabla de frecuencias y la representación
gráfica correspondiente.}
\item {Encontrar la puntuación que seleccione al 20\% de los
mejores candidatos.}
\end{enumerate}
\end{prob}
\begin{prob}En la población de estudiantes de la facultad, se
seleccionó  una muestra de 20 alumnos y se obtuvieron  las siguientes tallas en cm.:
$$
\begin{tabular}{ll}
&162,168,174,168,166,170,168,166,170,172,\\ & 188,182,178,180,176, 168,164, 166,164,172
\end{tabular}
$$

Se pide:
\begin{enumerate}[a)]
\item{Descripción numérica y representación gráfica.}
\item {Media aritmética, mediana y moda.}
\end{enumerate}
\end{prob}
\begin{prob}Agrupando los datos del ejercicio anterior en intervalos
de amplitud 10 cm., se pide:
\begin{enumerate}[a)]
\item {Descripción numérica y representación gráfica.}
\item {Media aritmética, mediana  y moda.}
\item {Analizar los cálculos hechos y los errores de agregación
comparándolos con el ejercicio anterior.}
\end{enumerate}
\end{prob}

\begin{prob}Las tres factorías de una industria han producido en
el último año el siguiente  numero de motocicletas por trimestre:

$$
\vbox{\halign{\offinterlineskip\strut\hskip0.25cm # \hskip0.25cm& \hskip0.25cm\hfill #
\hskip0.25cm&\hskip0.25cm\hfill # \hskip0.25cm& \hskip0.25cm\hfill # \hskip0.25cm\cr &
factoría 1 & factoría 2 & factoría 3 \cr $1^{\underline{o}}$. trimestre & 600 & 650  &
550  \cr $2^{\underline{o}}$. trimestre & 750 & 1200 & 900 \cr $3^{\underline{o}}$.
trimestre & 850 &1250 & 1050 \cr $4^{\underline{o}}$. trimestre & 400 & 800 & 650 \cr }}
$$

Obtener:
\begin{enumerate}[a)]
\item{Producción media trimestral de cada factoría y de toda la
industria.}
\item{Producción media diaria de cada factoría y de toda la
industria teniendo  en cuenta que durante el primer trimestre hubieron  68 días
laborables, el segundo, 78, el tercero, 54 y el cuarto, 74.}
\end{enumerate}
\end{prob}

\begin{prob}Una empresa ha pagado por un cierto articulo: 225, 250,
300 y 200 pts. de precio. Determinar el precio medio  en las siguientes hipótesis:
\begin{enumerate}[a)]
\item {Compraba por valor de 38250, 47500, 49500 y 42000 pts.
respectivamente.}
\item {Compraba cada vez un mismo importe global.}
\item {Compraba 174, 186, 192 y 214 unidades respectivamente.}
\end{enumerate}
\end{prob}

\begin{prob}Sobre una muestra de 56 tiendas distintas, se obtuvieron
los siguientes precios de venta de un determinado articulo:

$$
\vbox{\halign{\offinterlineskip\strut\hskip0.25cm # \hskip0.25cm& \hskip0.25cm #
\hskip0.25cm&\hskip0.25cm # \hskip0.25cm& \hskip0.25cm # \hskip0.25cm&\hskip0.25cm #
\hskip0.25cm& \hskip0.25cm # \hskip0.25cm&\hskip0.25cm # \hskip0.25cm\cr
 3260 & 3510 & 3410 & 3180 & 3300 & 3540 & 3320 \cr
3450 & 3840 & 3760 & 3340 & 3260 & 3720 & 3430 \cr 3320 & 3460 & 3600 & 3700 & 3670 &
3610 & 3910 \cr 3610 & 3610 & 3620 & 3150 & 3520 & 3430 & 3330 \cr 3370 & 3620 & 3750 &
3220 & 3400 & 3520 & 3360 \cr 3300 & 3340 & 3410 & 3600 & 3320 & 3670 & 3420 \cr 3320 &
3290 & 3550 & 3750 & 3710 & 3530 & 3500 \cr 3290 & 3410 & 3100 & 3860 & 3560 & 3440 &
3620 \cr }}
$$
Se pide:
\begin{enumerate}[a)]
\item {Agrupar la información en seis intervalos de igual
amplitud y hacer la representación gráfica correspondiente}
\item {Media aritmética y desviación típica}
\item {Desviación media respecto de la media aritmética y la
mediana.}
\end{enumerate}
\end{prob}

\begin{prob} La siguiente  distribución corresponde al capital
pagado por las 420 empresas de la construcción con domicilio fiscal en una región
determinada:
$$\vbox{\halign{\offinterlineskip\strut\hskip0.25cm \hfill #
\hskip0.25cm& \hskip0.25cm \hfill # \hskip0.25cm\cr
 Capital (millones de pts.) & Número de empresas\cr
 menos de 5 &12\cr
 de 5 a 13 &66\cr
 de 13 a 20 &212\cr
 de 20 a 30 &84\cr
 de 30 a 50 &30\cr
 de 50 a 100 &14\cr
 más de 100 &2\cr}}
$$
\begin{enumerate}[a)]
\item{Haciendo  servir como  marcas de clase del primer y
último intervalo 4 y 165 respectivamente, encontrar la media aritmética y la desviación
típica}
\item {Calcular la moda y la mediana}
\item {Estudiar gráficamente su simetría}
\end{enumerate}
\end{prob}

\begin{prob}La distribución de los ingresos en España en el
inicio y final del segundo plan de desarrollo (1967-70) era:
$$\vbox{\halign{\offinterlineskip\strut\hskip0.25cm \hfill #
\hskip0.25cm& \hskip0.25cm \hfill # \hskip0.25cm&\hskip0.25cm \hfill # \hskip0.25cm\cr
 Ingresos medios( en miles de pts)& \%  Hogares 1967& \% Hogares
1970\cr hasta a 60 & 20.02 & 13.87\cr de 60 a 120 & 49.46 & 39.20\cr de 120 a 180 & 17.27
& 24.31\cr de 180 a 240 & 6.48 & 11.44\cr de 240 a 500 & 5.14 & 8.54\cr de 500 a 1000 &
1.46 & 1.42\cr de 1000 a 2000 & 0.88 & 0.80\cr de 2000 a 5000 & 0.21 & 0.30\cr
 más de 5000 & 0.08 & 0.12\cr
}}$$

Utilizando el  $Q_{1}$ (percentil 25) y $Q_{3}$ (percentil 75) como umbrales de pobreza y
riqueza entre los  que se encuentra la clase media de la población y usarlo en la
clasificación siguiente:
$$
\vbox{\halign{\strut\offinterlineskip\strut #&\hskip1cm#&#&#\cr \vbox{\halign{\strut #\cr
Intervalo \cr al que \cr pertenecen\cr}}& clase&&\cr \noalign{\hrule} hasta
$Q_{1}$\hfill & baja&&\cr \vbox{\halign{\strut \offinterlineskip#\cr de $Q_{1}$ a $M_{e}$
(Media)\hfill \cr de  $M_{e}$  a $Q_{3}$\hfill\cr}}&
$
%\left.
\vbox{\halign{\strut \offinterlineskip#\cr media baja\hfill\cr
 media alta\hfill\cr}}
%\right\}
$&&\cr
 más de $Q_{3}$\hfill & alta\hfill&&\cr}}
$$
Discutir la veracidad de los siguientes conclusiones relativas al segundo plan de
desarrollo:
\begin{enumerate}[a)]
\item {La diferencia entre les clase baja y alta aumentó.}
\item  {El recorrido entre las clases media baja  y media  alta
también aumentó, siendo  menor el incremento en el primer caso que en el segundo.}
\end{enumerate}
\end{prob}


\begin{prob}La siguiente tabla muestra la distribución de les
cargas máximas que soportan los hilos producidos en una cierta fábrica:
$$\vbox{\halign{\offinterlineskip\strut
\vrule \hfill#\hfill\vrule&\hfill#\hfill\vrule \cr \noalign{\hrule} Carga
máxima(Tm)&Número de hilos\cr \noalign{\hrule} 9.25-9.75&2\cr 9.75-10.25&5\cr
10.25-10.75&12\cr 10.75-11.25&17\cr 11.25-11.75&14\cr 11.75-12.25&6\cr 12.25-12.75&3\cr
12.75-13.25&1\cr \noalign{\hrule} }}$$ Encontrar la media y la varianza. Dar un intervalo
donde estén  al menos el 90\% de los datos. \
\end{prob}

\begin{prob}Las calificaciones finales  de 20
estudiantes de estadística son:
$$
\begin{array}{ll}&59,
60, 62, 68, 68, 71, 73, 73, 75, 75,\\ & 76, 79, 82, 84, 85, 88, 88, 90, 93, 93.
\end{array}
$$
Hacer la distribución de frecuencias, y los histogramas de frecuencias relativas y
relativas acumuladas en tantos por cien.
\end{prob}


\begin{prob}
La siguiente tabla muestra los precios por persona y noche en hoteles
 y pensiones del área metropolitana de una ciudad española en pts:


\begin{tabular}{ccccccccccccc}
6500 & 3800 & 5450 & 2795 & 2500 & 3200 & 8400 & 4700 & 4500 & 3295 \\ 7000 & 3700 & 6400
& 2600 & 4000 & 4500 & 3400 & 4700 & 6100 & 6600 \\ 4300 & 4600 & 6200 & 5600 & 4700 &
5200 & 2800 & 2800 & 2600 & 3200 \\ 9400 & 4000 & 5700 & 3600 & 3000 & 5400 & 6000 & 2395
& 2395 & 2395 \\ 2395 & 2525 & 5000 & 6500 & 3495 & 6000 & 3200 & 3200 & 2600 & 2500 \\
3300 & 10000
\end{tabular}


\begin{enumerate}[a)]
\item Realizar el diagrama de árbol y hojas
(\textit{stem-and-leaf}) de los datos anteriores (con profundidad), en cientos de pesetas
(despreciando decimales).
\item Calcular la distribución de frecuencias (agrupando de forma
oportuna) de los precios.
\item Dibujar el histograma de frecuencias absolutas y absolutas
acumuladas y sus polígonos asociados.
\item Dibujar el histograma de frecuencias relativas  y relativas
acumuladas y sus polígonos asociados.
\item Dibujar el diagrama de caja asociado a los datos.
\item Dibujar el diagrama de tarta de los precios.
\item Comentar todos los gráficos.
\end{enumerate}
\end{prob}


\begin{prob}Supongamos que seis vendedores necesitan vender un total
de 50 aspiradoras en un mes. El señor A vende 7 durante los primeros 4 días; el señor B
vede 10 durante los siguientes  5 días; el señor C vende 12 durante los siguientes 5
días; el señor D vende  10 durante los siguientes  4 días; el señor E vende 6 durante los
siguientes  3 días y el señor F vende 5 durante los siguientes  3 días. Encontrar el
promedio de aspiradoras vendidas por día.
\end{prob}

\begin{prob}Consideremos la siguiente variable discreta que nos da el
numero de veces que la gente se examina para  aprobar el examen de conducir. Los
resultados, dada  una muestra de 30 aspirantes, son:

$$
\vbox{\halign{\offinterlineskip\strut\hskip0.05cm #\hskip0.05cm& \hskip0.05cm
#\hskip0.05cm&\hskip0.05cm #\hskip0.05cm& \hskip0.05cm #\hskip0.05cm&\hskip0.05cm
#\hskip0.05cm& \hskip0.05cm #\hskip0.05cm&\hskip0.05cm #\hskip0.05cm& \hskip0.05cm
#\hskip0.05cm&\hskip0.05cm #\hskip0.05cm& \hskip0.05cm #\hskip0.05cm\cr 4, & 3, & 1, & 1,
& 1, & 1, & 2, & 2, & 2, & 3, \cr  2, & 1, & 2, & 2, & 1, & 2, & 1, & 1, & 1, & 2, \cr 2,
& 3, & 1, & 3, & 3, & 3, & 2, & 1, & 4, & 1, \cr}}
$$
Encontrar $\overline{X}$ y $s_X^2$.
\end{prob}

\begin{prob}
Los pesos en Kg. de 120  langostas compradas en una pescadería fueron:

\begin{tabular}{ccccccccccc}
  0.66  & 0.47  & 0.58 & 0.68   & 0.56   & 0.52   & 0.54   & 0.59   &
0.56   &
  0.72   & 0.63\\
  0.59  & 0.56  & 0.56 & 0.49   & 0.63   & 0.53   & 0.56   & 0.55   &
0.50   &
  0.75   & 0.56 \\
  0.59  & 0.66  & 0.61 & 0.56   & 0.52   & 0.48   & 0.56   & 0.68   &
0.77   &
  0.59   & 0.53 \\
  0.56  & 0.65  & 0.51 & 0.59   & 0.49   & 0.62   & 0.54   & 0.56   &
0.56   &
  0.61   & 0.50 \\
  0.61  & 0.45  & 0.65 & 0.55   & 0.54   & 0.61   & 0.64   & 0.56   &
0.71   &
  0.59   & 0.56 \\
  0.59  & 0.64  & 0.49 & 0.56   & 0.48   & 0.64   & 0.56   & 0.62   &
0.54   &
  0.53   & 0.55 \\
  0.56  & 0.63  & 0.56 & 0.52   & 0.66   & 0.68   & 0.62   & 0.56   &
0.59   &
  0.54   & 0.50 \\
  0.56  & 0.62  & 0.49 & 0.56   & 0.64   & 0.60   & 0.53   & 0.55   &
0.64   &
  0.59   & 0.60 \\
  0.52  & 0.56  & 0.66 & 0.54   & 0.68   & 0.59   & 0.56   & 0.48   &
0.54   &
  0.56   & 0.67 \\
  0.63  & 0.46  & 0.48 & 0.68   & 0.61   & 0.56   & 0.54   & 0.49   &
0.65   &
  0.56   & 0.61 \\
  0.45  & 0.73  & 0.60 & 0.68   & 0.65   & 0.56   & 0.54   & 0.55   &
0.60   & 0.60
\end{tabular}

\begin{enumerate}[a)]
\item Realizar el diagrama de tallo y hojas
(\textit{stem-and-leaf}) de los datos anteriores (con profundidad), gramos.
\item Calcular la distribución de frecuencias (agrupando de forma
oportuna) de los pesos.
\item Dibujar el histograma de frecuencias absolutas y absolutas
acumuladas y sus polígonos asociados.
\item Dibujar el histograma de frecuencias relativas  y relativas
acumuladas y sus polígonos asociados.
\item Dibujar el diagrama de caja asociado a los datos.
\item Dibujar el diagrama de tarta de los precios.
\item Comentar todos los gráficos.
\end{enumerate}
\end{prob}


\begin{prob}
Los siguientes pesos en Kg. corresponden a langostas  compradas en la misma
pescadería pero en un mes distinto:

\begin{tabular}{cccccccccccc}
  & 0.76   & 0.81   & 0.72   & 0.80   & 0.57   & 0.52   & 0.67   &
    0.59   & 0.67   & 0.85   & 1.10 \\
  & 0.60   & 0.82   & 1.19   & 0.61   & 0.77   & 0.83   & 1.15   &
    0.56   & 0.75   & 0.96   & 0.57 \\
  & 0.95   & 0.81   & 0.97   & 0.64   & 0.62   & 0.86   & 0.70   &
    0.79   & 1.00   & 0.70   & 1.06 \\
  & 0.79   & 0.67   & 0.95   & 0.81   & 0.53   & 0.92   & 0.73   &
    0.64   & 0.65   & 0.71   & 0.68 \\
  & 0.92   & 0.56   & 0.76   & 1.04   & 0.61   & 0.62   & 0.93   &
    0.81   & 0.87   & 0.76   & 0.77 \\
  & 0.75   & 0.89   & 0.53   & 0.82   & 0.95   & 0.88   & 0.65   &
    0.85   & 0.76   & 0.85   & 0.64 \\
  & 0.84   & 0.74   & 0.76   & 0.90   & 0.96   & 0.94   & 1.10   &
    0.69   & 0.62   & 0.58   & 0.52 \\
  & 0.57   & 0.88   & 0.69   & 0.79   & 0.66   & 0.92   & 0.93   &
    0.74   & 1.17   & 0.67   & 0.61 \\
  & 0.81   & 0.87   & 1.15   & 0.66   & 0.87   & 0.87   & 0.68   &
    0.49   & 0.89   & 1.21   & 0.92 \\
  & 0.72   & 0.48   & 1.03   & 1.05   & 0.70   & 0.58   & 0.70   &
    1.04   & 0.76   & 0.65   & 0.68 \\
  & 0.52   & 0.79 &   1.03   & 0.77   & 0.99   & 1.24   & 0.59   &
    0.91   & 0.66   & 0.71
\end{tabular}

  Realizar un  estudio comparativo de los dos grupos de langostas
  mediante gráficas.
\end{prob}

\begin{centerline}
{{\bf ESTAD\'ISTICA DESCRIPTIVA BIVARIANTE}}
\end{centerline}


\begin{prob}
{Las puntuaciones obtenidas por 26 concursantes  a un puesto  de trabajo en las pruebas
de procesador de textos (PT) y  hoja de cálculo (C) han sido, en este orden:

$$
\begin{array}{l}
(1,2)\ (1,3)\  (2,1)\  (2,3)\ (2,2)\  (2,1)\  (1,2)\  (2,1)\
(1,3)\  (3,2)\  (2,2)\  (2,3)\  (1,3)\
\\  (3,1)\
(3,2)\  (1,1)\  (3,2)\  (2,1) \  (3,3)\  (1,1)\  (2,1)\  (1,3)\
(1,2)\ (2,2)\  (2,1)\  (1,3)
\end{array}
$$
Calcular la tabla de contingencia $n_{i,j}$.}
\end{prob}

\begin{prob}
{Un servicio regular de transportes a larga distancia dispone  del
siguiente modelo relativo a las variables:
$$
\begin{array}{l}
X=\mbox{retraso en horas sobre la hora de llegada prevista,}\\ Y=\mbox{velocidad modal en
el recorrido.}
\end{array}
$$
$$\vbox{\halign{\offinterlineskip\strut\vrule\ $#$ &\vrule
\ $#$ &\vrule\ $#$ &\vrule\  $#$
\vrule\cr\noalign{\hrule}X\backslash Y & 40-50 & 50-60 &
60-80\cr\noalign{\hrule}0-1&0&0&0,32\cr\noalign{\hrule}
1-2&0&0,13&0,08\cr\noalign{\hrule}2-3&0,16&0,10&0\cr\noalign{\hrule}
3-4&0,15&0,06&0\cr\noalign{\hrule}}}$$ Estudiar la independencia
entre $X$ e $Y$ calcular el coeficiente de contingencias de
Pearson $C_P$.}
\end{prob}

\begin{prob}
{En un proceso de manufacturación de un artículo de vestir se han controlado dos
características: tiempo empleado y perfeccionamiento en el acabado, teniendo la siguiente
distribución de frecuencias conjunta sobre una muestra de 120 unidades:
$$\vbox{\halign{\offinterlineskip\strut\vrule\ $#$ &\vrule
\ $#$ &\vrule\ $#$ &\vrule\ $#$ &\vrule\  $#$ \vrule \cr\noalign{\hrule}\hbox{Errors
encontrados}\backslash\hbox{minutos
empleados}&3&4&5&6\cr\noalign{\hrule}0&2&5&10&12\cr\noalign{\hrule}
1&6&10&28&8\cr\noalign{\hrule}2&15&12&6&6\cr\noalign{\hrule}}}$$ Se pide:
\begin{itemize}
\item[a)] {Distribuciones marginales.}
\item[b)] {Media aritmética, moda y desviación típica de las distribuciones
marginales.}
\end{itemize}
}
\end{prob}


\begin{prob}
{Las 130 agencias de una entidad bancaria presentaban, en el
ejercicio 1984, las observaciones siguientes:
$$
\begin{array}{l}
X=\mbox{tipo de cuenta (corriente, a plazo fijo,...)/total de cuentas,}\\ Y=\mbox{saldo
medio de las cuentas a 31-XII (en cientos de euros.).}
\end{array}
$$
$$\vbox{\halign{\offinterlineskip\strut\vrule\ $#$ &\vrule
\ $#$ &\vrule\ $#$ &\vrule\  $#$
\vrule\cr\noalign{\hrule}Y\backslash X & \hbox{menos de
0,1}&\hbox{de 0,1 a 0,3}& \hbox{más de
0,3}\cr\noalign{\hrule}\hbox{menos de
20}&48&0&0\cr\noalign{\hrule}\hbox{de 20 a
50}&21&11&0\cr\noalign{\hrule}\hbox{de 50 a
100}&14&8&2\cr\noalign{\hrule}\hbox{de 100 a
250}&7&5&1\cr\noalign{\hrule}\hbox{más de
250}&6&6&1\cr\noalign{\hrule}}}$$
 Se pide:
\begin{itemize}
\item[a)] {Distribuciones marginales.}
\item[b)] {Mediana de $Y$ y tercer cuartil de $X$.}
\item[c)] {Distribución de las agencies según $Y$,cuando el ratio
$X$ esta entre $0,1$ y $0,3$.}
\end{itemize}
}
\end{prob}

\begin{prob}
{De una muestra de 24 puestos de venta en un mercado de provisiones se  recogió
información sobre $X:$ número de balanzas e $Y:$ número de dependientes.
$$\vbox{\halign{\offinterlineskip\strut\vrule\ $#$ &\vrule
\ $#$ &\vrule\ $#$ &\vrule\ $#$ &\vrule\  $#$
\vrule\cr\noalign{\hrule}X\backslash Y &
1&2&3&4\cr\noalign{\hrule}1&1&2&0&0\cr\noalign{\hrule}
2&1&2&3&1\cr\noalign{\hrule}3&0&1&2&6\cr\noalign{\hrule}
4&0&0&2&3\cr\noalign{\hrule}}}$$ Calcular la covarianza entre
estas dos características.}
\end{prob}

\begin{prob}
{La siguiente distribución corresponde a los controls a los  que han sido sometidas 42
piezas por dos secciones del equipo de control de calidad:
$$\vbox{\halign{\offinterlineskip\strut\vrule\ $#$ &\vrule
\ $#$ &\vrule\ $#$ &\vrule\ $#$ &\vrule\  $#$
\vrule\cr\noalign{\hrule} \hbox{Controles sección 2}\backslash
\hbox{Controles sección 1} &
0&1&2&3\cr\noalign{\hrule}0&0&3&6&6\cr\noalign{\hrule}
1&2&4&3&4\cr\noalign{\hrule}2&6&2&0&0\cr\noalign{\hrule}
3&3&2&1&0\cr\noalign{\hrule}}}$$

Se pide:
\begin{itemize}
\item[a)] {Distribución de los controles efectuados por la sección 2,
media, moda y mediana.}
\item[b)] {Coeficiente de correlación lineal entre estas
variables.}
\end{itemize}
}
\end{prob}

\begin{prob}
{Las seis cooperativas agrarias de una comarca presentaban las
siguientes cifras correspondientes a las variables:
$$
\begin{array}{l}
X=\mbox{stock medio diario en naves de almacenamiento (miles de euros.)}\\ Y= \mbox{cifra
comercializada diariamente (en miles de euros.)}\\ Z=\mbox{empleados fijos en
plantilla.}\\ V=  \mbox{empleados eventuales.}
\end{array}
$$
$$\vbox{\halign{\offinterlineskip\strut\vrule\ $#$ &\vrule
\ $#$ &\vrule\ $#$ &\vrule\ $#$ &\vrule\  $#$ \vrule\cr\noalign{\hrule}
\hbox{Cooperativa}&X&Y&Z&V\cr\noalign{\hrule}A&26&146&6&8 \cr\noalign{\hrule}
B&33&167&8&6\cr\noalign{\hrule}C&12&92&6&8\cr\noalign{\hrule}
D&18&125&8&6\cr\noalign{\hrule}E&18&118&10&4
\cr\noalign{\hrule}F&25&132&10&4\cr\noalign{\hrule}}}$$ Calcular el coeficiente de
correlación lineal entre las variables $(X,Y)$, $(Y,Z)$ y $(Z,V)$.}
\end{prob}

\begin{prob}
{Se pidió a dos usuarios de detergentes que clasificaran 6
detergentes de acuerdo con sus preferencias. Los resultados
fueron:
$$\vbox{\halign{\offinterlineskip\strut\vrule\ $#$ &\vrule
\ $#$ &\vrule\ $#$ \vrule\cr\noalign{\hrule}
\hbox{Detergente}&\hbox{Usu. A}&\hbox{Usu. B}\cr\noalign{\hrule}
A&2&3\cr\noalign{\hrule}B&4&2\cr\noalign{\hrule}
C&5&4\cr\noalign{\hrule}D&1&1\cr\noalign{\hrule}
E&6&6\cr\noalign{\hrule}F&3&5\cr\noalign{\hrule}}}$$ Calcular el
coeficiente de correlación de Spearman, e interpretar el
resultado.}
\end{prob}

\begin{prob}
{Los siguientes datos corresponden a las calificaciones otorgadas
a 18 empleados después de unos cursillos de especialización
realizados por una agencia de ventas:
$$\vbox{\halign{\offinterlineskip\strut\vrule\ # &\vrule\ $#$
&\vrule\ $#$ &\vrule\ $#$ &\vrule\ $#$ &\vrule\ $#$ &\vrule\ $#$
 &\vrule\ $#$ &\vrule\ $#$ &\vrule\ $#$ &\vrule\ $#$ &\vrule\ $#$
 &\vrule\ $#$ &\vrule\ $#$ &\vrule\ $#$ &\vrule\ $#$ &\vrule\ $#$
 &\vrule\ $#$ &\vrule\ $#$ \vrule\cr\noalign{\hrule}
empleado&1&2&3&4&5&6&7&8&9&10&11&12&13&14&15&16&17&18
\cr\noalign{\hrule}\noalign{\hrule}persuasión&0&0&1&2&1&2&0&1&2&1&1&1&2&1&0&1&1&0
\cr\noalign{\hrule}retentiva&1&0&1&0&1&0&2&1&0&1&2&1&1&2&0&1&1&2
\cr\noalign{\hrule}prudencia&1&1&1&2&2&1&0&1&2&0&1&1&0&0&0&1&1&2
\cr\noalign{\hrule}}}$$ Se pide:
\begin{itemize}
\item[a)] {Distribución de las puntuaciones de retentiva y prudencia.}
\item[b)] {Distribución de las puntuaciones en persuasión y prudencia
para aquellos que no han obtenido un cero en retentiva.}
\item[c)] {Distribución de las puntuaciones en persuasión  de aquellos
que han sacado  un 1 en las pruebas de retentiva y prudencia.}
\end{itemize}
}
\end{prob}

\begin{prob}
{Una cartera de valores puede estar compuesta por dos tipos d e acciones con una
rentabilidad dada por la siguiente distribución conjunta, según una escala subjetiva que
les concedemos ``a priori'':
$$\vbox{\halign{\offinterlineskip\strut\vrule\ $#$ &\vrule
\ $#$ &\vrule\ $#$ \vrule\cr\noalign{\hrule} \hbox{Rentabilidad
$X_2$}\backslash\hbox{Rentabilidad $X_1$}&\hbox{De 5 a
10}&\hbox{De 10 a 15}\cr\noalign{\hrule} \hbox{De 0 a
5}&0,06&0,02\cr\noalign{\hrule}\hbox{De 5 a
10}&0,14&0,38\cr\noalign{\hrule} \hbox{De 10 a
15}&0,30&0,10\cr\noalign{\hrule}}}$$ Se pide:
\begin{itemize}
\item[a)] {?`Cuál  de las dos acciones presenta mayor rentabilidad
media esperada?}
\item[b)] {?`Cuál de las dos acciones presenta menos varianza en su
rentabilidad?}
\item[c)] {Calcular la covarianza entre las rentabilidades.}
\end{itemize}
}
\end{prob}


\newpage



\begin{centerline}
{{\bf PROBABILIDAD}}
\end{centerline}


\newcommand{\pr}[1]{P(#1)}

%\newcounter{problema}
%\newcommand{\prb}{\addtocounter{problema}{1}
%\noindent\vskip 2mm {\textbf{\theproblemes  }}


%\setcounter{problema}{105}
 %
 %  %tema 7
%%%fulla 4

\probl  En una carrera en la que participan diez caballos ?`de
cuántas maneras diferentes se pueden dar los cuatro primeros
lugares? \sol{\bf 5040}

\probl  Una empresa de reciente creación encarga a un diseñador gráfico la elaboración
del su logotipo, indicando que ha de seleccionar exactamente tres colores de una lista de
seis. ?`Cuántos grupos tienen para elegir el diseñador? \sol{\bf 20}

\probl  ?`Cuántas palabras diferentes, de cuatro letras, se pueden
formar con la palabra {\bf byte}? \sol{\bf 24}

\probl  ?`De cuantas maneras diferentes se pueden elegir el director y el subdirector de
un departamento formado por 50 miembros? \sol{\bf 2450}

\probl  Con once empleados ?`cuántos comités de  empresa de cinco
personas se pueden formar? \sol{\bf  462}

\probl  ?`Cuántas maneras distintas hay de colocar quince libros diferentes en una
estantería si queremos  que el  de Probabilidades esté el  primero y  el de Estadística
en el tercero? \sol{\bf 6227020800}

\probl  ?`Cuántos caracteres diferentes podemos formar utilizando a lo sumo  a tres
símbolos  de los utilizados en el alfabeto Morse? \sol{\bf 14}

\probl  Un supermercado  organiza una rifa con un premio de  una botella de cava para
todas las papeletas que tengan las dos últimas cifras iguales a las correspondientes dos
últimas cifras del número premiado en el sorteo de Navidad. Supongamos que todos los
décimos tienen cuatro cifras y que existe un único décimo de cada numeración ?`Cuántas
botellas repartirá el supermercado? \sol{\bf 100}

\probl  ?`Cuántas  palabras diferentes podemos formar con todas las letras de la palabra
estadística? \sol{\bf 2494800}

\probl  En una tienda de regalos hay relojes de  arena con cubetas
de colores, y no hay diferencia alguna entre las  dos cubetas que
forman cada reloj. Si hay cuatro  colores posibles y el color de
los dos recipientes puede coincidir ?`cuántos  modelos de reloj de
arena puede ofrecer el establecimiento? \sol{\bf 10}

\probl  En una partida de parchís gana aquel jugador que consigue alcanzar antes  con sus
cuatro fichas la llegada. Si hay  cuatro jugadores y la partida continua hasta que todos
han completado el recorrido?`cuántos   órdenes de llegadas hay para la dieciséis fichas?
\sol{\bf 63063000}

\probl  Se han de repartir cinco becas entre diez españoles  y seis extranjeros, de
manera que se den tres a españoles y dos a extranjeros ?`De cuántas maneras se puede hacer
el reparto? \sol{\bf 1800}

\probl  ?`Cuantas  fichas tiene un dominó? \sol{\bf 28}




\probl  Calcular la probabilidad de que al lanzar a la vez $5$
dados se obtenga:
\begin{enumerate}[a)]
\item repóker (5 resultados iguales);
\item póker (4 resultados iguales);
\item full (3 resultados iguales y los otros distintos pero iguales entre si);
\item trio (3  resultados iguales y los otros dos  diferentes);
\item doble pareja (2 resultados  iguales, otros 2 iguales y diferentes de los anteriores y el restante diferente
);
\item pareja (exactamente 2 resultados iguales);
\item nada (5 resultados distintos).
\end{enumerate}
\sol{ a) $\bf 6/6^5$; b) $\bf 150/6^5$;c) $\bf 300/6^5$;  d) $\bf
1200/6^5$; e) $\bf 1800/6^5$; f) $\bf 3600/6^5$; g) $\bf 720/6^5$}

\probl  Tenemos 12 radios de las   que  5 son defectuosas. Elegimos  3 radios al azar.
?`Cuál  es la probabilidad de que sólo una de las 3 sea defectuosa? \sol{{\bf 21/44}}

\probl  Lanzamos al aire 6 dados.
\begin{enumerate}[a)]
\item ?`Cuál es la probabilidad de que todos ellos den resultados distintos?
\item ?`Cuál es la probabilidad de obtener 3 parejas?
\end{enumerate}
\sol{ a)$\mathbf{120/6^5}$; b)  $\mathbf{300/6^5}$ }

\probl  Supongamos que en una empresa de fabricación de componentes electrónicos se sabe
que en un lote de 550 almacenados el 2\% son   defectuosos ?`Cuál es la probabilidad de
encontrar 2 de defectuosos si cogemos de forma equiprobable 25? \sol{$\mathbf{0.074}$}

\probl  Si mezclamos   suficientemente una baraja de 52 cartas ?`cuál es la probabilidad
de que los 4 ases queden colocados consecutivamente? \sol{{\bf 24/132600}}

\probl  Una forma de incrementar la  fiabilidad de un sistema es la introducción de una
copia de los componentes en una configuración paralela. Supongamos  que la N:A.S.A.
quiere un vuelo  con una probabilidad no inferior a $0.99999$ de  que el transbordador
espacial entre en órbita alrededor de la Tierra con éxito. ?`Cuántos   motores se han de
montar en  paralelo para que se  alcance esta fiabilidad, si se sabe que la probabilidad
de que cada uno de los motores funcione adecuadamente es $0.95$? Suponer que los motores
funcionan de manera independiente entre si. \sol{{\bf 4}}

\probl  ?`Cuál es la probabilidad de  que  entre
 $n$ personas, que no han nacido el 29 de
febrero,  haya  como a mínimo dos que hayan nacido el mismo día del año? (no
necesariamente del mismo año). Calcular la probabilidad para los siguientes valores de
$n: 10, 15, 22, 23, 30, 40, 50, 55$. \sol{$\mathbf 0.12; 0.25; 0.48; 0.51; 0.71; 0.89;
0.97; 0.99$}

\probl  Cuatro cartas numeradas de $1$ a $4$ están colocadas  boca abajo sobre una  mesa.
Una persona, supuestamente clarividente, irá adivinando  los valores de las $4$ cartas
una a una. Si suponemos que es un farsante y que lo que hace  es decir los cuatro números
al azar ?`cuál es la probabilidad de   que acierte como mínimo una ? (Obviamente, no
repite ningún número) \sol{\bf 15/24}

\probl  En una lotería hay $500$ billetes y $5$ premios. Si una persona compra $10$
billetes ?`cuál es la probabilidad de obtener?:
\begin{enumerate}[a)]
\item el primer premio?
\item como  mínimo un premio?
\item exactamente un premio?
\end{enumerate} \sol{{\bf 0.02; 0.096; 0.093}}

\probl  Se elige de forma equiprobable un número del $1$ al
$6000$. Calcular la probabilidad de que sea múltiplo de $2$ o de
$3$ o de $4$ o de $5$. \sol{\bf 0.73}


\probl  Si elegimos un número de entre los 120 primeros enteros
  positivos  ?`cuál es la probabilidad de  que  sea
múltiplo de $3$, no sea divisible por 5, y sea divisible por 4 o
por 6? \sol{\bf 2/15}

\probl  Una cuarta parte de la población ha sido  vacunada contra
una enfermedad contagiosa. Durante una epidemia, se observa que de
  uno de entre cada cuatro enfermos ha sido vacunado.
\begin{enumerate}[a)]
\item ?`Ha tenido alguna eficacia la vacuna?
\item Por otra parte, se sabe que hay un enfermo  entre cada $12$ personas
vacunadas ?`Cuál es la probabilidad de que  esté enferma una
persona que no se ha vacunado?\sol{a) {\bf Sí}; b) {\bf 1/9}}
\end{enumerate}

\probl La probabilidad de  que un estudiante  acabe una carrera determinada es $0.4$.
Dado un grupo de $5$ estudiantes de esta carrera, calcular la probabilidad de que:

\begin{enumerate}[a)]
\item  ninguno acabe la carrera,
\item sólo uno acabe la carrera,
\item al menos dos acaben la carrera;
\item todos la acaben.
\end{enumerate}
\sol{\bf a) 0.07776; b) 0.2592; c) 0.66304; d) 0.01024}

\probl  Un  mensaje se ha codificado con un alfabeto de dos símbolos $A$ y $B$ para poder
transmitirse a través de un canal de comunicación. La codificación es tal que $A$ aparece
el doble de veces que $B$ en el mensaje codificado. El ruido del canal es tal que cuando
$A$ se transmite, se recibe  $A$ con una probabilidad de $0.8$ y  $B$ con una
probabilidad de  $0.2$; cuando se transmite $B$ se recibe  $B$ con una probabilidad de
$0.7$ y se recibe $A$ con probabilidad $0.3$.
\begin{enumerate}[a)]
\item ?`Cuál es la frecuencia relativa de $A$ en el mensaje recibido?
\item Si última letra del mensaje recibido es una $A$ ?`cuál es la probabilidad de que
 se haya enviado una $A$?
\end{enumerate}\sol{{\bf a) 0.633; b) 0.84}}

\probl  En una ciudad se publican 3 diarios $A$, $B$ y $C$. El $30
\%$ de la población lee $A$, el $20 \%$ lee $B$ y el $15 \%$ lee
C; el $12 \%$ lee $A$ y $B$, el $9 \%$ lee $A$ y $C$, y el $6 \%$
lee $B$ y $C$; finalmente, el $3 \%$ lee $A$, $B$ y $C$.
Calcular:\
\begin{enumerate}[a)]
\item El porcentaje de gente que lee al menos uno de los tres diarios.
\item El porcentaje de gente que sólo lee $A$.
\item  El porcentaje de gente que lee $B$ o $C$, pero no $A$.
\item  El porcentaje de gente que lee $A$ o no lee ni $B$ ni $C$.
\sol{ \bf{ a) 0.41 ; b) 0.12; c) 0.11; d) 0.89}}
\end{enumerate}


 \probl  Supongamos que en un dado la probabilidad de cada una de sus seis
 caras es proporcional al número inscrito en ella. Calcular la probabilidad
de obtener un número par. \sol{\bf 4/7}

\probl  En una reunión, $n$ personas ($n \geq 3$) lazan una moneda
al  aire. Si una de ellas da diferente de todas las otras, su
propietario paga una ronda ?`Cuál es la probabilidad de que pase
esto? \sol{$\mathbf{(n \cdot \left( {1 \over 2} \right)^{n-1})}$}



%%%%fulla 3


\probl  Un matrimonio planifica  su descendencia  considerando los
siguientes esquemas (se supone que tener una varón o una hembra es
equiprobable):
\begin{description}
\item[Esq. A)] Tener $3$ varones.
\item[Esq. B)] Tener varones hasta que nazca la primera hembra, o ya tengan tres
niños (lo que pase primero).
\item[Esq. C)] Tener niños hasta que tengan una pareja de ambos sexos,
 o ya tengan tres niños (lo que pase primero).
\end{description}
Sea $B_i$ el suceso  han nacido $i$ niños ($i=1, 2, 3$) y $C$ el suceso tener más varones
que hembras.
\begin{enumerate}[1)]
\item Calcular $p(B_1)$ y $p(C)$ en cada uno de  los tres esquemas.
\item Calcular $p(B_2)$ y $p(B_3)$ en cada uno de los tres esquemas.
\item Sea $E$ el suceso que la familia completa tenga igual número de
varones que de hembras. Encontrar $p(E)$ en cada uno de los tres
esquemas.

\sol{a) $\bf(p(B_1)=3/8,1/4,5/8; p(C)=1/2,1/2,1/4)$; b) $\bf
(p(B_2)=3/8,1/8,1/8; p(B_3)=1/8,1/8,1/8)$; c)
$\mathbf{(0,1/4,1/2)}$.}
\end{enumerate}

\probl  Un comerciante ha de viajar en avión de Bangkok a Bagdad. Preocupado, pide a la
compañía aérea cuál es la probabilidad de que haya como mínimo una bomba en el avión y le
dicen que es de $0.1$. Más preocupado aún pide cuál es la probabilidad de que haya como
mínimo dos bombas y le dicen que es $0.01$. Más tranquilo, decide llevar una bomba en su
equipaje. Haciendo las suposiciones adicionales oportunas ?`qué  valoración estadística
podemos hacer de su decisión? \sol{\bf Decisión absurda por sentido común y
estadísticamente. }

\probl  Dos sistemas con cuatro componentes independientes con
fiabilidades respectivas $p_1, p_2, p_3$ y $p_4$ se configuran de
las dos maneras siguientes: En el sistema $A$, la combinación en
serie de los components $1$  y $2$ se configura en paralelo con la
combinación en serie de los componentes $3$ y $4$; en el sistema
$B$, la combinación en paralelo de $1$ y $3$ se configura en serie
con la combinación en paralelo de $2$ y $4$. Determinar el sistema
más fiable. \sol{\bf B}

\probl  Si un sistema que consiste en tres components independientes con la misma
fiabilidad ($p_1=p_2=p_3$) tiene una fiabilidad de $0.8$, determinar $p_1$ en los
siguientes casos:

\begin{enumerate}[a)]
\item el componente 3 está configurado en serie con la combinación en paralelo
$1$ y $2$.
 \item el componente 3 está configurado en paralelo
con la combinación en serie de $1$ y $2$.
\end{enumerate}
 \sol{\bf a) 0.825; b)
0.652}
%%
%%\probl  N'\`Oscar diu la veritat nou vegades de cada deu i n'Ivan
%%set de cada nou. S'extreu a l'atzar una bolla d'una bossa on hi
%%havia 5 bolles blanques i 20 negres. Tots dos observen el color de
%%la bolla extreta i llavors diuen de manera independent que la
%%bolla extreta és blanca. Quina és la probabilitat que aix\`o sigui
%%cert? \sol{\bf 0.89}



%%%%%%%%%%%%%tema 8
%
\newpage

%\problemes=0
\begin{centerline}
{{\bf VARIABLES ALEATORIAS}}
\end{centerline}


\probl{En los ocho problemas siguientes determinar la función de probabilidad y la de
distribución de las variables aleatorias que aparezcan}
\begin{enumerate}[a)]
\item Consideremos el experimento  consistente en lanzar simultáneamente dos dados; repetimos
el ex\-pe\-ri\-mento dos veces. Sea $X$ la variable aleatoria que da  el número de
lanzamientos en que los dos dados han mostrado un número par. Sea $Y$ la variable aleatoria
que nos da el número de lanzamientos en que la suma de los dos dados es par.

\item  Supongamos que tenemos almacenadas $10$ piezas, de las que sabemos que hay $8$ del
tipo I y $2$ del tipo II; se toman dos al azar de forma equiprobable. Sea $X$ la variable
aleatoria que da el número de piezas de tipo I que hemos cogido.

\item  Supongamos  que un estudiante realiza el tipo de examen siguiente: El profesor le va
formulando preguntas hasta que el estudiante falla una (no os preguntéis como se evalúa ni
yo lo sé). La probabilidad  de que el estudiante acierte  una pregunta cualquiera es $0.9$
(examen fácil). Sea $X$ la variable aleatoria que nos da el número de preguntas formuladas
por el profesor ?`Cuál es el número más probable de preguntas formuladas?

\item   Consideremos  dos cañones que van disparando alternativamente hacia el mismo
objetivo. El primer cañón tiene una probabilidad de acertar el objetivo de $0.3$ mientas
que en el segundo es de $0.7$. El primer cañón comienza la serie de lanzamientos y no se
detienen hasta que uno de los cañones desintegre el blanco (es suficiente darle una vez).
Sea $X$ la variable aleatoria que nos da el número de proyectiles lanzados por el primer
cañón e $Y$ la que nos da los proyectiles lanzados por el segundo cañón.

\item  En la misma situación que en el ítem anterior,
considerar la variable aleatoria $X$ que da el número de proyectiles lanzados por el primer
cañón condicionado a que gana  y sea  $Y$ la variable aleatoria que cuenta el número de
proyectiles lanzados por segundo cañón cuando gana.

\item  Supongamos que se hace una tirada de $100000$ ejemplares de un determinado libro
. La probabilidad de que una encuadernación sea incorrecta es  $0.0001$ ?`Cuál es la
probabilidad de que haya $5$ libros de la tirada mal encuadernados?

\item  Dos compañeros de estudios  se encuentran en un conocido bar de copas de Palma
y deciden jugar a dardos de una manera especial: Lanzarán consecutivamente un dardo cada
hasta que uno de los dos acierte el triple $10$ (centro de la diana).  El que lanza en
primer lugar tiene una probabilidad de $0.7$de acertar  y el que lo hace en segundo lugar
$0.8$. Sea $X$ la variable aleatoria que da el número total de lanzamientos de dardos
hechos por los dos compañeros.

\item  Un examen tipo test  consta de $5$ preguntas, cada una  con tres opciones de
respuesta, sólo hay una opción correcta. Un estudiante contesta al azar  a las $5$
cuestiones. Sea $X$ la variable aleatoria que da el número de puntos obtenidos por el
alumno.
\begin{enumerate}[i)]
\item Si les respuestas erróneas  no restan puntos.
\item Si cada respuesta errónea resta 1 punto.
\end{enumerate}
\end{enumerate}



\probl  Un coche tiene que  pasar por cuatro semáforos. En cada uno de ellos el coche tiene
la misma probabilidad de seguir su marcha que de detenerse. Hallar la función de
distribución del número de semáforos que pasa el coche sin detenerse.


\probl Un individuo quiere invertir un capital de medio millón de euros en un negocio que
tiene una rentabilidad del $50\%$, pero con el riesgo de perder toda la inversión. Su
asesor financiero le informa que este negocio tiene una probabilidad de ser rentable del
$0.8$ ?`Cuál es el beneficio esperado? $\mathbf{(100000)}$

\probl Un juego se dice justo si la ganancia esperada de cada jugador es $0$. Dos jugadores
A y  B tiran un dado por turnos, y gana el primero que obtiene un $5$. Cada jugador apuesta
una cantidad $ c_j \ (j=1,2),  $  y el total se lo queda el ganador. Si suponemos que
comienza  a jugar A ?`qué relación tienen que verificar $c_1$ y $c_2$ para que el juego sea
justo?


\probl  Supongamos que jugamos a la ruleta en un casino. Sea $p<1/2$ la probabilidad de que
salga un número rojo. Supongamos que apostamos a la par (lo que quiere decir, que si
apostamos $k$ dólares, cuando sale rojo nos dan $k$ dólares más los que hemos entregado al
apostar y perdemos los
 $k$ dólares que hemos apostado si no sale rojo). La primera apuesta es de 1 dólar. Si
ganamos, nos retiramos. Si perdemos, hacemos una segunda apuesta de $2$ dólares. Si
ganamos, nos retiramos. Si perdemos, apostamos $2^2$ dólares,  y así sucesivamente. La
$n$-ésima apuesta será de $2^{n-1}$ dólares.

\begin{enumerate}[a)]
\item Probar  que con este sistema es seguro que ganamos 1 dólar.
\item Calcular el importe esperado de la apuesta ganadora. $\mathbf{(\infty)}$
\end{enumerate}


Supongamos ahora que la casa tiene un límite de $2^L$ dólares (el máximo que está permitido
apostar), de manera que, si no hemos ganado antes, esa será nuestra apuesta final.

\begin{enumerate}[a)]
\item  ?`Cuál es la ganancia esperado cuando paramos de jugar?
$\mathbf{(1-(2(1-p))^{L+1})}$
\item ?`Qué es mejor un límite alto o bajo?  \textbf{(bajo)}
\end{enumerate}

\probl Se venden 5000 billetes de lotería a 100 pts. cada uno, para  un sorteo con un
premio de  300000 pts. ?`Cuál es la ganancia (pérdida) esperada de una persona que compra
tres billetes? $\mathbf{(-120\quad \textrm{pts})}$


\probl Un sistema de transmisión emite los dígitos -1, 0, 1. Cuando se transmite el símbolo
{\it i}, se recibe el símbolo {\it j} con las probabilidades siguientes: $ \pr{r_1 / t_1} =
1, \ \pr{r_{-1} / t_{-1}} = 1, \ \pr{r_1 / t_0} = 0.1, \ \pr{r_{-1} / t_0} = 0.1, \ \pr{r_0
/ t_0} = 0.8. $ Se dice que en este caso se ha producido una distorsión $ \displaystyle
(i-j)^2. $ ?`Cuál es el valor medio de la distorsión? $\mathbf{(1/15)}$

%%%%va la 13
\probl Una fuente  binaria emite de manera equiprobable e independiente un bloque  de 3
dígitos (0 ó 1) cada segundo. De cada bloque se envía a una canal de transmisión un 0 si en
el bloque hay más ceros que unos y un 1 en caso contrario. El canal transmite el dígito con
una probabilidad de error $p$. El receptor reconstruye la terna de dígitos repitiendo tres
veces el dígito que ha recibido ?`Cuál es el número medio de bits erróneos por bloque?
$\mathbf{(\frac{3}{4}+\frac{3\cdot p}{2})}$ ?`Cuál tendría que ser la probabilidad $p$ para
que este valor medio no fuera más grande que 1?  $\mathbf{(p \leq 1/6)}$

\probl  Dos personas juegan a cara o cruz, y han decidido continuar la partida hasta que se
obtengan como mínimo $3$ caras  y $3$ cruces. Hallar la probabilidad de que el juego no se
acabe en $10$ tiradas y el número esperado de tiradas.
\sol{$\mathbf{\frac{7}{64}=0.109375}$; $\mathbf{E(X)=\frac{63}{8}}$.}

%%%\probl  Calcular la esperanza y la varianza del número de puntos obtenidos en el
%%%lanzamiento de un dado. \sol{$\mathbf{(7/2,\quad 35/12)}$}



\probl (Opcional) Es un buen ejercicio calcular la esperanza y la varianza de todas las
variables que aparecen en el problema  68.

%%%%%ggggggggg
%\headline{\hfill \bf Fulla 6}

% {\bf PROBABILITATS I PROCESSOS ALEATORIS. 2on TELEM\`ATICA}
% \vskip 1 cm
%
% \magnification \magstep 1

\begin{center}
\textbf{Variables aleatorias continuas}
\end{center}

\probl  Sea  $X$ una variable aleatoria continua con función de densidad $f(x)$ dada por:
$$f(x) = \cases{ k \cdot (1+x^2) & si $x \in (0,3)$\cr 0 & si $ x \not \in (0,3)$\cr}$$
\begin{enumerate}[a)]
\item Calcular la constante $k$  y la función de distribución de $X$.
\item Calcular la probabilidad de que $X$ esté comprendida entre $1$ y $2$
\item Calcular la probabilidad de  que $X $ sea menor que $1.$
\item Sabiendo que $X$ es mayor que $1$, calcular la probabilidad de
que sea menor que $2$. \sol{a)$\mathbf{k=1/12}$; b) $\mathbf{5/18}$; c)$\mathbf{1/9}$; e)
$\mathbf{5/16}$}
\end{enumerate}

\probl  La función de densidad de una variable aleatoria continua es: $$f(x) = \cases{ a
\cdot x^2 + b & si $ x \in (0,3)$\cr 0 & si $ x \not \in (0,3)$\cr}$$.

 Determinar $ a$ y $b$ , sabiendo que
$P(1 < X \leq 2) = 2/3$.  \sol{$\mathbf{a=-1/2, b=11/6}$}

\probl  La duración en minutos de unas ciertas comunicaciones telefónicas es una variable
aleatoria con función de distribución:
$$F(x) = \cases{ 0 & si $ x \leq 0$\cr 1-{1 \over 2} e^{-x/3} - {1 \over 2} e^{-R[x/3]} &
si $ x > 0$\cr}$$ donde $ R[x] $ es la parte entera de $x.$ Calcular la probabilidad de la
comunicación  dure:
\begin{enumerate}[a)]
\item Más de $6$ minutos.
\item Menos de $4$ minutos.
\item Exactamente $3$ minutos.
\item Menos de $9$ minutos, sabiendo que ha durado más de $5$.
\item Más de $5$ minutos, sabiendo que ha durado menos de $9$.\sol{a)$\mathbf{e^{-2}}$;
b)$\mathbf{1-{1 \over 2} e^{-4/3} -{1 \over 2} e^{-1}}$  ; c)
$\mathbf{{1 \over 2}(1-e^{-1})}$ }
\end{enumerate}

 \probl  Sea $X$ una variable aleatoria
 continua con densidad:

$$f(x) = \cases{1-|x| & si $ |x| \leq 1$\cr 0 & en caso
contrario\cr}$$
\begin{enumerate}[a)]
\item  Encontrar  la función de distribución de $ X.$
\item Calcular $ \pr{X \geq 0}$  y $\pr{|X| < 1/2}.$  \sol{b) $\mathbf{1/2,
3/4}$}
\end{enumerate}




% \headline{\hfill \bf Fulla 7}
%
% {\bf PROBABILITATS I PROCESSOS ALEATORIS. 2on TELEM\`ATICA}
%
% \vskip 1 cm
% \magnification \magstep 1


\probl  Se llama {\bf distribución triangular} a cualquier distribución continua tal que su
densidad es cero salvo en un cierto intervalo  $ (a,b), $ en el que su gráfica tiene forma
de triángulo isósceles. Hallar la función de densidad y de distribución de una distribución
triangular.



%%%%%%%%distribuciones notables

%%%%%%pppppppp


\probl  Sea $X$ una variable aleatoria  continua con densidad (Laplaciana)

$$f(x) = {1 \over 2}  e^{- |x|} \mbox{ si } x\in\RR$$

\begin{enumerate}[a)]
\item Calcular $ \pr{|X| > 2}.$  \sol{$\mathbf{e^{-2}}$}
\item Calcular $E(X)$ y $Var(X)$. \sol{$E(X)=0$, $Var(X)=2.$}
\end{enumerate}

\probl  Consideremos $ f:{\R}\to {\R}  $ dada por

$$f(x) = \cases{ 0 & si $ x \leq 0$\cr a (1+x) & si $ 0 < x \leq
1$\cr 2/3 & si $ 1 < x \leq 2$\cr 0 & si $ x > 2$\cr}$$
\begin{enumerate}[a)]
\item  Determinar el valor de $a$ para que $f$  sea una densidad.
\item  En este caso, si $X$ es una variable aleatoria
 continua con densidad $f$, calcular $ \pr{1/2 < X\leq 3/2}$.
 \item Calcular, para el valor de $a$ encontrado $E(X)$ y $Var(X)$.
 \sol{a)$\mathbf{a=2/9}$ ; b) $\mathbf{19/36}$; c) $E(X)=\frac{32}{27}$,
    $Var(X)=\frac{409}{1458}$.}
\end{enumerate}

% \nopagenumbers
%
% \parindent = 0 pt
%
% \headline{\hfill \bf Full 8}
%
% {\bf PROBABILITATS I PROCESSOS ALEATORIS. 2on TELEM\`ATICA}
%
% \vskip 1 cm
% \magnification \magstep 1

\newpage

\begin{center}
\textbf{Transformación de variables aleatorias}
\end{center}

\probl Sea $X$ la variable que nos da la puntuación obtenida al lanzar un dado. Calcular la
distribución de las variables $Y=X^2$, $Z=X^2-6x+6$. Calcular las esperanzas y las
varianzas de las variables $Y$ y $Z$.

\probl  Conocida  la función de distribución de una variable aleatoria continua $X$, hallar
la función de densidad de $\displaystyle Y = X^2$ y de $ Z = e^X.$

\probl  La función de distribución de una variable aleatoria  $X$ es: $$F_{X}(x) = \cases{
0 & si $ x \leq 0$\cr 1-e^{-x^2} & si $ x
> 0$\cr}$$ Encontrar la función de densidad de la variable aleatoria
$Y = \ln (X+1).$


\probl  La función de densidad de una variable aleatoria $X$ es:

$$f_{X}(x) = \cases{ x+1 & si $ x \in (-1,0]$\cr -x+1 & si $ x \in
(0,1]$\cr 0 & si $x \in (-\infty,-1] \cup (1,\infty)$\cr}$$

Definimos la variable aleatoria $Y=g(X)$,  donde $g$ es la función

$$g(x) = \cases{ 1 & si $ x \in (1/2,\infty)$\cr 0 & si $ x \in
(-1/2,1/2]$\cr -1 & si $x \in (-\infty,-1/2]$\cr}$$

Determinar la función de masa de probabilidad y la de distribución de $Y.$ Calcular las
esperanzas y varianzas de $X$ e $Y$.




\probl  El precio por estacionar un vehículo en un aparcamiento es de $75$ pts. por  la
primera hora o fracción, y de $60$ pts. a partir de la segunda hora o fracción. Supongamos
que el tiempo, en horas, que un vehículo cualquiera permanece en el aparcamiento se
modeliza según la siguiente función de densidad

$$f_{X}(x) = \cases{e^{-x} & si $ x \geq 0$\cr 0 & si $ x <
0$\cr}.$$

Calcular  el ingreso medio por vehículo. $\mathbf{(109.919)}$







%%%\probl Calculau l'esperan\c{c}a i la vari\`ancia del nombre de punts
%%%obtingut en el llan\c{c}ament d'un dau. $\mathbf{(21/6,\quad 35/12)}$





\probl Calcular $E(X)$ y $Var(X)$ para una v.a. $X$ que tiene por  funció de densidad $f_X$
dada por:
$$f_X(x) = \cases{{1\over 2\sqrt{x}} & si $ x \in (0,1)$\cr 0 & si $ x \not \in
(0,1)$\cr}$$ $\mathbf{(1/3,\quad 4/45)}$


\probl Consideremos una variable aleatoria $X$ con función de densidad:
$$f_X(x) = \cases{1/2 & si $ 0<x<2$\cr 0 & en caso contrario\cr}$$ determinar
$E(Y)$, donde $Y = \ln( X).$ $\mathbf{(-0.3069)}$


\probl Sea $X$ una variable aleatoria con función de densidad:

$$f_{X}(x) = \cases{2 x & si $x \in (0,1)$\cr 0 & si $x \not \in
(0,1)$\cr}$$
\begin{enumerate}[a)]
\item Determinar $E(\sqrt{X})$ a partir de $f_{\sqrt{X}}.$
$\mathbf{(4/5)}$
\item Hacer lo mismo a partir de $f_{X}.$
\end{enumerate}


%%%\probl Un sistema de transmissió emet els dígits -1, 0, 1. Quan se
%%%transmet el símbol {\it i}, se rep el símbol {\it j} amb les
%%%segŸents probabilitats: $ \pr{r_1 / t_1} = 1, \ \pr{r_{-1} /
%%%t_{-1}} = 1, \ \pr{r_1 / t_0} = 0.1, \ \pr{r_{-1} / t_0} = 0.1, \
%%%\pr{r_0 / t_0} = 0.8.  $ Se diu en aquest cas que s'ha produ•t una
%%%distorsió $ \displaystyle (i-j)^2.  $ Quin és el valor mitj\`a de la
%%%distorsió? $\mathbf{(1/15)}$

%%%%va la 13
%%%\probl Una font bin\`aria emet de manera equiprobable i
%%%independentment un bloc de 3 dígits (0 o 1) cada segon. De cada
%%%bloc envia a un canal de transmissió un 0 si al bloc hi ha més 0's
%%%que 1's, i un 1 en cas contrari. El canal transmet el dígit amb
%%%una probabilitat d'error {\it p}. El receptor reconstrueix la
%%%terna de dígits repetint tres vegades el dígit que ha rebut. Quin
%%%és el nombre mitj\`a de bits erronis per bloc?
%%%$\mathbf{(\frac{3}{4}+\frac{3\cdot p}{2})}$ Quina hauria de ser la
%%%probabilitat {\it p} per tal que aquest valor mitj\`a no fos més
%%%gran que 1? $\mathbf{(p \leq 1/6)}$


\begin{center}
 \textbf{Desigualdad Chebychef}
\end{center}




\probl Sea $X$ una variable aleatoria que toma los valores $-k,0,k$ con probabilidades
$\frac{1}{8}, \frac{3}{4}$ y $\frac{1}{8}$ respectivamente.

\begin{enumerate}[a)]
\item Calcular $ \pr{| X - \mu | \geq 2 \sigma} \ (\mu =
E(X), \sigma^2 =Var(X)).$ $\mathbf{(1/4)}$
\item Obtener una cota superior de la probabilidad
anterior utilizando la desigualdad de Chebychef. $\mathbf{(1/4)}$
\end{enumerate}

\probl Sea $X$ una variable aleatoria con función de densidad

$$f(x) = \cases{ 1 - |x| & si $ |x| < 1$\cr 0 & si $ |x| \geq
1$\cr}$$
\begin{enumerate}[a)]
\item Calcular $ \pr{| X | \geq k}, \ \  $ donde $ 0 < k < 1.$
$\mathbf{((1-k)^2)}$
\item  Obtener una cota superior de la probabilidad anterior utilizando
la desigualdad de Chebychef. $\mathbf{(\frac{1}{6k^2})}$
\item  Comparar los dos resultados anteriores
para $ k = 0.1, 0.2, \ldots , 0.9$
\end{enumerate}





%%%%%% va la 14
\probl El número de periódicos vendidos en un kiosko es una variable aleatoria de media
$200$ y desviación típica $10$. ?`Cuántos ejemplares diarios tiene que encargar el dueño del
kiosko para tener una seguridad de que el menos el $99$\% de los días no se quedará sin
existencias? \sol{$\mathbf{(300)}$.}

\probl El número de días al año que un trabajador de un pequeño comercio está de baja por
enfermedad es  una variable aleatoria de media $10$ y  desviación típica $2$. Si cada uno
de estos días la empresa pierde $10000$ pts., determinar los límites inferior y superior de
las pérdidas anuales por trabajador con un grado de fiabilidad no inferior a $95$\%.
\sol{$\mathbf{(10.560)}$ y $\mathbf{(189.440)}$ pts.}

\probl El ayuntamiento de una ciudad ha decido establecer una zona de aparcamiento
limitado, en la que aparcan $500$ vehículos  diarios por termino medio. Si al menos el
$99$\% de los días el número de coches que utilizan esta servicio está entre $475$ y $525$,
estimar la desviación típica de la variable aleatoria que da el número  de vehículos
diarios que ocupan plazas de estacionamiento limitado. \sol{$\mathbf{(2.5)}$}

\probl Probar  que $E((X-C)^2)$ toma el valor mínimo respecto de un valor real $C$ para
 $C = \mathrm{E}(X).$ ?`Cuál es el valor del mínimo?



\probl  El número medio de personas que van a un local es $1000$, con una desviación típica
de $20$ ?`Cuál es el número de sillas necesarias para que sea seguro, con una probabilidad
no inferior a $0.75$ , que todos los asistentes podrán sentarse? \sol{$\mathbf{(1040)}$}

%%%\probl D'una urna formada per 3 bolles blanques i 2 negres
%%%s'extreuen tres bolles sense devolució. Obteniu:
%%%
%%%\begin{enumerate}[a)]
%%%\item La funció característica de la variable aleat\`oria que dóna el
%%%nombre de bolles negres extretes fins a obtenir la primera bolla
%%%blanca.
%%%\item L'esperan\c{c}a i la vari\`ancia de la variable aleat\`oria anterior.
%%%$\mathbf{(1/2,\quad 9/20)}$
%%% \end{enumerate}
%%%
%%%\probl Sigui $ X  $ una variable aleat\`oria amb densitat
%%%
%%%$$f(x) = \cases{1 & si $ 0 < x < 1$\cr 0 & si $ x \not \in
%%%(0,1)$\cr}$$
%%%
%%%\begin{enumerate}[a)]
%%%\item Calculau la seva funció característica.
%%%\item  Obteniu $\mathrm{E}(X)$ a partir de $\phi_{X}.$ $\mathbf{(1/2)}$
%%% \end{enumerate}
%%%
%%%\probl Sigui $X$ una variable aleat\`oria amb densitat
%%%
%%%$$f(x) = \cases{2x & si $ 0 < x < 1$\cr 0 & si $ x \not \in
%%%(0,1)$\cr}$$
%%%
%%%Determinau $ \Phi_{X}.$
%%%
%%%\probl Sigui $X$ una v.a. amb densitat $f_{X}(x)=p a e^{-a
%%%x}+(1-p) b e^{-b x}$ si $x\geq 0$ i zero en qualsevol altre cas on
%%%$0<p<1$ trobau $\Phi_{X}$.
%%%
%%%\probl Una v.a. $X$  té distribució gamma amb par\`ametres
%%%$\alpha>0,\lambda>0$ si la seva funció de densitat es
%%%
%%%$f_{X}(x)=\left\{\begin{array}{ll}\frac{\lambda (\lambda
%%%x)^{\alpha-1} e^{-\lambda x}}{\Gamma(\alpha)} & \mbox{ si } x>0\\
%%%0 & \mbox{en qu alsevol altre cas}\end{array}\right.$
%%%
%%%On $\Gamma(\alpha)=\int_{0}^{+\infty} x^{\alpha-1} e^x dx$  (amb
%%%$\alpha>0$. La funció gamma verifica les seg\"{u}ents propietats: a)
%%%$\Gamma(\frac{1}{2}=\sqrt{\pi}$; b)
%%%$\Gamma(\alpha+1)=\alpha\Gamma(\alpha)$ si $\alpha>0$; c)
%%%$\Gamma(n+1)=n!$ si $n\in \Z^{+}$
%%%
%%%  si $\alpha=m\in\Z^{+}$ la distribució gamma es diu distribució
%%%$m$-Erlang.
%%%
%%%\begin{enumerate}[a)]
%%%    \item Trobau la transformada de Laplace de la distribució gamma.
%%%\item Trobau la esperan\c{c}a  i la varian\c a de la distribució gamma.
%%%\end{enumerate}
\newpage

\begin{centerline}
{{\bf DISTRIBUCIONES NOTABLES}}
\end{centerline}
 %\textbf{Distribuciones notables}

 \begin{prob} Usar la desigualdad de Chebychef para estimar el número de veces
 que se tiene que lanzar al aire una moneda bien balanceada si queremos tener un probabilidad
 de al menos  $0.9$ de que la frecuencia relativa de caras esté comprendida entre $0.45$ y $0.55$
 $\mathbf{(1000)}$
 \end{prob}
%  
% \begin{resolucion}
%  
%  Sea $X$ el número de caras esn $n$ lanzamientos indpendientes de una moneda conprobabilidad de cara $p$.
%  La v.a. sigue una ley $B(n,\frac{1}{2})$. Sabemos que $E(X)=\frac{n}{2}$ y que $Var(X)=\frac{n}{4}$.
%  La frecuencia realtiva de caras en los $n$ lanzamientos  será $\frac{X}{n}$
% Nos piden que calculemos mediante la desigualdad de Chebychef el menor valor de $n$ tal que:

% $$P(0.45\leq \frac{X}{n}\leq 0.55)>0.9$$


% $P(0.45\leq \frac{X}{n}\leq 0.55)= P(n\cdot 0.45\leq X \leq n\cdot 0.55)=
% P(n\cdot 0.45- \frac{n}{2}\leq X - \frac{n}{2} \leq n\cdot 0.55- \frac{n}{2})= 
% P(n\cdot 0.45- \frac{n}{2}\leq X - \frac{n}{2} \leq n\cdot 0.55- \frac{n}{2})= $
%\end{resolucion}


%%%
%%%\end{document}
%%%% {\bf PROBABILITATS I PROCESSOS ALEATORIS. 2on TELEM\`ATICA}
%%%\newpage
%%%%\problemes=0
%%%\begin{centerline}
%%%{{\bf MOMENTS D'UNA V.A.}}
%%%\end{centerline}


%%%\probl  Un individu vol invertir un capital de mig milió de pts.
%%%en un negoci que té una rendibilitat del $50\%$, per\`o amb el
%%%possible risc de perdre tota la inversió. El seu assessor financer
%%%l'ha informat que aquest negoci té una probabilitat 0.8 de ser
%%%rendible. Quin és el benefici esperat? \sol{$\mathbf{(100.000)}$}
%%%
%%%\probl  Un joc se diu que és just si el guany esperat de cada
%%%jugador és 0. Dos jugadors A i B tiren un dau per torns, i guanya
%%%el primer que obté un 5. Cada jugador aposta una quantitat $ c_j \
%%%(j=1,2),  $ i el total se'l quedar\`a el guanyador. Si suposam que
%%%comen\c{c}a a jugar A, quina condició han de verificar $ c_1$ i $c_2 $
%%%perqu\`e el joc sigui just? \sol{$\mathbf{(6 c_2 = 5 c_1)}$}
%%%
%%% \probl   El preu per estacionament en un aparcament és
%%%de 75 pts. per a la primera hora o fracció, i de 60 pts. a partir
%%%de la segona hora o fracció. Suposem que el temps, en hores, que
%%%un vehicle qualsevol roman a l'aparcament se modelitza segons la
%%%funció de densitat
%%%
%%%$$f_{X}(x) = \cases{e^{-x} & si $ x \geq 0$\cr 0 & si $ x <
%%%0$\cr}$$
%%%
%%%Calculau l'ingrés mitj\`a per vehicle. \sol{$\mathbf{(109.8)}$}
%%%
%%%\probl   Sigui $ X  $ una variable aleat\`oria amb funció de
%%%densitat
%%%
%%%$$f(x) = \cases{0 & si $ x \leq 0$\cr 2(1+x)/9 & si $ 0 < x \leq
%%%1$\cr 2/3 & si $ 1 < x \leq 2$\cr 0 & si $ x > 2$\cr}$$
%%%
%%%Calculau $\mathrm{E}(X )$ i  $\mathrm{Var}(X)$.
%%%\sol{$\mathbf{(32/27,\quad 0.28)}$}
%%%
%%%\probl   Suposem que jugam a la ruleta. Sigui $p<1/2$ la
%%%probabilitat que surti un número vermell. Suposem que apostam a la
%%%par (vol dir que, si apostam $k$ d\`ol.lars, mos donen $k$ d\`ol.lars
%%%més els que hem apostat si surt vermell i perdem els $k$ d\`ol.lars
%%%que hem apostat si surt negre). La primera aposta és de 1 d\`ol.lar.
%%%Si guanyam, mos retiram. Si perdem, feim una segona aposta de 2
%%%d\`ol.lars. Si guanyam, mos retiram. Si perdem, apostam $2^2$
%%%d\`ol.lars, i així successivament. La $n$-\`essima aposta ser\`a de
%%%$2^{n-1}$ d\`ol.lars.
%%%
%%%\begin{enumerate}[a)]
%%%\item Provau que amb aquest sistema és segur que guanyarem 1 d\`ol.lar.
%%%\item Trobau l'import esperat de l'aposta guanyadora.
%%%\end{enumerate}
%%%
%%%
%%%Suposem ara que la casa té un límit de $2^L$ d\`ol.lars (el m\`axim
%%%que permet apostar), de manera que, si no hem guanyat abans,
%%%aquesta ser\`a la nostra aposta final.
%%%
%%%\begin{enumerate}[a)]
%%%\item  Quin és el guany esperat quan mos aturem de jugar?
%%%
%%%\item Qu\`e és millor un límit alt o baix?  \sol{1b)
%%%$\mathbf{(\infty)}$; 2a)$\mathbf{(1-(2(1-p))^{L+1})}$; 2b)baix }
%%%\end{enumerate}
%%%%%%%%va la 11
%%%
%%%\probl  El temps de vida, en anys, d'un cert component d'una
%%%m\`aquina se modelitza mitjan\c{c}ant la seg\"{u}ent funció de densitat
%%%
%%%$$f_{X}(x) = \cases{e^{-x} & si $ x \geq 0$\cr 0 & si $ x <
%%%0$\cr}$$
%%%
%%% Si el cost $ Y  $ de funcionament del component, en milions de
%%%pts., és funció del temps de vida, $ Y = 2 X^2 + 1,  $ calculau la
%%%quantitat que espera gastar l'empresa en concepte de manteniment.
%%%\sol{$\mathbf{(5)}$}
%%%
%%%%%%%%%%%\probl  Calculau l'esperan\c{c}a i la vari\`ancia de les variables
%%%%%%%%%%%aleat\`ories estudiades en el tema 2.
%%%
%%%\probl  Una dona a qui robaren la seva bossa va descriure el seu
%%%assaltant com un homo de 2m 10cm d'alt, amb un capell color
%%%taronja, camisa vermella, cal\c{c}ons verds i sabates grogues. Al cap
%%%d'una estona, una persona que responia a aquesta descripció va
%%%esser detinguda a uns quants blocs de dist\`ancia, i se la va acusar
%%%del robatori. Creis que hi ha prou proves perqu\`e sigui declarat
%%%culpable? Per qu\`e?\sol{No}

\probl   Sea $X$ el número de éxitos en $n$ repeticiones independientes de un experimento
con probabilidad de éxito $p$.
\begin{enumerate}[a)]
\item  Si $k$ es el valor más probable de $X$, probar que
 $$(n+1)p - 1 \leq k \leq (n+1)p$$
\item  Si lanzamos $10$ veces un dado bien balanceado, ?`cuál es el número de veces más probable
en el que obtendremos un $2$.\sol{b) $\mathbf{(1)}$}
\end{enumerate}

\probl  Una cadena de producción da salida a $10000$ unidades diarias,  el número medio de
unidades incorrectas es $200$. Una vez al día, se inspecciona un lote de $100$ unidades.
Determinar la probabilidad de que el lote contenga más de $3$ unidades incorrectas
\begin{enumerate}[a)]
\item Utilizando la distribución binomial.
\item Utilizando la aproximación de Poisson.
\sol{a) $\mathbf{(0.1410)}$ ; b)  $\mathbf{(0.1429)}$}
\end{enumerate}

\probl  En una planta de fabricación de circuitos integrados, la proporción de circuitos
defectuosos es $p$. Supongamos que la incidencia de circuitos defectuosos es completamente
aleatoria.
\begin{enumerate}[a)]
\item  Determinar la distribución del número $X$ de circuitos
aceptables producidos antes del primer circuito defectuoso.
\item  ?`Cuál es la longitud media de una cadena de producción exitosa? si
$p = 0.05.$\sol{b) $\mathbf{(19)}$}
\end{enumerate}

\probl Un servidor de mensajería esta en funcionamiento. Los clientes acceden a él de forma
independiente. La probabilidad de que el servidor caiga cuando accede el cliente es $p$.
Calcular la distribución de probabilidad del número de clientes a los que se dará servicio
antes de que el servidor caiga.

\begin{enumerate}[a)]
\item Calcular el valor esperado y la varianza de esta variable.
\item  ?`Cuál es la probabilidad de que se de servicio a más de $1000$ clientes sin que se
caiga el servidor?
\end{enumerate}
\sol{a) $\mathbf{(\frac{q}{p};\, \frac{q}p^2)}$ b) $\mathbf{(q^{1001})}$}

\probl Un sistema informático  dispone de un sistema de seguridad compuesto por tres claves
de $3$ dígitos (del $0$ al $9$) cada una. Para entrar en el sistema hay que averiguar la
primera clave, luego la segunda y por último la tercera. Un pirata informático intenta
entrar ilegalmente en el sistema, para ello va introduciendo al azar distintas claves de
forma independiente, olvidando las que ha introducido antes. Calcular el valor esperado y
la varianza del número de intentos antes de romper el sistema.

Comparar el resultado anterior cuando se ataca el sistema de forma similar pero cuando el
sistema de seguridad sólo consta de una clave de $9$ dígitos. ?`Cuál es el sistema más
seguro, desde el punto de vista del número de intentos necesarios para violarlo?




\probl De un grupo de $10$ personas se eligen $5$ de forma equiprobable y se les pregunta
si están a favor de una cierta ley. Sea $X$ la variable aleatoria que cuenta el número de
personas a favor de la ley entre las $5$. Supongamos que hay una persona a favor por cada
nueve en contra.

\begin{enumerate}[a)]
\item Calcular la función de probabilidad de $X$.
\item Calcular la esperanza y la varianza de $X$.
\item Supongamos ahora que la población es de tamaño $10000$ ?`podemos suponer un modelo
binomial para $X$?, en este caso ?`varía mucho la esperanza y la varianza con respecto al
modelo hipergeométrico?
\end{enumerate}




\probl  Los taxis llegan aleatoriamente (según un proceso Poisson) a la terminal de un
aeropuerto con un ritmo medio de un taxi cada $3$ minutos. ?`Cuál es la probabilidad de que
el último pasajero de una cola de $4$ tenga que esperar un taxi más de un cuarto de hora?
\sol{$\mathbf{(0.265)}$}


%%%%%%%%%%va la 12



\probl  La variable aleatoria $X$ sigue una ley  $\displaystyle N(\mu, \sigma^2). $ Sabemos
que   $ \mu = 5  \sigma,
 $ y  que $ \pr{X < 6} = 0.8413.$

\begin{enumerate}[a)]
\item Determinar la esperanza y la varianza de $X$.
\item ?`Cuál es la función de distribución de $ Y = 3
- X^2  $ y su esperanza? \sol{a)$\mathbf{(5,\quad 1)}$ ; b) $\mathbf{E(Y) = -23}$}
\end{enumerate}

%%%\probl  Sigui $ X  $ una variable aleat\`oria amb funció de
%%%densitat:
%%%
%%%$$f_{X}(x) = \cases{2x & si $x \in (0,1)$\cr 0 & si $x \not \in
%%%(0,1)$\cr}$$
%%%\begin{enumerate}[a)]
%%%\item Determinau $\mathrm{E}(\sqrt{X})$ a partir de $f_{\sqrt{X}}.$
%%%\sol{a) $\mathbf{(4/5)}$}
%%%\item El mateix a partir de $f_{X}.$
%%%\end{enumerate}




\probl  Consideremos una variable aleatoria $X$ con función de densidad $f_X$ dada por:
$$f_X(x) = \left\{\begin{array}{ll}
1/2 & \mbox{si }  0<x<2\\ 0 & \mbox{en otro caso}\end{array}\right.$$
 determinar
$\mathrm{E}( Y)$, donde $Y= \ln X.$ \sol{$\mathbf{(-0.3069)}$}

\probl  Sea $X$ una variable aleatoria continua con distribución uniforme sobre el
intervalo  $(0,1).$
\begin{enumerate}[a)]
\item Encontrar las funciones de densidad de las variables aleatorias
 $Y
= g(X)$ y $Z = h(X),$ donde $g(x) = 8 x^3$ y $h(x) = (x-1/2)^2.$
\item  Calcular la esperanza de las variables aleatorias $Y$ y
$Z$ como esperanzas de v.a. y como esperanzas de funciones de la v.a. $X$. \sol{b)
$\mathbf{(2,\quad 1/12)}$}
 \end{enumerate}





\probl  El percentil $90$ de una variable aleatoria $X$ es el valor $x_{90}$ para el que $
\displaystyle F_{X}(x_{90}) = \pr{X \leq x_{90}} = 0.9. $ De manera similar, el percentil
$50$ es el valor $x_{50}$ que satisface $ \pr{X \leq x_{50}} = 0.5 $ que recibe el nombre
de  {\bf mediana} (poblacional). Determinar estos dos valores para  una variable aleatoria
exponencial de valor medio $10$. \sol{$\mathbf{(23.0585,\quad 6.93147)}$}

\probl  Usar la desigualdad de Chebyshef para estimar el número de veces que hay  que
lanzar una moneda, no trucada, al aire si queremos tener una probabilidad de al menos
 el $90$\% de que la frecuencia de caras este comprendida entre $0.45$ y $0.55$.
\sol{$\mathbf{(1000)}$}

\probl  Una centralita recibe llamadas telefónicas con un ritmo medio de $\mu$ llamadas por
minuto. Calcular la probabilidad de que el intervalo entre dos llamadas consecutivas supere
al intervalo  medio en más de dos desviaciones típicas, comparar este resultado con la cota
superior dada por la desigualdad de Chebyshef \sol{$\mathbf{0.05,\quad 0.25}$}.




\newpage

\begin{centerline}
{{\bf Variables aleatorias vectoriales}}
\end{centerline}

\begin{prob}
Los estudiantes de una universidad se clasifican de acuerdo a sus años en la universidad
($X$) y el número de visitas a un museo el último año($Y=0$ si no  hizo ninguna visita
$Y=1$ si hizo una visita, $Y=2$ si hizo más de una visita). En la tabla siguiente aparecen
la probabilidades conjuntas que se estimaron para estas dos variables:

    $$\begin{tabular}{c|cccc}
        \hline\\
        Núm. de Visitas (Y) & &Núm. de años (X) & & \\
        \hline
          & 1& 2 & 3  & 4 \\
         \hline\\
         0 & 0.07 & 0.05 & 0.03 & 0.02\\
         1 & 0.13 & 0.11 & 0.17 & 0.15\\
         2 & 0.04 & 0.04 & 0.09 & 0.10\\
         \hline
        \end{tabular}
        $$
        \begin{enumerate}[a)]
            \item Hallar la probabilidad de que un estudiante elegido
            aleatoriamente no haya visitado ningún museo el último año.
            \item Hallar las medias de las variables aleatoria $X$ e $Y$.
            \item Hallar e interpretar la covarianza y la correlación
            entre las variables aleatorias $X$ e $Y$.
            \end{enumerate}
        \sol{a) 0.17; b) $E(X)= 2.59$, $E(Y)=1,10$; c) $Cov(X,Y)=0.191$,
        $r_{XY}$=0.259291}
        \end{prob}

\begin{prob}
    Un vendedor de libros  de texto realiza llamadas  a los despachos de
    lo profesores, y tiene la impresión que éstos suelen ausentarse
    más de los despachos los viernes  que cualquier otro día
    laborable. Un repaso a las llamadas, de las cuales un quinto se
    realizan los viernes, indica que  para el 16\% de las llamadas
    realizadas en viernes, el profesor no estaba en su despacho, mientras
    que esto ocurre sólo  para el 12\% de llamadas que se realizan en
    cualquier otro día laborable. Definamos las variables aleatorias
    siguientes:
    $$X=\left\{\begin{array}{ll}
    1 & \mbox{si la llamada es realizada el viernes}\\
    0 & \mbox{en cualquier otro caso}
    \end{array}\right.
    $$
    $$Y=\left\{\begin{array}{ll}
    1 & \mbox{si  el profesor está en el despacho}\\
    0 & \mbox{en cualquier otro caso}
    \end{array}\right.
    $$
    \begin{enumerate}[a)]
    \item Hallar la función de probabilidad conjunta de $X$ e $Y$.
    \item Hallar la función de probabilidad condicional de $Y$, dado
    que $X=0$.
    \item Hallar las funciones de probabilidad marginal de $X$ e $Y$.
    \item Hallar e interpretar la covarianza de $X$ e $Y$.
    \end{enumerate}
    \sol{\begin{tabular}{l|ll|l|}
     & & $X$ & \\
    \hline
    $Y$ & 0 & 1 & $P_{Y}(y)$\\
    \hline
    0 & 0.096 & 0.032 & 0.128\\
    1 & 0.704 & 0.168&  0.872\\
    \hline
    $P_{X}(x)$ & 0.8 & 0.2 & 1
    \end{tabular};
    b) \begin{tabular}{l|ll}
    $Y$ & 0 & 1 \\
    \hline
     $P_{Y/X}(y|0)$&0.12& 0.88
    \end{tabular}
    d) $Cov(X,Y)=-0.0064$}
    \end{prob}


    \begin{prob} Se lanzan al aire dos dados de diferente color, uno es blanco y el otro rojo.
     Sea $X$ la variable aleatoria "número de puntos obtenidos con el dado blanco, e $Y$
     la variable aleatoria "número más grande de puntos obtenido entre los dos dados".
\begin{enumerate}[a)]
\item Determinar la función de probabilidad conjunta.
\item Obtener las funciones de probabilidad marginales.
\item ?`Son independientes? (\textbf{No})
\end{enumerate}

\end{prob}


  \begin{prob}  Si $X_1 \mbox{ y } X_2 $ son dos  variables aleatorias con distribución Poisson,
   independientes y con medias respectivas $\alpha \mbox{ y } \beta$, probar que $Y = X_1 +
X_2 $ también una variable aleatoria Poisson (con media $\> \alpha + \beta$).

\end{prob}

    \begin{prob}
Las variables aleatorias continuas $X$ e $Y$ tienen por función de densidad

$$
f(x,y)=\left\{\begin{array}{ll} k(3x^2+2y) & \mbox{si } 0\leq x \leq 1 \mbox{ y }  0\leq y
\leq 1\\ 0 & \mbox{en cualquier otro caso}
\end{array}\right.
$$
Se pide:
\begin{enumerate}[a)]
    \item El valor de la constante $k$.
    \item Las funciones de densidad marginales.
    \item Las probabilidades $P(X\leq 0.5)$ y $P(Y\leq 0.3)$
    \item Las medias y las varianzas  de $X$ y de $Y$.
    \item La covarianza de $X$ e $Y$.
    \item La matriz de varianzas-covarianzas y la de correlaciones.
    \end{enumerate}
    \sol{a) $k=\frac{1}{2}$;
     b) $f_{X}(x)=\left\{\begin{array}{ll}
    \frac{1}{2} (3x^2+1) & \mbox{si } 0\leq x\leq 1\\
    0 & \mbox{en el resto de casos}\end{array}\right.$;
    $f_{Y}(y)=\left\{\begin{array}{ll}
    \frac{1}{2} (1+2y) & \mbox{si } 0\leq y\leq 1\\
    0 & \mbox{en el resto de casos}\end{array}\right.$;
    c) $P(X\leq 0.5)=0.312$ y $P(Y\leq 0.3)=0.195$;
    d) $E(X)=\frac{5}{8}$; $E(Y)=\frac{7}{12}$; $Var(X)=\frac{73}{960}$;
    $Var(Y)=\frac{11}{144}$.
    e) $Cov(X,Y)=\frac{-1}{96}$; f)$\left(\begin{array}{rr}
     \frac{73}{960} & \frac{-1}{96}\\
      \frac{-1}{96} & \frac{11}{144}\end{array}\right)$}
\end{prob}


\begin{prob}
    Los gastos $X$  e ingresos $Y$ de una familia se consideran como una
    variable bidimensional con función de densidad dada por:


$$
f(x,y)=\left\{\begin{array}{ll} k (x+y)  & \mbox{si } 0\leq x \leq 100 \mbox{ y }  0\leq y
\leq 100\\ 0 & \mbox{en cualquier otro caso}
\end{array}\right.
$$
Se pide:
\begin{enumerate}[a)]
    \item El valor de la constante $k$ para que $f(x,y)$ sea densidad.
    \item La probabilidad $P(0\leq X\leq 60, 0\leq Y\leq 50)$
    \item Las funciones de densidad marginales.
    \item Los gastos e ingresos medios.
    \item La covarianza de $X$ e $Y$.
    \item La matriz de varianzas-covarianzas.
    \item \textbf{Opcional.} La densidad (condicionada) de los gastos
    de las familias con ingresos $Y=50$. La esperanza de los gastos
    condicionados a que los ingresos valen $Y=50$.
    \end{enumerate}
    \sol{a) $k=\frac{1}{100^3}$; b) 0.165;
    c) $f_{X}(x)=\left\{\begin{array}{ll}
    \frac{x+50}{100^2} & \mbox{ si } 0<x<100\\
    0 & \mbox{ en el resto de casos}
    \end{array}\right.$, la de $Y$ es similar;
    d)$E(X)=E(Y)=\frac{175}{3}$;
    e) $Cov(X,Y)=\frac{-625}{9}$;
    f) $\left(\begin{array}{cc}
      \frac{6875}{9} & \frac{-625}{9}\\
      \frac{-625}{9} &  \frac{6875}{9} \end{array}\right)$;
    g) $f_{X/Y}(x|50)=
    \left\{ \begin{array}{ll}
\frac{1}{100^2} (x+50) & \mbox{si } 0\leq x \leq 100 \\ 0 & \mbox{en cualquier otro caso}
\end{array}\right.$;
 $E(X/Y=50)=\frac{175}{3}$}

\end{prob}


\begin{prob}
 Dos amigos desayunan cada mañana en una cafetería entre la 8 y las 8:30 de la mañana. La
distribución  conjunta de sus tiempos de llegada es uniforme en dicho intervalo, es decir
(y para simplificar tomando el tiempo en minutos):
$$f(x,y)=\left\{\begin{array}{ll}
k  & \mbox{ si } 0\leq x \leq 30 \mbox{ y } 0\leq y \leq 30\\ 0 & \mbox{ en el resto de
casos}
\end{array}\right. .$$
Si los amigos esperan un máximo de 10 minutos, calcular la probabilidad de que se
encuentren (sugerencia: resolverlo gráficamente). Calcular las distribuciones marginales.
\sol{(Para la solución completa véase Daniel Peña ``\textit{Estadística Modelos y Métodos.
Vol 1. 2 Ed pág.160})  \  $\frac{5}{9}$; las marginales son uniformes en el intervalo
$(0,30)$ }

\end{prob}

% \begin{prob}
% (*)En el problema anterior calcular la función de
% distribución conjunta, las densidades y distribuciones marginales.
% \end{prob}

\begin{prob}
La variable $X$ representa la proporción de errores tipo A en ciertos documentos y la
variable $Y$ la proporción de errores de tipo B. Se verifica que $X+Y\leq 1$ ( es decir
puede haber más tipos de errores posibles) y la densidad conjunta de ambas variables es
$$f(x,y)=\left\{\begin{array}{ll}
k  & \mbox{ si } 0\leq x \leq 1\mbox{; } 0\leq y\leq 1\mbox{ y } x+y\leq 1\\ 0 & \mbox{ en
el resto de casos}
\end{array}\right. .$$
\begin{enumerate}[a)]
\item Calcular el valor de la constante $k$.
\item \textbf{Opcional.} Calcular la densidad condicional de $X$ a $Y=y_{0}$  con
$0<y_{0}<1$.
\item \textbf{Opcional.}  Calcular la esperanza de la variable condicionada del
apartado anterior.
\item Calcular el vector de medias y  la matriz de correlaciones de $(X,Y)$
\end{enumerate}
\sol{a) $k=2$; b) Dado $0<y_{0}<1$ la densidad condicional de $X$ a $Y=y_{0}$ es
$f_{X|Y}(x|y_{0})= \left\{\begin{array}{ll} \frac{1}{(1-y)} & \mbox{si } 0<x<1 \mbox{y }
x<1-y_{0}\\ 0 & \mbox{en el resto de casos}
\end{array}\right.$; c) $\frac{1}{3}$}
\end{prob}

% \begin{prob} En un proceso administrativo ciertos documentos se
% clasifican como: sin errores ($A_{1}$), con errores leves ($A_{2}$) y
% con errores graves ($A_{3}$). Se ha estimado que $p(A_{1})=0.7$,
% $P(A_{2})=0.2$ y $P(A_{3})=0.1$. Utilizando la distribución
% multinomial responder a las siguientes preguntas:
% \begin{enumerate}[a)]
% \item Si se toman al azar 3 documentos, calcular la probabilidad de que
% haya sólo uno de la clase $A_{3}$.
% \item En una muestra de 7 documentos se obtienen 5 sin errores.
% ?`Cuál es la probabilidad de que en dicha muestra haya 1 documento
% con errores graves?
% \end{enumerate}
% \sol{pág 172 Peña}
% \end{prob}

% teorema central del límite
% \begin{prob}
%   En el comienzo del invierno, la propietaria de un piso estima la
%   probabilidad de que su factura de calefacción  para todo el invierno
%   sea inferior a $38000$ ptas. es $0.4$. Además estima que la
%   probabilidad de que la factura sea inferior a $46.000$ ptas. es $0.6$.
%   \begin{enumerate}[a)]
%       \item ?`Cuál es la probabilidad de que la factura esté entre
%       $38000$ y $46000$ pesetas?
%       \item Sin conocer más información  ?`qué puede decirse sobre
%       la probabilidad de que la factura sea inferior a $40000$ pesetas?
%       \end{enumerate}
%       \sol{a) $0.2; b)
%   \end{prob}

\begin{prob}  La función de densidad conjunta de dos variables aleatorias  continuas
es:
$$f (x,y) = \cases{k(x+xy) & si $\> (x,y) \in (0,1)^2$\cr 0 & en otro caso\cr}$$

\begin{enumerate}[a)]
\item Determinar $\> k.$ ($\mathbf{4/3}$)
\item Encontrar las funciones de densidad marginales.
\item ?`Son independientes? (\textbf{ Sí})
\end{enumerate}
\end{prob}

\begin{prob} Las variables aleatorias $X_1 \mbox{ y  } X_2$ son independientes y ambas tienen la misma densidad
$$f(x) = \cases{1 & si $0 \leq x \leq 1$\cr 0 & en caso  contrario\cr}$$
\begin{enumerate}[a)]
\item Determinar la densidad de $Y = X_1 + X_2.$
\item Determinar la densidad de $Z = X_1 - X_2.$
\item Calcular la esperanza y varianza de $Y$ y de $Z$.
\end{enumerate}

\end{prob}


\begin{prob} Un proveedor de servicios informáticos tiene  una cantidad $X$ de cientos de unidades
de un cierto producto al principio de cada mes. Durante el mes se venden $Y$ cientos de
unidades del producto. Supongamos que  $X$
 e $Y$ tienen una densidad conjunta dada por

$$f(x,y) = \cases{2/9 & si $\> 0 < y < x < 3$\cr 0 & en caso
contrario\cr}$$
\begin{enumerate}[a)]
\item Comprobar que {\it f} es una densidad.
\item Determinar $F_{X,Y}.$
\item \textbf{Opcional.}  Calcular la probabilidad de que al final de mes se hayan vendido como mínimo la mitad de las unidades
que había inicialmente. ($\mathbf{1/2}$)
\item \textbf{Opcional.}  Si se han vendido 100 unidades, ?`cuál es la probabilidad de que hubieran, como mínimo 200
a principio de mes? ($\mathbf{1/2}$)
\end{enumerate}
\end{prob}

\begin{prob} \textbf{Opcional.}  Sean $X$ e $Y$ dos variables aleatorias conjuntamente absolutamente continuas.
Supongamos que

$$ f_{X}(x) = \cases{4x^3 & si $\> 0 < x < 1$\cr 0 & en caso
contrario\cr}$$

y que

$$ f_{Y}(y|x) = \cases{{2y \over x^2} & si $\> 0 < y < x$\cr 0 &
en caso contrario\cr}$$
\begin{enumerate}[a)]
\item  Determinar $f_{X,Y}$.
\item Obtener la distribución de $Y.$
\item  Calcular $f_{X}(x|y).$
\end{enumerate}

\end{prob}


\begin{prob}  Sea $W = X + Y + Z$, donde  $X$, $Y$ y $Z$ son variables aleatorias con media
0 y varianza 1,
\begin{enumerate}[a)]
\item Sabiendo que  $ Cov(X,Y) = 1/4, Cov(X,Z) = 0, Cov(Y,Z) = -1/4.$
calcular la esperanza y la varianza de $W$. (Sol.:$\mathbf{0; 3}$)
\item Sabiendo  que $X, Y$ y $Z$ 
son incorreladas calcular la esperanza y la varianza de $W$. (Sol.:$\mathbf{0; 3}$)
\item Calcular la esperanza y la varianza de $W$ si  $ Cov(X,Y) = 1/4, Cov(X,Z) = 1/4,
 Cov(Y,Z) = 1/4.$ (Sol.:$\mathbf{0; 15/4 }$)
\item Sabiendo  que $X, Y$ y $Z$ con independientes calcular $Var(W)$ y $E(W)$.
\end{enumerate}
\end{prob}

\begin{centerline}
{{\bf Teorema del Límite Central}}
\end{centerline}


\begin{prob}% teorema central del límite
En cierta fabricación mecánica el 96\% de las piezas resultan con longitudes admisibles
(dentro de las toleradas), un 3\% defectuosas cortas y un 1\% defectuosas largas. Calcular
la probabilidad de:

\begin{enumerate}[a)]
\item En un lote de 250 piezas sean admisibles 242 o más
\item En un lote de 500 sean cortas 10 o menos.
\item En 1000 piezas haya entre 6 y 12 largas.
\end{enumerate}
%%\sol{Aproximando  por el T.C.L. a) 0.9131; b) 0.119; c) 0.7088}
\end{prob}

\begin{prob} Una organización de investigación de mercados ha
encontrado que el 40\% de los clientes de un supermercado no quieren contestar cuando son
encuestados. Si se pregunta a 1000 clientes, ?`cuál es la probabilidad de que menos de 500
de ellos se nieguen a contestar? \sol{Aproximando por el T.C.L. prácticamente es 1}

\end{prob}

\begin{prob}
Un servicio de grúa de auxilio en carretera recibe diariamente una media de 70 llamadas.
Para un día cualquiera, ?`cuál es la probabilidad de que se reciban menos de 50 llamadas?
\sol{Aproximadamente 0.3897}
\end{prob}


\begin{prob}   Supongamos que  el 10\% de los votantes,
 de un determinado cuerpo electoral, están a favor de una cierta legislación. Se hace  una
encuesta entre la población y se obtiene una frecuencia relativa $f_n(A)$ que estima la
proporción poblacional anterior. Determinar, aplicando la desigualdad de Cheb. ¿Cuántos
votantes se tendrían que encuestar para que  la probabilidad de que $f_n(A)$ difiera de $
0.1$ menos de $0.02$ sea al menos de $0.95$? (Sol.:$\mathbf{4500}$.)?`Qué podemos decir si no
conocemos el valor de la proporción? (Sol.:$\mathbf{12500}$) Repetir el ejercicio pero
aplicando el T.L.C. (Sol.:$\mathbf{865}$ si sabemos que  $\mathbf{p=0.1}$ y $\mathbf{2401}$
en otro caso)
\end{prob}

\begin{prob}  Se lanza al aire una dado regular $100$ veces. Aplicar la desigualdad de Cheb.
para obtener una cota de la probabilidad de  que el número total de puntos obtenidos esté
entre $300$ y $400$. (Sol.:$\mathbf{0.883}$). ?`Cuál es la probabilidad que se obtiene
aplicando el T.L.C? (Sol.:$\mathbf{0.9964}$)
\end{prob}

\begin{prob} Se sabe que, en una población, la altura de los individuos machos adultos es
 una variable aleatoria $X$ con media $\mu_x = 170$ cm y desviación típica
 $\sigma_x = 7 $ cm. Se toma una muestra aleatoria  de 140 individuos.
  Calcular la probabilidad de que la media muestral
$\overline{x}$ difiera de $\mu_x$ en menos de $1$ cm. (Sol.:$\mathbf{0.909}$)
\end{prob}
\begin{prob}  ?`Cuántas veces hemos de lanzar un dado bien balanceado para tener como mínimo un  $95$\% de  certeza de que la frecuencia
 relativa del salir ``6'' diste menos de $0.01$ de la
probabilidad teórica $1/6$? (Sol.:$\mathbf{5336}$)
\end{prob}
\begin{prob}
Se lanza al aire una moneda sin sesgo $n$ veces. Estimar el valor de  $n$ de manera que la
frecuencia relativa del número de caras difiera de $1/2$ en menos de $0.01$ con
probabilidad $0.95$. (Sol.:$\mathbf{9604}$)
\end{prob}

\begin{prob} El número de mensajes llegan a un multiplexor es una variable aleatoria que
sigue una ley  Poisson con una  media de $10$ mensajes por segundo. Estimar la probabilidad
de  que lleguen más de $650$ mensajes en un minuto. (Ind.: Utilizar el teorema del límite
central) (Sol.: El valor exacto es $\mathbf{0.0207}$, aproximando por el TLC con corrección de continuidad $\mathbf{0.0197}$, aproximando por TLC sin correción de continuidad  $\mathbf{0.0206}$.)
\end{prob}




%%%\probl  D'una urna formada per 3 bolles blanques i 2 negres
%%%s'extreuen tres bolles sense devolució. Obteniu:
%%%
%%%\begin{enumerate}[a)]
%%%\item La funció característica de la variable aleat\`oria que dóna el
%%%nombre de bolles negres extretes fins a obtenir la primera bolla
%%%blanca.
%%%\item L'esperan\c{c}a i la vari\`ancia de la variable aleat\`oria anterior.
%%%\sol{b) $\mathbf{(1/2,\quad 9/20)}$}
%%% \end{enumerate}
%%%
%%%\probl  Sigui $ X  $ una variable aleat\`oria amb densitat
%%%
%%%$$f(x) = \cases{1 & si $ 0 < x < 1$\cr 0 & si $ x \not \in
%%%(0,1)$\cr}$$
%%%
%%%\begin{enumerate}[a)]
%%%\item Calculau la seva funció característica.
%%%\item  Obteniu $\mathrm{E}(X)$ a partir de $\phi_{X}.$
%%%\sol{$\mathbf{(1/2)}$}
%%% \end{enumerate}
%%%
%%%\probl  Sigui $ X  $ una variable aleat\`oria amb densitat
%%%
%%%$$f(x) = \cases{2x & si $ 0 < x < 1$\cr 0 & si $ x \not \in
%%%(0,1)$\cr}$$
%%%
%%%Determinau $ \phi_{X}.$
%%%
%%%\probl  Trobau la funció característica de una v.a. Laplaciana.
%%%Trobau la mitjana i la variancia utilitzant al funció
%%%característica.
%%%
%%%\probl Idem que l'anterior per una v.a. normal de praamtres $\mu$
%%%i $\sigma^2$.
%%%
%%%\probl Trobau la funció característica de una v.a. amb distribució
%%%de Cauchy. Demostrau que no existeix la seva espera\c{c}a ni la seva
%%%variancia.
%%%
%%%\probl Calculau la funció gernadora de probailitats  d'una v.a.
%%%Geom\`etrica. Calculeu la seva espranza i la seva variancia
%%%utilitzant la funció generadora de probabilitats.
%%%
%%%\probl Idem que l'anterior per una v.a. binomial.
%%%
%%%\probl Idem que l'anterior per una v.a. Poisson.
%%%
%%%\probl Trobau $P(X=r)$ per a una v.a. binomial negativa amb
%%%probabilitat de exit $p$ y nombre de exits $r$, sabent que la seva
%%%funció generadora de probabilitats es
%%%$G_X(z)=\left(\frac{pz}{1-qz}\right)^r$
%%%
%%%\probl Trobau el moment d'ordre $n$  de una v.a.  amb distribució
%%%Gamma utilizant la seva transformada de Laplace.


%%%%%%%%\probl La transformada de Laplace d'una certa v.a. es
%%%%%%%%
%%%%%%%%$$X^{*}(z)=\left(\frac{a}{s+a}\right)\left(\frac{b}{s+b}\right)$$
%%%%%%%%
%%%%%%%%
%%%%%%%%Trobau la seva funció de densitat. (Indicació:

%\end{document}%%%%%primera entrega

\newpage

\textbf{Muestreo. Distribuciones muestrales}

\begin{prob}
    El precio medio del $m^2$ en  la venta de casas nuevas durante el último año
    en una determinada ciudad  fue de 115000 pts. La desviación
    típica de la población fue de 25000 pts. Se toma una muestra
    aleatoria de 100 casas nuevas de esta ciudad.
    \begin{enumerate}[a)]
        \item ?`Cuál es la probabilidad de que la media muestral de los
        precios de venta sea menor que 110000 pts?
        \item ?`Cuál es la probabilidad de que la media muestral de los
        precios de venta esté entre 113000 pts. y 117000 pts.??
        \item?`Cuál es la probabilidad de que la media muestral de los
        precios de venta esté entre 114000 pts. y 116000 pts.?
        \item Sin hacer cálculos, razonar en cuál de los siguientes rangos
        resulta más probable que se encuentre la media muestral de los
        precios de venta:\break
        \begin{center}
        \begin{tabular}{ll}
        113000 pts-&115000 pts\\
        114000 pts-&116000 pts\\
        115000 pts-&117000 pts\\
        116000 pts-&118000 pts
            \end{tabular}

            \end{center}
        \end{enumerate}
    \sol{a) 0.0228; b) 0.5762; c) 0.3108: d) el intervalo  114000 pts-116000 pts}
    \end{prob}
    \begin{prob}
        Se ha tomado una muestra de 16 directores de oficina  de
        corporaciones de una gran ciudad, con el fin de estimar el tiempo
        medio que emplean en desplazarse para ir a su trabajo. Supongamos
        que la distribución de dichos tiempos en la población sigue una
        normal con media 87 minutos y desviación típica  22.
        \begin{enumerate}[a)]
\item ?`Cuál es el error estándar de la media muestral de los
tiempos de desplazamiento?
\item ?`Cuál es la probabilidad de que la media muestral sea
inferior a 100 minutos?
\item ?`Cuál es la probabilidad de que la media muestral sea
superior  a 80 minutos?
\item ?`Cuál es la probabilidad de que la media muestral esté entre
85 y 95 minutos?
\item Supongamos que se toma una segunda muestra de quince directores,
independiente de la anterior. Sin hacer los cálculos, razonar si la probabilidades
calculadas en los apartados b), c) y d)  serán mayores, menores o iguales para esta segunda
muestra. Utilizar gráficos para ilustrar las respuestas.
        \end{enumerate}
        \sol{a) 5.5 ; b) 0.9909; c) 0.8980; d) 0.5671; e) es menor en los
        tres casos.}
    \end{prob}

\begin{prob}
    Una compañía produce cereales para el desayuno. La media del
    peso que contienen las cajas de estos cereales es de doscientos
    gramos y su desviación típica de seis gramos. La distribución de
    los pesos de la población es normal. Se eligen cuatro cajas, que
    pueden considerarse como una muestra aleatoria del total de la
    producción.
    \begin{enumerate}[a)]
        \item ?`Cuál es el error estándar de la media muestral del peso de
        las cuatro cajas?
        \item ?`Cuál es la probabilidad de que la media del peso de esas cuatro cajas
        sea inferior  que 197 gramos?
        \item ?`Cuál es la probabilidad , en media, el peso de estas
        cuatro cajas esté entre 105 y 195 gramos?
        \item ?`Cuál es la probabilidad de que la suma del peso de estas
        cuatro cajas sea menor de 800 gr.?
        \item Se eligen al azar dos de estas cuatro cajas ?`Cuál es la
        probabilidad de que, en media, el contenido de estas dos cajas
        pese entre 195 y 200 gramos?
        \end{enumerate}
        \sol{a) 3; b) 0.1587; c) 0.0475; d) 0.5; e) 0.3810}
    \end{prob}

    \begin{prob} La tasa de rentabilidad de ciertos tipos de acciones
        sigue una distribución con desviación típica 3.8. Se extrae una
        muestra de tales acciones con el fin de estimar el precio medio.
        \begin{enumerate}[a)]
        \item ?`Qué tamaño ha de tener la muestra para asegurarnos que la
        probabilidad de que la media muestral difiera de la media
        poblacional en una cantidad superior a 1 sea menor que 0.1?
        \item Sin realizar los cálculos razonar si será preciso  un tamaño
        muestral mayor o menor que el requerido  en el apartado a) para
        garantizar que la probabilidad de que la media muestral difiera
        de la media poblacional en más de 1 sea inferior a 0.05.
        \item Sin realizar los cálculos razonar si será preciso  un tamaño
        muestral mayor o menor que el requerido  en el apartado a) para
        garantizar que la probabilidad de que la media muestral difiera
        de la media poblacional en más de 1.5 sea inferior a 0.1.
        \end{enumerate}
        \sol{a) $n\geq 40$; b) mayor; c) menor}
    \end{prob}

    \begin{prob}
        De acuerdo con los datos del ministerio de Economía y Hacienda, el
        15\% de las declaraciones del IRPF del último año darán lugar a
        una devolución. Se toma una muestra aleatoria de 10 declaraciones.
        \begin{enumerate}[a)]
        \item ?`Cuál es la media de la distribución en el muestreo de la
        proporción muestral de declaraciones que darán lugar a una
        devolución?
        \item ?`Cuál es la varianza de la proporción muestral?
        \item ?`Cuál es el error estándar de la proporción muestral?
        \item ?`Cuál es la probabilidad de que la proporción muestral sea
        mayor que 0.8?
        \end{enumerate}
        \sol{a) 0.15; b) 0.01275; c) 0.1129; d) casi nula.}
        \end{prob}

        %PPPPPPPPPPPPP

        \begin{prob}
    El dueño de una portal de ventas de discos por internet ha comprobado que el 20\% de los
    clientes que acceden a su portal realizan una compra. Cierta mañana
    entraron en el portal 180 personas, que pueden ser consideradas como
    una muestra aleatoria de todos sus clientes.
        \begin{enumerate}[a)]
        \item ?`Cuál será la media de la proporción muestral de clientes
        que realizaron alguna compra?
        \item ?`Cuál es la varianza de la proporción muestral?
        \item ?`Cuál es el error estándar de la proporción muestral?
        \item ?`Cuál es la probabilidad de que la proporción muestral sea
        mayor que 0.15?
        \end{enumerate}
    \sol{a) 0.2; b) $\approx 0.0009$; c) $0.03$; d) 0.9525}
\end{prob}

                \begin{prob}
    El administrador de una gran cadena de hospitales opina que, de
    entre todos sus pacientes, el 30\% generará facturas que se pagarán
    con más de dos meses de retraso. Se toma una muestra aleatoria de 200
    pacientes.
        \begin{enumerate}[a)]
        \item ?`Cuál es el error estándar de la proporción muestral de
        pacientes con facturas cuyo pago se retrasará dos meses?
        \item ?`Cuál es la probabilidad de que la proporción muestral sea
        inferior a   0.25?
        \item  ?`Cuál es la probabilidad de que la proporción muestral sea
        mayor que 0.33?
        \item  ?`Cuál es la probabilidad de que la proporción muestral esté
        entre 0.27 y 0.33?
        \item Sin realizar los cálculos, razonar en cuál de los siguientes
        intervalos es más probable que se encuentre la proporción
        muestral: 0.29-0.31; 0.30-0.32; 0.31-0.33; 0.32-0.34.
        \item Supongamos que se toma al azar una muestra de 500 pacientes.
        Sin realizar los cálculos razonar si las probabilidades de los
        apartados b), c)  y d) resultarán en este caso mayores, menores o iguales
        que las calculadas para la muestra anterior.
        \end{enumerate}
        \sol{a) 0.0324; b) 0.0618; c) 0.1762; d) 0.6476; e) 0.29-0.31;  f)
        menor, menor, mayor}
        \end{prob}

\begin{prob}
    Se toma una muestra aleatoria de 100 votantes con el fin
    de estimar la proporción de los mismos que están a favor de
    un aumento en los impuestos sobre la gasolina para contar
    así con un ingreso adicional para reparaciones de las
    autopistas.  ?`Cuál es el mayor valor que puede tomar el
    error estándar de la proporción muestral de esta medida?
    \sol{0.05}
    \end{prob}

\begin{prob}
    Continuando en la situación del problema anterior, se decide que
    una muestra de 100 votantes es muy pequeña para obtener una
    estimación de la proporción poblacional que resulte
    suficientemente creíble. Se decide exigir que la probabilidad de
    que la proporción muestral difiera de la proporción poblacional
    (cualquiera que sea su valor) en más de 0.03 no debe ser superior
    a 0.05. ?`Qué tamaño ha de tener la muestra para poder garantizar
    que se cumple este requisito?
    \sol{$n\geq 757$}
    \end{prob}

    \begin{prob}
Una compañía quiere estimar la proporción de personas que son posibles compradores de
máquinas de afeitar eléctricas que ven retransmisiones partidos de La Liga de Campeones. Se
toma una muestra de 120 individuos que se identificaron como posibles compradores de
afeitadoras eléctricas. Supongamos que la proporción de posibles compradores de afeitadoras
eléctricas  en la población que ven estas retransmisiones es 0.25.
\begin{enumerate}[a)]
    \item 0.10 es la probabilidad de que la proporción muestral
    exceda a la proporción poblacional ?`en qué valor?
        \item 0.05 es la probabilidad de que la proporción muestral
    esté por debajo de la proporción poblacional ?`en qué cantidad?
        \item 0.30 es la probabilidad de que la proporción muestral
    difiera de la proporción poblacional ?`en menos de qué cantidad?
    \end{enumerate}
    \sol{a)  0.0506; b) 0.0648; c) 0.0154}
    \end{prob}
    \begin{prob}
        Supongamos que el 50\% de los españoles adultos opina que es
        necesaria una revisión del sistema nacional público de hospitales.
        ?`Cuál es la probabilidad de que más del 56\% de los componentes
        de una muestra de 150 españoles adultos tenga esa opinión?
            \sol{ 0.0708}
        \end{prob}

        \begin{prob} Las rentabilidades mensuales de cierto tipo de
        acciones son independientes unas de otras, y siguen una
        distribución normal con desviación típica 1,7. Se toma una
        muestra de 12 meses.
        \begin{enumerate}[a)]
            \item Hallar la probabilidad de que la desviación típica muestral
            sea menor que 2.5.
            \item Hallar la probabilidad de que la desviación típica
            muestral sea mayor que 1.
            \end{enumerate}
            \sol{a) 0.8; b) 0.975}
            \end{prob}

        \begin{prob} El número de horas que dedican a ver la televisión
        los estudiantes en la semana anterior a los exámenes finales
        sigue una distribución normal con una desviación típica de 4.5
        horas. Se toma una muestra aleatoria de 30 estudiantes.

        \begin{enumerate}[a)]
            \item La probabilidad  de que la desviación típica muestral sea
            mayor que 3.5 horas, ?`es mayor que 0.95?
            \item La probabilidad de que la desviación típica muestral sea
            menor que seis horas, ?`es mayor que 0.95?
            \end{enumerate}
            \sol{ a) Sí; b) Sí }
            \end{prob}
                \begin{prob} Se extrae una muestra aleatoria de 15 economistas
                    y se les pregunta acerca de su predicción sobre la tasa de
                    inflación para el próximo año. Supongamos que las predicciones
                    para la población completa de economistas sigue una
                    distribución normal con una desviación típica de 1.8.
                \begin{enumerate}[a)]
            \item 0.01 es la probabilidad de que la desviación típica sea
            mayor que ?`qué número?
            \item 0.025 es la probabilidad de que la desviación típica sea
            menor que ?`qué número?
            \item Encontrar una par de números, a y b, tales que la
            probabilidad de que la desviación típica muestral se encuentre
            entre ellos sea 0.9.
            \end{enumerate}
            \sol{a) 2.5969; b) 1.1415; c) 1.2331; 2.341}
            \end{prob}


\newpage
\textbf{Estimación puntual}

\begin{prob}
    Se toma una muestra de ocho lotes de un producto químico para
    comprobar la concentración de impurezas. Los niveles porcentuales
    de impurezas encontrados en la muestra fueron
    $$3.2 \ 4.3 \ 2.1 \ 2.8\ 3.2 \ 3.6\ 4.0\ 3.8$$
    \begin{enumerate}[a)]
        \item Hallar la media y la varianza muestrales.
        Hallar la proporción muestral de lotes con nivel
        porcentual de impurezas mayor que 3.75\%.
        \item ?`Para qué parámetros poblacionales  se han hallado en la
        parte  a) estimadores por procedimientos insesgados?
        \end{enumerate}
        \sol{a) $\overline{X}=3.375$; $S_{X}^2=0.4993$; $\hat{p}=
        0.3557$; b) para todos.}
\end{prob}
% \begin{prob}
%   Una muestra de diez economistas hicieron las siguientes predicciones
%   para el crecimiento del producto interior bruto el próximo año.
%
%
%   \begin{tabular}{lllll}
%       2.2 & 2.82& 2.02 & 2.52& 2.42\\
%       2.62 & 2.52& 2.42 & 2.72& 2.6
%       \end{tabular}
%   \end{prob}

\begin{prob}
        Sea $\hat{\theta}_{1}$ un estimador insesgado de $\theta_{1}$, y
        $\hat{\theta}_{2}$ un estimador insesgado de $\theta_{2}$.
        \begin{enumerate}[a)]
            \item Probar que $(\hat{\theta}_{1}+\hat{\theta}_{2})$ es un
            estimador insesgado de $(\theta_{1}+\theta_{2})$.
            \item  Probar que $(\hat{\theta}_{1}-\hat{\theta}_{2})$ es un
            estimador insesgado de $(\theta_{1}-\theta_{2})$.
            \end{enumerate}
        \end{prob}

\begin{prob}
    Sea $X_{1}$ y $X_{2}$ una muestra aleatoria de dos observaciones
    independientes de
    una población con media $\mu$ y varianza $\sigma^2$. Considerar los
    siguientes tres estimadores puntuales de $\mu$:
    $$\begin{array}{l}
    \overline{X}=\frac{1}{2} X_{1}+\frac{1}{2} X_{2}\\
    \hat{\mu}^{(1)}=\frac{1}{4}X_{1}+\frac{3}{4} X_{2}\\
    \hat{\mu}^{(2)}=\frac{1}{3}X_{1}+\frac{2}{3} X_{2}
\end{array}
    $$
        \begin{enumerate}[a)]
        \item Probar que los tres estimadores son insesgados.
        \item ?`Cuál de los tres estimadores es más eficiente?
        \item Hallar la eficiencia relativa de $\overline{X}$ con respecto
        a los otros estimadores.
        \end{enumerate}
        \sol{ b) $\overline{X}$; c) $\frac{Var(\overline{X})}{Var(\hat{\mu}^{(1)})}=0.8$;
          $\frac{Var(\overline{X})}{Var(\hat{\mu}^{(2)})}=0.9$}
    \end{prob}

    \begin{prob}
        A una clase de estadística asisten estudiantes de Informática de Gestión y de
        Sistemas. En una muestra de
         diez estudiantes de Gestión  se observaron las
         siguientes calificaciones en el examen final
         $$62 \ 57 \ 85 \ 59\ 64\ 63\ 71\ 58\ 77\ 72$$
         En una muestra independiente de ocho estudiantes de Sistemas se
         observaron las siguientes calificaciones en el mismo examen
         $$73 \ 79 \ 85 \ 73\ 62\ 51\ 60\ 57$$
         \begin{enumerate}[a)]
             \item Utilizar un método de estimación insesgado para
             obtener una estimación puntual de la diferencia de las
             calificaciones medias entre los estudiantes de Gestión y
             los de Sistemas. (Indicación: Utilizar el problema 151)
             \item Utilizar un método de estimación insesgado para obtener una
             estimación puntual de la diferencia entre la proporción poblacional de
             estudiantes que obtuvieron una calificación  mayor que 70 en el
             grupo de estudiantes de Gestión y el grupo de Sistemas.
               (Indicación: Utilizar el problema 151)
             \end{enumerate}
             \sol{a) 0.2444; b)$-\frac{1}{10}$}
        \end{prob}
    \begin{prob}
Se toma una muestra aleatoria $X_{1},X_{2},\ldots,X_{n}$  de una población con media $\mu$
y varianza $\sigma^2$. Se considera el siguiente estimador de $\mu$ :
$$\hat{\mu}=\frac{2}{n(n+1)}(X_{1}+2X_{2}+3X_{3}+\ldots+nX_{n})$$
\begin{enumerate}[a)]
    \item Probar que $\hat{\mu}$ es un estimador insesgado de $\mu$.
    \item Hallar la eficiencia relativa de $\hat{\mu}$  respecto a
    $\overline{X}$, la media muestral.
    \end{enumerate}
    \sol{ b) $Var(\hat{\mu})=\left(\frac{2}{n(n+1)}\right)^2 \frac{n (n+1) (2n+1)}{6}
    \sigma^2$; $Eff.rel=\frac{Var(\hat{\mu})}{Var(\overline{X})}=\frac{2\,\left( 1 + 2\,n \right) }
  {3\,\left( 1 + n \right)}.$ }
        \end{prob}

\begin{prob}
\begin{enumerate}[a)]
\item (Examen junio 2003) Calcular el estimador máximo verosímil (MLE\footnote{Del inglés
 ``Maximun Likelihood  Estimator''}) para el parámetro $\lambda$ de una población que
sigue una ley $Exp(\lambda)$ para una muestra aleatoria simple de tamaño $n$.
\item (Examen septiembre 2004) Calcular el MLE para el parámetro $\lambda$ de una población
que sigue una ley $Po(\lambda)$ para una muestra aleatoria simple de tamaño $n$.
\item  Calcular el MLE para el parámetro $\mu$ de una población
que sigue una ley $N(\mu,\sigma^2)$ para una muestra aleatoria simple de tamaño $n$.
\item  Calcular el MLE para el parámetro $\sigma^2$ de una población
que sigue una ley $N(\mu,\sigma^2)$ para una muestra aleatoria simple de tamaño $n$.
\item Estudiar si los estimadores MLE de los apartados anteriores son insesgados.
\end{enumerate}
\end{prob}




\textbf{Estimación por intervalos}

\begin{prob}
De una población de barras de hierro se extrae una muestra de $64$ barras y  se calcula la
resistencia a la rotura por tracción se obtiene que $\overline{X}=1012\ Kg/cm^{2}$. Se sabe
por experiencia que en este tipo de barras $\sigma=25$. Calcular un intervalo de confianza
para $\mu$ al nivel 0.95. \sol{$\left(1005.88 , 1018.13\right)$}
\end{prob}

\begin{prob}
Para investigar el C.I. medio de una cierta población de estudiantes, se realiza un test a
$400$ estudiantes. La media y la desviación típica  muestrales obtenidas son
$\overline{x}=86$ y $\tilde{s}_{X}=10.2$. Calcular un intervalo para $\mu$ con un nivel de
significación del 98\%. \sol{$\left(84.8117, 87.1883\right)$}
\end{prob}

\begin{prob}
Para investigar un nuevo tipo de combustible para cohetes espaciales, se disparan cuatro
unidades y se miden las velocidades iniciales. Los resultados obtenidos, expresados en
Km/h, son :19600, 20300, 20500, 19800. Calcular un intervalo para la velocidad media $\mu$
con un nivel de confianza del 95\%, suponiendo que las velocidades son normales.
\sol{$\left(19381.7, 20718.3\right)$} 20718.3
\end{prob}

\begin{prob}
Un fabricante de cronómetros quiere calcular un intervalo de estimación de la desviación
típica del tiempo marcado en $100$ horas por todos los cronómetros  de un cierto modelo.
Para ello pone en marcha $10$ cronómetros del modelo durante $100$ horas y encuentra que
$\tilde{s}_{X}=50$ segundos. Encontrar un intervalo para el parámetro $\sigma^2$ con
$\alpha=0.01$, suponiendo que la población del tiempo marcado por los cronómetros es
normal. \sol{$\left(953.834,12968.3\right)$}
\end{prob}

% \begin{prob}
% El mismo fabricante del problema anterior quiere comparar ahora las
% varianzas de dos modelos de cronómetros (modelo 1 y modelo 2)
% tomando dos muestras con los siguientes resultados: $n_{1}=10$,
% $\tilde{S}_{1}=50$ $n_{2}=16$ y $\tilde{S}_{2}=45$. Utilizar
% $\alpha=0.01$
% \sol{$[0.52,2.72]$}
% \end{prob}

\begin{prob}
Un auditor informático quiere investigar la proporción de rutinas de un programa que
presentan una determinada irregularidad. Para ello observa $120$ rutinas, resultando que
$30$ de ellas presentan alguna irregularidad. Con estos datos buscar unos límites de
confianza para la proporción $p$ de rutinas de la población que presentan esa irregularidad
con probabilidad del 95\%.

\sol{$\left(0.1725,0.3275\right)$}
\end{prob}

% \begin{prob} En una ciudad A de 400 propietarios de coches 125 tienen
% una marca $X$; y en otra población B, de 600 propietarios, 180
% tienen la marca $X$. Calcular un intervalo de confianza del 95\% para
% la diferencia de proporciones entre la ciudad A y la B.
% \sol{ $\left(-0.111,0.136\right)$}
% \end{prob}
%
%
%%%%%%%%%%%%%%%
\begin{prob}(Examen septiembre 2003) Una infección por un virus puede haber perjudicado a muchos ordenadores con
 \emph{Windwos}. Desde el
 Centro de Alerta Temprana (CAT) se quiere calcular la
 proporción de ordenadores infectados. El jefe del centro  os pide que calculéis
 el tamaño de una muestra para que el intervalo de confianza de la proporción muestral de
 ordenadores infectados tenga amplitud de a lo sumo $0.01$ con una probabilidad del 90\%.
  \end{prob}

\begin{prob}(Examen junio 2003) Se han medido los siguientes valores (en miles de personas) para la audiencia de un
programa de televisión en distintos días (supuestos igualmente distribuidos e
independientes):

$$521, 742, 593, 635, 788, 717, 606, 639, 666, 624.$$

Construir un intervalo de confianza del $90$\%, para la audiencia poblacional media y otro
para la varianza poblacional, bajo la hipótesis de que la población de audiencias sigue una
ley normal.

Nota Suma de las audiencias=$6531$, Suma de los cuadrados de las audiencias=$4320401$.
\end{prob}

\begin{prob}
Supongamos que la empresa para la que trabajamos está en un proyecto de investigación,
financiado con fondos de la Comunidad Europea, que pretende extender una nueva aplicación
de las TIC. Una de las tareas del proyecto es realizar una encuesta de opinión sobre el
grado de aceptación que tendría esta nueva tecnología en el mercado europeo. De entre todas
las universidades y empresas participantes en el proyecto, es a tu empresa a la que le toca
hacer el protocolo de la encuesta, llevarla a cabo y redactar esta parte del informe final.
Como eres el último que llegó a la empresa y el resto de miembros del equipo no se acuerda
de la estadística que vio en la carrera, te toca a ti cargar con la responsabilidad. Claro
que el coste de la encuesta depende del número $n$ de entrevistas que se realicen y el
error de las proporciones de las contestaciones disminuye cuando $n$ aumenta. Como no sabes
cuánto dinero está dispuesto a gastar tu jefe, tabula los tamaños muestrales para los
errores $\pm 5\%, \pm 3\%, \pm 2\%,\pm 1\%$, y para niveles de confianza $0.95$ y $0.99$,
suponiendo el peor caso. Añade un comentario para que el equipo de dirección del proyecto,
en el que hay componentes ignorantes en  materia de encuestas, vea como quedarían redactado
los datos técnicos de la encuesta, y pueda decidir  el tamaño muestral leyendo tu informe.
\end{prob}

\begin{prob}
El número de reservas semanales de billetes de cierto vuelo de una compañía aérea sigue una
distribución aproximadamente normal. Se toma un muestra aleatoria de $81$ observaciones de
números de reservas de este vuelo: el número medio de reservas muestral resulta ser $112$,
mientras que la desviación típica muestral es  $36$. Además de estos $81$ vuelos, $30$
llegaron a su destino con un retraso de más de $15$ minutos.
\begin{enumerate}[a)]
    \item Calcular un intervalo de confianza del $95\%$ para el número medio
    poblacional de reservas en este vuelo.
    \item Calcular un intervalo de confianza de $95\%$ para la varianza
    poblacional de las reservas.
    \item Calcular un intervalo de confianza del $95\%$ para la proporción
    poblacional de vuelos que llegan con un retraso de más de $15$ minutos.
     \item Calcular el tamaño muestral que asegura un intervalo de
     confianza de amplitud $0.1$ para la proporción de vuelos que llegan
     con un retraso de más de $15$ minutos al nivel de confianza $95\%$.
\end{enumerate}
\sol{a) $\left(104.16,119.84)\right)$; b) $\left(972.343,1814.08)\right)$; c)
$\left(0.265,0.475)\right)$; d) $n=385$}
\end{prob}

\begin{prob}
    Una empresa cervecera sabe que las cantidades de cerveza que contienen
sus latas sigue una distribución normal con desviación típica poblacional $0.03$ litros.
\begin{enumerate}[a)]
    \item Se extrae una muestra aleatoria de $25$ latas y, a partir de la
    misma, un experto en estadística construye un  intervalo de confianza
    para la media poblacional del contenido en litros de las latas que
    discurre entre $0.32$ y $0.34$ ?`Cuál es el nivel de confianza de este
    intervalo?
    \item Un gerente de esta empresa exige un intervalo de confianza del
    $99\%$  que tenga una amplitud máxima de $0.03$ litros a cada lado de la
    media  muestral ?`Cuántas observaciones son necesarias, como mínimo,
    para alcanzar este objetivo?
\end{enumerate}
\sol{a) $90.3\%$; b) $n=7$}
\end{prob}
\newpage




\textbf{Contraste de hipótesis}

\begin{prob}%10.1 pag 185 J.Amon
Siendo $\overline{x}=63.5$ la media de una muestra aleatoria simple de tamaño 36 extraída
de una población normal con $\sigma^2=144$, poner
 a prueba, con un nivel de significación $\alpha=0.05$, la hipótesis
 nula $\mu=60$ y decir si se rechaza en favor de la alternativa
 $\mu<60$. Calcular el $p$-valor.
\end{prob}

\begin{prob}%10.2 pag 185 J.Amon
Siendo $\overline{x}=72.5$ la media de una muestra aleatoria simple de tamaño 100 extraída
de una población normal con $\sigma^2=900$, poner
 a prueba, con un nivel de significación $\alpha=0.10$, la hipótesis
 nula $\mu=77$ y decir si se rechaza en favor de las hipótesis
 alternativas
 $\mu\not= 70$, $ \mu>70$, $\mu<70$. Calcular el $p$-valor en cada caso.
\end{prob}


\begin{prob}%10.9 pag 185 J.Amon
    En un contraste bilateral, con $\alpha=0.01$, ?`para qué valores
    de $\overline{X}$ rechazaríamos  la hipótesis nula $H_{0}:\mu=70$, a
    partir de una muestra aleatoria simple de tamaño 64 extraída de una
    población normal con $\sigma^2=256$?
\end{prob}

\begin{prob}%11.1 pag 324 J.Amon
    El salario anual medio de 1600 personas, elegidas aleatoria e
    independientemente de cierta población de economistas con $\sigma=20000$ euros, ha
    valido 45000 euros  ?`Es compatible  con este resultado la hipótesis
    nula, $H_{0}:\mu=43500$, suponiendo $\alpha=.01$? ?`Cuál es el intervalo
    de confianza para $\mu$? Calcular el $p$-valor.
\end{prob}

\begin{prob}%11.2 pag 324 J.Amon
    Con los datos del ejercicio anterior , ?`son compatibles con el
    resultado obtenido los siguientes contrastes?:
    \begin{enumerate}[a)]
        \item $\left\{\begin{array}{ll} H_{0}:\mu=44000\\
        H_{1}:\mu>44000\end{array}\right.$
        \item $\left\{\begin{array}{ll} H_{0}:\mu=46250\\
        H_{1}:\mu>46250\end{array}\right.$
        \end{enumerate}
\end{prob}

\begin{prob}%11.6 pag 324 J.Amon
El peso medio de los paquetes de café puestos a la venta por la casa comercial CAFEINASA es
supuestamente de 1 Kg. Para comprobar esta suposición, elegimos una muestra aleatoria
simple de 100 paquetes y encontramos que su peso  medio es de 0.978 Kg. y su desviación
típica $s=0.10$ kg. Siendo $\alpha=0.05$ ?`es compatible este resultado con la hipótesis
nula $H_{0}:\mu=1$ frente a $H_{1}:\mu\not=1$? ?`Lo es frente a $H_{1}:\mu>1$? Calcular el $p$-valor.
\end{prob}

\begin{prob}
El fabricante de la marca de tornillos FDE afirma que el diámetro medio de sus tornillos
vale 20 mm. Para comprobar dicha afirmación, extraemos aleatoria e independientemente 16
tornillos , y vemos que la media de sus diámetros  es 22 mm. y la desviación típica 4 mm.
?`Podemos aceptar la pretensión del fabricante, suponiendo $\alpha=0.05$ y siendo el
contraste bilateral? Calcular el $p$-valor.
\end{prob}

\begin{prob}
    Para evitar basarse en su intuición los jefes de admisión de
    personal de las grandes empresas discriminan mediante un test
    diseñado por un gabinete de psicólogos, supuestamente especializado
    en selección de personal, a los aspirantes a trabajar en la empresa.
    La varianza del test de selección solía venir siendo $100$.
    Aplicando un nuevo test a una muestra aleatoria simple de tamaño
    $n=31$, se obtiene que $S=129$. Suponiendo que la población se
    distribuye normalmente, ?`es compatible la hipótesis nula
    $H_{0}:\sigma^2=100$, frente a la alternativa $H_{1}:\sigma^2>100$,
    con $\alpha=0.01$? Calcular el $p$-valor.
\end{prob}

\begin{prob}
    Una máquina produce cierto tipo de piezas mecánicas. El tiempo en
    producirlas se distribuye normalmente con varianza desconocida
    $\sigma^2$. Elegida una muestra aleatoria simple de 21 de dichas
    piezas ($x_{1},\ldots,x_{21}$), se obtiene que $\overline{x}=30$ y
    $\sum_{i=1}^{21}x_{i}^2=19100.$ Comprobar si es compatible la
    hipótesis nula $H_{0}:\sigma^2=22$ frente $H_{1}:\sigma^2\not=22$,
    con $\alpha=0.1$, y construir un intervalo de confianza del
    $(1-\alpha)100\%$ para el verdadero valor de $\sigma^2$. Calcular el $p$-valor.
\end{prob}

\begin{prob}
A partir de las puntuaciones 15, 22, 20, 21, 19 ,23, construir el intervalo de confianza de
$\sigma^2$ y decir si es compatible con estos resultado la hipótesis $H_{0}:\sigma=2$,
siendo $\alpha=0.01$ contra una $H_1$ bilateral. Decir si se utiliza alguna hipótesis adicional. Calcular el $p$-valor.
\end{prob}

\begin{prob}
    Sabiendo que con $\hat{p}=0.52$ ha sido rechazada $H_{0}:p=0.50$, al
    nivel de significación $\alpha=0.05$, ?`cuál ha tenido que ser el
    tamaño mínimo de la muestra mediante la cual fue rechazada
    $H_{0}$

    \begin{enumerate}[a)]
        \item frente a $H_{1}:p\not=0.5$?
    \item frente a $H_{1}:p>0.5$?
    \end{enumerate}

    \begin{prob}
        Lanzamos una moneda al aire 10 veces consecutivas . ?`Con qué número
        de caras rechazaremos la hipótesis nula de que la moneda está bien
        balanceada, siendo $\alpha=0.05$?
        \end{prob}
\end{prob}



\begin{prob}% mío
Un fabricante de productos farmacéuticos tiene que mantener un estándar de impurezas en el
proceso de producción de sus píldoras. Hasta ahora el número medio poblacional de impurezas
es correcto pero está preocupado porque las impurezas en algunas de las partidas se salen
del rango admitido de forma que provocan devoluciones y posibles reclamaciones por daños a
la salud. El gabinete de control de calidad afirma que si la distribución de las impurezas
es normal y que si el proceso de producción mantiene  una varianza inferior a 1 no tendría
que existir ningún problema pues las píldoras tendrían una concentración aceptable.
Preocupado por esta tema la dirección encarga una prueba externa en la que se toma una
muestra aleatoria de 100 de las partidas obteniéndose $S^2=1.1$. ?`Puede aceptar el director
de la prueba externa que el proceso de producción cumple la recomendación del gabinete de
control?
\end{prob}


\begin{prob}
Un IAP está preocupado por su estándar de calidad y quiere compararlo con el medio europeo.
El estándar medio europeo dice que una empresa de este sector tiene una calidad aceptable
si tiene un número de quejas que no excede del 3\%.

Se sabe que la varianza de las quejas es $0.16$. Examinando $64$ clientes escogidos al azar
se encuentra con que el porcentaje de quejas es del $3.07\%$. Calcular el $p$-valor.

\begin{enumerate}[a)]
    \item Contrastar al nivel de significación del 5\%, la hipótesis nula
    de que la media poblacional del porcentaje de quejas es del 3\%
    frente a la alternativa de que es superior al 3\%.
    \item Hallar el $p$-valor del contraste.
    \item Supongamos que la hipótesis alternativa fuese bilateral en
    lugar de unilateral( con hipótesis nula $H_{0}:\mu=3$). Deducir, sin
    hacer ningún cálculo, si el $p$-valor del contraste sería mayor,
    menor o igual que el del apartado anterior. Construir un gráfico
    para ilustrar el razonamiento.
    \item En el contexto de este problema, explicar por qué una hipótesis
    alternativa unilateral es más apropiada que una bilateral.
\end{enumerate}
\end{prob}

\begin{prob}
     A partir de una muestra aleatoria se contrasta:

     $\left\{\begin{array}{l}
     H_{0}:\mu=\mu_{0}\\
     H_{1}:\mu>\mu_{0}\end{array}\right.$

     y se acepta la hipótesis nula al nivel de significación del 5\%.

     \begin{enumerate}[a)]
         \item ?`Implica esto necesariamente que $\mu_{0}$ está contenido
         en el intervalo de confianza del 95\% para $\mu$?
         \item Si la media muestral observada es mayor que $\mu_{0}$,
         ?`implica necesariamente que $\mu_{0}$ está contenido en el
         intervalo de confianza  del 90\% para $\mu$?
\end{enumerate}
\end{prob}

\begin{prob}
Una compañía que se dedica a la venta de franquicias afirma que, por término medio, los
delegados obtienen un redimiendo del 10\% en sus inversiones iniciales. Una muestra
aleatoria de diez de estas franquicias presentaron los siguientes rendimientos el primer
año de operación:

6.1, 9.2, 11.5, 8.6, 12.1, 3.9 , 8.4, 10.1, 9.4, 8.9

Asumiendo que los rendimientos poblacionales tienen distribución normal, contrastar la
afirmación de la compañía.
\end{prob}


\begin{prob}
Una distribuidora de bebidas refrescantes afirma que una buena fotografía de tamaño real de
un conocido actor, incrementará las ventas de un producto en los supermercados en una media
de $50$ cajas semanales. Para una muestra de $20$ supermercados, el incremento medio fue de
$41.3$ cajas con una desviación típica de $12.2$ cajas. Contrastar al nivel de
significación $\alpha=0.05$, la hipótesis nula de que la media poblacional del incremento
en las ventas es al menos $50$ cajas, indicando cualquier supuesto que se haga. Calcular el
$p$-valor del contraste e interpretarlo.
\end{prob}



\newpage



\textbf{Problemas de bondad de ajuste}
\begin{prob}
Una compañía de gas afirma, basándose en experiencias anteriores, que normalmente, al final
del invierno, el 80\% de las facturas han sido ya cobradas, un 10\% se cobrará con pago
aplazado a un mes, un 6\% se cobrará a 2 meses y un 4\% se cobrará a más de dos meses. Al
final del invierno actual, la compañía selecciona una muestra aleatoria de $400$ facturas,
resultando que 287 de estas facturas cobradas, 49 a cobrar en un mes, $30$ a cobrar en dos
meses y $34$ a cobrar en un periodo superior a dos meses. ?`Podemos concluir, a raíz de los
resultados, que la experiencia de años anteriores se ha vuelto a repetir este invierno?
\end{prob}

\begin{prob}
El Rector de una Universidad opina que el $60\%$ de los estudiantes consideran los cursos
que realizan como muy útiles, el $20\%$ como algo útiles y el $20\%$ como nada útiles. Se
toma una muestra aleatoria de $100$ estudiantes, y se les pregunta sobre la utilidad de los
cursos. Resultando que $68$ estudiantes consideran que los cursos son muy útiles, $18$
consideran que son poco útiles y $14$ consideran que no son nada útiles. Contrastar la
hipótesis nula de que los resultados obtenidos se corresponden con la opinión personal del
Rector.
\end{prob}

\begin{prob}
Considerense los fondos de inversión ordenados en función de su rendimiento en el periodo
1995-99. Se realizó un seguimiento del rendimiento en los cinco años posteriores de una
muestra aleatoria de 65 fondos entre el $25\%$ más rentable del periodo 1995-99. En este
segundo periodo se observó que $11$ de los fondos de la muestra se hallan entre el $25\%$
más rentable en este segundo periodo, $17$ en el segundo $25\%$, $18$ en el tercer 25\% y
$19$ en el $25\%$ menos rentable. Contrastar la hipótesis de que un fondo de inversión
escogido azar del $25\%$ más rentable en 1995-99 tenga la misma probabilidad de hallarse en
cualquiera de las cuatro categorías de rendimiento en el periodo 2000-2004.
\end{prob}

\begin{prob}
    A una muestra aleatoria de $502$ consumidores se les preguntó la
    importancia que se le daba al precio a la hora de elegir un
    ordenador. Se les pidió que valoraran entre: ``ninguna
    importancia'', ``alguna importancia'' y ``principal importancia''. El
    número respectivo de respuestas en cada tipo fueron 169, 136 y 197.
    Contrastar la hipótesis nula de que la probabilidad de que un
    consumidor elegido al azar conteste cualquiera de las tres
    respuestas es la misma.
    \end{prob}


\begin{prob}
Durante cien semanas se ha venido observando el número de veces a la semana que se ha fuera
de servicio  un servidor de una pequeña empresa de informática, presentándose los
resultados de la siguiente tabla:

\begin{tabular}{|l|cccccc|}
    \hline
    Núm. Fuera Servicio & 0 & 1 & 2 & 3 & 4 & 5 o más\\
    \hline
    Núm. Semanas & 10 & 24 & 32 & 23 & 6 & 5\\
    \hline
\end{tabular}

El número medio de veces que quedo fuera de servicio por semana durante este periodo fue de
$2.1$. Contrastar la hipótesis nula de que la distribución de averías es una Poisson.
    \end{prob}


    \begin{prob}
        A lo largo de $100$ minutos, llegaron a una web de un periódico $100$
        internautas. La siguiente tabla muestra la frecuencia de llegadas a
        lo largo de ese intervalo de tiempo.

        \begin{tabular}{|l|ccccc|}
    \hline
    Núm. llegadas/min. & 0 & 1 & 2 & 3 & 4 o más\\
    \hline
    Frec. Observada & 10 & 26& 35 & 24 & 5 \\
    \hline
\end{tabular}

        Contrastar la hipótesis nula de
        que la distribución es Poisson.
        \end{prob}




\textbf{KS test.}

\begin {prob} Se quiere saber si el tiempo entre accesos, en una determinada franja horaria, a  una cierta página web sigue una ley exponencial. Se dispone de la siguiente muestra de 25 intervalos entre tiempos de acceso:

$140.7,13.7,67.6,7.8,49.3,128.5,59.6,234,171.1,205.8,99.3,199.8,100.8,13.5,12,33.9,$
$44.1,12.3,56.4,9.4,112.1,8.2,110.5,79,55.4$

Resolver manualmente o utilizando R las siguientes cuestiones.

\begin{enumerate}[a)]
\item  Contrastar la hipótesis de que la distribución  sigue una ley exponencial de parámetro $\lambda=100$, al nivel $\alpha=0.05$.
\item  Contrastar la hipótesis de que la distribución  sigue una ley Poisson de parámetro $\lambda=105$, al nivel $\alpha=0.05$.
\item  Contrastar la hipótesis de que la distribución  sigue una ley Poisson de parámetro $\lambda=110$, al nivel $\alpha=0.05$.
\item  Contrastar la hipótesis de que la distribución  sigue una ley Poisson, estimando el parámetro parámetro  partir de la muestra, al nivel $\alpha=0.05$.
\end{enumerate}
\end{prob}

\begin{prob}
Resolver las mismas cuestiones que en el problema anterior para la muestra (decir si se viola algunas de las condiciones del test KS, pero resolver igualmente el ejercicio):

$$69.9,  31.5, 130.2,  80.5, 236.1, 151.2,  74.8,  13.8,  54.5, 147.6$$

En esta ocasión realizar los cálculos manualmente.

\end{prob}

\begin{prob}
 Nos hemos bajado un generador de números aleatorios normales de internet. Queremos contrastar si funciona correctamente. Para ello generamos una muestra de 10 números aleatorios de una normal estándar:

$$ -1.18, -0.77, -0.59, -0.27, -0.12,  0.27,  0.29,  0.40,  1.27,  1.60
$$

\begin{enumerate}[a)]
\item Contrastar si provienen de una normal estándar al nivel de significación $\alpha=0.05$ mediante el test KS. Decir si ha violado alguna de las suposiciones de este test.
\item Contrastar la hipótesis de normalidad contra una distribución normal de media y varianza la estimadas a partir de la muestra.
\end{enumerate}

\end{prob}

\begin{prob} Con la muestra:

$$0.60, -1.42,  1.05, -0.14,  0.57,  0.11, -0.59, 1.11, -1.55, -1.41
$$

Contrastar con un test KS si los datos provienen de una distribución uniforme en el intervalo $(-2,2)$ al nivel $\alpha=0.05$
\end{prob}




%    \begin{prob}
%
%   El estadístico de \emph{Bowman-Seldon} es:
%
%   $B=n\left(\frac{(\mbox{Coefiente de
%   asimetría})^2}{6}+\frac{(\mbox{Curtosis}-3)^2}{24}\right)$
%
%   Que para un tamaño muestral suficientemente grande sigue
%   aproximadamente una distribución muestral $\chi^2$ con 2 g.l.
%
%   Criterio de rechado para \newline
%   $\left\{\begin{array}{l}
%   H_{0}:\mbox{La distribución poblacional es
%   normal}\\
%   H_{1}:\mbox{La distribución poblacional no es
%   normal}\end{array}\right. $
%
%   es
%
%   Rechazar $H_{0}$ al nivel de significación $\alpha$ si\newline
%   $B>\chi_{2,\alpha}^2$
%
%     Resolver, utilizando el estadístico de \emph{Bowman-Seldon} el
%     siguiente problema.
%
%       Durante 268 días  escogidos al azar se observaron los beneficios de
%       un contrato de futuro de cerdos y se observó un coeficiente de
%       asimetría de $0.04033$ y una curtosis $3.15553$ en la muestra. ?`Es
%       la distribución de os beneficios normal?
% %         \textbf{Solución}
% %         El
% %         valor del estadístico de \emph{Bowman-Shelton} es:\newline
% %         $B=n\left(\frac{(\mbox{0.04033})^2}{6}+
% %         \frac{(\mbox{3.15553}-3)^2}{24}\right)=0.36$
% %
% %     Si tomamos un nivel $\alpha=0.1$ tenemos que $\chi_{2,0.1}=4.61$
% %     y como $B=0.36\not>4.16$ no podemos rechazar $H_{0}$ a este nivelk
% %     de significación.
% %
% %         \end{example}

\newpage

\textbf{Contrastes de dos parámetros.}

\textbf{Comparación de medias.}

Los siguientes problemas tratan de contrastes de parámetros entre dos muestras. Para cada uno de los enunciados contratar contra las  hipótesis unilaterales y bilaterales. Calcular también el intervalo de confianza para la diferencia o el cociente de los parámetros. Tomar finalmente la decisión más correcta. Calcular todos los test e intervalos de confianza para $\alpha=0.05$. Calcular el $p$-valor en cada caso. 

\begin{prob}
Para comparar la producción  media de dos procedimientos de fabricación de cierto elemento
se toman dos muestras, una con los elementos fabricados durante 25 días con el primer
método y otra con los producidos durante 16 días con el segundo método. Por experiencia se
sabe que la varianza del primer procedimiento es $\sigma_{1}^2=12$ y al del segundo
$\sigma_{2}^2=10$. De las muestras obtenemos que $\overline{X}_{1}=136$ para el primer
procedimiento y $\overline{X}_{2}=128$ para el segundo. 
% Si $\mu_{1}$ y $\mu_{2}$ son los
% valores esperados para cada uno de los procedimientos, calcular un intervalo de confianza
% para $\mu_{1}-\mu_{2}$ al nivel 99\%. \sol{$\left(5.2989,10.7011\right)$}
\end{prob}

\begin{prob}
Estamos interesados en comparar la vida media, expresada en horas de dos tipos de
componentes electrónicos. Para ello se toma una muestra de cada tipo y se obtiene:

$$
\begin{tabular}{|c|c|c|c|}
\hline Tipo & tamaño & $\overline{X}$ & $S$\\ \hline \hline 1 & 50 & 1260 & 20\\ \hline 2 &
100 & 1240 & 18\\ \hline
\end{tabular}
$$

% Calcular un intervalo de confianza para $\mu_{1}-\mu_{2}$ ($\mu_{1}$ esperanza del primer
% grupo y $\mu_{2}$ esperanza del segundo grupo) al nivel 98\% 
Suponer si es necesario las
poblaciones aproximadamente normales. % \sol{$\left(12.19,27.81\right)$}
\end{prob}

\begin{prob}
Para reducir la concentración de ácido úrico en la sangre se prueban dos drogas. La primera
se aplica a un grupo de 8 pacientes y la segunda a un grupo de 10. Las disminuciones
observadas en las concentraciones de ácido úrico de los distintos pacientes expresadas en
tantos por cien de concentración después de aplicado el tratamiento son:

\begin{center}
\begin{tabular}{|c|c|c|c|c|c|c|c|c|c|c|}
droga 1 & 20 & 12 & 16 & 18 & 13 & 22 & 15 & 20\\ \hline droga 2 & 17 & 14 & 12 & 10 & 15 &
13 & 9 & 19 & 20 & 11
\end{tabular}
\end{center}

% Calcular un intervalo de confianza para la diferencia de medias entre la primera droga y la
% segunda al nivel del 99\%.
 Suponer que las reducciones de ácido úrico siguen una
distribución normal son independientes y de igual varianza. 
Ídem pero suponiendo que las varianza son distintas.%\sol{$\left(-2.09,8.09\right)$}
\end{prob}

\begin{prob}
Para comparar la dureza media de dos tipos de aleaciones (tipo 1 y tipo 2) se hacen 5
pruebas de dureza  con la de tipo 1 y 7 con la de tipo 2. Obteniéndose los resultados
siguientes:
$$\overline{X}_{1}=18.2,\quad S_{1}=0.2 \mbox{ y}$$

$$\overline{X}_{2}=17.8;\quad S_{2}=0.5$$

Suponer que la población de las durezas es normal y que las desviaciones típicas no son
iguales. Hacer lo mismo  si las varianzas son distintas.
% , buscar un intervalo de confianza para $\mu_{1}-\mu_{2}$ con  probabilidad $0.95$
% \sol{$\left(0.314,0.486\right)$}
\end{prob}
\newpage

\begin{prob} Se encuestó a dos muestras independientes de  internautas, una en Menorca y otra en Mallorca, sobre si utilizaban  telefonía por intenet. La encuesta de Menorca tuvo un tamaño $n_1=500$  y 
$100$ usuarios mientras que  en Mallorca se encuestarron a $n_2=750$ y se obtuvo un resultado de $138$ usuarios.
\end{prob}

\begin{prob}
Se pregunta a un grupo de 100 personas elegido al azar asiste a una conferencia sobre tecnologías de la comunicación. Antes de la conferencia se les pregunta si consideran a internet peligrosa, después de la conferencia se les vuelve a preguntar cual es su opinión. Los resultados fueron los siguientes:

$$
\begin{tabular}{|c|c|cc|}
\cline{3-4}
     \multicolumn{2}{c|}{}& \multicolumn{2}{|c|} {Después}\\\cline{3-4}
   \multicolumn{2}{c|}{} & Sí es peligrosa & No es peligrosa \\\hline
Antes & Sí es peligrosa&  50 &  30 \\
    & No es peligrosa  &  5 & 15 
\\\hline
\end{tabular}
$$




% \hline
% & Antes & Después \\\hline
% Sí es Peligrosa &  80 & 70\\
% No es Peligrosa  & 20 & 30 
% \\\hline
% \end{tabular}

\end{prob}


\begin{prob}  Tenemos $10$ ordenadores, deseamos optimizar su funcionamiento. Con este fin  se piensa en ampliar su memoria. Se les pasa una prueba de rendimiento antes  y después de  ampliar la memoria. Los resultados fueron:

\scriptsize{
\begin{tabular}{|c|llllllllll|}
\hline
 &\multicolumn{10}{|c|}{Ordenador} \\\hline
Muestra\slash Tiempo & 1 & 2 & 3 & 4 & 5 & 6 & 7 & 8 & 9 & 10\\\hline
Antes ampliación & 98.70 & 100.48 & 103.75 & 114.41 & 97.82&
91.13 & 85.42 & 96.8 & 107.76 & 112.94\\
\hline
Después ampliación & 99.51 & 114.44 & 108.74 & 97.92 & 103.54&
104.75 & 109.69 & 90.8 & 110.04 & 110.09\\

\hline
\end{tabular}
}
\end{prob}


\begin{prob} Las siguientes muestras provienen de dos poblaciones independientes y supuestamente normales.  Se desea comparar la igualdan de sus medias, pero antes debemos contrastar si podemos o no aceptar que sus varianza son iguales o distintas. Se pide hacer el contraste de las medias en el caso en que se se decida aceptar varianzas iguales o distintas al nivel de significación $\alpha=0.05$.





Contrastar tambien la hipótesis de igualdad de medias en el otro caso ( es decir si se decide varianzas ditintas contrastar para iguales y viceversa).


\end{prob}


% 98.7 & 100.48 & 103.75 & 114.41 & 97.82
% 91.13 & 85.42 & 96.8 & 107.76 & 112.94
% 99.51 & 114.44 & 108.74 & 97.92 & 103.54
% 104.75 & 109.69 & 90.8 & 110.04 & 110.09

%%%%%%%%\footnote{Además tenéis que hacer los problemas de las páginas 250 a 252 y 259 y 260 del
%%%%%%%%Newbold.}



% \begin{prob}
%
%   Los candidatos para un puesto de trabajo en una gran empresa deben
% superar un examen de aptitud escrito y realizar una entrevista con el
% director de personal. Después de la entrevista, el director otorga a
% cada candidato una puntuación entre 0 y 10. Para comprobar la
% consistencia de los criterios de los distintos directores, se extrae
% una muestra de 10 pares de candidatos. La formación de las parejas se
% organiza de forma que en cada una de ellas los dos candidatos han
% obtenido una puntuación idéntica en el examen escrito.
% Posteriormente, la entrevista a cada uno de los candidatos de cada
% par es realizada por cada uno de los dos directores de personal,
%  Sr. José y Sra. Isabel. Las puntuaciones obtenidas fueron las
%  siguientes:
%
%
%  \begin{tabular}{l|cccccccccc}
%    Sr.. José & 80 & 65 & 87 & 64 & 73 & 78 & 83 & 91 & 84 & 83\\
%    \hline
%    Sra.. Isabel & 74 & 63 & 91 & 65 & 64 & 71 & 69 & 90 & 79 & 87
%    \end{tabular}.
%
%    \begin{enumerate}[a)]
%   \item  Especificando las hipótesis necesarias calcular un intervalo
%   de confianza del 95\% para la diferencia entre las dos puntuaciones
%   medias poblacionales obtenidas (para fijar ideas hacer Sr.. José - Sra.
%   Isabel).
% \item Sin realizar los cálculos, determinar si un intervalo de
% confianza al 99\% para la diferencia de las medias poblacionales
% tendría mayor, menor o la misma amplitud que en el apartado a).
% \end{enumerate}
%   \end{prob}


    \newpage

\end{document}
