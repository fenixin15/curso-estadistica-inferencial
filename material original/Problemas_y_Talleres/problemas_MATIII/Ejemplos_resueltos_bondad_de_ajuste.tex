\documentclass[10pt,spanish,es-nodecimaldot]{article}
\usepackage{babel}
\usepackage{amsfonts,amssymb,amsmath,amsthm,graphicx,enumerate,multirow,adjustbox}
\usepackage{multirow}
%\usepackage{accents}
\usepackage[utf8]{inputenc} 
\decimalpoint




\advance\hoffset by -0.9in \advance\textwidth by 1.8in
\advance\voffset by -1.4in \advance\textheight by 2.8in \parskip= 1 ex
\parindent = 10pt \baselineskip= 13pt

\pagestyle{empty}

\newcounter{problemes}
\newcounter{punts} \def\thepunts{\arabic{punts}}
\def\probl{\addtocounter{problemes}{1} \setcounter{punts}{0}
\medskip\noindent{\bf \theproblemes) }}
\def\punt{\addtocounter{punts}{1} \smallskip{\emph{\thepunts) }}}
\newif\ifsol
\solfalse
%\solfalse



\newenvironment{solucion}{\textbf{Solución:}\sf}{\rm}
\usepackage{Sweave}
\begin{document}
\Sconcordance{concordance:Ejemplos_resueltos_bondad_de_ajuste.tex:Ejemplos_resueltos_bondad_de_ajuste.Rnw:%
1 29 1 1 0 12 1 1 6 7 1 1 2 1 0 2 1 5 0 1 2 10 0 1 2 1 1 1 2 1 0 1 1 5 %
0 1 1 6 0 1 2 17 1 1 2 4 0 2 2 1 0 2 1 5 0 3 1 5 0 1 2 1 0 4 1 5 0 2 1 %
6 0 2 2 18 0 2 2 11 0 2 2 1 0 3 1 5 0 1 1 5 0 2 1 5 0 1 1 5 0 2 1 5 0 1 %
1 5 0 3 1 6 0 2 2 1 0 1 1 11 0 2 1 6 0 1 2 7 1 1 2 1 0 1 1 5 0 4 1 5 0 %
2 1 5 0 2 1 11 0 1 2 4 1}




\emph{Nombre:}\hfill\hfill\hfill\hfill\hfill\ \emph{Grupo:}\hfill \medskip
\setcounter{problemes}{0}

\begin{center}
\textsc{\textbf{Matemáticas III. Entrega de Pascua. Cuestiones Grado Matemáticas UIB 16-17}}\\[1ex]%1
\end{center}
%{\footnotesize Tenéis que contestar en esta hoja. Todas las cuestiones valen \textbf{0.5 puntos}.}





\probl  Una compañía de gas afirma, bas-andose en experiencias anteriores, que normalment, al final del invierno el 80\%
de las facturas han sido ya cobradas, un 10\% se cobrará con pago aplazado un mes, 6\% se cobrará a  2 meses y un 4\% se cobrará a más de  dos meses. Al final del invierno actual, la compañía seleccionada una muestra aleatoria de 400 facturas, resultando que 287 de estas facturas cobradas, 49 a cobrar en un mes, 30 a cobrar en dosmeses y 34 a cobrar en un periodo  superior a dos meses.  ¿Podemos concluir, a raíz de los resultados, que la experiencia de ãnos anteriores se ha vuelto a repetir este invierno?


\begin{Schunk}
\begin{Sinput}
> oi=c(287,49,30,34)
> pi=c(0.8,0.1,0.06,0.04)
> sum(pi)
\end{Sinput}
\begin{Soutput}
[1] 1
\end{Soutput}
\begin{Sinput}
> #PONER EL PARAMETRO p= para bondad de ajuste , ensaco contrario hace un test de independencia
> chisq.test(oi,p=pi)
\end{Sinput}
\begin{Soutput}
	Chi-squared test for given probabilities

data:  oi
X-squared = 27.178, df = 3, p-value = 5.402e-06
\end{Soutput}
\end{Schunk}


\begin{Schunk}
\begin{Sinput}
> ei=pi*sum(oi)
> sum((oi-ei)^2/ei)
\end{Sinput}
\begin{Soutput}
[1] 27.17813
\end{Soutput}
\begin{Sinput}
> 1-pchisq(27.18,3)
\end{Sinput}
\begin{Soutput}
[1] 5.397347e-06
\end{Soutput}
\end{Schunk}

Todo da lo mismo


\begin{tabular}{l||llll} 
& Cobradas & un mes & dos meses & más de 2 meses \\
\hline
Prob bajo $H_0$ $p_i$ &0.8  &0.1 & 0.06 & 0.04\\
Prob bajo $H_0$ $p_i$ &0.8  &0.1 & 0.06 & 0.04\\
Prob bajo $H_0$ $p_i$ &0.8  &0.1 & 0.06 & 0.04\\
Prob bajo $H_0$ $p_i$ &0.8  &0.1 & 0.06 & 0.04\\
}




\probl

\begin{Schunk}
\begin{Sinput}
> knitr::opts_chunk$set(echo = TRUE)
\end{Sinput}
\end{Schunk}

\begin{Schunk}
\begin{Sinput}
> x=c(0.60,-1.42,1.05,-0.14,0.57,0.11,-0.59,1.11,-1.55,-1.41)
> x=sort(x)
> x
\end{Sinput}
\begin{Soutput}
 [1] -1.55 -1.42 -1.41 -0.59 -0.14  0.11  0.57  0.60  1.05  1.11
\end{Soutput}
\begin{Sinput}
> Fn=1:10/10
> Fx=(x+2)/4
> Fx
\end{Sinput}
\begin{Soutput}
 [1] 0.1125 0.1450 0.1475 0.3525 0.4650 0.5275 0.6425 0.6500 0.7625 0.7775
\end{Soutput}
\begin{Sinput}
> #punif(x,-2,2)
> df_ks=data.frame(i=1:10,x,Fn,Fx)
> df_ks$F1=abs(Fx-(0:9)/10)
> df_ks$F2=abs(Fx-(1:10)/10)
> df_ks$pmax=pmax(df_ks$F1,df_ks$F2)
> df_ks$pmax
\end{Sinput}
\begin{Soutput}
 [1] 0.1125 0.0550 0.1525 0.0525 0.0650 0.0725 0.0575 0.1500 0.1375 0.2225
\end{Soutput}
\begin{Sinput}
> D10=max(df_ks$pmax)
> D10
\end{Sinput}
\begin{Soutput}
[1] 0.2225
\end{Soutput}
\end{Schunk}

\begin{Schunk}
\begin{Sinput}
> knitr::kable(df_ks)
\end{Sinput}
\begin{Soutput}
|  i|     x|  Fn|     Fx|     F1|     F2|   pmax|
|--:|-----:|---:|------:|------:|------:|------:|
|  1| -1.55| 0.1| 0.1125| 0.1125| 0.0125| 0.1125|
|  2| -1.42| 0.2| 0.1450| 0.0450| 0.0550| 0.0550|
|  3| -1.41| 0.3| 0.1475| 0.0525| 0.1525| 0.1525|
|  4| -0.59| 0.4| 0.3525| 0.0525| 0.0475| 0.0525|
|  5| -0.14| 0.5| 0.4650| 0.0650| 0.0350| 0.0650|
|  6|  0.11| 0.6| 0.5275| 0.0275| 0.0725| 0.0725|
|  7|  0.57| 0.7| 0.6425| 0.0425| 0.0575| 0.0575|
|  8|  0.60| 0.8| 0.6500| 0.0500| 0.1500| 0.1500|
|  9|  1.05| 0.9| 0.7625| 0.0375| 0.1375| 0.1375|
| 10|  1.11| 1.0| 0.7775| 0.1225| 0.2225| 0.2225|
\end{Soutput}
\end{Schunk}

\begin{Schunk}
\begin{Sinput}
> ks.test(x,"punif",-2,2)
\end{Sinput}
\begin{Soutput}
	One-sample Kolmogorov-Smirnov test

data:  x
D = 0.2225, p-value = 0.629
alternative hypothesis: two-sided
\end{Soutput}
\end{Schunk}

\begin{Schunk}
\begin{Sinput}
> x=c(0:4)
> oi=c(10,26,35,25,5)
> pi=1.88^x/factorial(x)*exp(-1.88)
> ppois(x,1.88)
\end{Sinput}
\begin{Soutput}
[1] 0.1525901 0.4394595 0.7091167 0.8781019 0.9575250
\end{Soutput}
\begin{Sinput}
> sum(pi)
\end{Sinput}
\begin{Soutput}
[1] 0.957525
\end{Soutput}
\begin{Sinput}
> pi[5]=1-sum(pi[1:4])
> sum(pi)
\end{Sinput}
\begin{Soutput}
[1] 1
\end{Soutput}
\begin{Sinput}
> pi
\end{Sinput}
\begin{Soutput}
[1] 0.1525901 0.2868694 0.2696572 0.1689852 0.1218981
\end{Soutput}
\begin{Sinput}
> ei=round(100*pi,2)
> ei
\end{Sinput}
\begin{Soutput}
[1] 15.26 28.69 26.97 16.90 12.19
\end{Soutput}
\begin{Sinput}
> (oi-ei)^2
\end{Sinput}
\begin{Soutput}
[1] 27.6676  7.2361 64.4809 65.6100 51.6961
\end{Soutput}
\begin{Sinput}
> num_1=(oi-ei)^2
> num_2=(oi-ei)^2/ei
> sum(round((oi-ei)^2/ei,4))
\end{Sinput}
\begin{Soutput}
[1] 12.5792
\end{Soutput}
\end{Schunk}

\begin{Schunk}
\begin{Sinput}
> df_chi2test=data.frame(x,oi,pi,ei,num_1=num_1,num_2=num_2)
> knitr::kable(df_chi2test)
\end{Sinput}
\begin{Soutput}
|  x| oi|        pi|    ei|   num_1|     num_2|
|--:|--:|---------:|-----:|-------:|---------:|
|  0| 10| 0.1525901| 15.26| 27.6676| 1.8130799|
|  1| 26| 0.2868694| 28.69|  7.2361| 0.2522168|
|  2| 35| 0.2696572| 26.97| 64.4809| 2.3908380|
|  3| 25| 0.1689852| 16.90| 65.6100| 3.8822485|
|  4|  5| 0.1218981| 12.19| 51.6961| 4.2408614|
\end{Soutput}
\begin{Sinput}
> chi2=sum(df_chi2test$num_2)
> chi2
\end{Sinput}
\begin{Soutput}
[1] 12.57924
\end{Soutput}
\end{Schunk}

El número de clases es $k=5$ y  hemos estimado un paramétro la media de la Poisson. La muestra es grade y todas las frecuencias esperadas $e_i$ son mayores de 5 el estadístico de contraste sigue una ley $\chi^2_{k-1-1}=\chi^2_{3}.$  por lo tanto el p.valor del contraste es $P(\chi^2_3> 12.5792445984489)=1- P(\chi^2_3\leq 12.5792445984489)=1-pchisq(12.57924,3)=0.00564078914788602.$




problema 28 

\begin{Schunk}
\begin{Sinput}
> errores=c(5,3,2,1)/100
> errores
\end{Sinput}
\begin{Soutput}
[1] 0.05 0.03 0.02 0.01
\end{Soutput}
\begin{Sinput}
> amplitudes=2*errores
> nivel_confiaza=c(0.95,0.99)
> n95=ceiling(qnorm(1-0.05/2)^2/amplitudes)
> n95
\end{Sinput}
\begin{Soutput}
[1]  39  65  97 193
\end{Soutput}
\begin{Sinput}
> n99=ceiling(qnorm(1-0.01/2)^2/amplitudes)
> n99
\end{Sinput}
\begin{Soutput}
[1]  67 111 166 332
\end{Soutput}
\begin{Sinput}
> df=data.frame(n95,n99,amplitudes,errores)
> knitr::kable(df)
\end{Sinput}
\begin{Soutput}
| n95| n99| amplitudes| errores|
|---:|---:|----------:|-------:|
|  39|  67|       0.10|    0.05|
|  65| 111|       0.06|    0.03|
|  97| 166|       0.04|    0.02|
| 193| 332|       0.02|    0.01|
\end{Soutput}
\end{Schunk}


EL error en la estimació de la porporción es de $\pm 5\%$  con una probabilidad del 95\% para una muetras de tamaño $n=39$  supuesto que $p=q=0.5$


