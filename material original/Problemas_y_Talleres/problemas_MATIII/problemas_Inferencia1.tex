\documentclass[12pt]{article}\usepackage[]{graphicx}\usepackage[]{color}
%% maxwidth is the original width if it is less than linewidth
%% otherwise
\usepackage{framed}
\makeatletter
\newenvironment{kframe}{%
 \def\at@end@of@kframe{}%
 \ifinner\ifhmode%
  \def\at@end@of@kframe{\end{minipage}}%
  \begin{minipage}{\columnwidth}%
 \fi\fi%
 \def\FrameCommand##1{\hskip\@totalleftmargin \hskip-\fboxsep
 \colorbox{shadecolor}{##1}\hskip-\fboxsep
     % There is no \\@totalrightmargin, so:
     \hskip-\linewidth \hskip-\@totalleftmargin \hskip\columnwidth}%
 \MakeFramed {\advance\hsize-\width
   \@totalleftmargin\z@ \linewidth\hsize
   \@setminipage}}%
 {\par\unskip\endMakeFramed%
 \at@end@of@kframe}
\makeatother

\definecolor{shadecolor}{rgb}{.97, .97, .97}
\definecolor{messagecolor}{rgb}{0, 0, 0}
\definecolor{warningcolor}{rgb}{1, 0, 1}
\definecolor{errorcolor}{rgb}{1, 0, 0}
\newenvironment{knitrout}{}{} % an empty environment to be redefined in TeX

\usepackage{alltt}
\usepackage{enumerate}
\usepackage{eurosym}
\usepackage[utf8]{inputenc}
 %\input{8bitdefs}
% \textwidth 15cm
% \textheight 22cm
 \setlength{\textwidth}{16.5cm}
\setlength{\textheight}{24cm}
 \setlength{\oddsidemargin}{-0.3cm}
 \setlength{\evensidemargin}{1cm} \addtolength{\headheight}{\baselineskip}
\addtolength{\topmargin}{-3cm}
\newcommand{\RR}{\mbox{I\kern-.2em\hbox{R}}}
\def\N{I\!\!N}
\def\R{I\!\!R}
\def\Z{Z\!\!\!Z}
\def\Q{O\!\!\!\!Q}
\def\C{I\!\!\!\!C}

% \def\Q{{\mathchoice {\setbox0=\hbox{$\displaystyle\rm
% Q$}\hbox{\raise
% 0.15\ht0\hbox to0pt{\kern0.4\wd0\vrule height0.8\ht0\hss}\box0}}
% {\setbox0=\hbox{$\textstyle\rm Q$}\hbox{\raise
% 0.15\ht0\hbox to0pt{\kern0.4\wd0\vrule height0.8\ht0\hss}\box0}}
% {\setbox0=\hbox{$\scriptstyle\rm Q$}\hbox{\raise
% 0.15\ht0\hbox to0pt{\kern0.4\wd0\vrule height0.7\ht0\hss}\box0}}
% {\setbox0=\hbox{$\scriptscriptstyle\rm Q$}\hbox{\raise
% 0.15\ht0\hbox to0pt{\kern0.4\wd0\vrule height0.7\ht0\hss}\box0}}}}
\IfFileExists{upquote.sty}{\usepackage{upquote}}{}
\begin{document}

%\pagestyle{empty}
\font\sc=cmcsc10
\parskip=1ex
%%\newcount\problemes
\newcounter{problemes}
\setcounter{problemes}{0}

%%%\newcount\problemes
%%%\problemes=0

%%%\newenvironment{prob}{\vskip 0.25cm\addtocounter{problemes}{1}
%%%\noindent{\textbf{\thebloque.\theproblemes.- }}}
\newenvironment{prob}{\vskip 0.25cm\addtocounter{problemes}{1}
\noindent{\textbf{\theproblemes.- }}}
%\newenvironment{resolucion}{\hfill $\bullet$}


\newcommand{\sol}[1]{{\textbf{\footnotetext[\theproblemes]{Sol.: #1} }}}

\newcommand{\probl}{\vskip 0.25cm\addtocounter{problemes}{1}
\noindent{\textbf{\theproblemes.- }}}

%\newcommand{\sol}[1]{{\textbf{\footnotetext[\the\problemes]{Sol.: #1} }}}
%%%
%%%\newcommand{\probl}{\addtocounter{problemes}{1} \vskip 2ex \noindent
%%%{\bf \the\problemes)}}
%%%
%%%\def\probl{\advance\problemes by 1
%%%\vskip 1mm\noindent{\bf \the\problemes) }}
%%%\newcounter{pepe}



%%%%%PROBABILIDAD

%%%\newcount\problemes
%%%\problemes=0

%\chapter*{\large \textbf{BLOQUE 1 ESTAD?STICA DESCRIPTIVA (resumen)}}



\textbf{Muestreo. Distribuciones muestrales}

\begin{prob}
El precio medio del $m^2$ en  la venta de casas nuevas durante el último año
en una determinado barrio periférico de una  ciudad  fue de 115000  La desviación
típica de la población fue de 25000 \euro. Se toma una muestra aleatoria de 100 casas nuevas de esta ciudad.
\begin{enumerate}[a)]
\item ¿Cuál es la probabilidad de que la media muestral de los precios de venta sea menor que 110000 \euro?
\item ¿Cuál es la probabilidad de que la media muestral de los precios de venta esté entre 113000 \euro y 117000 \euro?
\item ¿Cuál es la probabilidad de que la media muestral de los precios de venta esté entre 114000 \euro y 116000 \euro?
\item Sin hacer cálculos, razonar en cuál de los 
siguientes rangos resulta más probable que se encuentre la media muestral de los precios de venta:\break
\begin{center}
\begin{tabular}{ll}
113000 \euro-&115000 \euro\\
114000 \euro-&116000 \euro\\
115000 \euro-&117000 \euro\\
116000 \euro-&118000 \euro
\end{tabular}
\end{center}
\end{enumerate}
\sol{a) 0.0228; b) 0.5762; c) 0.3108: d) el intervalo  114000 \euro-116000 \euro}
\end{prob}

\begin{prob}
Se ha tomado una muestra de 16 directores de oficina  de
corporaciones de una gran ciudad, con el fin de estimar el tiempo
medio que emplean en desplazarse para ir a su trabajo. Supongamos
que la distribución de dichos tiempos en la población sigue una
normal con media 87 minutos y desviación típica  22.
\begin{enumerate}[a)]
\item ¿Cuál es el error estándar de la media muestral de los
tiempos de desplazamiento?
\item ¿Cuál es la probabilidad de que la media muestral sea
inferior a 100 minutos?
\item ¿Cuál es la probabilidad de que la media muestral sea
superior  a 80 minutos?
\item ¿Cuál es la probabilidad de que la media muestral esté entre
85 y 95 minutos?
\item Supongamos que se toma una segunda muestra de quince directores,
independiente de la anterior. Sin hacer los cálculos, razonar si la probabilidades
calculadas en los apartados b), c) y d)  serán mayores, menores o iguales para esta segunda
muestra. Utilizar gráficos para ilustrar las respuestas.
\end{enumerate}
\sol{a) 5.5 ; b) 0.9909; c) 0.8980; d) 0.5671; e) es menor en los
tres casos.}
\end{prob}

\begin{prob}
Una compañía produce cereales para el desayuno. La media del
peso que contienen las cajas de estos cereales es de doscientos
gramos y su desviación típica de seis gramos. La distribución de
los pesos de la población es normal. Se eligen cuatro cajas, que
pueden considerarse como una muestra aleatoria del total de la
producción.
\begin{enumerate}[a)]
\item ¿Cuál es el error estándar de la media muestral del peso de
las cuatro cajas?
\item ¿Cuál es la probabilidad de que la media del peso de esas cuatro cajas
sea inferior  que 197 gramos?
\item ¿Cuál es la probabilidad , en media, el peso de estas
cuatro cajas esté entre 105 y 195 gramos?
\item ¿Cuál es la probabilidad de que la suma del peso de estas
cuatro cajas sea menor de 800 gr.?
\item Se eligen al azar dos de estas cuatro cajas ¿Cuál es la
probabilidad de que, en media, el contenido de estas dos cajas
pese entre 195 y 200 gramos?
\end{enumerate}
\sol{a) 3; b) 0.1587; c) 0.0475; d) 0.5; e) 0.3810}
\end{prob}

\begin{prob} La tasa de rentabilidad de ciertos tipos de acciones
sigue una distribución con desviación típica 3.8. Se extrae una
muestra de tales acciones con el fin de estimar el precio medio.
\begin{enumerate}[a)]
\item ¿Qué tamaño ha de tener la muestra para asegurarnos que la
probabilidad de que la media muestral difiera de la media
poblacional en una cantidad superior a 1 sea menor que 0.1?
\item Sin realizar los cálculos razonar si será preciso  un tamaño
muestral mayor o menor que el requerido  en el apartado a) para
garantizar que la probabilidad de que la media muestral difiera
de la media poblacional en más de 1 sea inferior a 0.05.
\item Sin realizar los cálculos razonar si será preciso  un tamaño
muestral mayor o menor que el requerido  en el apartado a) para
garantizar que la probabilidad de que la media muestral difiera
de la media poblacional en más de 1.5 sea inferior a 0.1.
\end{enumerate}
\sol{a) $n\geq 40$; b) mayor; c) menor}
\end{prob}

\begin{prob}
De acuerdo con los datos del ministerio de Economía y Hacienda, el
15\% de las declaraciones del IRPF del último año darán lugar a
una devolución. Se toma una muestra aleatoria de 10 declaraciones.
\begin{enumerate}[a)]
\item ¿Cuál es la media de la distribución en el muestreo de la
proporción muestral de declaraciones que darán lugar a una
devolución?
\item ¿Cuál es la varianza de la proporción muestral?
\item ¿Cuál es el error estándar de la proporción muestral?
\item ¿Cuál es la probabilidad de que la proporción muestral sea
mayor que 0.8?
\end{enumerate}
\sol{a) 0.15; b) 0.01275; c) 0.1129; d) casi nula.}
\end{prob}

%PPPPPPPPPPPPP

\begin{prob}
El dueño de una portal de ventas de discos por Internet ha comprobado que el 20\% de los
clientes que acceden a su portal realizan una compra. Cierta mañana
entraron en el portal 180 personas, que pueden ser consideradas como
una muestra aleatoria de todos sus clientes.
\begin{enumerate}[a)]
\item ¿Cuál será la media de la proporción muestral de clientes
que realizaron alguna compra?
\item ¿Cuál es la varianza de la proporción muestral?
\item ¿Cuál es el error estándar de la proporción muestral?
\item ¿Cuál es la probabilidad de que la proporción muestral sea
mayor que 0.15?
\end{enumerate}
\sol{a) 0.2; b) $\approx 0.0009$; c) $0.03$; d) 0.9525}
\end{prob}

\begin{prob}
El administrador de una gran cadena de hospitales opina que, de
entre todos sus pacientes, el 30\% generará facturas que se pagarán
con más de dos meses de retraso. Se toma una muestra aleatoria de 200
pacientes.
\begin{enumerate}[a)]
\item ¿Cuál es el error estándar de la proporción muestral de
pacientes con facturas cuyo pago se retrasará dos meses?
\item ¿Cuál es la probabilidad de que la proporción muestral sea
inferior a   0.25?
\item  ¿Cuál es la probabilidad de que la proporción muestral sea
mayor que 0.33?
\item  ¿Cuál es la probabilidad de que la proporción muestral esté
entre 0.27 y 0.33?
\item Sin realizar los cálculos, razonar en cuál de los siguientes
intervalos es más probable que se encuentre la proporción
muestral: 0.29-0.31; 0.30-0.32; 0.31-0.33; 0.32-0.34.
\item Supongamos que se toma al azar una muestra de 500 pacientes.
Sin realizar los cálculos razonar si las probabilidades de los
apartados b), c)  y d) resultarán en este caso mayores, menores o iguales
que las calculadas para la muestra anterior.
\end{enumerate}
\sol{a) 0.0324; b) 0.0618; c) 0.1762; d) 0.6476; e) 0.29-0.31;  f)
menor, menor, mayor}
\end{prob}

\begin{prob}
Se toma una muestra aleatoria de 100 votantes con el fin
de estimar la proporción de los mismos que están a favor de
un aumento en los impuestos sobre la gasolina para contar
así con un ingreso adicional para reparaciones de las
autopistas.  ¿Cuál es el mayor valor que puede tomar el
error estándar de la proporción muestral de esta medida?
\sol{0.05}
\end{prob}

\begin{prob}
Continuando en la situación del problema anterior, se decide que
una muestra de 100 votantes es muy pequeña para obtener una
estimación de la proporción poblacional que resulte
suficientemente creíble. Se decide exigir que la probabilidad de
que la proporción muestral difiera de la proporción poblacional
(cualquiera que sea su valor) en más de 0.03 no debe ser superior
a 0.05. ¿Qué tamaño ha de tener la muestra para poder garantizar
que se cumple este requisito?
\sol{$n\geq 757$}
\end{prob}

\begin{prob}
Una compañía quiere estimar la proporción de personas que son posibles compradores de
máquinas de afeitar eléctricas que ven retransmisiones partidos de La Liga de Campeones. Se
toma una muestra de 120 individuos que se identificaron como posibles compradores de
afeitadoras eléctricas. Supongamos que la proporción de posibles compradores de afeitadoras
eléctricas  en la población que ven estas retransmisiones es 0.25.
\begin{enumerate}[a)]
\item 0.10 es la probabilidad de que la proporción muestral
exceda a la proporción poblacional ¿en qué valor?
\item 0.05 es la probabilidad de que la proporción muestral
esté por debajo de la proporción poblacional ¿en qué cantidad?
\item 0.30 es la probabilidad de que la proporción muestral
difiera de la proporción poblacional ¿en menos de qué cantidad?
\end{enumerate}
\sol{a)  0.0506; b) 0.0648; c) 0.0154}
\end{prob}
\begin{prob}
Supongamos que el 50\% de los españoles adultos opina que es
necesaria una revisión del sistema nacional público de hospitales.
¿Cuál es la probabilidad de que más del 56\% de los componentes
de una muestra de 150 españoles adultos tenga esa opinión?
\sol{ 0.0708}
\end{prob}

\begin{prob} Las rentabilidades mensuales de cierto tipo de
acciones son independientes unas de otras, y siguen una
distribución normal con desviación típica 1,7. Se toma una
muestra de 12 meses.
\begin{enumerate}[a)]
\item Hallar la probabilidad de que la desviación típica muestral
sea menor que 2.5.
\item Hallar la probabilidad de que la desviación típica
muestral sea mayor que 1.
\end{enumerate}
\sol{a) 0.98; b) 0.975}
\end{prob}

\begin{prob} El número de horas que dedican a ver la televisión
los estudiantes en la semana anterior a los exámenes finales
sigue una distribución normal con una desviación típica de 4.5
horas. Se toma una muestra aleatoria de 30 estudiantes.

\begin{enumerate}[a)]
\item La probabilidad  de que la desviación típica muestral sea
mayor que 3.5 horas, ¿es mayor que 0.95?
\item La probabilidad de que la desviación típica muestral sea
menor que seis horas, ¿es mayor que 0.95?
\end{enumerate}
\sol{ a) Sí; b) Sí }
\end{prob}
\begin{prob} Se extrae una muestra aleatoria de 15 economistas
y se les pregunta acerca de su predicción sobre la tasa de
inflación para el próximo año. Supongamos que las predicciones
para la población completa de economistas sigue una
distribución normal con una desviación típica de 1.8.
\begin{enumerate}[a)]
\item 0.01 es la probabilidad de que la desviación típica sea
mayor que ¿qué número?
\item 0.025 es la probabilidad de que la desviación típica sea
menor que ¿qué número?
\item Encontrar una par de números, a y b, tales que la
probabilidad de que la desviación típica muestral se encuentre
entre ellos sea 0.9.
\end{enumerate}
\sol{a) 2.5969; b) 1.1415; c) 1.2331; 2.341}
\end{prob}


\newpage
\textbf{Estimación puntual}

\begin{prob}
Se toma una muestra de ocho lotes de un producto químico para
comprobar la concentración de impurezas. Los niveles porcentuales
de impurezas encontrados en la muestra fueron
$$3.2 \ 4.3 \ 2.1 \ 2.8\ 3.2 \ 3.6\ 4.0\ 3.8$$
\begin{enumerate}[a)]
\item Hallar la media y la varianza muestrales.
Hallar la proporción muestral de lotes con nivel
porcentual de impurezas mayor que 3.75\%.
\item ¿Para qué parámetros poblacionales  se han hallado en la
parte  a) estimadores por procedimientos insesgados?
\end{enumerate}
\sol{a) $\overline{X}=3.375$; $S_{X}^2=0.4993$; $\hat{p}=
0.3557$; b) para todos.}
\end{prob}
% \begin{prob}
%   Una muestra de diez economistas hicieron las siguientes predicciones
%   para el crecimiento del producto interior bruto el próximo año.
%
%
%   \begin{tabular}{lllll}
%       2.2 & 2.82& 2.02 & 2.52& 2.42\\
%       2.62 & 2.52& 2.42 & 2.72& 2.6
%       \end{tabular}
%   \end{prob}

\begin{prob}
Sea $\hat{\theta}_{1}$ un estimador insesgado de $\theta_{1}$, y
$\hat{\theta}_{2}$ un estimador insesgado de $\theta_{2}$.
\begin{enumerate}[a)]
\item Probar que $(\hat{\theta}_{1}+\hat{\theta}_{2})$ es un
estimador insesgado de $(\theta_{1}+\theta_{2})$.
\item  Probar que $(\hat{\theta}_{1}-\hat{\theta}_{2})$ es un
estimador insesgado de $(\theta_{1}-\theta_{2})$.
\end{enumerate}
\end{prob}

\begin{prob}
Sea $X_{1}$ y $X_{2}$ una muestra aleatoria de dos observaciones
independientes de
una población con media $\mu$ y varianza $\sigma^2$. Considerar los
siguientes tres estimadores puntuales de $\mu$:
$$\begin{array}{l}
\overline{X}=\frac{1}{2} X_{1}+\frac{1}{2} X_{2}\\
\hat{\mu}^{(1)}=\frac{1}{4}X_{1}+\frac{3}{4} X_{2}\\
\hat{\mu}^{(2)}=\frac{1}{3}X_{1}+\frac{2}{3} X_{2}
\end{array}
$$
\begin{enumerate}[a)]
\item Probar que los tres estimadores son insesgados.
\item ¿Cuál de los tres estimadores es más eficiente?
\item Hallar la eficiencia relativa de $\overline{X}$ con respecto
a los otros estimadores.
\end{enumerate}
\sol{ b) $\overline{X}$; c) $\frac{Var(\overline{X})}{Var(\hat{\mu}^{(1)})}=0.8$;
$\frac{Var(\overline{X})}{Var(\hat{\mu}^{(2)})}=0.9$}
\end{prob}

\begin{prob}
A una clase de estadística asisten estudiantes de Informática de Gestión y de
Sistemas. En una muestra de
diez estudiantes de Gestión  se observaron las
siguientes calificaciones en el examen final
$$62 \ 57 \ 85 \ 59\ 64\ 63\ 71\ 58\ 77\ 72$$
En una muestra independiente de ocho estudiantes de Sistemas se
observaron las siguientes calificaciones en el mismo examen
$$73 \ 79 \ 85 \ 73\ 62\ 51\ 60\ 57$$
\begin{enumerate}[a)]
\item Utilizar un método de estimación insesgado para
obtener una estimación puntual de la diferencia de las
calificaciones medias entre los estudiantes de Gestión y
los de Sistemas. (Indicación: Utilizar el problema 151)
\item Utilizar un método de estimación insesgado para obtener una
estimación puntual de la diferencia entre la proporción poblacional de
estudiantes que obtuvieron una calificación  mayor que 70 en el
grupo de estudiantes de Gestión y el grupo de Sistemas.
(Indicación: Utilizar el problema 151)
\end{enumerate}
\sol{a) 0.2444; b)$-\frac{1}{10}$}
\end{prob}
\begin{prob}
Se toma una muestra aleatoria $X_{1},X_{2},\ldots,X_{n}$  de una población con media $\mu$
y varianza $\sigma^2$. Se considera el siguiente estimador de $\mu$ :
$$\hat{\mu}=\frac{2}{n(n+1)}(X_{1}+2X_{2}+3X_{3}+\ldots+nX_{n})$$
\begin{enumerate}[a)]
\item Probar que $\hat{\mu}$ es un estimador insesgado de $\mu$.
\item Hallar la eficiencia relativa de $\hat{\mu}$  respecto a
$\overline{X}$, la media muestral.
\end{enumerate}
\sol{ b) $Var(\hat{\mu})=\left(\frac{2}{n(n+1)}\right)^2 \frac{n (n+1) (2n+1)}{6}
\sigma^2$; $Eff.rel=\frac{Var(\hat{\mu})}{Var(\overline{X})}=\frac{2\,\left( 1 + 2\,n \right) }
{3\,\left( 1 + n \right)}.$ }
\end{prob}

\begin{prob}
\begin{enumerate}[a)]
\item (Examen junio 2003) Calcular el estimador máximo verosímil (MLE\footnote{Del inglés
``Maximun Likelihood  Estimator''}) para el parámetro $\lambda$ de una población que
sigue una ley $Exp(\lambda)$ para una muestra aleatoria simple de tamaño $n$.
\item (Examen septiembre 2004) Calcular el MLE para el parámetro $\lambda$ de una población
que sigue una ley $Po(\lambda)$ para una muestra aleatoria simple de tamaño $n$.
\item  Calcular el MLE para el parámetro $\mu$ de una población
que sigue una ley $N(\mu,\sigma^2)$ para una muestra aleatoria simple de tamaño $n$.
\item  Calcular el MLE para el parámetro $\sigma^2$ de una población
que sigue una ley $N(\mu,\sigma^2)$ para una muestra aleatoria simple de tamaño $n$.
\item Estudiar si los estimadores MLE de los apartados anteriores son insesgados.
\end{enumerate}
\end{prob}




\textbf{Estimación por intervalos}

\begin{prob}
De una población de barras de hierro se extrae una muestra de $64$ barras y  se calcula la
resistencia a la rotura por tracción se obtiene que $\overline{X}=1012\ Kg/cm^{2}$. Se sabe
por experiencia que en este tipo de barras $\sigma=25$. Calcular un intervalo de confianza
para $\mu$ al nivel 0.95. \sol{$\left(1005.88 , 1018.13\right)$}
\end{prob}

\begin{prob}
Para investigar el C.I. medio de una cierta población de estudiantes, se realiza un test a
$400$ estudiantes. La media y la desviación típica  muestrales obtenidas son
$\overline{x}=86$ y $\tilde{s}_{X}=10.2$. Calcular un intervalo para $\mu$ con un nivel de
significación del 98\%. \sol{$\left(84.8117, 87.1883\right)$}
\end{prob}

\begin{prob}
Para investigar un nuevo tipo de combustible para cohetes espaciales, se disparan cuatro
unidades y se miden las velocidades iniciales. Los resultados obtenidos, expresados en
Km/h, son :19600, 20300, 20500, 19800. Calcular un intervalo para la velocidad media $\mu$
con un nivel de confianza del 95\%, suponiendo que las velocidades son normales.
\sol{$\left(19381.7, 20718.3\right)$} 20718.3
\end{prob}

\begin{prob}
Un fabricante de cronómetros quiere calcular un intervalo de estimación de la desviación
típica del tiempo marcado en $100$ horas por todos los cronómetros  de un cierto modelo.
Para ello pone en marcha $10$ cronómetros del modelo durante $100$ horas y encuentra que
$\tilde{s}_{X}=50$ segundos. Encontrar un intervalo para el parámetro $\sigma^2$ con
$\alpha=0.01$, suponiendo que la población del tiempo marcado por los cronómetros es
normal. \sol{$\left(953.834,12968.3\right)$}
\end{prob}

% \begin{prob}
% El mismo fabricante del problema anterior quiere comparar ahora las
% varianzas de dos modelos de cronómetros (modelo 1 y modelo 2)
% tomando dos muestras con los siguientes resultados: $n_{1}=10$,
% $\tilde{S}_{1}=50$ $n_{2}=16$ y $\tilde{S}_{2}=45$. Utilizar
% $\alpha=0.01$
% \sol{$[0.52,2.72]$}
% \end{prob}

\begin{prob}
Un auditor informático quiere investigar la proporción de rutinas de un programa que
presentan una determinada irregularidad. Para ello observa $120$ rutinas, resultando que
$30$ de ellas presentan alguna irregularidad. Con estos datos buscar unos límites de
confianza para la proporción $p$ de rutinas de la población que presentan esa irregularidad
con probabilidad del 95\%.

\sol{$\left(0.1725,0.3275\right)$}
\end{prob}

% \begin{prob} En una ciudad A de 400 propietarios de coches 125 tienen
% una marca $X$; y en otra población B, de 600 propietarios, 180
% tienen la marca $X$. Calcular un intervalo de confianza del 95\% para
% la diferencia de proporciones entre la ciudad A y la B.
% \sol{ $\left(-0.111,0.136\right)$}
% \end{prob}
%
%
%%%%%%%%%%%%%%%
\begin{prob}(Examen septiembre 2003) Una infección por un virus puede haber perjudicado a muchos ordenadores con
\emph{Windwos}. Desde el
Centro de Alerta Temprana (CAT) se quiere calcular la
proporción de ordenadores infectados. El jefe del centro  os pide que calculéis
el tamaño de una muestra para que el intervalo de confianza de la proporción muestral de
ordenadores infectados tenga amplitud de a lo sumo $0.01$ con una probabilidad del 90\%.
\end{prob}

\begin{prob}(Examen junio 2003) Se han medido los siguientes valores (en miles de personas) para la audiencia de un
programa de televisión en distintos días (supuestos igualmente distribuidos e
independientes):

$$521, 742, 593, 635, 788, 717, 606, 639, 666, 624.$$

Construir un intervalo de confianza del $90$\%, para la audiencia poblacional media y otro
para la varianza poblacional, bajo la hipótesis de que la población de audiencias sigue una
ley normal.

Nota Suma de las audiencias=$6531$, Suma de los cuadrados de las audiencias=$4320401$.
\end{prob}

\begin{prob}
Supongamos que la empresa para la que trabajamos está en un proyecto de investigación,
financiado con fondos de la Comunidad Europea, que pretende extender una nueva aplicación
de las TIC. Una de las tareas del proyecto es realizar una encuesta de opinión sobre el
grado de aceptación que tendría esta nueva tecnología en el mercado europeo. De entre todas
las universidades y empresas participantes en el proyecto, es a tu empresa a la que le toca
hacer el protocolo de la encuesta, llevarla a cabo y redactar esta parte del informe final.
Como eres el último que llegó a la empresa y el resto de miembros del equipo no se acuerda
de la estadística que vio en la carrera, te toca a ti cargar con la responsabilidad. Claro
que el coste de la encuesta depende del número $n$ de entrevistas que se realicen y el
error de las proporciones de las contestaciones disminuye cuando $n$ aumenta. Como no sabes
cuánto dinero está dispuesto a gastar tu jefe, tabula los tamaños muestrales para los
errores $\pm 5\%, \pm 3\%, \pm 2\%,\pm 1\%$, y para niveles de confianza $0.95$ y $0.99$,
suponiendo el peor caso. Añade un comentario para que el equipo de dirección del proyecto,
en el que hay componentes ignorantes en  materia de encuestas, vea como quedarían redactado
los datos técnicos de la encuesta, y pueda decidir  el tamaño muestral leyendo tu informe.
\end{prob}

\begin{prob}
El número de reservas semanales de billetes de cierto vuelo de una compañía aérea sigue una
distribución aproximadamente normal. Se toma un muestra aleatoria de $81$ observaciones de
números de reservas de este vuelo: el número medio de reservas muestral resulta ser $112$,
mientras que la desviación típica muestral es  $36$. Además de estos $81$ vuelos, $30$
llegaron a su destino con un retraso de más de $15$ minutos.
\begin{enumerate}[a)]
\item Calcular un intervalo de confianza del $95\%$ para el número medio
poblacional de reservas en este vuelo.
\item Calcular un intervalo de confianza de $95\%$ para la varianza
poblacional de las reservas.
\item Calcular un intervalo de confianza del $95\%$ para la proporción
poblacional de vuelos que llegan con un retraso de más de $15$ minutos.
\item Calcular el tamaño muestral que asegura un intervalo de
confianza de amplitud $0.1$ para la proporción de vuelos que llegan
con un retraso de más de $15$ minutos al nivel de confianza $95\%$.
\end{enumerate}
\sol{a) $\left(104.16,119.84)\right)$; b) $\left(972.343,1814.08)\right)$; c)
$\left(0.265,0.475)\right)$; d) $n=385$}
\end{prob}

\begin{prob}
Una empresa cervecera sabe que las cantidades de cerveza que contienen
sus latas sigue una distribución normal con desviación típica poblacional $0.03$ litros.
\begin{enumerate}[a)]
\item Se extrae una muestra aleatoria de $25$ latas y, a partir de la
misma, un experto en estadística construye un  intervalo de confianza
para la media poblacional del contenido en litros de las latas que
discurre entre $0.32$ y $0.34$ ¿Cuál es el nivel de confianza de este
intervalo?
\item Un gerente de esta empresa exige un intervalo de confianza del
$99\%$  que tenga una amplitud máxima de $0.03$ litros a cada lado de la
media  muestral ¿Cuántas observaciones son necesarias, como mínimo,
para alcanzar este objetivo?
\end{enumerate}
\sol{a) $90.3\%$; b) $n=7$}
\end{prob}
\newpage




\textbf{Contraste de hipótesis}

\begin{prob}%10.1 pag 185 J.Amon
Siendo $\overline{x}=63.5$ la media de una muestra aleatoria simple de tamaño 36 extraída
de una población normal con $\sigma^2=144$, poner
a prueba, con un nivel de significación $\alpha=0.05$, la hipótesis
nula $\mu=60$ y decir si se rechaza en favor de la alternativa
$\mu<60$. Calcular el $p$-valor.
\end{prob}

\begin{prob}%10.2 pag 185 J.Amon
Siendo $\overline{x}=72.5$ la media de una muestra aleatoria simple de tamaño 100 extraída
de una población normal con $\sigma^2=900$, poner
a prueba, con un nivel de significación $\alpha=0.10$, la hipótesis
nula $\mu=77$ y decir si se rechaza en favor de las hipótesis
alternativas
$\mu\not= 70$, $ \mu>70$, $\mu<70$. Calcular el $p$-valor en cada caso.
\end{prob}


\begin{prob}%10.9 pag 185 J.Amon
En un contraste bilateral, con $\alpha=0.01$, ¿para qué valores
de $\overline{X}$ rechazaríamos  la hipótesis nula $H_{0}:\mu=70$, a
partir de una muestra aleatoria simple de tamaño 64 extraída de una
población normal con $\sigma^2=256$?
\end{prob}

\begin{prob}%11.1 pag 324 J.Amon
El salario anual medio de 1600 personas, elegidas aleatoria e
independientemente de cierta población de economistas con $\sigma=20000$ euros, ha
valido 45000 euros  ¿Es compatible  con este resultado la hipótesis
nula, $H_{0}:\mu=43500$, suponiendo $\alpha=.01$? ¿Cuál es el intervalo
de confianza para $\mu$? Calcular el $p$-valor.
\end{prob}

\begin{prob}%11.2 pag 324 J.Amon
Con los datos del ejercicio anterior , ¿son compatibles con el
resultado obtenido los siguientes contrastes?:
\begin{enumerate}[a)]
\item $\left\{\begin{array}{ll} H_{0}:\mu=44000\\
H_{1}:\mu>44000\end{array}\right.$
\item $\left\{\begin{array}{ll} H_{0}:\mu=46250\\
H_{1}:\mu>46250\end{array}\right.$
\end{enumerate}
\end{prob}

\begin{prob}%11.6 pag 324 J.Amon
El peso medio de los paquetes de café puestos a la venta por la casa comercial CAFEINASA es
supuestamente de 1 Kg. Para comprobar esta suposición, elegimos una muestra aleatoria
simple de 100 paquetes y encontramos que su peso  medio es de 0.978 Kg. y su desviación
típica $s=0.10$ kg. Siendo $\alpha=0.05$ ¿es compatible este resultado con la hipótesis
nula $H_{0}:\mu=1$ frente a $H_{1}:\mu\not=1$? ¿Lo es frente a $H_{1}:\mu>1$? Calcular el $p$-valor.
\end{prob}

\begin{prob}
El fabricante de la marca de tornillos FDE afirma que el diámetro medio de sus tornillos
vale 20 mm. Para comprobar dicha afirmación, extraemos aleatoria e independientemente 16
tornillos , y vemos que la media de sus diámetros  es 22 mm. y la desviación típica 4 mm.
¿Podemos aceptar la pretensión del fabricante, suponiendo $\alpha=0.05$ y siendo el
contraste bilateral? Calcular el $p$-valor.
\end{prob}

\begin{prob}
Para evitar basarse en su intuición los jefes de admisión de
personal de las grandes empresas discriminan mediante un test
diseñado por un gabinete de psicólogos, supuestamente especializado
en selección de personal, a los aspirantes a trabajar en la empresa.
La varianza del test de selección solía venir siendo $100$.
Aplicando un nuevo test a una muestra aleatoria simple de tamaño
$n=31$, se obtiene que $S=129$. Suponiendo que la población se
distribuye normalmente, ¿es compatible la hipótesis nula
$H_{0}:\sigma^2=100$, frente a la alternativa $H_{1}:\sigma^2>100$,
con $\alpha=0.01$? Calcular el $p$-valor.
\end{prob}

\begin{prob}
Una máquina produce cierto tipo de piezas mecánicas. El tiempo en
producirlas se distribuye normalmente con varianza desconocida
$\sigma^2$. Elegida una muestra aleatoria simple de 21 de dichas
piezas ($x_{1},\ldots,x_{21}$), se obtiene que $\overline{x}=30$ y
$\sum_{i=1}^{21}x_{i}^2=19100.$ Comprobar si es compatible la
hipótesis nula $H_{0}:\sigma^2=22$ frente $H_{1}:\sigma^2\not=22$,
con $\alpha=0.1$, y construir un intervalo de confianza del
$(1-\alpha)100\%$ para el verdadero valor de $\sigma^2$. Calcular el $p$-valor.
\end{prob}

\begin{prob}
A partir de las puntuaciones 15, 22, 20, 21, 19 ,23, construir el intervalo de confianza de
$\sigma^2$ y decir si es compatible con estos resultado la hipótesis $H_{0}:\sigma=2$,
siendo $\alpha=0.01$ contra una $H_1$ bilateral. Decir si se utiliza alguna hipótesis adicional. Calcular el $p$-valor.
\end{prob}

\begin{prob}
Sabiendo que con $\hat{p}=0.52$ ha sido rechazada $H_{0}:p=0.50$, al
nivel de significación $\alpha=0.05$, ¿cuál ha tenido que ser el
tamaño mínimo de la muestra mediante la cual fue rechazada
$H_{0}$

\begin{enumerate}[a)]
\item frente a $H_{1}:p\not=0.5$?
\item frente a $H_{1}:p>0.5$?
\end{enumerate}

\begin{prob}
Lanzamos una moneda al aire 10 veces consecutivas . ¿Con qué número
de caras rechazaremos la hipótesis nula de que la moneda está bien
balanceada, siendo $\alpha=0.05$?
\end{prob}
\end{prob}



\begin{prob}% mío
Un fabricante de productos farmacéuticos tiene que mantener un estándar de impurezas en el
proceso de producción de sus píldoras. Hasta ahora el número medio poblacional de impurezas
es correcto pero está preocupado porque las impurezas en algunas de las partidas se salen
del rango admitido de forma que provocan devoluciones y posibles reclamaciones por daños a
la salud. El gabinete de control de calidad afirma que si la distribución de las impurezas
es normal y que si el proceso de producción mantiene  una varianza inferior a 1 no tendría
que existir ningún problema pues las píldoras tendrían una concentración aceptable.
Preocupado por esta tema la dirección encarga una prueba externa en la que se toma una
muestra aleatoria de 100 de las partidas obteniéndose $S^2=1.1$. ¿Puede aceptar el director
de la prueba externa que el proceso de producción cumple la recomendación del gabinete de
control?
\end{prob}


\begin{prob}
Un IAP está preocupado por su estándar de calidad y quiere compararlo con el medio europeo.
El estándar medio europeo dice que una empresa de este sector tiene una calidad aceptable
si tiene un número de quejas que no excede del 3\%.

Se sabe que la varianza de las quejas es $0.16$. Examinando $64$ clientes escogidos al azar
se encuentra con que el porcentaje de quejas es del $3.07\%$. Calcular el $p$-valor.

\begin{enumerate}[a)]
\item Contrastar al nivel de significación del 5\%, la hipótesis nula
de que la media poblacional del porcentaje de quejas es del 3\%
frente a la alternativa de que es superior al 3\%.
\item Hallar el $p$-valor del contraste.
\item Supongamos que la hipótesis alternativa fuese bilateral en
lugar de unilateral( con hipótesis nula $H_{0}:\mu=3$). Deducir, sin
hacer ningún cálculo, si el $p$-valor del contraste sería mayor,
menor o igual que el del apartado anterior. Construir un gráfico
para ilustrar el razonamiento.
\item En el contexto de este problema, explicar por qué una hipótesis
alternativa unilateral es más apropiada que una bilateral.
\end{enumerate}
\end{prob}

\begin{prob}
A partir de una muestra aleatoria se contrasta:

$\left\{\begin{array}{l}
H_{0}:\mu=\mu_{0}\\
H_{1}:\mu>\mu_{0}\end{array}\right.$

y se acepta la hipótesis nula al nivel de significación del 5\%.

\begin{enumerate}[a)]
\item ¿Implica esto necesariamente que $\mu_{0}$ está contenido
en el intervalo de confianza del 95\% para $\mu$?
\item Si la media muestral observada es mayor que $\mu_{0}$,
¿implica necesariamente que $\mu_{0}$ está contenido en el
intervalo de confianza  del 90\% para $\mu$?
\end{enumerate}
\end{prob}

\begin{prob}
Una compañía que se dedica a la venta de franquicias afirma que, por término medio, los
delegados obtienen un redimiendo del 10\% en sus inversiones iniciales. Una muestra
aleatoria de diez de estas franquicias presentaron los siguientes rendimientos el primer
año de operación:

6.1, 9.2, 11.5, 8.6, 12.1, 3.9 , 8.4, 10.1, 9.4, 8.9

Asumiendo que los rendimientos poblacionales tienen distribución normal, contrastar la
afirmación de la compañía.
\end{prob}


\begin{prob}
Una distribuidora de bebidas refrescantes afirma que una buena fotografía de tamaño real de
un conocido actor, incrementará las ventas de un producto en los supermercados en una media
de $50$ cajas semanales. Para una muestra de $20$ supermercados, el incremento medio fue de
$41.3$ cajas con una desviación típica de $12.2$ cajas. Contrastar al nivel de
significación $\alpha=0.05$, la hipótesis nula de que la media poblacional del incremento
en las ventas es al menos $50$ cajas, indicando cualquier supuesto que se haga. Calcular el
$p$-valor del contraste e interpretarlo.
\end{prob}



\newpage



\textbf{Contrastes de dos parámetros.}

\textbf{Comparación de medias.}

Los siguientes problemas tratan de contrastes de parámetros entre dos muestras. Para cada uno de los enunciados contrastar contra las  hipótesis unilaterales y bilaterales. Calcular también el intervalo de confianza para la diferencia o el cociente de los parámetros. Tomar finalmente la decisión más correcta. Calcular todos los test e intervalos de confianza para $\alpha=0.05$. Calcular el $p$-valor en cada caso. 

\begin{prob}
Para comparar la producción  media de dos procedimientos de fabricación de cierto elemento
se toman dos muestras, una con los elementos fabricados durante 25 días con el primer
método y otra con los producidos durante 16 días con el segundo método. Por experiencia se
sabe que la varianza del primer procedimiento es $\sigma_{1}^2=12$ y al del segundo
$\sigma_{2}^2=10$. De las muestras obtenemos que $\overline{X}_{1}=136$ para el primer
procedimiento y $\overline{X}_{2}=128$ para el segundo. 
% Si $\mu_{1}$ y $\mu_{2}$ son los
% valores esperados para cada uno de los procedimientos, calcular un intervalo de confianza
% para $\mu_{1}-\mu_{2}$ al nivel 99\%. \sol{$\left(5.2989,10.7011\right)$}
\end{prob}

\begin{prob}
Estamos interesados en comparar la vida media, expresada en horas de dos tipos de
componentes electrónicos. Para ello se toma una muestra de cada tipo y se obtiene:

$$
\begin{tabular}{|c|c|c|c|}
\hline Tipo & tamaño & $\overline{X}$ & $S$\\ \hline \hline 1 & 50 & 1260 & 20\\ \hline 2 &
100 & 1240 & 18\\ \hline
\end{tabular}
$$

% Calcular un intervalo de confianza para $\mu_{1}-\mu_{2}$ ($\mu_{1}$ esperanza del primer
% grupo y $\mu_{2}$ esperanza del segundo grupo) al nivel 98\% 
Suponer si es necesario las
poblaciones aproximadamente normales. % \sol{$\left(12.19,27.81\right)$}
\end{prob}

\begin{prob}
Para reducir la concentración de ácido úrico en la sangre se prueban dos drogas. La primera
se aplica a un grupo de 8 pacientes y la segunda a un grupo de 10. Las disminuciones
observadas en las concentraciones de ácido úrico de los distintos pacientes expresadas en
tantos por cien de concentración después de aplicado el tratamiento son:

\begin{center}
\begin{tabular}{|c|c|c|c|c|c|c|c|c|c|c|}
droga 1 & 20 & 12 & 16 & 18 & 13 & 22 & 15 & 20\\ \hline droga 2 & 17 & 14 & 12 & 10 & 15 &
13 & 9 & 19 & 20 & 11
\end{tabular}
\end{center}

% Calcular un intervalo de confianza para la diferencia de medias entre la primera droga y la
% segunda al nivel del 99\%.
Suponer que las reducciones de ácido úrico siguen una
distribución normal son independientes y de igual varianza. 
Ídem pero suponiendo que las varianza son distintas.%\sol{$\left(-2.09,8.09\right)$}
\end{prob}

\begin{prob}
Para comparar la dureza media de dos tipos de aleaciones (tipo 1 y tipo 2) se hacen 5
pruebas de dureza  con la de tipo 1 y 7 con la de tipo 2. Obteniéndose los resultados
siguientes:
$$\overline{X}_{1}=18.2,\quad S_{1}=0.2 \mbox{ y}$$

$$\overline{X}_{2}=17.8;\quad S_{2}=0.5$$

Suponer que la población de las durezas es normal y que las desviaciones típicas no son
iguales. Hacer lo mismo  si las varianzas son distintas.
% , buscar un intervalo de confianza para $\mu_{1}-\mu_{2}$ con  probabilidad $0.95$
% \sol{$\left(0.314,0.486\right)$}
\end{prob}
\newpage

\begin{prob} Se encuestó a dos muestras independientes de  internautas, una en Menorca y otra en Mallorca, sobre si utilizaban  telefonía por intente. La encuesta de Menorca tuvo un tamaño $n_1=500$  y 
$100$ usuarios mientras que  en Mallorca se encuestaron a $n_2=750$ y se obtuvo un resultado de $138$ usuarios.
\end{prob}

\begin{prob}
Se pregunta a un grupo de 100 personas elegido al azar asiste a una conferencia sobre tecnologías de la comunicación. Antes de la conferencia se les pregunta si consideran a internet peligrosa, después de la conferencia se les vuelve a preguntar cual es su opinión. Los resultados fueron los siguientes:

$$
\begin{tabular}{|c|c|cc|}
\cline{3-4}
\multicolumn{2}{c|}{}& \multicolumn{2}{|c|} {Después}\\\cline{3-4}
\multicolumn{2}{c|}{} & Sí es peligrosa & No es peligrosa \\\hline
Antes & Sí es peligrosa&  50 &  30 \\
& No es peligrosa  &  5 & 15 
\\\hline
\end{tabular}
$$




% \hline
% & Antes & Después \\\hline
% Sí es Peligrosa &  80 & 70\\
% No es Peligrosa  & 20 & 30 
% \\\hline
% \end{tabular}

\end{prob}


\begin{prob}  Tenemos $10$ ordenadores, deseamos optimizar su funcionamiento. Con este fin  se piensa en ampliar su memoria. Se les pasa una prueba de rendimiento antes  y después de  ampliar la memoria. Los resultados fueron:

\scriptsize{
\begin{tabular}{|c|llllllllll|}
\hline
&\multicolumn{10}{|c|}{Ordenador} \\\hline
Muestra\slash Tiempo & 1 & 2 & 3 & 4 & 5 & 6 & 7 & 8 & 9 & 10\\\hline
Antes ampliación & 98.70 & 100.48 & 103.75 & 114.41 & 97.82&
91.13 & 85.42 & 96.8 & 107.76 & 112.94\\
\hline
Después ampliación & 99.51 & 114.44 & 108.74 & 97.92 & 103.54&
104.75 & 109.69 & 90.8 & 110.04 & 110.09\\

\hline
\end{tabular}
}
\end{prob}


\begin{prob} Las siguientes muestras provienen de dos poblaciones independientes y supuestamente normales.  Se desea comparar la igualdad de sus medias, pero antes debemos contrastar si podemos o no aceptar que sus varianza son iguales o distintas. Se pide hacer el contraste de las medias en el caso en que se se decida aceptar varianzas iguales o distintas al nivel de significación $\alpha=0.05$.





Contrastar también la hipótesis de igualdad de medias en el otro caso ( es decir si se decide varianzas distintas contrastar para iguales y viceversa).


\end{prob}


% 98.7 & 100.48 & 103.75 & 114.41 & 97.82
% 91.13 & 85.42 & 96.8 & 107.76 & 112.94
% 99.51 & 114.44 & 108.74 & 97.92 & 103.54
% 104.75 & 109.69 & 90.8 & 110.04 & 110.09

%%%%%%%%\footnote{Además tenéis que hacer los problemas de las páginas 250 a 252 y 259 y 260 del
%%%%%%%%Newbold.}



% \begin{prob}
%
%   Los candidatos para un puesto de trabajo en una gran empresa deben
% superar un examen de aptitud escrito y realizar una entrevista con el
% director de personal. Después de la entrevista, el director otorga a
% cada candidato una puntuación entre 0 y 10. Para comprobar la
% consistencia de los criterios de los distintos directores, se extrae
% una muestra de 10 pares de candidatos. La formación de las parejas se
% organiza de forma que en cada una de ellas los dos candidatos han
% obtenido una puntuación idéntica en el examen escrito.
% Posteriormente, la entrevista a cada uno de los candidatos de cada
% par es realizada por cada uno de los dos directores de personal,
%  Sr. José y Sra. Isabel. Las puntuaciones obtenidas fueron las
%  siguientes:
%
%
%  \begin{tabular}{l|cccccccccc}
%    Sr.. José & 80 & 65 & 87 & 64 & 73 & 78 & 83 & 91 & 84 & 83\\
%    \hline
%    Sra.. Isabel & 74 & 63 & 91 & 65 & 64 & 71 & 69 & 90 & 79 & 87
%    \end{tabular}.
%
%    \begin{enumerate}[a)]
%   \item  Especificando las hipótesis necesarias calcular un intervalo
%   de confianza del 95\% para la diferencia entre las dos puntuaciones
%   medias poblacionales obtenidas (para fijar ideas hacer Sr.. José - Sra.
%   Isabel).
% \item Sin realizar los cálculos, determinar si un intervalo de
% confianza al 99\% para la diferencia de las medias poblacionales
% tendría mayor, menor o la misma amplitud que en el apartado a).
% \end{enumerate}
%   \end{prob}


%\newpage


\textbf{Problemas de bondad de ajuste}
\begin{prob}
Una compañía de gas afirma, basándose en experiencias anteriores, que normalmente, al final
del invierno, el 80\% de las facturas han sido ya cobradas, un 10\% se cobrará con pago
aplazado a un mes, un 6\% se cobrará a 2 meses y un 4\% se cobrará a más de dos meses. Al
final del invierno actual, la compañía selecciona una muestra aleatoria de $400$ facturas,
resultando que 287 de estas facturas cobradas, 49 a cobrar en un mes, $30$ a cobrar en dos
meses y $34$ a cobrar en un periodo superior a dos meses. ¿Podemos concluir, a raíz de los
resultados, que la experiencia de años anteriores se ha vuelto a repetir este invierno?
\end{prob}

\begin{prob}
El Rector de una Universidad opina que el $60\%$ de los estudiantes consideran los cursos
que realizan como muy útiles, el $20\%$ como algo útiles y el $20\%$ como nada útiles. Se
toma una muestra aleatoria de $100$ estudiantes, y se les pregunta sobre la utilidad de los
cursos. Resultando que $68$ estudiantes consideran que los cursos son muy útiles, $18$
consideran que son poco útiles y $14$ consideran que no son nada útiles. Contrastar la
hipótesis nula de que los resultados obtenidos se corresponden con la opinión personal del
Rector.
\end{prob}

\begin{prob}
Considérense los fondos de inversión ordenados en función de su rendimiento en el periodo
1995-99. Se realizó un seguimiento del rendimiento en los cinco años posteriores de una
muestra aleatoria de 65 fondos entre el $25\%$ más rentable del periodo 1995-99. En este
segundo periodo se observó que $11$ de los fondos de la muestra se hallan entre el $25\%$
más rentable en este segundo periodo, $17$ en el segundo $25\%$, $18$ en el tercer 25\% y
$19$ en el $25\%$ menos rentable. Contrastar la hipótesis de que un fondo de inversión
escogido azar del $25\%$ más rentable en 1995-99 tenga la misma probabilidad de hallarse en
cualquiera de las cuatro categorías de rendimiento en el periodo 2000-2004.
\end{prob}

\begin{prob}
A una muestra aleatoria de $502$ consumidores se les preguntó la
importancia que se le daba al precio a la hora de elegir un
ordenador. Se les pidió que valoraran entre: ``ninguna
importancia'', ``alguna importancia'' y ``principal importancia''. El
número respectivo de respuestas en cada tipo fueron 169, 136 y 197.
Contrastar la hipótesis nula de que la probabilidad de que un
consumidor elegido al azar conteste cualquiera de las tres
respuestas es la misma.
\end{prob}


\begin{prob}
Durante cien semanas se ha venido observando el número de veces a la semana que se ha fuera
de servicio  un servidor de una pequeña empresa de informática, presentándose los
resultados de la siguiente tabla:

\begin{tabular}{|l|cccccc|}
\hline
Núm. Fuera Servicio & 0 & 1 & 2 & 3 & 4 & 5 o más\\
\hline
Núm. Semanas & 10 & 24 & 32 & 23 & 6 & 5\\
\hline
\end{tabular}

El número medio de veces que quedo fuera de servicio por semana durante este periodo fue de
$2.1$. Contrastar la hipótesis nula de que la distribución de averías es una Poisson.
\end{prob}


\begin{prob}
A lo largo de $100$ minutos, llegaron a una web de un periódico $100$
internautas. La siguiente tabla muestra la frecuencia de llegadas a
lo largo de ese intervalo de tiempo.

\begin{tabular}{|l|ccccc|}
\hline
Núm. llegadas/mín. & 0 & 1 & 2 & 3 & 4 o más\\
\hline
Frec. Observada & 10 & 26& 35 & 24 & 5 \\
\hline
\end{tabular}

Contrastar la hipótesis nula de
que la distribución es Poisson.
\end{prob}




\textbf{KS test.}

\begin {prob} Se quiere saber si el tiempo entre accesos, en una determinada franja horaria, a  una cierta página web sigue una ley exponencial. Se dispone de la siguiente muestra de 25 intervalos entre tiempos de acceso:

\begin{knitrout}
\definecolor{shadecolor}{rgb}{0.969, 0.969, 0.969}\color{fgcolor}\begin{kframe}
\begin{alltt}
\hlstd{x}\hlkwb{=}\hlkwd{c}\hlstd{(}\hlnum{140.7}\hlstd{,}\hlnum{13.7}\hlstd{,}\hlnum{67.6}\hlstd{,}\hlnum{7.8}\hlstd{,}\hlnum{49.3}\hlstd{,}\hlnum{128.5}\hlstd{,}\hlnum{59.6}\hlstd{,}\hlnum{234}\hlstd{,}\hlnum{171.1}\hlstd{,}\hlnum{205.8}\hlstd{,}\hlnum{99.3}\hlstd{,}\hlnum{199.8}\hlstd{,}
    \hlnum{100.8}\hlstd{,}\hlnum{13.5}\hlstd{,}\hlnum{12}\hlstd{,}\hlnum{33.9}\hlstd{,}\hlnum{44.1}\hlstd{,}\hlnum{12.3}\hlstd{,}\hlnum{56.4}\hlstd{,}\hlnum{9.4}\hlstd{,}\hlnum{112.1}\hlstd{,}\hlnum{8.2}\hlstd{,}\hlnum{110.5}\hlstd{,}\hlnum{79}\hlstd{,}\hlnum{55.4}\hlstd{)}
\hlstd{x}
\end{alltt}
\begin{verbatim}
##  [1] 140.7  13.7  67.6   7.8  49.3 128.5  59.6 234.0 171.1 205.8  99.3
## [12] 199.8 100.8  13.5  12.0  33.9  44.1  12.3  56.4   9.4 112.1   8.2
## [23] 110.5  79.0  55.4
\end{verbatim}
\end{kframe}
\end{knitrout}

Resolver realizando los cálculos de forma manual y utilizando funciones de R las siguientes cuestiones.

\begin{enumerate}[a)]
\item  Contrastar la hipótesis de que la distribución  sigue una ley exponencial de parámetro $\lambda=100$, al nivel $\alpha=0.05$.
\item  Contrastar la hipótesis de que la distribución  sigue una ley Poisson de parámetro $\lambda=105$, al nivel $\alpha=0.05$.
\item  Contrastar la hipótesis de que la distribución  sigue una ley Poisson de parámetro $\lambda=110$, al nivel $\alpha=0.05$.
\item  Contrastar la hipótesis de que la distribución  sigue una ley Poisson, estimando el parámetro parámetro  partir de la muestra, al nivel $\alpha=0.05$.
\end{enumerate}
\end{prob}

\begin{prob}
Resolver las mismas cuestiones que en el problema anterior para la muestra (decir si se viola algunas de las condiciones del test KS, pero resolver igualmente el ejercicio):

$$69.9,  31.5, 130.2,  80.5, 236.1, 151.2,  74.8,  13.8,  54.5, 147.6$$

En esta ocasión realizar los cálculos manualmente.

\end{prob}

\begin{prob}
Nos hemos bajado un generador de números aleatorios normales de Internet. Queremos contrastar si funciona correctamente. Para ello generamos una muestra de 10 números aleatorios de una normal estándar:

$$ -1.18, -0.77, -0.59, -0.27, -0.12,  0.27,  0.29,  0.40,  1.27,  1.60
$$

\begin{enumerate}[a)]
\item Contrastar si provienen de una normal estándar al nivel de significación $\alpha=0.05$ mediante el test KS. Decir si ha violado alguna de las suposiciones de este test.
\item Contrastar la hipótesis de normalidad contra una distribución normal de media y varianza la estimadas a partir de la muestra.
\end{enumerate}

\end{prob}

\begin{prob} Con la muestra:

$$0.60, -1.42,  1.05, -0.14,  0.57,  0.11, -0.59, 1.11, -1.55, -1.41
$$

Contrastar con un test KS si los datos provienen de una distribución uniforme en el intervalo $(-2,2)$ al nivel $\alpha=0.05$
\end{prob}




%    \begin{prob}
%
%   El estadístico de \emph{Bowman-Seldon} es:
%
%   $B=n\left(\frac{(\mbox{Coefiente de
%   asimetría})^2}{6}+\frac{(\mbox{Curtosis}-3)^2}{24}\right)$
%
%   Que para un tamaño muestral suficientemente grande sigue
%   aproximadamente una distribución muestral $\chi^2$ con 2 g.l.
%
%   Criterio de rechado para \newline
%   $\left\{\begin{array}{l}
%   H_{0}:\mbox{La distribución poblacional es
%   normal}\\
%   H_{1}:\mbox{La distribución poblacional no es
%   normal}\end{array}\right. $
%
%   es
%
%   Rechazar $H_{0}$ al nivel de significación $\alpha$ si\newline
%   $B>\chi_{2,\alpha}^2$
%
%     Resolver, utilizando el estadístico de \emph{Bowman-Seldon} el
%     siguiente problema.
%
%       Durante 268 días  escogidos al azar se observaron los beneficios de
%       un contrato de futuro de cerdos y se observó un coeficiente de
%       asimetría de $0.04033$ y una curtosis $3.15553$ en la muestra. ¿Es
%       la distribución de os beneficios normal?
% %         \textbf{Solución}
% %         El
% %         valor del estadístico de \emph{Bowman-Shelton} es:\newline
% %         $B=n\left(\frac{(\mbox{0.04033})^2}{6}+
% %         \frac{(\mbox{3.15553}-3)^2}{24}\right)=0.36$
% %
% %     Si tomamos un nivel $\alpha=0.1$ tenemos que $\chi_{2,0.1}=4.61$
% %     y como $B=0.36\not>4.16$ no podemos rechazar $H_{0}$ a este nivelk
% %     de significación.
% %
% %         \end{example}


\end{document}
